\section{线性回归分析部分}\label{SecLinearRegressionAnalysis}
\begin{point}
    Steps in Regression Analysis
\end{point}

\begin{enumerate}[topsep=2pt,itemsep=2pt]
    \item Exploratory Data Analysis (EDA)\index{EDA (Exploratory Data Analysis)}
    \item Statement of the problem;
    \item Selection of potentially relevant variables;
    \item Data collection;
    \item Model specification;
    \item Choice of fitting method;
    \item Model fitting;
    \item Model validation and criticism;
    \item Using the chosen model(s) for the solution of the posed problem.
\end{enumerate}

    

\subsection{Linear Regression Model}
\begin{itemize}[topsep=6pt,itemsep=4pt]
    \item Assume a Model
    \begin{enumerate}[topsep=6pt,itemsep=4pt]
        \item Parameter of the model
        \item Basic Assumptions
        \item Dsitribution of error
    \end{enumerate}
    \item Parametric Estimation
    \begin{enumerate}[topsep=6pt,itemsep=4pt]
        \item Ordinary Least Squares Estimation
        \item Maximun Likelihood Estimation
    \end{enumerate}
    \item Statistics Inference
    \begin{enumerate}[topsep=6pt,itemsep=4pt]
        \item Hypotheses Testing
        \item Interval Estimation
    \end{enumerate}

\end{itemize}


    


        
\subsubsection{Data for simple linear regression}

    We will observe pairs of variables, called 'cases'(样本点)
        
    A sample is $ (X_1,Y_1),\ldots,(X_n,Y_n) $

    Linear Model: \footnote{Here in linear regression, we consider $ X_i $ only as real number, \textbf{without} randomness. So here $ Y_i $ can be considered as an r.v. with $ X_i $ as parameter, i.e. $ Y_i|_{X_i=x_i} $}

%% 关于线性模型的X_i性质的假定
    \[
        Y_i=\beta _0+\beta _1X_i+\varepsilon _i 
    \]
    where $ \varepsilon_i  $ i.i.d.$ \sim \varepsilon  $ is a random error term, satisfies    \footnote{Note: Why we need $ \varepsilon $ as 'random error term'?
    \begin{itemize}[topsep=6pt,itemsep=4pt]
        \item It represents the intrinsic random property of the model.
        \item Based on $ \varepsilon  $, we can take r.v. into our statistic model.
    \end{itemize}
    }
    
    \[
        E(\varepsilon _i)=0\qquad var(\varepsilon _i)=\sigma ^2 
    \]
    
    

    Normal Error Assumption: Further in most cases, we consider $ \varepsilon \sim N(0,\sigma^2) $ ----because of its well-property distribution, $ \varepsilon _1,\varepsilon _2,\ldots,\varepsilon _n $ i.i.d. $ N(0,\sigma ^2) $.\footnote{i.e. $ Y_i $ are independent
    \[
        Y_i\sim N(\beta _0+\beta _1X_i,\sigma^2)\quad i=1,2,\ldots ,n 
    \]
    
    }
        
    What does Linear Regression do? Under Linear Model, try to estimate 
    \begin{itemize}[topsep=0pt,itemsep=-2pt]
        \item $ \beta _0\text{ (intercept) }$;
        \item $\beta _1\text{ (slope) }$;
        \item $\sigma ^2\text{ (variance of error)} $.
    \end{itemize}
    
    
    (Thus Linear Regression is also a Statistics Inference process: deduce properties of model from data)
        
\subsubsection{The Ordinary Least Square Estimation}
    Aim: use $ (x_i,y_i) $  to estimate $ \beta _0,\beta _1,\sigma^2 $. The idea is to define a 'loss function' to reflect the 'distance' from sample point to estimation point.

    Estimate Principle: \footnote{Detailed Definition and Derivation see sec.\ref{SubSectionMoM_MLE_LinearRegression}.}
    \begin{itemize}[topsep=2pt,itemsep=2pt]
        \item Ordinary Least Squares\index{Ordinary Least Squares}:
        \[
            (\hat{\beta  }_0,\hat{\beta _1})=\arg\min\sum_{i=1}^n (y-\beta _0-\beta _1x_i)^2
        \]
        \item MLE or MoM Estimation.
    \end{itemize}
    

    
    And get $ \hat{\beta _1},\hat{\beta _0}$ as well as $ \hat{\sigma^2} $(see eqa(\ref{EqaOLSEstimatorOfSigma}):\footnote{A memory trick: use $ \dfrac{Y}{\sqrt{s_Y}}=r_{XY}\dfrac{X}{\sqrt{s_X}} $ to get formular of $ Y\sim X $:
    \[
        \hat{\beta }_1=r_{XY}\dfrac{\sqrt{s_Y}}{\sqrt{s_X}}=\dfrac{{\displaystyle\sum (x_i-\bar{x})(y_i-\bar{y})}}{{\displaystyle\sum (x_i-\bar{x})^2}} 
    \]}

%LSE beta_0 beta_1
\begin{equation}\label{EqaOLSEstimatorOfBeta}
    \begin{aligned}
        \hat{\beta }_1=&\dfrac{\sum\limits_{i=1}^n (x_i-\bar{x})(y_i-\bar{y})}{\sum\limits_{i=1}^n (x_i-\bar{x})^2}\\
        \hat{\beta }_0=&\bar{y}-\hat{\beta _1}\bar{x}\\
        \hat{\sigma^2}=&\dfrac{1}{n-p-1}\sum_{i=1}^n(y_i-\hat{\beta }_0-\hat{\beta }_1x_i)^2
    \end{aligned}
\end{equation}


    
    Def. \index{Residual}\textbf{Residual}: distance from sample point to estimate point, to reflect how the sample points fit the model.
    \[
        e_i=y_i-\hat{y}_i=\text{observed value of }\varepsilon _i 
    \]
    
    Note: under least square estimation, we have\footnote{Intuitively, they each means '$ E(\varepsilon )=0 $' and '$ X\parallel \varepsilon  $'.}
\begin{equation}\label{Limit_to_Residual}
        \sum e_i=0\qquad \sum x_ie_i=0 
\end{equation}
    

    Then use $ e_i $ to estimate $ \sigma ^2 $ (because it is $ \varepsilon _0 $ that are i.i.d., not $ Y_i $), where $ (n-p-1) $ is Degree of Freedom (df or dof)\footnote{Generally, MLE and LSE are different.

    Comment from R.A.Fisher: $ \sum e_i^2 $ should be divided by 'number of $ e_i^2 $ that contribute to variance'. Here $ (n-p-1) $ corresponds to 'degree of freedom' $ =(n-2) $, $ p=1 $ corresponds to `one' variable (see sec.\ref{SubSectionMoM_MLE_LinearRegression}, eqa(\ref{EqaEstimatorSigmaWithDoF})), and correponds to the two equations of $ e_i $, eqa(\ref{Limit_to_Residual})}
\begin{equation}\label{EqaOLSEstimatorOfSigma}
    \begin{aligned}
        \hat{\sigma _n^2}&=\dfrac{1}{n}\sum e_i^2 \quad\text{(use MLE or MoM)}\\
        \hat{\sigma^2}&=\dfrac{1}{n-p-1}\sum e_i^2=\dfrac{1}{n-2}\sum e_i^2\quad\text{(use OLS, unbiased)}
\end{aligned}
\end{equation}

    % MSE SSE dof






    % Review: Statistical Inference
    % \begin{itemize}[topsep=6pt,itemsep=4pt]
    %     \item Basic concepts: HT CI;
    %     \item Inference about $ \beta _1$;
    %     \item Inference about $ \beta _0 $.
    % \end{itemize}

    % Note: the distribution of $ \hat{\beta }_0,\hat{\beta }_1 $ is sampling distribution(抽样分布): distribution of statistics.



    % Power function of testing
    % \begin{itemize}[topsep=6pt,itemsep=4pt]
    %     \item Definition;
    %     \item Calculation;
    %     \item Sample<->Power (Calculation of sampling).
    % \end{itemize}


\subsubsection{Statistical Inference to $ \beta _0 $,$ \beta _1 $}

\begin{point}
    Sampling Distribution of $ \hat{\beta} _1,\hat{\beta} _0  $
\end{point}

    Consider $ \hat{\beta} _1,\hat{\beta} _0 $ as statistics of sample, then we can examine the sampling distribution of $  \hat{\beta} _1,\hat{\beta} _0 $. Their randomness comes from
    \[
        Y_i=\beta_0+\beta_1X_i+\varepsilon _i 
    \]
    
    

    (The following part treats $\hat{\beta} _1,\hat{\beta} _0 $ as r.v., and note that $ X_i $ are \textbf{not }r.v.. And  for convenience and conciseness, denote $ S_{XX}={\displaystyle\sum_{i=1}^n(X_i-\bar{X})^2} $)

   
\begin{align*}
        \hat{\beta }_1&=\beta _1+\sum_{i=1}^n\dfrac{X_i-\bar{X}}{S_{XX}}\varepsilon _i\\
        \hat{\beta }_0&=\beta _0+\sum_{i=1}^n\left(\dfrac{1}{n}-\dfrac{(X_i-\bar{X})\bar{X}}{S_{XX}}\right)\varepsilon _i
\end{align*}
 
    Denote corresponding variance as $ \sigma^2_{\hat{\beta}_1} $ and $ \sigma^2_{\hat{\beta}_0} $, using eqa(\ref{EqaDistributionOfSumOfiidNormal}) to get:
    \[
        \sigma^2_{\hat{\beta}_1}= \dfrac{\sigma^2}{S_{XX}}\qquad \sigma^2_{\hat{\beta}_0}=\sigma^2(\dfrac{1}{n}+\dfrac{\bar{X}^2}{S_{XX}})
    \] 
    
     And under normal error assumption, distribution of $ \hat{\beta} _1,\hat{\beta} _0  $ are
    \begin{align*}
        \hat{\beta }_1&\sim N(\beta _1,\sigma^2_{\hat{\beta}_1}) =N(\beta_1,\dfrac{\sigma^2}{S_{XX}})\\
        \hat{\beta}_0&\sim N(\beta_0,\sigma^2_{\hat{\beta }_0}) =N(\beta_0,\sigma^2(\dfrac{1}{n}+\dfrac{\bar{X}^2}{S_{XX}}))
    \end{align*}
    
    Based on sampling distribution of $ \hat{\beta} _1,\hat{\beta} _0  $, we can conduct statistical inference, including CI and HT.\footnote{Detail see sec.\ref{SectionHypothesisTesting}, estimating/testing $ \hat{\beta} _1,\hat{\beta} _0  $ usually corresponds to 'estimate $ \mu $, with $ \sigma^2 $ unknown'.}
    
    % \begin{itemize}[topsep=2pt,itemsep=2pt]
    %     \item LSE of $ \beta _1 $ gives 
    %     \[
    %         \hat{\beta _1}=\dfrac{\sum (x_i-\bar{x})(y_i-\bar{y})}{\sum (x_i-\bar{x})^2}
    %     \]
        
    %     and satisfies $ E(\hat{\beta}_1)=\beta_1 $. Can prove that $ \hat{\beta }_1\sim N(\beta _1,\dfrac{\sigma ^2}{\sum (x_i-\bar{x})^2})=N(\beta_1,\sigma^2(\hat{\beta}_1))$
       
    % \end{itemize}
    
    Note: In linear regression model, we usually focus more on $ \beta_1 $. And note that when $ 0 $ is \textbf{not} within the fitting range,$ \beta_0 $ is not so important.\footnote{Two reason:\begin{itemize}[topsep=2pt,itemsep=2pt]
        \item The etimation error of $ Y $ from $ \hat{\beta}_1 $ increases with $ X_h-\bar{X} $;
        \item $ \beta_1==0  $ is important: decides whether linear model can be used. 
    \end{itemize}
    
        }

    Why we choose OLS to get regression coefficients?

\begin{point}
    \index{Gauss-Markov Thm.`'}Gauss–Markov Thm.: the OLS estimator has the lowest sampling variance within the class of linear unbiased estimators, i.e. OLS is the Best Linear Unbiased Estimator(BLUE).\footnote{This Thm. does \textbf{not }require normal error assumption.}
\end{point}
    



\subsubsection{Prediction to $ Y_h $}
    For a new $ X_h $ at which we wish to \textbf{predict }the corresponding $ Y_h $ (based on other known point $ (X_i,Y_i) $), denote the estimator as $ \hat{\mu}_h $:
    \[
        \hat{\mu}_h=\hat{\beta}_1X_h+\hat{\beta}_0 =\beta_1X_h+\beta _0+\sum_{i=1}^n\left( \dfrac{1}{n}+\dfrac{(X_i-\bar{X})(X_h-\bar{X})}{S_{XX}} \right)\varepsilon _i
    \]
    
    Thus we can get\footnote{So $ \sigma ^2(\hat{\mu }_h) $ increases with $ X_h-\bar{X} $. Intuitively it make sense, because $ (\bar{X},\bar{Y})$ must falls on regression line.}
    \[
        E(\hat{\mu}_h)= \beta _1X_h+\beta _0\qquad \sigma ^2_{\hat{\mu}_h}=\left( \dfrac{1}{n}+\dfrac{(X_h-\bar{X})^2}{S_{XX}} \right)\sigma^2
    \]
    
    Under Normal assumption:
    \[
        \hat{\mu}_h\sim N(\beta _1X_h+\beta _0,\left( \dfrac{1}{n}+\dfrac{(X_h-\bar{X})^2}{S_{XX}} \right)\sigma^2) 
    \]
    
    Base on distribution we can give CI and HT.

    Note: Remember that when we consider the estimator $ \hat{\mu } $, we \textbf{must } have the randomness of $ \hat{\beta }_0,\hat{\beta }_1 $ considered(if they are unknown).
    
    Prediction Error: $ Y_h $ itself is an $ Y $ of the linear model, i.e. $ Y_i=\beta_0+\beta_1X_h+\varepsilon _h $, we can consider \textbf{$ Y_h $ itself as an r.v. }v.s.\textbf{ predicted $ Y_h $ from other sample points} and define \textbf{Prediction Error}: 
    \[
        d_h=Y_h-\hat{\mu}_h 
    \]

    
    \[
        E(d_h)=0\qquad \sigma^2_{d_h}=var(Y_h-\hat{\mu }_h)=\sigma^2\left[ 1+\dfrac{1}{n}+\dfrac{(X_h-\bar{X})}{S_{XX}} \right] > \sigma ^2_{\hat{\mu}_h}
    \]
    


    \begin{point}
       Simultaneous Confidence Band(SCB)\index{SCB (Simultaneous Confidence Band)}\index{CB (Confidence Band)}
    \end{point}

    Confidence Band is \textbf{not} the CI at each point, but really a \textbf{band} for the \textbf{entire} regression line.\footnote{Why they are different? We require the confidence band have a \textbf{simultaneous} converage probability. For the same band $ (L(x),U(x)) $, $ P(\text{the whole line})< P(\text{each point})$, so Confidence Band is wider than $ \bigcup $CIs to hold the same $ 1-\alpha $.
    
    Also, we will see that for linear model, the boundary of SCB forms hyperbola, which make sense considering its asymptotic line.}
    
    
    Aim: Find lower and upper function $ L(x) $ and $ U(x) $ such that
    \[
        P[L(x)<(\beta _0+\beta _1x)<U(x),\,\forall x\in I_x]=1-\alpha  
    \]
    
    and get \textbf{Confidence Band}:
    \[
        \{(x,y)|L(x)<y<U(x)|\forall x\in I_x\} 
    \]
    
    % Note: \textbf{Cannot} use CI at each point to form Confidence Band. Band is wider. And we are actually conduce CI \textbf{simoutanesly} to all $ x $.

    Where $ (L(x),U(x)) $ can be derived as
    \[
        (L(x),U(x))=\hat{\mu}_x\pm s_{\hat{\mu}_x}W_{2,n-2,1-\alpha}
    \]

    Where $ W $ correponds to $ W $ distribution: $ W_{m,n}=\sqrt{2F_{m,n}} $
    
    
    
    Small sample case: Bonferroni correction.
    




% 
% 
% 
% 
% 
% 
% 


\subsection{Analysis of Variance}
    \index{ANOVA (Analysis of Variance)}\textbf{AN}alysis \textbf{O}f \textbf{VA}riance (ANOVA): One-sample $ t $ test$ \rightsquigarrow $ Two sample $ t $ test$ \rightsquigarrow $ ANOVA
\begin{itemize}[topsep=2pt,itemsep=2pt]
    \item Partition of Totla Sum of Squares;
    \item Partition of Degree of Freedom;
    \item MSS$ \rightsquigarrow $ F-test;
    \item ANOVA table;
    \item General linear test. --to be examined further in later sections.
    \item (Pearson) Correlation Coefficient $ \leftrightarrow \, R^2$
\end{itemize}

    SST: Total Sum of Squares\index{SST (Total Sum of Squares)}
    \[
        \mathrm{SST}=\sum_{i=1}^n(Y_i-\bar{Y})^2 
    \]
    
    Note: Here $ Y_i $ are not i.i.d. (different mean).

    Idea: take partition of SST. For instance
    \[
        Y_i-\bar{Y}=(Y_i-\hat{Y})+(\hat{Y}-\bar{Y})=e_i 
    \]
    
    Note: $ \bar{Y}=\bar{\hat{Y}} $

    then we partition SST into\footnote{\textbf{IMPORTANT: }In some books \begin{itemize}[topsep=2pt,itemsep=2pt]
        \item SSError $ \to $ SSResidual;
        \item SSRegression $ \to $ SSExplained.
    \end{itemize}

    And Cause \textbf{Confusion}! In this summary we take the former.
         }

         \begin{itemize}[topsep=2pt,itemsep=2pt]
        \item variation due to model \index{SSR (Regression Sum of Squares)}(SSRegression) (which is explained by regression line);
        \[
            \mathrm{SSR}= \sum_{i=1}^n(\hat{Y}_i-\bar{Y})^2
        \]
        
        \item variation attribtes to $ \varepsilon  $ \index{SSE (Error Sum of Squares)}(SSError).
        \[
            \mathrm{SSE}= \sum_{i=1}^n(Y_i-\hat{Y_i})
        \]
        
        
    \end{itemize}
    
    can prove
    \[
        \mathrm{SST}=\sum_{i=1}^n(Y_i-\bar{Y})^2=\sum_{i=1}^n(\hat{Y}_i-\bar{Y})^2+\sum_{i=1}^n(Y_i-\hat{Y_i})^2=\mathrm{SSR+SSE} 
    \]

    That is: we \textbf{partition} SST into two parts, so that we can examine them seperately.

    \begin{point}
        ANOVA Table\footnote{$ \mathrm{SSR}=\hat{\beta }_1^2\sum_{i=1}^n(X_i-\bar{X})^2$, so $ dof_R=1 $}
    \end{point}
    \begin{table}[H]
        \centering
        \renewcommand\arraystretch{1}
        \begin{tabular}{c|ccc}
            \hline
            Source&$ dof $&SS&MS\\
            Regression&1&$ \sum_{i=1}^n(\hat{Y}_i-\bar{Y})^2  $&SSR/$ dof_R $\\
            Error&$ n-2 $&$ \sum_{i=1}^n(Y_i-\hat{Y}_i)^2  $&SSE/$ dof_E $\\
            Total&$ n-1 $&$ \sum_{i=1}^n(Y_i-\bar{Y})^2  $&SST/$ dof_T $\\
            \hline
        \end{tabular}
    \end{table}
    
    Properties:
    
    \[
        E(\mathrm{MSE})=\sigma ^2\qquad E(\mathrm{MSR})=\sigma ^2+\beta _1^2S_{XX} 
    \]
    
\begin{point}
    Hypotheses Testing to $ H_0:\beta _1=0 $
\end{point}
\begin{itemize}[topsep=2pt,itemsep=2pt]
    \item 
    We can examine $ F=\dfrac{\mathrm{MSR}}{\mathrm{MSE}}\sim F_{dof_R,dof_E}=F_{1,n-2} $
    \item    Or: General Linear Test (GLT)\index{GLT (General Linear Test)}\begin{itemize}[topsep=2pt,itemsep=2pt]
        \item Full model: $ Y_i=\beta _0+\beta _1X_i+\varepsilon _i $.
        \item Reduced model: $ Y_i=\beta _0+\varepsilon _i $.
    \end{itemize}

    and examine
    \[
        F=\dfrac{(\mathrm{SSE_R-SSE_F})/dof_{R-F} }{\mathrm{SSE_F}/dof_F} \sim F_{dof_{R}-dof_F,dof_F}
    \]
\end{itemize}

\begin{point}
    Pearson Correlation Coefficient $ R^2 $
\end{point}

\[
    R^2=\dfrac{\mathrm{SSR}}{\mathrm{SST}} 
\]

    Note: under simple linear model, $ r^2=R^2 $, where $ r=\hat{\beta}_1\dfrac{\sigma _X}{\sigma _Y} $



\subsection{Model Assumption and Diagnostics}
    
\begin{point}
    Diagonostics to $ X $
\end{point}


    Considering the dependence of $ Y_i $ on $ X_i $, we cannot just focus on the (marginal) distribution of $ Y_i $. Thus we also need a better 'distribution' of $ X_i $
    \begin{itemize}[topsep=2pt,itemsep=2pt]
        \item 4 statistics(parameters);\footnote{See sec.\ref{SubSectionStatistics}}
        \begin{itemize}[topsep=2pt,itemsep=2pt]
            \item Mean: Location;
            \[
                \bar{X}=\dfrac{1}{n}\sum_{i=1}^nX_i 
            \]
            \item Standard Deviation: Variability;
            \[
                S^2=\dfrac{1}{n-1}\sum_{i=1}^n(X_i-\bar{X}) ^2
            \]
            
            
            \item Skewness: Lack of Symmertry;
            \[
                \hat{g}_1=\dfrac{m_{n,3}}{m_{n,2}^{3/2}}=\dfrac{\frac{1}{n}\sum\limits_{i=1}^n(X_i-\bar{X})^3}{\left( \frac{1}{n}\sum\limits_{i=1}^n(X_i-\bar{X}) \right)^{3/2}} 
            \]

            Adjusted Skewness (Least MSE):
            \[
                \dfrac{\sqrt{n(n-1)}}{n-2}\hat{g}_1 
            \]
            
            \begin{itemize}[topsep=2pt,itemsep=2pt]
                \item $ \hat{g}_1>0 $: Right skewness, longer right tail;
                \item $ \hat{g}_1<0 $: Left skewness, longer left tail.
            \end{itemize}
            
                
            Fisher-Pearson coefficient of skewness.


            \item Kurtosis: Heavy/Light Tailed.
            \[
                \hat{g}_2=\dfrac{m_{n,4}}{m_{n,2}^2}-3= \dfrac{\frac{1}{n}\sum\limits_{i=1}^n(X_i-\bar{X})^4}{\left( \frac{1}{n}\sum\limits_{i=1}^n(X_i-\bar{X}) \right)^{2}} -3
            \]

            $ \hat{g}_2=0 \Rightarrow $ similar to normal.
            \begin{itemize}[topsep=2pt,itemsep=2pt]
                \item $ \hat{g}_2>0 $: Leptokurtic, heavy tail slender;
                \item $ \hat{g}_2<0 $: Platykurtic, light tail  broad.
            \end{itemize}
            
            Note: In expression of $ \hat{g}_1 $ and $ \hat{g}_2 $, we already divide the variance. So Skewness and Kurtosis only reflect the difference from normal, but not related to variance!
                
            Best tool to determine Kurtosis: QQ-Plot.
            
        \end{itemize}
        \item Useful Plots:
        \begin{itemize}[topsep=2pt,itemsep=2pt]
            \item BoxPlot: a rough distribution.
            
            25\%-quantile$ - $ 1.5IQR$ \vdash \sqsubset  $ 25\%-quantile$ \square  $ 75\%-quantile $ \sqsupset \dashv  $ 75\%-quantile$ + $ 1.5IQR\footnote{IQR:InterQuartile Range\index{IQR (InterQuartile Range)}}

            \item Histogram Plots: Frequency distribution (can deal with many-peak)
            \item Quantile-Quantile Plots\index{QQ-Plot (Quantile-Quantile Plots)}: Examine the similarity  between distribution.
            
            For two CDF $ q=F(x) $ and $ q=G(x) $(where $ q $ for quantile), with $ x=F^{-1}(q) $, $ x=G^{-1}(q) $. And Plot $ F^{-1}(q) $-$ G^{-1}(q) $.

            Usually test normality, take $ G=\Phi  $
        \end{itemize}
        
            
        \item Normality;
        \item Bias:
        \begin{itemize}[topsep=2pt,itemsep=2pt]
            \item Selection Bias: Not completely random sampling;
            \item Information Bias: Difference between 'designed' and 'get', e.g. no response;
            \item Confounding: Exist another important variable, while the model actually focuses on a less important variable, or even reverse the causality.
        \end{itemize}
        
            
    \end{itemize}
    
\begin{point}
    Diagnostics to Residual
\end{point}

    Residual Plot: Reflect the linearity and variance assumption

    Testing of 
\begin{itemize}[topsep=2pt,itemsep=2pt]
    \item The Assumption of Equal Variances:
    \begin{itemize}[topsep=2pt,itemsep=2pt]
        \item Bartlett's test: Comes from UMPT, useful when normality assumption satisfied.
        \item Levene's test: 
        \item Brown-Forsythe test (Modified Levene's test): 
        \item Breusch-Pagan test:
    \end{itemize}


    \item The Assumption of Normality:
    \begin{itemize}[topsep=2pt,itemsep=2pt]
        \item Shapiro-Wilk Test (Most Powerful):
    \end{itemize}
    


    \item The Assumption of Independence:
        
    
        
\end{itemize}

    

    