\section{线性回归分析部分}\label{SecLinearRegressionAnalysis}
\begin{center}
    Instructor: Zaiying Zhou
\end{center}
\begin{point}
    Steps in Regression Analysis
\end{point}

\begin{enumerate}[topsep=2pt,itemsep=2pt]
    \item Statement of the problem;
    \item Selection of potentially relevant \textbf{variables};
    \item Data collection;
    \item Exploratory Data Analysis (\textbf{EDA} )\index{EDA (Exploratory Data Analysis)}
    \item \textbf{Model} specification;
    \item Choice of fitting method;
    \item Model fitting;
    \item Model validation and criticism;
    \item Using the chosen model(s) for the solution of the posed problem;
    \item \textbf{Explain} the result.
\end{enumerate}

    \lstinline|R.| Code for EDA
\begin{lstlisting}[language=R]
libaray('GGally')
head(df)
ggpairs(df)
str(df)
summary(df)
\end{lstlisting}



\begin{point}
    Used Packages in \lstinline|R.|
\begin{lstlisting}[language=R]
library('ggplot2')
libaray('GGally')
library('car')
library('moments')
library('lmtest')
library('nortest')
library('MASS')
library('tseries')

source('package.r')
\end{lstlisting}

\end{point}


\subsection{Linear Regression Model}
% \begin{itemize}[topsep=6pt,itemsep=4pt]
%     \item Assume a Model
%     \begin{enumerate}[topsep=6pt,itemsep=4pt]
%         \item Parameter of the model
%         \item Basic Assumptions
%         \item Dsitribution of error
%     \end{enumerate}
%     \item Parametric Estimation
%     \begin{enumerate}[topsep=6pt,itemsep=4pt]
%         \item Ordinary Least Squares Estimation
%         \item Maximun Likelihood Estimation
%     \end{enumerate}
%     \item Statistics Inference
%     \begin{enumerate}[topsep=6pt,itemsep=4pt]
%         \item Hypotheses Testing
%         \item Interval Estimation
%     \end{enumerate}

% \end{itemize}


    


        
\subsubsection{Data and Model for Simple Linear Regression}

    We will observe pairs of variables, called 'cases'(样本点). A sample is $ (X_1,Y_1),\ldots,(X_n,Y_n) $

\begin{rcode}
    Example data import:
\begin{lstlisting}[language=R]
df <- read.table('dataset/CH01PR27.txt',header=FALSE,
    sep=',',col.names = c('y','x'))
\end{lstlisting}
\end{rcode}

    Linear Model: \footnote{Here in linear regression, we consider $ X_i $ only as real number, \textbf{without} randomness. So here $ Y_i $ can be considered as an r.v. with $ X_i $ as parameter, i.e. $ Y_i|_{X_i=x_i} $}
    \footnote{Note: Why we need $ \varepsilon $ as 'random error term'?
    \begin{itemize}[topsep=6pt,itemsep=4pt]
        \item It represents the intrinsic random property of the model.
        \item Based on $ \varepsilon  $, we can take r.v. into our statistic model.
    \end{itemize}
    }
    
%% 关于线性模型的X_i性质的假定
    \begin{equation}
        Y_i=\beta _0+\beta _1X_i+\varepsilon _i 
    \end{equation}

    with Guass-Markov Assumption:
    \begin{equation}\label{EqaGaussMarkovAssumption}
        \begin{aligned}
            \text{Zero-Mean: }&E(\epsilon_i|X_i)=0 \\
            \text{Homogeneity of Variance: }&var(\epsilon_i)=\sigma^2\\
            \text{Independent: }&\epsilon_i\text{ i.i.d. }\sim \varepsilon
        \end{aligned}
    \end{equation}
 
  

    Normal Error Assumption: Further in most cases, we consider $ \varepsilon \sim N(0,\sigma^2) $ ----because of its well-property distribution, $ \varepsilon _1,\varepsilon _2,\ldots,\varepsilon _n $ i.i.d. $ N(0,\sigma ^2) $.\footnote{i.e. $ Y_i $ are independent
    \begin{equation}
        Y_i\sim N(\beta _0+\beta _1X_i,\sigma^2)\quad i=1,2,\ldots ,n 
    \end{equation}
    
    }
        
    What does Linear Regression do? Under Linear Model, try to estimate 
    \begin{itemize}[topsep=0pt,itemsep=-2pt]
        \item $ \beta _0\text{ (intercept) }$;
        \item $\beta _1\text{ (slope) }$;
        \item $\sigma ^2\text{ (variance of error)} $.
    \end{itemize}
    
    
    (Thus Linear Regression is also a Statistics Inference process: deduce properties of model from data)
        
\subsubsection{The Ordinary Least Square Estimation}
    Aim: use $ (x_i,y_i) $  to estimate $ \beta _0,\beta _1,\sigma^2 $. The idea is to define a 'loss function' to reflect the 'distance' from sample point to estimation point.

    Estimate Principle: \footnote{Detailed Definition and Derivation see sec.\ref{SubSectionMoM_MLE_LinearRegression}.}
    \begin{itemize}[topsep=2pt,itemsep=2pt]
        \item Ordinary Least Squares\index{Ordinary Least Squares}:
        \begin{equation}
            (\hat{\beta  }_0,\hat{\beta _1})=\arg\min\sum_{i=1}^n (y-\beta _0-\beta _1x_i)^2
        \end{equation}
        \item MLE or MoM Estimation.
    \end{itemize}
    

    
    And get $ \hat{\beta _1},\hat{\beta _0}$ as well as $ \hat{\sigma^2} $(see eqa(\ref{EqaOLSEstimatorOfSigma}):\footnote{A memory trick: use $ \dfrac{Y}{\sqrt{s_Y}}=r_{XY}\dfrac{X}{\sqrt{s_X}} $ to get formular of $ Y\sim X $:
    \begin{equation}
        \hat{\beta }_1=r_{XY}\dfrac{\sqrt{s_Y}}{\sqrt{s_X}}=\dfrac{{\displaystyle\sum (x_i-\bar{x})(y_i-\bar{y})}}{{\displaystyle\sum (x_i-\bar{x})^2}} 
    \end{equation}}

%LSE beta_0 beta_1
\begin{equation}\label{EqaOLSEstimatorOfBeta}
    \begin{aligned}
        \hat{\beta }_1=&\dfrac{\sum\limits_{i=1}^n (x_i-\bar{x})(y_i-\bar{y})}{\sum\limits_{i=1}^n (x_i-\bar{x})^2}\\
        \hat{\beta }_0=&\bar{y}-\hat{\beta _1}\bar{x}\\
        \hat{\sigma^2}=&\dfrac{1}{n-p-1}\sum_{i=1}^n(y_i-\hat{\beta }_0-\hat{\beta }_1x_i)^2
    \end{aligned}
\end{equation}


    
    Def. \index{Residual}\textbf{Residual}: distance from sample point to estimate point, to reflect how the sample points fit the model.
    \begin{equation}
        e_i=y_i-\hat{y}_i=\text{observed value of }\varepsilon _i 
    \end{equation}
    
    Note: under least square estimation, we have\footnote{Intuitively, they each means '$ E(\varepsilon )=0 $' and '$ X\parallel \varepsilon  $'.}
\begin{equation}\label{Limit_to_Residual}
        \sum e_i=0\qquad \sum x_ie_i=0 
\end{equation}
    

    Then use $ e_i $ to estimate $ \sigma ^2 $ (because it is $ \varepsilon _0 $ that are i.i.d., not $ Y_i $), where $ (n-p-1) $ is Degree of Freedom (df or dof)\footnote{Generally, MLE and LSE are different.

    Comment from R.A.Fisher: $ \sum e_i^2 $ should be divided by 'number of $ e_i^2 $ that contribute to variance'. Here $ (n-p-1) $ corresponds to 'degree of freedom' $ =(n-2) $, $ p=1 $ corresponds to `one' variable (see sec.\ref{SubSectionMoM_MLE_LinearRegression}, eqa(\ref{EqaEstimatorSigmaWithDoF})), and correponds to the two equations of $ e_i $, eqa(\ref{Limit_to_Residual})}
\begin{equation}\label{EqaOLSEstimatorOfSigma}
    \begin{aligned}
        \hat{\sigma _n^2}&=\dfrac{1}{n}\sum e_i^2 \quad\text{(use MLE or MoM)}\\
        \hat{\sigma^2}&=\dfrac{1}{n-p-1}\sum e_i^2=\dfrac{1}{n-2}\sum e_i^2\quad\text{(use OLS, unbiased)}
\end{aligned}
\end{equation}

\textbf{Degree of Freedom}\index{$ dof $/$ df $ (Degree of Freedom)} of a Quadric Form:
\begin{itemize}[topsep=2pt,itemsep=2pt]
    \item Intuitively: the number of independent variable;
    \item Rigorously: for quadric $ \mathrm{SS}=x'Ax $:
    \begin{equation}\label{EqaDefinitionOfDegreeOfFreedom}
        dof_{SS}=\mathrm{rank}(A)
    \end{equation}
    
    
    
\end{itemize}

     

\begin{rcode}
\begin{lstlisting}[language=R]
lmfit <- lm(formula,df)
summary(lmfit,cor=TRUE)
ggcoef(lmfit)
\end{lstlisting}

    \lstinline|lmfit| includes parameters \lstinline|lmfit$coefficient| and \lstinline|lmfit$residuals|

    Example \lstinline|lm()| output:
\begin{lstlisting}[language=R]
    Call:
    lm(formula = y ~ x, data = df)
    
    Residuals:
         Min       1Q   Median       3Q      Max 
    -16.1368  -6.1968  -0.5969   6.7607  23.4731 
    
    Coefficients:
                Estimate Std. Error t value Pr(>|t|)    
    (Intercept) 156.3466     5.5123   28.36   <2e-16 ***
    x            -1.1900     0.0902  -13.19   <2e-16 ***
    ---
    Signif. codes:  0 '***' 0.001 '**' 0.01 '*' 0.05 '.' 0.1 ' ' 1
    
    Residual standard error: 8.173 on 58 degrees of freedom
    Multiple R-squared:  0.7501,    Adjusted R-squared:  0.7458 
    F-statistic: 174.1 on 1 and 58 DF,  p-value: < 2.2e-16
\end{lstlisting}

\end{rcode}

    % MSE SSE dof






    % Review: Statistical Inference
    % \begin{itemize}[topsep=6pt,itemsep=4pt]
    %     \item Basic concepts: HT CI;
    %     \item Inference about $ \beta _1$;
    %     \item Inference about $ \beta _0 $.
    % \end{itemize}

    % Note: the distribution of $ \hat{\beta }_0,\hat{\beta }_1 $ is sampling distribution(抽样分布): distribution of statistics.



    % Power function of testing
    % \begin{itemize}[topsep=6pt,itemsep=4pt]
    %     \item Definition;
    %     \item Calculation;
    %     \item Sample<->Power (Calculation of sampling).
    % \end{itemize}


\subsubsection{Statistical Inference to $ \beta _0 $,$ \beta _1 $,$ e_i $}

\begin{point}
    Sampling Distribution of $ \hat{\beta} _1,\hat{\beta} _0  $
\end{point}

    Consider $ \hat{\beta} _1,\hat{\beta} _0 $ as statistics of sample, then we can examine the sampling distribution of $  \hat{\beta} _1,\hat{\beta} _0 $. Their randomness comes from
    \begin{equation}
        Y_i=\beta_0+\beta_1X_i+\varepsilon _i 
    \end{equation}
    
    

    (The following part treats $\hat{\beta} _1,\hat{\beta} _0 $ as r.v., and note that $ X_i $ are \textbf{not }r.v.. And  for convenience and conciseness, denote $ S_{XX}={\displaystyle\sum_{i=1}^n(X_i-\bar{X})^2} $)

   
\begin{align*}
        \hat{\beta }_1&=\beta _1+\sum_{i=1}^n\dfrac{X_i-\bar{X}}{S_{XX}}\varepsilon _i\\
        \hat{\beta }_0&=\beta _0+\sum_{i=1}^n\left(\dfrac{1}{n}-\dfrac{(X_i-\bar{X})\bar{X}}{S_{XX}}\right)\varepsilon _i
\end{align*}
 
    Denote corresponding variance as $ \sigma^2_{\hat{\beta}_1} $ and $ \sigma^2_{\hat{\beta}_0} $, using eqa(\ref{EqaDistributionOfSumOfiidNormal}) to get:
    \begin{equation}
        \sigma^2_{\hat{\beta}_1}= \dfrac{\sigma^2}{S_{XX}}\qquad \sigma^2_{\hat{\beta}_0}=\sigma^2(\dfrac{1}{n}+\dfrac{\bar{X}^2}{S_{XX}})
    \end{equation} 
    
     And under normal error assumption, distribution of $ \hat{\beta} _1,\hat{\beta} _0  $ are
    \begin{align*}
        \hat{\beta }_1&\sim N(\beta _1,\sigma^2_{\hat{\beta}_1}) =N(\beta_1,\dfrac{\sigma^2}{S_{XX}})\\
        \hat{\beta}_0&\sim N(\beta_0,\sigma^2_{\hat{\beta }_0}) =N(\beta_0,\sigma^2(\dfrac{1}{n}+\dfrac{\bar{X}^2}{S_{XX}}))
    \end{align*}
    
    Based on sampling distribution of $ \hat{\beta} _1,\hat{\beta} _0  $, we can conduct statistical inference, including CI and HT.\footnote{Detail see sec.\ref{SectionHypothesisTesting}, estimating/testing $ \hat{\beta} _1,\hat{\beta} _0  $ usually corresponds to 'estimate $ \mu $, with $ \sigma^2 $ unknown'.}
    
    % \begin{itemize}[topsep=2pt,itemsep=2pt]
    %     \item LSE of $ \beta _1 $ gives 
    %     \begin{equation}
    %         \hat{\beta _1}=\dfrac{\sum (x_i-\bar{x})(y_i-\bar{y})}{\sum (x_i-\bar{x})^2}
    %     \end{equation}
        
    %     and satisfies $ E(\hat{\beta}_1)=\beta_1 $. Can prove that $ \hat{\beta }_1\sim N(\beta _1,\dfrac{\sigma ^2}{\sum (x_i-\bar{x})^2})=N(\beta_1,\sigma^2(\hat{\beta}_1))$
       
    % \end{itemize}
    
    Note: In linear regression model, we usually focus more on $ \beta_1 $. And note that when $ 0 $ is \textbf{not} within the fitting range,$ \beta_0 $ is not so important.\footnote{Two reason:\begin{itemize}[topsep=2pt,itemsep=2pt]
        \item The etimation error of $ Y $ from $ \hat{\beta}_1 $ increases with $ X_h-\bar{X} $;
        \item $ \beta_1==0  $ is important: decides whether linear model can be used. 
    \end{itemize}}


\begin{point}
    Sampling Distribution of $ e_i $ 
\end{point}
    Consider $ e_i $ as r.v. satisfies
    \begin{equation}
        e_i= Y_i-\hat{Y}_i=Y_i-\hat{\beta }_0-\hat{\beta }_1X_i
    \end{equation}

    and get the expression of $ \hat{e}_i $
    \begin{equation}
        \begin{aligned}
            \hat{e}_i=\varepsilon _i-\sum_{k=1}^n\left( \dfrac{1}{n}+\dfrac{(X_i-\bar{X})^2}{S_{XX}} \right)\varepsilon _k
        \end{aligned}
    \end{equation}
    
    
    \begin{equation}
        E(e_i)=0\qquad \sigma ^2_{e_i}=\sigma ^2 \left( 1-\dfrac{1}{n}-\dfrac{(X_i-\bar{X})^2}{S_{XX}} \right)
    \end{equation}

    Under normal assumption:
    \begin{equation}\label{EqaSamplingDistributionOfResiduals}
        e_i\sim N(0,\sigma ^2\left( 1-\dfrac{1}{n}-\dfrac{(X_i-\bar{X})^2}{S_{XX}} \right) ) 
    \end{equation}
    

    Further we can get $ \hat{\sigma }^2=E(\dfrac{1}{n-2}\sum_{i=1}^ne_i^2) $ where $ e_i^2\sim \sigma ^2\left( 1-\dfrac{1}{n}-\dfrac{(X_i-\bar{X})^2}{S_{XX}} \right)\chi^2 $
    \begin{equation}
        \hat{\sigma }^2=\dfrac{1}{n-2}\sigma ^2\sum_{i=1}^n(1-\dfrac{1}{n}-\dfrac{(X_i-\bar{X})^2}{S_{XX}})=\sigma ^2
    \end{equation}
    
    More definition of refined residuals see sec.\ref{SubSecDiagnostics} in page \ref{SubSecDiagnostics}.
    


\begin{point}
    Why we choose OLS to get regression coefficients?


    \index{Gauss-Markov Thm.`'}Gauss–Markov Thm.: the OLS estimator has the lowest sampling variance within the class of linear unbiased estimators, i.e. OLS is the Best Linear Unbiased Estimator(BLUE).\footnote{This Thm. does \textbf{not }require normal error assumption.}
\end{point}
    



\subsubsection{Prediction to $ Y_h $}
    For a new $ X_h $ at which we wish to \textbf{predict }the corresponding $ Y_h $ (based on other known point $ (X_i,Y_i) $), denote the estimator as $ \hat{\mu}_h $:
    \begin{equation}
        \hat{\mu}_h=\hat{\beta}_1X_h+\hat{\beta}_0 =\beta_1X_h+\beta _0+\sum_{i=1}^n\left( \dfrac{1}{n}+\dfrac{(X_i-\bar{X})(X_h-\bar{X})}{S_{XX}} \right)\varepsilon _i
    \end{equation}
    
    Thus we can get\footnote{So $ \sigma ^2(\hat{\mu }_h) $ increases with $ X_h-\bar{X} $. Intuitively it make sense, because $ (\bar{X},\bar{Y})$ must falls on regression line.}
    \begin{equation}
        E(\hat{\mu}_h)= \beta _1X_h+\beta _0\qquad \sigma ^2_{\hat{\mu}_h}=\left( \dfrac{1}{n}+\dfrac{(X_h-\bar{X})^2}{S_{XX}} \right)\sigma^2
    \end{equation}
    
    Under Normal assumption:
    \begin{equation}
        \hat{\mu}_h\sim N(\beta _1X_h+\beta _0,\left( \dfrac{1}{n}+\dfrac{(X_h-\bar{X})^2}{S_{XX}} \right)\sigma^2) 
    \end{equation}
    
    Base on distribution we can give CI and HT.

    Note: We can either consider 
    \begin{itemize}[topsep=2pt,itemsep=2pt]
        \item \textbf{$ Y_h $ itself as an r.v. }: Confidence Interval of $ Y_h $;
        
        And we can just use $  \sigma ^2_{\hat{\mu}_h} $ to construct CI;

        \begin{rcode}
\begin{lstlisting}[language=R]
predict(lmfit,data.frame(x=c(df$x,40)),
    interval="confidence",level=0.95)
\end{lstlisting}
        \end{rcode}
        \item \textbf{predicted $ Y_h $ from other sample points}: Prediction Interval of $ Y_h $
        
        Need to  have the randomness of $ \hat{\beta }_0,\hat{\beta }_1 $ considered(if they are unknown).

        Def. Prediction Error: $ Y_h $ itself is an $ Y $ of the linear model, i.e. $ Y_i=\beta_0+\beta_1X_h+\varepsilon _h $, we can  and define \textbf{Prediction Error}: 
        \begin{equation}
            d_h=Y_h-\hat{\mu}_h 
        \end{equation}
    
        
        \begin{equation}
            E(d_h)=0\qquad \sigma^2_{d_h}=var(Y_h-\hat{\mu }_h)=\sigma^2\left[ 1+\dfrac{1}{n}+\dfrac{(X_h-\bar{X})}{S_{XX}} \right] > \sigma ^2_{\hat{\mu}_h}
        \end{equation}
\begin{rcode}
\begin{lstlisting}[language=R]
predict(lmfit,data.frame(x=c(df$x,40)),
    interval="prediction",level=0.95)
\end{lstlisting}
\end{rcode}
    
    \end{itemize}
    
    % Remember that when we consider the estimator $ \hat{\mu } $, we \textbf{must } have the randomness of $ \hat{\beta }_0,\hat{\beta }_1 $ considered(if they are unknown).
    


    \begin{point}
       Simultaneous Confidence Band(SCB)\index{SCB (Simultaneous Confidence Band)}\index{CB (Confidence Band)}
    \end{point}

    Confidence Band is \textbf{not} the CI at each point, but really a \textbf{band} for the \textbf{entire} regression line.\footnote{Why they are different? We require the confidence band have a \textbf{simultaneous} converage probability. For the same band $ (L(x),U(x)) $, $ P(\text{the whole line})< P(\text{each point})$, so Confidence Band is wider than $ \bigcup $CIs to hold the same $ 1-\alpha $.
    
    Also, we will see that for linear model, the boundary of SCB forms hyperbola, which make sense considering its asymptotic line.}
    
    
    Aim: Find lower and upper function $ L(x) $ and $ U(x) $ such that
    \begin{equation}
        P[L(x)<(\beta _0+\beta _1x)<U(x),\,\forall x\in I_x]=1-\alpha  
    \end{equation}
    
    and get \textbf{Confidence Band}:
    \begin{equation}
        \{(x,y)|L(x)<y<U(x)|\forall x\in I_x\} 
    \end{equation}
    
    % Note: \textbf{Cannot} use CI at each point to form Confidence Band. Band is wider. And we are actually conduce CI \textbf{simoutanesly} to all $ x $.

    Where $ (L(x),U(x)) $ can be derived as
    \begin{equation}
        (L(x),U(x))=\hat{\mu}_x\pm s_{\hat{\mu}_x}W_{2,n-2,1-\alpha}
    \end{equation}

    Where $ W $ correponds to $ W $ distribution: $ W_{m,n}=\sqrt{2F_{m,n}} $
    
    
    
    Small sample case: Bonferroni correction.
    
\begin{rcode}
\begin{lstlisting}[language=R]
library(ggplot2)
ggplot(df,aes(x,y))+geom_point()+geom_smooth(method='lm',formula=y~x)
\end{lstlisting}
\end{rcode}



% 
% 
% 
% 
% 
% 
% 


\subsection{Analysis of Variance}
    \index{ANOVA (Analysis of Variance)}\textbf{AN}alysis \textbf{O}f \textbf{VA}riance (ANOVA): \hyperlink{OneSampletTest}{One-sample $ t $ test} $\rightsquigarrow $ \hyperlink{TwoSampletTest}{Two sample $ t $ test} $\rightsquigarrow $ ANOVA

\begin{point}
    \textbf{Key Point Of ANOVA}: Take Partition of Total Sum of Square To Examine \textbf{Variation}.  

    Because $ Y_i $ are not i.i.d. (different mean), so has different parts of variation from Regression Model/Error Term.
\end{point}

% \begin{itemize}[topsep=2pt,itemsep=2pt]
%     \item Partition of Totla Sum of Squares;
%     \item Partition of Degree of Freedom;
%     \item MSS$ \rightsquigarrow $ F-test;
%     \item ANOVA table;
%     \item General linear test. --to be examined further in later sections.
%     \item (Pearson) Correlation Coefficient $ \leftrightarrow \, R^2$
% \end{itemize}

\subsubsection{Monovariate ANOVA}

    Measure of Variation: Sum of Square (SS) \& Mean Sum of Square (MS).

    MS: Divide each SS by corresponding $ dof $. Definition of $ dof $ see eqa(\ref{EqaDefinitionOfDegreeOfFreedom}).
    \begin{equation}
        \mathrm{MS}=\dfrac{\mathrm{SS}}{dof} 
    \end{equation}

\begin{itemize}[topsep=2pt,itemsep=2pt]
    \item SST: Total Sum of Squares\index{SST (Total Sum of Squares)}
    \begin{equation}
        \mathrm{SST}=\sum_{i=1}^n(Y_i-\bar{Y})^2 \qquad dof_{\mathrm{SST}}=n-1
    \end{equation}
    \item SSRegression: Variation due to Regression Model \index{SSR (Regression Sum of Squares)} (which is explained by regression line);\footnote{$ \mathrm{SSR}=\hat{\beta }_1^2\sum_{i=1}^n(X_i-\bar{X})^2$, so $ dof_R=1 $}
    \begin{equation}
        \mathrm{SSR}= \sum_{i=1}^n(\hat{Y}_i-\bar{Y})^2 \qquad dof_{\mathrm{SSR}}=1
    \end{equation}
    
    \item SSError: Variation attribtes to $ \varepsilon  $ \index{SSE (Error Sum of Squares)} (which is reflected by residuals).
    \begin{equation}
        \mathrm{SSE}= \sum_{i=1}^n(Y_i-\hat{Y_i}) \qquad dof_{\mathrm{SSE}}=n-2
    \end{equation}
\end{itemize}

\fbox{
    \begin{minipage}{0.9\linewidth}
        $ \Delta $ \textbf{IMPORTANT: }In some books \begin{itemize}[topsep=2pt,itemsep=2pt]
        \item SSRegression $ \to $ SSExplained of SSModel;
        \item SSError $ \to $ SSResidual.
    \end{itemize}

    And Cause \textbf{Confusion}! In this summary we take the former.
    \end{minipage}
}\\



    Idea: take partition of SST. i.e.
    \begin{equation}
        Y_i-\bar{Y}=(Y_i-\hat{Y})+(\hat{Y}-\bar{Y})=e_i 
    \end{equation}
    
    And we can prove that
    \begin{equation}
        \mathrm{SST}=\sum_{i=1}^n(Y_i-\bar{Y})^2=\sum_{i=1}^n(\hat{Y}_i-\bar{Y})^2+\sum_{i=1}^n(Y_i-\hat{Y_i})^2=\mathrm{SSR+SSE} 
    \end{equation}

    That is: we \textbf{partition} SST into two parts, so that we can examine them seperately.
    
Properties:
    
\begin{equation}
    E(\mathrm{MSE})=\sigma ^2\qquad E(\mathrm{MSR})=\sigma ^2+\beta _1^2S_{XX} 
\end{equation}


\subsubsection{Multivariate ANOVA}
    Sampling Notation see eqa(\ref{EqaSampleNotationOfMultiLinear}), still consider $ p+1 $ -dim $ (\mathbf{1}_n,X_i) $ v.s. $ 1 $-dim $ Y $, and $ \beta=(\beta _0,\beta _1,\beta _2,\ldots,\beta _p) $

\begin{itemize}[topsep=2pt,itemsep=2pt]
    \item SST:
    \begin{equation}
        \mathrm{SST}=(Y-\bar{Y}\mathbf{1}_n)'(Y-\bar{Y}\mathbf{1}_n)\qquad dof_{\mathrm{SST}}=n-1
    \end{equation}
    \item SSR:
    \begin{equation}
         \mathrm{SSR}=(\hat{Y}-\bar{Y}\mathbf{1}_n)'(\hat{Y}-\bar{Y}\mathbf{1}_n)\qquad dof_{\mathrm{SSR}}=p
    \end{equation}

    Denoted in hat matrix $ H $ and $ \mathcal{J} $ in eqa(\ref{EqaAllOneMatrix})
    
    \begin{equation}\label{EqaSSMInMatrixNotation}
        \mathrm{SSM}=Y'(H-\dfrac{1}{n}\mathcal{J})Y 
    \end{equation}
    
    
    \item SSE:
    \begin{equation}
         \mathrm{SSE}=(Y-\hat{Y})'(Y-\hat{Y})\qquad dof_\mathrm{SSE}=n-p-1
    \end{equation}

    Denoted in residual $ e $ and hat matrix $ H $:
    \begin{equation}
        \mathrm{SSE}=e'e=Y'(I-H)Y 
    \end{equation}
    
    
    
\end{itemize}

\subsubsection{ANOVA Table}
    \begin{table}[H]
        \centering
        \renewcommand\arraystretch{1}
        \begin{tabular}{c|cccc}
            \hline
            Source&$ dof $&SS&MS&$ F $-Statistic\\\hline
            SSRegression&$ p $&$ \sum_{i=1}^n(\hat{Y}_i-\bar{Y})^2  $&SSR/$ dof_R $& $ \mathrm{MSR}/\mathrm{MSE} $\\
            SSError&$ n-p-1 $&$ \sum_{i=1}^n(Y_i-\hat{Y}_i)^2  $&SSE/$ dof_E $& \\
            SSTotal&$ n-1 $&$ \sum_{i=1}^n(Y_i-\bar{Y})^2  $&SST/$ dof_T $& \\
            \hline
        \end{tabular}
    \end{table}
\begin{rcode}
    \begin{lstlisting}[language=R]
anova(lmfit)
    \end{lstlisting}
\end{rcode}    

% \begin{rcode}
% \begin{lstlisting}[language=R]
% anova(lmfit)
% \end{lstlisting}

% Example output:
% \begin{lstlisting}[language=R]
% Analysis of Variance Table

% Response: y
%             Df  Sum Sq Mean Sq F value    Pr(>F)    
% x          1 11627.5 11627.5  174.06 < 2.2e-16 ***
% Residuals 58  3874.4    66.8
% ---
% Signif. codes:  0 '***' 0.001 '**' 0.01 '*' 0.05 '.' 0.1 ' ' 1
% \end{lstlisting}
% \end{rcode}


\subsubsection{Hypotheses Testing to Slope}
    Main focus: whether the linear relation exist:
\begin{equation}
    H_0:\beta _1=\beta _2=\ldots=\beta _p=0\longleftrightarrow H_1:\exists \beta _i\neq 0,\, i=1,2,\ldots,p
\end{equation}
\begin{itemize}[topsep=2pt,itemsep=2pt]
\item ANOVA $ F $-Test:\index{ANOVA $ F $-test}
    
    We can examine  
    \[
        F=\dfrac{\mathrm{MSR}}{\mathrm{MSE}}\sim F_{p,n-p-1} 
    \]
    
\item General Linear Test (GLT)\index{GLT (General Linear Test)}
    
    First we introduce the examine models:
    \begin{itemize}[topsep=2pt,itemsep=2pt]
        \item Full model: 
        \[
            Y_i=X_i'\beta +\varepsilon _i=\beta _0+\sum_{j=1}^nX_{ij}\beta _j+\varepsilon _i
        \]

        And define $ \mathrm{SSE}_\mathrm{F} $ with $ dof_\mathrm{F}=n-p-1 $ under Full Model.
        \item Reduced model: 
        \[
         Y_i=\beta _0+\varepsilon _i 
        \]
        
        And define $ \mathrm{SSE}_\mathrm{R} $ with $ dof_\mathrm{R}=n-1 $ under Reduced Model.
    \end{itemize}

    and examine
    \begin{equation}
        F=\dfrac{(\mathrm{SSE_R-SSE_F})/(dof_\mathrm{R}-dof_\mathrm{F} )}{\mathrm{SSE_F}/dof_F} \sim F_{p,n-p-1}
    \end{equation}
\begin{rcode}
\begin{lstlisting}[language=R]
nullmodel <- lm(y ~ 1,df)
anova(nullmodel,lmfit)
\end{lstlisting}
\end{rcode}

\item Pearson Correlation Coefficient $ r $ and Coefficient of Multiple Determination $ R^2 $\index{CMD (Coefficient of Multiple Determination)}:

    Pearson's $ r $:
    \[
        r=\hat{cov}(Y,\hat{Y})=\dfrac{\sum\limits_{i=1}^n(Y_i-\bar{Y})(\hat{Y}_i-\bar{Y})}{\sqrt{\sum\limits_{i=1}^n(Y_i-\bar{Y})^2}\sqrt{\sum\limits_{i=1}^n(\hat{Y}_i-\bar{Y})^2}}=\sqrt{\dfrac{\sum\limits_{i=1}^n(\hat{Y}-\bar{Y})^2}{\sum\limits_{i=1}^n(Y_i-\bar{Y})^2}}
    \]
    
    CMD $ R^2 $:

    \[
        R^2=\dfrac{\mathrm{SSR}}{\mathrm{SST}}=1-\dfrac{\mathrm{SSE}}{\mathrm{SST}}
    \]

    Adjusted $ R^2 $:
    \[
        R^2_\mathrm{a}=1-\dfrac{\mathrm{MSE}}{\mathrm{MST}} =1-\dfrac{n-1}{n-p}\dfrac{\mathrm{SSE}}{\mathrm{SST}}
    \]
    
    \begin{itemize}[topsep=2pt,itemsep=2pt]
        \item Relation between $ r $ and $ R^2 $: Under Simple Linear Model, we have 
        \[
            R^2=r^2 
        \]
        \item Relation between $ R^2 $ and $ F $-Statistic:
        \[
            F=\dfrac{R^2}{1-R^2}\dfrac{n-p}{n-1} 
        \]
    \end{itemize}
\end{itemize}






\subsection{Model Assumption, Diagnostics and Remedies}

    To apply OLS, we need the basic Gauss–Markov Assumption eqa(\ref{EqaGaussMarkovAssumption}); or we further need better properties of the model, so need Normal Assumption.
    
    Assumptions:
    \begin{equation}
        \begin{aligned}
            \text{Zero-Mean: }&E(\epsilon_i|X_i)=0 \\
            \text{Homogeneity of Variance: }&var(\epsilon_i)=\sigma^2\\
            \text{Independent: }&\epsilon_i\text{ i.i.d. }\sim \varepsilon\\
            \text{Normal: }&Y_i\sim N(\beta _0+\beta _1X_i,\sigma^2)
        \end{aligned}
    \end{equation}
    
    Or sum up as 
    \begin{equation}
        \vec{\varepsilon }\sim N_n(\vec{0},\sigma^2I_n) 
    \end{equation}
    
    
    
    Thus we need to conduct Diagnostics and Remedies to 
    \begin{itemize}[topsep=2pt,itemsep=2pt]
        \item examine whether these assumptions are satisfies;
        \item perform correction to regression method.
    \end{itemize}
\subsubsection{Diagnostics}\label{SubSecDiagnostics}


    Preliminary Diagnostics:
\begin{rcode}
\begin{lstlisting}[language=R]
lmfit <- lm(y~x,lmfit)
par(mfrow = c(2, 2))
plot(lmfit)
\end{lstlisting}

\end{rcode}
    

\begin{point}
    Diagnostics to $ X $
\end{point}

    Considering the dependence of $ Y_i $ on $ X_i $, to get a more reliable $ \hat{\beta }_1 $, we cannot just focus on the (marginal) distribution of $ Y_i $, we would also need a better 'distribution' of $ X_i $
    \begin{itemize}[topsep=2pt,itemsep=2pt]
        \item 4 statistics(parameters);\footnote{See sec.\ref{SubSectionStatistics}}
        \begin{itemize}[topsep=2pt,itemsep=2pt]
            \item Mean: Location;
            \begin{equation}
                \bar{X}=\dfrac{1}{n}\sum_{i=1}^nX_i 
            \end{equation}
            \item Standard Deviation: Variability;
            \begin{equation}
                S^2=\dfrac{1}{n-1}\sum_{i=1}^n(X_i-\bar{X}) ^2
            \end{equation}
            
            
            \item Skewness: Lack of Symmertry;
            \begin{equation}
                \hat{g}_1=\dfrac{m_{n,3}}{m_{n,2}^{3/2}}=\dfrac{\frac{1}{n}\sum\limits_{i=1}^n(X_i-\bar{X})^3}{\left( \frac{1}{n}\sum\limits_{i=1}^n(X_i-\bar{X}) \right)^{3/2}} 
            \end{equation}

            Adjusted Skewness (Least MSE):
            \begin{equation}
                \dfrac{\sqrt{n(n-1)}}{n-2}\hat{g}_1 
            \end{equation}
            
            \begin{itemize}[topsep=2pt,itemsep=2pt]
                \item $ \hat{g}_1>0 $: Right skewness, longer right tail;
                \item $ \hat{g}_1<0 $: Left skewness, longer left tail.
            \end{itemize}
            
                
            Fisher-Pearson coefficient of skewness.


            \item Kurtosis: Heavy/Light Tailed.
            \begin{equation}
                \hat{g}_2=\dfrac{m_{n,4}}{m_{n,2}^2}-3= \dfrac{\frac{1}{n}\sum\limits_{i=1}^n(X_i-\bar{X})^4}{\left( \frac{1}{n}\sum\limits_{i=1}^n(X_i-\bar{X})^2 \right)^{2}} -3
            \end{equation}

            $ \hat{g}_2=0 \Rightarrow $ similar to normal.
            \begin{itemize}[topsep=2pt,itemsep=2pt]
                \item $ \hat{g}_2>0 $: Leptokurtic, heavy tail, slender;
                \item $ \hat{g}_2<0 $: Platykurtic, light tail, broad.
            \end{itemize}
            
            Note: In expression of $ \hat{g}_1 $ and $ \hat{g}_2 $, we already divide the variance. So Skewness and Kurtosis only reflect the difference from normal, but \textbf{not}  related to variance.
                
            Best tool to determine Kurtosis: \hyperlink{QQplot}{QQ-Plot}.
            
        \end{itemize}

\begin{rcode}
\begin{lstlisting}[language=R]
summary(df$x)
\end{lstlisting}

    Other moments use package \lstinline|moments|
\end{rcode}
        \item Useful Plots:
        \begin{itemize}[topsep=2pt,itemsep=2pt]
            \item BoxPlot: to examine the similarity of  distribution.
            
            Notation:
            \begin{enumerate}[topsep=2pt,itemsep=2pt]
                \item min point above 25\% quantile-1.5IQR;
                \item 25\% quantile;
                \item median;
                \item 75\% quantile;
                \item max point below 75\% quantile+1.5IQR.
            \end{enumerate}
            
                
            \begin{center}
                \begin{tikzpicture}
                    \draw(-2,-0.6)rectangle(2,0.6);
                    \draw (-5.5,0)--(-2,0);
                    \draw (2,0)--(5.5,0);
                    \draw(-5.5,-0.7)--(-5.5,0.7);
                    \draw(5.5,-0.7)--(5.5,0.7);
                    \draw(0,-0.6)--(0,0.6);
                    \node at (-2,1){$ 2 $};
                    \node at (2,1){$ 4 $};
                    \node at (-5.5,1.1){$ 1 $};
                    \node at (5.5,1.1){$ 5 $};
                    \node at (0,1){$ 3 $};
                \end{tikzpicture}
            \end{center}


    
            

            \item Histogram Plots: Frequency distribution (can deal with many-peak)
            \item Quantile-Quantile Plots\index{QQ-Plot (Quantile-Quantile Plots)}: Examine the similarity  between distribution.
            
            For two CDF $ q=F(x) $ and $ q=G(x) $(where $ q $ for quantile), with $ x=F^{-1}(q) $, $ x=G^{-1}(q) $. And Plot $ F^{-1}(q) $-$ G^{-1}(q) $.

            Usually test normality, take $ G=\Phi  $
        \end{itemize}
\begin{rcode}
\begin{lstlisting}[language=R]
boxplot(df$x)

hist(df$x)

hist(df$x,freq=FALSE)
lines(density(df$x))

stem(df$x)

qqnorm(df$x)
qqline(df$x,col='red)
\end{lstlisting}

\end{rcode}
            
        \item Normality;
        \item Bias:
        \begin{itemize}[topsep=2pt,itemsep=2pt]
            \item Selection Bias: Not completely random sampling;
            \item Information Bias: Difference between 'designed' and 'get', e.g. no response;
            \item Confounding: Exist another important variable, while the model actually focuses on a less important variable, or even reverse the causality.
        \end{itemize}
        
            
    \end{itemize}
    
\begin{point}
    Diagnostics to Residual
\end{point}

   

\begin{itemize}[topsep=2pt,itemsep=2pt]
    \item Linearity: use Residual Plot: Reflect the linearity and variance assumption.
\begin{rcode}
\begin{lstlisting}[language=R]
lmfit <- lm(y~x,df)
scatter(df$x,lmfit$residuals)
abline(h=0)
\end{lstlisting}

\end{rcode}
    \item The Assumption of Equal Variances:
    \begin{itemize}[topsep=2pt,itemsep=2pt]
        \item Bartlett's test:\index{Bartlett's test}
        
        Idea: divide the sample into groups $ g $, and get each MSE
        \begin{equation}
             \mathrm{MSE}_g=\dfrac{1}{n_g}\sum_{i=1}^{n_g}(Y_{gi}-\hat{Y}_g)^2
        \end{equation}
        
        and take statistic
        \begin{equation}
            S=-\dfrac{(N-g)\ln\left[ \sum\limits_g \dfrac{n_g}{N-n_g}\mathrm{MSE}_g \right]-\sum\limits_{g}(n_g-1)\ln \dfrac{n_g}{N-n_g}\mathrm{MSE}_g }{1+\dfrac{1}{3(G-1)}\sum\limits_g\left( \dfrac{1}{n_g-1}-\dfrac{1}{N-G} \right)} \sim \chi^2
        \end{equation}

        to conduct test. 

        Note: \textbf{sensitive}  to normal assumption, not robust. Used when normal assumption is satisfied.
        \item Levene's test: \index{Levene's Test}Divide the sample into $ G $ groups. Denote \textbf{mean}  of residual within each group as $ \tilde{e}_g $, and in each group compute
        \begin{equation}
            d_{ig}=|e_{ig}-\tilde{e}_g| \Rightarrow \bar{d}_{g}=\dfrac{1}{n_g}\sum_{j=1}^{n_g}d_{ig}
        \end{equation}

        Then conduct ANOVA to $ d_{ig} $.

        If $ G=2 $: 2-sample $ t $-test,
        \begin{equation}
            T=\dfrac{\bar{d}_1-\bar{d}_2}{s\sqrt{\frac{1}{n_1}+\frac{1}{n_2}}}\xrightarrow[]{\mathscr{L}} t_{n-2}\qquad s^2=\dfrac{\sum(d_{i1}-\bar{d}_1)^2+\sum(d_{i2}-\bar{d}_2 )^2}{n-2}
        \end{equation}
        

        
        
        \item Brown-Forsythe's Test\index{Brown-Forsythe's Test} (Modified Levene's test): For skewed sample, take the \textbf{mean} as \textbf{median}, more robust. 


        \item Breusch-Pagan Test:\index{B-P Test (Breusch-Pagan Test)}
        
        Assume variance of $ \varepsilon _i $ dependent on $ X_i $ as $ m^{\mathrm{th}} $ polynomial:
        \begin{equation}
            \sigma_i^2=\alpha _0+\sum_{k=1}^m\alpha _kX_i^k
        \end{equation}
        
        and test 
        \begin{equation}
            H_0:\alpha _k=0\,\forall k=1,2,\ldots,m\longleftrightarrow H_1 
        \end{equation}

        Method: First conduct OLS to get regression line $ \hat{l}_1 $ and residuals $ e_i $ and SSE, and conduct regression of $ e_i^2 $ over $ X_i $ to get another regression line $ \hat{l}_2 $ and corresponding SSR$ ^* $.

        Then statistic
        \begin{equation}
            S=\dfrac{\mathrm{SSR^*}/2}{(\mathrm{SSE}/n)^2}\xrightarrow[]{\mathscr{L}} \chi^2_m
        \end{equation}
        
    \end{itemize}
\begin{rcode}
    Example for $ G=2 $:
\begin{lstlisting}[language=R]
group <-factor(rep(c(1,2),length.out=length(df$x),
    each=(ceiling(length(df$x)/2))))

bartlett.test(lmfit$residuals~group,group)

library(car)
leveneTest(lmfit$residuals~group,group,center=mean)
leveneTest(lmfit$residuals~group,group,center=median)

library(lmtest)
bptest(lmfit)
\end{lstlisting}

\end{rcode}

    \item The Assumption of Normality:
    
    In most case we use S-W Test($ n<2000 $) and K-S Test($ n>2000 $):
    \begin{itemize}[topsep=2pt,itemsep=2pt]
        \item QQ-plot of ordered residuals;
        \item[$ \star  $] Shapiro-Wilk Test\index{S­-W Test(Shapiro-­Wilk Test)} (Most Powerful)\footnote{Detail of S-W Test and K-S Test see \hyperlink{testofnormality}{Test of Normality} in sec.\ref{SubSectionIntroToNonParametricHypothesisTesting}}:
        \begin{equation}
            R^2=\dfrac{\left(\sum_{i=1}^n(X_{(i)}-\bar{X})(m_i-\bar{m})\right)^2}{\sum_{i=1}^n(X_{i}-\bar{X})^2\sum_{i=1}^n(m_i-\bar{m})^2}=corr(X_{(i)},m_i) 
        \end{equation}
        
        \item Kolmogorov-Smirnov Test\index{K-S Test (Kolmogorov-Smirnov Test)}: 
        \begin{equation}
            D_n=\sum_{x}|F_n(x)-\Phi(x)|
        \end{equation}
        
        
        \item Cramér-von Mises Test\index{CvM Test (Cramér-von Mises Test)}:
        \begin{equation}
            T=n\int_{-\infty}^\infty (F_n(x)-\Phi (x)) ^2\,\mathrm{d}\Phi(x)
        \end{equation}
        \item Anderson-Darling Test:\index{A-D Test (Anderson-Darling Test)}
        \begin{equation}
            A^2-n\int _{-\infty}^\infty (F_n(x)-\Phi(x))^2\dfrac{1}{\Phi(x)(1-\Phi(x))} \,\mathrm{d}\Phi(x)
        \end{equation}
    \end{itemize}
\begin{rcode}
\begin{lstlisting}[language=R]
qqnorm(lmfit$residuals)
qqline(lmfit$residuals)

qqp <- qqnorm(lmfit$residuals)
cor(qqp$x,qqp$y)

shapiro.test(lmfit$residuals)

ks.test(jitter(lmfit$residuals),pnorm,mean(lmfit$residuals),
    sd(lmfit$residuals))

library(nortest)
cvm.test(lmfit$residuals)

ad.test(lmfit$residuals)
\end{lstlisting}

\end{rcode}


    \item The Assumption of Independence:
    \begin{itemize}
        \item Durbin-Watson Test:  \index{DW Test (Durbin-Watson Test)}
        \begin{equation}
            d=\dfrac{\sum_{j=2}^n(e_j-e_{j-1})^2}{\sum_{j=1}^ne_j^2} 
        \end{equation}
        
        $ d\in (1.5,2.5) $ is fine.
        \item Ljung-Box Test:\index{Ljung-Box Test}
        
        \begin{equation}
            Q=n(n+2)\sum_{k=1}^n\dfrac{\hat{\rho}_k^2}{n-k} 
        \end{equation}
        
        
    \end{itemize}

\begin{rcode}
\begin{lstlisting}[language=R]
dwtest(lmfit)
\end{lstlisting}

\end{rcode}   
        
\end{itemize}

\begin{point}
    Diagnostics to Influentials
\end{point}

    An intuitive explanation to extreme values:
    \begin{itemize}[topsep=2pt,itemsep=2pt]
        \item Outliers: Extreme case for $ Y $;
        \item High Leverage: Extreme case for $ X $;
        \item Influentials: Cases that influence the regression line.
    \end{itemize}

    Influentials = Outliers $ \cap $ High Leverage

    In OLS part, we got the $ \hat{\beta}  $ as $\hat{ \beta} = (X'X)^{-1}X'Y $ and got $ \hat{Y} $ as 
    \begin{equation}
        \hat{Y}= X\hat{\beta }=X(X'X)^{-1}X'y=\hat{H}Y
    \end{equation}
    
    where $ \hat{H} $ is the \textbf{Hat Matrix}\footnote{It can also be considered as the projection matrix onto span$ \{X\} $.} 

    Denote in matrix derivation as $ H=\dfrac{\partial^{} \hat{Y}}{\partial Y^{}} $. The diagonal elements of $ \hat{H} $ is self-sensitivity:
    \begin{equation}
    h_{ii}=\dfrac{\partial^{} \hat{Y}_i}{\partial Y_i^{}} =\dfrac{1}{n}+\dfrac{(X_i-\bar{X})^2}{S_{XX}}
    \end{equation}

    Note: the distribution of $ e_i $ in eqa.(\ref{EqaSamplingDistributionOfResiduals}) thus can be written in $ h_{ii} $ as 
    \begin{equation}
        e_i\sim (0,\sigma ^2(1-h_{ii})) 
    \end{equation}

    Some refined residuals to help conduct Diagnostics:
\begin{itemize}[topsep=2pt,itemsep=2pt]
    \item Standardized Residual:
    \begin{equation}
         \dfrac{e_i}{\sigma _{e_i}}=\dfrac{e_i}{\sigma \sqrt{1-h_{ii}}}
    \end{equation}
    \item (Internal) Studentized Residual: replace $ \sigma  $ with $ s=\hat{\sigma } $
    
    \begin{equation}
         r_i=\dfrac{e_i}{\hat{\sigma }\sqrt{1-h_{ii}}}
    \end{equation}
    \item (External) Studentized Residual: To avoid self-influence, take \textbf{deleted} residual:
    
    Delete the $ i^{\mathrm{th}} $ case and conduct regression to the $ n-1 $ sample cases, denote the regression parameter as 
    \begin{equation}
        \hat{\beta }_{1(\wedge i)}\qquad \hat{\beta }_{0(\wedge i)} 
    \end{equation}
    
    and deleted residual defined as 
    \begin{equation}
        d_i=Y_i-Y_{i(\wedge i)} =\dfrac{e_i}{1-h_{ii}}
    \end{equation}

    external studentized residual:
    \begin{equation}
         t_i=\dfrac{d_i}{\sigma _{(\wedge i)\sqrt{1-h_{ii}}}}
    \end{equation}
    
    
\end{itemize}


    Cook's Distance: 
    \begin{equation}
        D_i=\dfrac{\sum_{k=1}^n(Y_k-\hat{Y}_{k(\wedge i)})^2}{p\hat{\sigma }^2} =\dfrac{e_i^2}{p\hat{\sigma }^2}\left[ \dfrac{h_{ii}}{(1-h_{ii})^2} \right]
    \end{equation}

    Comment:
    \begin{equation}
        D_i=\dfrac{e_i^2}{p\hat{\sigma }^2}\left[ \dfrac{h_{ii}}{(1-h_{ii})^2} \right]=\dfrac{1}{p}\dfrac{h_{ii}}{1-h_{ii}}\times r_i^2
    \end{equation}

    where $ \dfrac{1}{p}\dfrac{h_{ii}}{1-h_{ii}} $ correponds to hige leverage, and $ r_{i}^2 $ correponds to outliers, multiply to get influentials.
    
    
    
    
    

    
    
    
    
    
    
        










\subsubsection{Remedies}
\begin{point}
    General Linear Model
\end{point}
    
    \begin{equation}
        E(Y)=g(\beta _0+\beta _1X_1+\beta _2X_2+\ldots) 
    \end{equation}
    
\begin{point}
    Remedies: Conduct Transformation
\end{point}

    \begin{itemize}[topsep=2pt,itemsep=2pt]
        \item Stablize Variance;
        \item Improve Normality;
        \item Simplify the Model.
    \end{itemize}

    Transformation Methods:
    \begin{itemize}[topsep=2pt,itemsep=2pt]
        \item Variance Stabilizing Transformations:
            For $ E(Y_X)=\mu_X$, $ var(Y_X)=h(\mu_X) $, take transformation $ f(Y) $ such that $ var(f(Y))=\mathrm{const} $, satisfies
            \begin{equation}
                f(\mu)=\int\dfrac{c\,\mathrm{d}\mu}{\sqrt{h(\mu)}} 
            \end{equation}

            Examples:
            \begin{align*}
                h(\mu)=&\mu^2\Rightarrow f(\mu )=\ln\mu\\
                h(\mu)=&\mu^{2\nu}\Rightarrow f(\mu )=\mu ^{1-\nu}
            \end{align*}
        
        \item Box-Cox Transformation: Take 
    \begin{equation}
        Y^*=\dfrac{Y^\lambda -1}{\lambda }
    \end{equation}

            Examples:
        \begin{align*}
            \lambda =1&\Rightarrow Y^*\sim Y\\
            \lambda =0.5&\Rightarrow Y^*\sim \sqrt{Y}\\
            \lambda =0&\Rightarrow Y^*\sim \ln Y\\
            \lambda =-1&\Rightarrow Y^*\sim 1/Y
        \end{align*}
    
        And conduct regression to model
        \begin{equation}
            Y^* =\beta _0+\beta _1X+\varepsilon_i 
        \end{equation}
        
        Likelihood Function
        \begin{equation}
            L(\beta ,\sigma ^2;\lambda )=\dfrac{1}{(2\pi\sigma ^2)^{n/2}}\exp\left( -\dfrac{1}{2\sigma ^2}\sum_{i=1}^n (Y_i^*-\beta _0-\beta _1X_i)^2 \right) J(\dfrac{\partial^{} Y^*}{\partial Y^{}})
        \end{equation}

        where the Jacobi Matrix denoted in Geometric Mean $ \mathrm{GM}(Y)=\prod_{i=1}^n Y_i^{1/n}$
        \begin{equation}
            J(\dfrac{\partial^{} Y^*}{\partial Y^{}})=\prod_{i=1}^nY_i^{\lambda -1}=\mathrm{GM}(Y)^{n(\lambda -1)}
        \end{equation}
        
        

        MLE Estiamtor:
        \begin{align*}
            \hat{\beta }^*&= (X'X)^{-1}X'Y^*\\
            \hat{\sigma }^2_n&=\dfrac{1}{n}\mathrm{SSE}^*\\
            \mathrm{SSE}^*&=\sum_{i=1}^n(Y_i^*-\hat{Y}_i^*)^2
        \end{align*}

        And when $ \beta  $, $ \sigma ^2 $ take MLE estimator, $ L(\beta ,\sigma ^2;\lambda ) $ can be regarded a function of $ \lambda  $:
        \begin{equation}
            \ln L(\beta ,\sigma ^2;\lambda )=l(\lambda )=-\dfrac{n}{2}\ln \dfrac{\hat{\sigma}^2_n}{\mathrm{GM}(Y)^{2(\lambda -1)}}+\mathrm{const}
        \end{equation}

        For simplification, denote $ Z=Y*/J^{1/n} $ and get
        \begin{equation}
            l(\lambda )=-n\ln \sigma^2_{n_Z}+\mathrm{const} 
        \end{equation}
        
        where 

        \begin{equation}
            Z_i^* =\begin{cases}
                \dfrac{Y_i^\lambda-1 }{\lambda }\dfrac{1}{\prod\limits_{k=1}^n Y_k^{\frac{\lambda -1}{n}}},&\lambda \neq 0\\
                \ln Y_i\prod\limits_{k=1}^n Y_k^{\frac{1}{n}},&\lambda =0
            \end{cases}
        \end{equation}

        Plot $ l(\lambda ) $-$ \lambda  $ to determine a proper $ \lambda  $ and transform $ Y^*=\dfrac{Y^\lambda -1}{\lambda } $:
        \begin{itemize}[topsep=2pt,itemsep=2pt]
            \item Selected $ \lambda $ should be closed to $ \lambda_{\arg\max l} $, at least within CI\footnote{Here CI can be derived using Wilk's Thm.}
            \begin{equation}
                \{\lambda |l(\lambda )\geq l(\lambda_{\arg\max l})-\dfrac{1}{2}\chi^2_{1,1-\alpha }\}
            \end{equation}
            \item Should pick a $ \lambda  $ which is \textbf{Interpretable}. e.g. If $ \lambda =1 $ is within range, then take $ \lambda =1 $ (does not transform).
            
            
        \end{itemize}
        
            
        
% \begin{rcode}
% \begin{lstlisting}[language=R]
% library(MASS)
% bctrans <- boxcox(y~x,df,lambda = seq(-1.5, 1.5, length = 15))
% bctrans$x[which.max(bctrans$y)]
% \end{lstlisting}

% \end{rcode}
        
        
%     Note: we can transform on $ X $ or $ Y $ or simultaneously to get better regression model.
    
\end{itemize}



    
\subsection{Multiple Linear Regression}

\begin{point}
    Sample Geometry Notation
\end{point}

    In sample matrix notation:
    \begin{equation}
        Y=X\beta+\varepsilon \,\leftrightharpoons\, Y_i=X\beta _i+\varepsilon _i,\,\forall i=1,2,\ldots,q
    \end{equation}
    
    where 
\begin{subequations}\label{EqaSampleNotationOfMultiLinear}
    \begin{align}
        \mathop{Y}\limits_{n\times q} =&\begin{bmatrix}
        y_{11}&y_{12}&\ldots&y_{1q}\\
        y_{21}&y_{22}&\ldots&y_{2q}\\
        \vdots&\vdots&\ddots&\vdots\\
        y_{n1}&y_{n2}&\ldots&y_{nq}\\
        \end{bmatrix}
        =\begin{bmatrix}
            y_1,y_2,\ldots,y_q
        \end{bmatrix}
        &
        y _i=&\begin{bmatrix}
                y _{1i}\\
                y _{2i}\\
                \vdots\\
                y _{ni}
            \end{bmatrix}\\
        \mathop{X}\limits_{n\times (p+1)}=&\begin{bmatrix}
        1&x_{11}&x_{12}&\ldots&x_{1p}\\
        1&x_{21}&x_{22}&\ldots&x_{2p}\\
        \vdots&\vdots&\ddots&\vdots\\
        1&x_{n1}&x_{n2}&\ldots&x_{np}\\
        \end{bmatrix}
        =
        \begin{bmatrix}
        x_1'\\x_2'\\\vdots\\x_n'   
        \end{bmatrix} 
        &
        x_i=&\begin{bmatrix}
            1\\
            x_{i1}\\
            \vdots\\
            x_{ip}
        \end{bmatrix}\\
        \mathop{\beta }\limits_{(p+1)\times q}=&\begin{bmatrix}
        \beta _{01}&\beta _{02}&\ldots&\beta _{0q}\\
        \beta _{11}&\beta _{12}&\ldots&\beta _{1q}\\
        \beta _{21}&\beta _{22}&\ldots&\beta _{2q}\\
        \vdots&\vdots&\ddots&\vdots\\
        \beta _{p1}&\beta _{p2}&\ldots&\beta _{pq}\\
        \end{bmatrix} =\begin{bmatrix}
            \beta _1,\beta _2,\ldots,\beta _q
        \end{bmatrix}
        & \beta _i=&\begin{bmatrix}
            \beta _{i0}\\
            \beta_{i1}\\
            \vdots\\
            \beta_{ip}
        \end{bmatrix}\\
        \mathop{\varepsilon }\limits_{n\times q} =&
        \begin{bmatrix}
        \varepsilon _{11}&\varepsilon _{12}&\ldots&\varepsilon _{1q}\\
        \varepsilon _{21}&\varepsilon _{22}&\ldots&\varepsilon _{2q}\\
        \vdots&\vdots&\ddots&\vdots\\
        \varepsilon _{n1}&\varepsilon _{n2}&\ldots&\varepsilon _{nq}\\
        \end{bmatrix}=
        \begin{bmatrix}
            \varepsilon _1,\varepsilon _2,\ldots,\varepsilon _q
        \end{bmatrix}
        &
        \varepsilon _i=&\begin{bmatrix}
                \varepsilon _{1i}\\
                \varepsilon _{2i}\\
                \vdots\\
                \varepsilon _{ni}
            \end{bmatrix}
    \end{align}

    
\end{subequations}


    Under matrix notation, model and assumptions eqa(\ref{EqaGaussMarkovAssumption}) can be expressed in condensed notation:

    \begin{equation}
        Y_i=X\beta_i +\varepsilon_i  \sim N_n(X\beta_i ,\sigma_i^2I_n),\quad i=1,2,\ldots,q
    \end{equation}

    To conduct OLS
    \begin{equation}
        \hat{\beta }=\mathop{ \arg\min }\limits_{\beta \in \mathbb{R}^{p+1} } (Y-X\beta )^T(Y-X\beta )
    \end{equation}
    
    Here we introduce two approaches:
\begin{itemize}[topsep=2pt,itemsep=2pt]
    \item Analytical: Take matrix differciation (See \hypertarget{MatrixDifferenciation}{sec.\ref{SubSubSectionMatrixNotationAndLemma}})
    

\begin{align*}
    0&=\dfrac{\partial^{} (Y-X\beta )^T(Y-X\beta ) }{\partial \beta ^{}} =\dfrac{\partial^{} }{\partial\beta  ^{}}(Y^TY- Y^TX\beta -\beta ^TX^TY+\beta ^TX^TX\beta )\\ 
    &=-X^TY-X^TY+(X^TX+XX^T)\beta 
    =-2X^T(Y-X\beta )
\end{align*}
    
    Thus we get OLS:
    \begin{equation}
        \hat{\beta }=(X'X)^{-1}X'Y 
    \end{equation}
    
    
    \item Geometric/Algebraical: Use hyper-projection.
    \begin{equation}
        \hat{\beta }=\mathop{ \arg\min }\limits_{\beta \in \mathbb{R}^{p+1} } d(Y,X\beta )
    \end{equation}

    i.e. $ \hat{\beta } $ is the (hyper-)projection of $ Y $ onto $ X $ (within Euclidean Space), naturally we have
    \begin{equation}
        (X\beta )^T(Y-X\beta )=0\Rightarrow \hat{\beta }=(X'X)^{-1}X'Y 
    \end{equation}

\end{itemize}

\begin{point}
    Matrix Notation of OLS Estimator:
    \begin{equation}
        \hat{\beta }=(X'X)^{-1}X'Y 
    \end{equation}
\end{point}

    (For simplificaiton, the following part consider multivariate $ \mathop{X}\limits_{n\times (p+1)}  $ with one $ \mathop{Y}\limits_{n\times 1}  $)

    Properties \& Extrapolation
\begin{itemize}[topsep=2pt,itemsep=2pt]
    \item Sampling Districution of $ \hat{\beta } $: (Here consider normal case $ Y\sim N(X\beta ,\sigma^2I_n) $, and use eqa(\ref{EqaTransformOfMultiNormal})) 
    \begin{equation}
        \hat{\beta }=(X'X)^{-1}X'Y \sim N_n(\beta,\sigma^2(X'X)^{-1})
    \end{equation}

    Comment: $ cov(\beta_i,\beta_j ) $ are generally not 0, $ \Rightarrow $ $ \beta _i,\beta _j $ dependent.
    \item Predicted Response \& Hat Matrix $ H $:
    \begin{equation}
        \hat{Y}=X\hat{\beta }=X(X'X)^{-1}X'Y\equiv  HY=P_XY
    \end{equation}

    where \textbf{Hat Matrix}/Influence matrix/Projection matrix $ H=P_X=X(X'X)^{-1}X' $, with properties
    \begin{itemize}[topsep=2pt,itemsep=2pt]
        \item Symmertric: $ H^T=H $;
        \item Idempotence: $ H^2=H $
        \item $ H $ and self-influene factor $ h_{ii} $: Note the linearity of $ \hat{Y} $ on $ Y $
        \begin{equation}
            \hat{Y}=HY \Rightarrow H=\dfrac{\partial^{} \hat{Y}}{\partial Y^{}}
        \end{equation} 
    
        The diagonal elements of $ H $ is 
        \begin{equation}
            h_{ii}=\dfrac{\partial^{}\hat{y}_i}{\partial y_i^{}}=X_i(X'X)^{-1}X'_i
        \end{equation}

        Comment on $ h_{ii} $: $ var(e_i) =\sigma ^2(1-h_{ii})$, for $ h_{ii}\to 1 $, i.e. the regression line always pass $ y_i $, thus it's `influential'.
        \item $ H $ and Residual $ e $
    \end{itemize}




    \item Residual:
    \begin{equation}
        e=Y-\hat{Y}=(I-H)Y\sim N_n\left(0 , \sigma ^2(I-H) \right)
    \end{equation}

    Covariance Matrix of Residual:
    \begin{equation}
        cov(e)=\sigma ^2(I-H)=
        \sigma ^2\begin{bmatrix}
        1-h_{11}&-h_{12}&\ldots&-h_{1n}\\
        -h_{21}&1-h_{22}&\ldots&-h_{2n}\\
        \vdots&\vdots&\ddots&\vdots\\
        -h_{n1}&-h_{n2}&\ldots&1-h_{nn}\\
        \end{bmatrix}
    \end{equation}
    
    
    \item Estimator and Distribution of  $ \sigma ^2 $:
    
    First use eqa(\ref{EqaExpectationOfQuadric}) to get \footnote{Also we need the property of idmpotnet matrix
    \begin{equation}
        \lambda_i=0\text{ or }1\Rightarrow tr(H)=\mathrm{rank}(H)=\sum_{i=1}^n\lambda _i=\# (\lambda =1) 
    \end{equation}
    }
    \begin{equation}
        E(\mathrm{SSE})=E(e'e)=E(Y'(I-H)Y)=(X\beta )'(I-H)X\beta +tr((I-H)\sigma ^2I_n) =\sigma ^2(n-p-1)
    \end{equation}

    $ dof $ of Residual $ e $ (use definition eqa(\ref{EqaDefinitionOfDegreeOfFreedom})):
    \begin{equation}
        dof_e=dof_{(I-H)Y}=\mathrm{rank}(I-H)=n-p-1 
    \end{equation}
    
    
    
    Thus the unbiased estimator of $ \sigma ^2 $ is 
    \begin{equation}
        \hat{\sigma }^2=\mathrm{MSE}=\dfrac{e'e}{n-p-1}=\dfrac{Y'(I-H)Y}{n-p-1}
    \end{equation}

    Distribution (under normal assumption):
    
    \begin{equation}
        \dfrac{(n-p-1)\hat{\sigma }^2}{\sigma ^2}\sim \chi^2_{n-p-1}
    \end{equation}
    
    \item Gauss–Markov Thm.: OLS Estimator of $ \beta  $ is the BLUE Estimator.
    \item Leverage and Mahalanobis Distance:
    
    Mahalanobis Distance between $ X $ and $ Y $ as defined in eqa(\ref{MahalanobisDistance})
    \begin{equation}
         d_M(\vec{x})=\sqrt{(\vec{x}-\vec{\mu})^TS ^{-1}(\vec{x}-\vec{\mu})} 
    \end{equation}

    And we can proof $ d_M $ of a case item $ X_{ i\cdot}=(1,X_{i1},X_{i2},\ldots,X_{ip}) $ is
    \begin{equation}
        d_{M}^2(X_{i\cdot})=(n-1)(h_{ii}-\dfrac{1}{n}) 
    \end{equation}
    
    
    
    


\end{itemize}

    






Test of Normality: Jarque-Bera Test \index{JB-test (Jarque-Bera test)}, using skewness $ \hat{g}_1 $ and kurtosis $ \hat{g}_2 $
\begin{equation}
    \mathrm{JB}=\dfrac{n}{6}(\hat{g}_1^2+\dfrac{1}{4}\hat{g}_2^2) \xrightarrow[]{\mathscr{L}} \chi^2_2
\end{equation}

\begin{rcode}
\begin{lstlisting}[language=R]
library(tseries)
jarque.bera.test(df$y)
\end{lstlisting}

\end{rcode}











        

