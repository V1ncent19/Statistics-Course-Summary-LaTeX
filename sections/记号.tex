\phantomsection
\addcontentsline{toc}{chapter}{记号表}





\renewcommand{\nomname}{记号表} % 记号表标题

\renewcommand\nomgroup[1]{%记号表中的符号分组排序方式
  \item[\bfseries
  \ifstrequal{#1}{A}{General Convention}{%
  \ifstrequal{#1}{C}{Used in Most Cases}{%
  \ifstrequal{#1}{E}{Used Frequently}{}}}%
]}

% 标题与符号说明之间的文字
\renewcommand{\nompreamble}{Here is a list of frequently used symbols and their meanings in this summary note. I sometimes use different notation from people used to in the literature, or from the notation in lecture notes of Tsinghua University Statistics minor courses, or simply follow convention in my Physics major. 

Symbol listed here are notations `by default' in this summary note. Specially defined symbols, especially those different from convention, would be explained in the text.}
% 输出符号说明

\printnomenclature[3cm]

% 在正文中的任意位置都可以添加符号

\nomenclature[A]{Greek / Latin}{Greek alphabet is used to describe the intrinsic property while Latin alphabet is used for estimator. e.g. $ \sigma ^2_X=var(X) $, $ \hat{\sigma }^2_X=\hat{var}(X)=S^2 $. e.g. $ \theta  $ is used to denote the parameter of distribution (family).}

\nomenclature[C]{$ \mathbb{P} $}{Probability measure}

\nomenclature[E]{$ \f(\, \cdot \, ),\,p(\, \cdot \, ),\,\pi(\, \cdot \, ) $}{Probability distribution functions. Specially I usually use $ \f(\, \cdot \, ) $ for continuous random variable PDF, and $ p(\, \cdot \, ) $ for discrete random variable PMF, and $ \pi(\, \cdot \, ) $ for priori distribution. Sometimes subscript is used to denote the r.v. that obey the distribution, and $ f(\, \cdot \, |\, \cdot \, ) $ or $ f(\, \cdot \, ;\, \cdot \, ) $ to denote conditional distribution. e.g. $ X\sim N(\mu ,\sigma ^2) $ could be denoted $ f_X(x|\mu ,\sigma ^2) $ where $ f=\phi  $.}

\nomenclature[C]{$ \mathbb{E},\,var,\,corr $}{r.v. Expectation, variance (in multivariate case, covariance matrix ), and correlation coefficient. e.g. say $ \mathbb{E}\left[ X \right],\,var(X)  $. Sometimes subscript is used to clarify to which r.v. we are considering expectation. e.g. $ \mathbb{E}_{X\sim f_X(x)}\left[ \log X \right]  $. Sometimes I simply use $ \mu _X $ for expectation and $ \sigma ^2_X $ or $ \Sigma _X $ for variance, $ \rho_X  $ for correlation coefficient.}

\nomenclature[C]{$ \bar{\, \cdot \, },\, S^2_{\cdot },\,S_{\cdot } $}{Sample mean, sample variance, sample standard deviation. e.g. $ \bar{X} $, $ S^2_X $, $ S_X $. \textbf{But in multivariate case I directly use $ S_X $ for sample covariance matrix.}}

\nomenclature[C]{$ \vec{\, \cdot \, } $}{is used to stress that $ \, \cdot \,  $ is a multi-dim vector.}


% \nomenclature[]{$ \mathscr{F} $}{Usually used to denote function family. e.g. $ \mathscr{F}=\{ f(\, \cdot \, )\,|\,f\in \mathcal{C}^{\infty}(\mathbb{R}) \}  $.}

\nomenclature[C]{$ \# $}{number of, or number of elements in, $ \cdots $}



\nomenclature[E]{$ Z $}{If not used as r.v. In most cases used as the normalize constant in unnormalized distribution, say $ f=\dfrac{ 1 }{ Z }\tilde{f}=\dfrac{1}{\int \tilde{f}(x) \,\mathrm{d}x } \tilde{f}   $}

\nomenclature[C]{$ \independent $}{independent of}
\nomenclature[C]{$ \mathbb{I}_{\cdot } $}{Indicator Function, in which subscript is the set that the indicator function takes value $ 1 $.}

\nomenclature[E]{$ M_\cdot (s) $}{Moment generating function. Usually functions with $ s $ as argument are all generating functions, say $ g(s) $ for probability generating function; $ \phi (s) $ for characteristic function.}

\nomenclature[E]{$ N(x|0 ,1),\, \phi (x),\,\Phi(x) $}{Distribution, PDF, CDF of standard normal distribution.}

\nomenclature[E]{$ T;\chi^2,F $}{(Usually) $ T $ for test statistic; occasionally $ \chi^2 $ for when testing is $ \chi^2 $ test, $ F $ for when testing is $ F $-test.}

\nomenclature[C]{i.i.d. or $ \mathop{ \sim }\limits_{i.i.d.}  $}{independent identically distributed.}

\nomenclature[C]{$ A = \mathop{ \arg\max }\limits_{\alpha } f(\alpha )  $}{$ A $ is the value of $ \alpha $ that maximizes $ f(\alpha ) $.}

\nomenclature[C]{$ X_{(1)},\ldots,X_{(n)} $}{Order statistics.}

\nomenclature[C]{$ X^{(1)},\ldots,X^{(t)} $}{Superscript with bracket is used in iteration algorithms to denote the value of this $ X $ in the $ t^\mathrm{ th }  $ iteration.}

\nomenclature[C]{$ N_{\alpha },\,F_{m,n,\alpha } $, etc.}{(Upper $ \alpha  $) Quantile of distributions. I use $ N_\alpha  $ for quantiles of normal distribution instead of $ z_\alpha  $.}

\nomenclature[E]{$ L,\,l,\ell $}{Likelihood, log-likelihood.}
\nomenclature[E]{$ \mathcal{L} $}{Loss function.}\nomenclature[E]{$ \mathcal{D } $}{Dataset.}
\nomenclature[E]{$ S(\theta ),\, I(\theta ),\, J(\theta ) $}{Score Function, Fisher Information, Observed Information.}
\nomenclature[C]{$ \mathrm{SSE},\,\mathrm{SSR},\,\mathrm{SST}   $}{For $ \mathrm{SSError} $, $ \mathrm{SSRegression} $, $ \mathrm{SSTotal} $.}

\nomenclature[E]{$ \mathop{ X }\limits_{n\times (p+1)}  $}{Design matrix in regression, with a default intercept column.}
\nomenclature[E]{$ \mathop{ \beta  }\limits_{(p+1)\times 1}  $}{Regression coefficient vector, with $ \beta _0 $ as the first element.}

\nomenclature[C]{$ H_0; \,H_1,H_a $}{Null hypothesis and alternative hypothesis.}
\nomenclature[C]{$ (\wedge i) $}{Used in a sequence, means that we dropout the $ i^\mathrm{ th }  $ item in the sequence indexed $ 1,2,\ldots $}
\nomenclature[C]{$ \equiv, := $}{is defined as}
\nomenclature[C]{$ \delta _{ij},\,\delta (x) $}{Kronecker delta, Dirac delta.}
\nomenclature[C]{$ \langle\, \cdot \, |\, \cdot \, \rangle,\,\langle\, \cdot \, ,\, \cdot \, \rangle $}{Inner product.}
\nomenclature[C]{$ \mathrm{ Re}(\, \cdot \, ),\,\mathrm{ Im }(\, \cdot \, )   $}{real part of, imaginary part of.}
\nomenclature[C]{$ \mathscr{B}(\, \cdot \, ) $}{Backshift operator.}
\nomenclature[C]{$ \mathscr{F}[\, \cdot \, ], \fallingdotseq; \risingdotseq $}{Fourier transform; Inversed fourier transform.}
\nomenclature[C]{$ O(n),\mathcal{O}(n);o(n) $}{Remainder in function series. Or used for complexity of algorithm}
