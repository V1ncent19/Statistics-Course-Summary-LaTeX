\section{R语言命令部分}
%     This part base on R x64 4.0.3 on VSCode 1.52.1

% \subsection{Basic}

\subsection{Basic Command}

\begin{table}[htbp]
    \centering
    \renewcommand\arraystretch{0.9}
    \begin{tabular}{l|l}
        \hline
        Action$\qquad\qquad\qquad\qquad\qquad\qquad\qquad$&Command$\qquad\qquad\qquad\qquad\qquad$\\
        \hline
        Get help&Hover on function name\\
        Get \textbf{example}&\lstinline|example('')|\\
        \textbf{Get} current \textbf{w}orking \textbf{d}irectory&\lstinline|getwd()|\\
        \textbf{L}i\textbf{s}t all objects&\lstinline|ls()|\\
        \textbf{R}e\textbf{m}ove objects&\lstinline|rm()|\\
        Remove all objects&\lstinline|rm(list <- ls())|\\
        Install package&\lstinline|install.packages('')|\\
        Load package&\lstinline|library('')|\\
        \hline
    \end{tabular}
\end{table}

% \subsubsection{Data Structures}
% \begin{itemize}[topsep=6pt,itemsep=4pt]
%     \item Vector: One-dimensional array
    
%     \begin{point}
%         Def. a vector
%     \end{point}
% \begin{lstlisting}[language=R]
%     myvector <- c(elements)
% \end{lstlisting}

%     Note: here \lstinline|c()| for '\textbf{c}ombine'.
    
%     \begin{point}
%         Index reference
%     \end{point}
    
% \begin{lstlisting}[language=R]
%     myvector[3]
%     myvector[c(2,4)]
%     myvector[2:4]
% \end{lstlisting}
    
%     \item Matrix: Two-dimensional array
%     \begin{point}
%         Def. a matrix
%     \end{point}
% \begin{lstlisting}[language=R]
%     mymatrix <- matrix(vector, nrow, ncol, byrow, dimname)
% \end{lstlisting}

% \begin{point}
%     Index reference
% \end{point}
% \begin{lstlisting}[language=R]
%     mymatrix[2,4]
% \end{lstlisting}


%     \item Array
%     \item \textbf{Data.Frame} 
%     \item List 
% \end{itemize}

    



% \subsection{Statistics Method}




% \subsection{Graph}

    matrix multiply: \lstinline|%*%|



\subsection{Basic Statistics Tools}
\begin{table}[H]
    \centering
    \renewcommand\arraystretch{0.9}
    \begin{tabular}{l|l}
        \hline
        Action$\qquad\qquad\qquad\qquad\qquad\qquad\qquad$&Command$\qquad\qquad\qquad\qquad\qquad$\\
        \hline
        \textbf{L}inear \textbf{m}odel &\lstinline|lm(yvar~xvar)|\\
        Show \textbf{summary} & \lstinline|summary()| 
        \hline
    \end{tabular}
\end{table}

\subsection{Useful Packages}

\subsubsection{\consolas{ggplot2} package}
    \lstinline|ggplot2| for `the \textbf{G}rammar of \textbf{G}raphics \textbf{Plot}' (\textbf{2}$^\mathrm{nd}$ version). \\
    Key thought: separate statistics operation (\lstinline|stat_|) and geometric operation (\lstinline|geom_|) and other parts.

    Basic syntax structures for a \lstinline|ggplot2| figure:
\begin{lstlisting}[language=R]
ggplot(data = , aes(x = , y = , ...)) +
    geom_XXX(...) + stat_XXX(...) + ... +
    annotate(...)+ ... +
    scale_XXX(...) + coord_XXX(...) + guides(...) + theme(...)    
\end{lstlisting}


\begin{itemize}[topsep=6pt,itemsep=4pt]
    \item 
\end{itemize}

    
\begin{point}
    \lstinline|aes()| for `aesthetic': Describe how variables in the data are mapped to visual properties (aesthetics) of \lstinline|geom| layer.
\end{point}



    \lstinline|geom_| functions describes
\begin{table}[H]
    \centering
      \begin{tabular}{l|l|l}
        \hline
      Functions  & Adds  & Options \\
      \hline
      \lstinline|geom_bar()|  & Bar chart  & \lstinline|color; fill; alpha | \\
      \lstinline|geom_boxplot()|  & Box plot  & \lstinline|color; fill; alpha; notch; width | \\
      \lstinline|geom_density()|  & Density plot  & \lstinline|color; fill; alpha; linetype | \\
      \lstinline|geom_histogram()|  & Histogram  & \lstinline|color; fill; alpha; linetype; binwidth | \\
      \lstinline|geom_hline()|  & Horizontal lines  & \lstinline|color; alpha; linetype; size | \\
      \lstinline|geom_jitter()|  & Jittered points  & \lstinline|color; size; alpha; shape | \\
      \lstinline|geom_line()|  & Line graph  & \lstinline|colorvalpha; linetype; size | \\
      \lstinline|geom_point()|  & Scatterplot  & \lstinline|color; alpha; shape; size | \\
      \lstinline|geom_rug()|  & Rug plot  & \lstinline|color; side | \\
      \lstinline|geom_violin()|  & Violin plot  & \lstinline|color; fill; alpha; linetype | \\
      \lstinline|geom_vline()|  & Vertical lines  & \lstinline|color; alpha; linetype; size | \\
      \hline
      \lstinline|geom_smooth()|  & Fitted line  & \lstinline|method; formula; color; fill; linetype; size | \\
      \lstinline|geom_text()|  & Text annotations & See the help for this function \\
      \hline
      \end{tabular}
  \end{table}
  
  Some useful \lstinline|geom_| function options (More options see `R in Action P445'):
\begin{table}[H]
    \centering
    \renewcommand\arraystretch{1}
    \begin{tabular}{l|l}
        \hline
        Options&Specifies\\\hline
        \lstinline|color|&Color of points\&lines\&borders of regions.\\
        \lstinline|fill|&Color of filled areas.\\
        \lstinline|alpha|&Transparency (0-1)\\
        \lstinline|linetype|&Line pattern see table.\\%%%%
        \lstinline|size|&Size of points\&lines.\\
        \lstinline|shape|&Point shape see table.\\%%%%%
        \lstinline|width|&Width of box plots.\\
        \lstinline|position|&\lstinline|dodge|/\lstinline|stack|/\lstinline|jitter| etc. position relation.\\
        \hline
        \lstinline|method|&Smoothing function to use.\\
        \lstinline|formula|&Smoothing formula, e.g. \lstinline|y~log(x)|.\\
        \lstinline|se|&Confidence interval (\lstinline|TRUE|/\lstinline|FALSE|).\\
        \lstinline|level|&Level of confidence interval.\\
    \end{tabular}
\end{table}



\lstinline|facet_| trellis command.
\begin{table}[H]
    \centering
    \renewcommand\arraystretch{1}
    \begin{tabular}{l|l}
        \hline
        \lstinline|facet_warp(~var,ncol)|&Separate plots by column.\\
        \lstinline|facet_warp(~var,nrow)|&Sparate plots by row.\\
        \lstinline|facet_grid(rowvar~colvar)|&Separate plots into combination of \lstinline|rowvar| and \lstinline|colvar|.\\
        \lstinline|facet_grid(row|&\\

        \hline
    \end{tabular}
\end{table}





\subsubsection{\consolas{data.frame} and \consolas{dplyr} package}

