\documentclass[x11names,11pt,a4paper]{ctexart}
%以下为所使用的宏包
\usepackage{ulem}%下划线
\usepackage{amsmath,amsfonts,amssymb,amsthm,amsbsy}%数学符号
\usepackage{graphicx}%插入图片
\usepackage{booktabs}%三线表
%\usepackage{indentfirst}%首行缩进
\usepackage{tikz}%作图
\usetikzlibrary{shapes,arrows,chains}
\usepackage{appendix}%附录
\usepackage{array}%多行公式/数组
\usepackage{tabularx}%绘制定宽表格
\usepackage{makecell}%表格缩并
\usepackage{siunitx}%SI单位
\usepackage{mathrsfs}%数学字体
\usepackage{enumitem}%列表间距
\usepackage{multirow}%列表横向合并单元格
\usepackage[colorlinks,linkcolor=red,anchorcolor=blue,citecolor=green]{hyperref}%超链接引用
\usepackage{float}%图片、表格位置排版
\usepackage{pict2e,keyval,fp,diagbox}%带有斜线的表格
\usepackage{fancyvrb,listings}%设置代码插入环境
\usepackage{minted}%代码环境设置
\usepackage{fontspec}%字体设置
\usepackage{color,xcolor}%颜色设置
\usepackage{titlesec}%标题格式设置
\usepackage{makeidx}\makeindex%索引 编译命令 makeindex summary.idx
\usepackage{authblk}%titlepage作者信息



%页边距设置
\usepackage[left=0.5in,right=0.5in,top=0.7in,bottom=0.57in]{geometry}

%段行设置
\linespread{1.4}%行距
\setlength{\parskip}{0.1\baselineskip}%段距
\setlength{\parindent}{2em}%缩进

%字体设置
\setmainfont{Times New Roman}
\newfontfamily\consolas{Consolas}

%其他设置
\numberwithin{equation}{section}%公式按照章节编号
\newenvironment{point}{\raggedright$\Box $ }{}%新环境,用于标示重点内容
\newenvironment{rcode}{\raggedright$  \vartriangleright  $ \lstinline|R.| \textbf{Code}  \\}{}%新环境,用于建立R语言命令段落
\newcolumntype{Y}{>{\centering\let\newline\\\arraybackslash\hspace{0pt}}X}%tabularx中的居中表格格式
\allowdisplaybreaks[2]%跨页公式

%代码环境\lst设置
\definecolor{CodeBlue}{HTML}{268BD2}
\definecolor{CodeBlue2}{HTML}{0000CD}
\definecolor{CodeGreen}{HTML}{2AA1A2}
\definecolor{CodeRed}{HTML}{CB4B16}
\definecolor{CodeYellow}{HTML}{B58900}
\definecolor{CodePurPle}{HTML}{D33682}
\definecolor{CodeGreen2}{HTML}{859900}
\lstset{
    language=R,
    basicstyle=\fontspec{Consolas},
    numbers=left, %设置行号位置
    numberstyle=\tiny\color{black}, %设置行号大小
    keywordstyle=\color{black}, %设置关键字颜色
    stringstyle=\color{gray}, %设置字符串颜色
    commentstyle=\color{CodeGreen}, %设置注释颜色
    frame=single, %设置边框格式
    escapeinside=``, %逃逸字符(1左面的键),用于显示中文
    %breaklines, %自动折行
    extendedchars=false, %解决代码跨页时,章节标题,页眉等汉字不显示的问题
    xleftmargin=2em,xrightmargin=2em, aboveskip=1em, %设置边距
    tabsize=4, %设置tab空格数
    showspaces=false, %不显示空格
    emph={TRUE,FALSE,NULL,NAN,NA,\$,<-,T,F,:},emphstyle=\color{CodeBlue2}, %其他高亮
}

%节标题格式设置
\titleformat{\section}[block]{\centering\Large\bfseries}{Chapter. \Roman{section} }{1em}{}[]
\titleformat{\subsection}[block]{\Large\bfseries}{Section \arabic{section}.\arabic{subsection}}{1em}{}[]
% \titleformat{\subsubsection}[block]{\normalsize\bfseries}{    \arabic{subsection}-\alph{subsubsection}}{1em}{}[]
% \titleformat{\paragraph}[block]{\small\bfseries}{[\arabic{paragraph}]}{1em}{}[]

% \\titleformat{\\sectioncommand}[shape]{format}{title-label}{sep}{before-title}[after-title]


%SumatraPDF反向搜索命令行
%"C:\Users\彭拓锐Vincent\AppData\Local\Programs\Microsoft VS Code\Code.exe" "C:/Users/彭拓锐Vincent/AppData/Local/Programs/Microsoft VS Code\resources\app\out\cli.js" -r -g "%f:%l"




\begin{titlepage}
    \title{\textbf{A Brief Summary of Statistics Course}\\统计学课程知识总结}
  \author[]{Vincent\thanks{V1ncent19@outlook.com}}
    \affil[]{Department of Physics, Tsinghua University}
\end{titlepage}

\begin{document}

\maketitle\thispagestyle{myheadings}\markright{Complied using \LaTeX}

\tableofcontents
\newpage




% =================================================
% Set up a few colours
\colorlet{lcfree}{Green3}
\colorlet{lcnorm}{Blue3}
\colorlet{lccong}{Red3}
% -------------------------------------------------
% Set up a new layer for the debugging marks, and make sure it is on
% top
\pgfdeclarelayer{marx}
\pgfsetlayers{main,marx}
% A macro for marking coordinates (specific to the coordinate naming
% scheme used here). Swap the following 2 definitions to deactivate
% marks.
\providecommand{\cmark}[2][]{%
  \begin{pgfonlayer}{marx}
    \node [nmark] at (c#2#1) {#2};
  \end{pgfonlayer}{marx}
  } 
\providecommand{\cmark}[2][]{\relax} 
% -------------------------------------------------





\section{概率论部分}\label{Section1Probability}
\begin{center}
    Instructor: Wanlu Deng
\end{center}

%     Chapter Overview
% \begin{itemize}[topsep=2pt,itemsep=2pt]
%     \item \hyperlink{Basic axioms}{Basic axioms}
% \end{itemize}

    









%     Cover:Basic axioms, random events, $\sigma$-field; random variable/vector and their properties, some special distributions; $E$\,\&\,$\sigma^2$\,\&\,$cov$ and their properties; probability-generating/moment-generating/characteristic function; weak/strong law of large number, central limit thm.; intro. to multivariate normal distribution.



\subsection{Some Important Distributions}\label{SectionImportantDistributions}

\begin{table}[htbp]
    \centering
    \begin{tabular}{c|ccccc}
        \hline
        $X$&$p_X(k)\big/f_X(x)$&$\quad \mathbb{E}\quad$&$var$&PGF&MGF\\
        \hline
        $\mathrm{Bern} (p)$& &$p$&$pq$&&$q+pe^s$\\
        $B (n,p)$&$C_n^k p^k(1-p)^{n-k}$&$np$&$npq$&$(q+ps)^n$&$(q+pe^s)^n$\\
        $\mathrm{Geo} (p)$&$(1-p)^{k-1}p$&$\dfrac{1}{p}$&$\dfrac{q}{p^2}$&$\dfrac{ps}{1-qs}$&$\dfrac{pe^s}{1-qe^s}$\\
        $H(n,M,N)$&$\dfrac{C_M^kC_{N-M}^{n-k}}{C_N^n}$&$n\dfrac{M}{N}$&$\dfrac{nM(N-n)(N-M)}{N^2(n-1)}$&&\\
        $P(\lambda)$&$\dfrac{\lambda^k}{k!}e^{-\lambda}$&$\lambda$&$\lambda$&$e^{\lambda(s-1)}$&$e^{\lambda(e^s-1)}$\\
        $U(a,b)$&$\dfrac{1}{b-a}$&$\dfrac{a+b}{2}$&$\dfrac{(b-a)^2}{12}$&&$\dfrac{e^{sb}-e^{sa}}{(b-a)^s}$\\
        $N(\mu,\sigma^2)$&$\dfrac{1}{\sigma \sqrt{2\pi}}e^{-\frac{(x-\mu)^2}{2\sigma^2}}$&$\mu$&$\sigma^2$&&$e^{\frac{\sigma^2s^2}{2}+\mu s}$\\
        $\epsilon(\lambda)$&$\lambda e^{-\lambda x}$&$\dfrac{1}{\lambda}$&$\dfrac{1}{\lambda^2}$&&$\frac{\lambda}{\lambda-s}$\\
        $\Gamma(\alpha,\lambda)$&$\dfrac{\lambda^\alpha}{\Gamma(\alpha)}x^{\alpha-1}e^{-\lambda x}$&$\dfrac{\alpha}{\lambda}$&$\dfrac{\alpha}{\lambda^2}$&&$\left(\frac{\lambda}{\lambda-s}\right)^\alpha $\\
        $B(\alpha,\beta)$&$\dfrac{1}{B(\alpha,\beta)}x^{\alpha-1}(1-x)^{\beta-1}$&$\dfrac{\alpha}{\alpha+\beta}$&$\dfrac{\alpha\beta}{(\alpha+\beta)^2(\alpha+\beta+1)}$&&\\
        $\chi^2_n$&$\dfrac{1}{2^{\frac{n}{2}}\Gamma(\frac{n}{2})}x^{\frac{n}{2}-1}e^{-\frac{x}{2}}$&$n$&$2n$&&$ (1-2s)^{-n/2} $\\
        $t_\nu$&$\dfrac{\Gamma(\frac{\nu+1}{2})}{\sqrt{\nu\pi}\Gamma(\frac{\nu}{2})}(1+\frac{x^2}{\nu})^{-\frac{\nu+1}{2}}$&$0$&$\dfrac{\nu}{\nu-2}$&&\\
        $F_{m,n}$&$\dfrac{\Gamma(\frac{m+n}{2})}{\Gamma(\frac{m}{2})\Gamma(\frac{n}{2})}\dfrac{m^\frac{m}{2}n^\frac{n}{2}x^{\frac{m}{2}-1}}{(mx+n)^{\frac{m+n}{2}}}$&$\dfrac{n}{n-2}$&$\dfrac{2n^2(m+n-2)}{m(n-2)^2(n-4)}$&&\\
        \hline
    \end{tabular}
\end{table}

    Definition of PGF, MGF, CF see \autoref{SectionPGFMGFCF}.

    More Properties of $\chi^2,t,F$ see {\autoref{chi2_t_F_properties}}.

    Relation between distributions and more properties see \url{http://www.math.wm.edu/~leemis/chart/UDR/UDR.html}. Distribution support in \lstinline|R.| see \url{https://CRAN.R-project.org/view=Distributions}

Use the following command for all distributions supported in \lstinline|R. stats::|.
\begin{lstlisting}[language=R]
?Distributions
\end{lstlisting}


\subsection{Probability and Probability Model}

    What is \textbf{Probability}? A `belief' in `what would happen'.


\subsubsection{Sample Space and $\sigma$-Field}

\begin{point}
    Experiment and Sample Space
\end{point}

    Def. sample space $\Omega$: The set of \text{all} possible outcomes of one particular \textbf{experiment} . Conducting the experiment would result in a result/sample point $ \omega  $ in sample space $ \Omega  $. These results should be mutually exclusive, e.g. Tossing two coins simultaneously, the sample space is the set of all possible results
    \begin{align}
        \Omega = \{(0,0),(0,1),(1,0),(1,1)\}, \quad \omega \in \Omega 
    \end{align}

    On the sample space, the `belief' in results happening is measured by probability $ \mathbb{P}\left( \omega  \right),\,\omega \in\Omega   $

    \textbf{Note}: Randomness comes from the random result $ \omega  $ that an experiment generates.
    
\begin{point}
    Event 
\end{point}

    We may care about a conbination of some results, say `at least one of the coin lands tails-up'. It's like a kind of `structure' on sample space describing how we put results together to form \textbf{Events}. The definition is \index{Sigma-field@$ \sigma $-Field} a $\sigma$-field(or a $\sigma$-algebra) $\mathscr{F}$ as a collection of some subsets of $\Omega$, with properties:
    \begin{itemize}[topsep=2pt,itemsep=0pt]
        \item $\Omega\in\mathscr{F}$
        \item if $A\in\mathscr{F}$,then $A^\complement \in\mathscr{F}$
        \item if $A_n\in\mathscr{F}$, then ${\displaystyle\bigcup_{n=1}^\infty} A_n\in\mathscr{F}$
    \end{itemize}

    And $(\Omega,\mathscr{F})$ is a measurable space, on which we can select the events that we care about.

    Events (and their properties) can be described in the language of set, e.g. for events $ A $, $ B\in\mathscr{F} $
\begin{itemize}[topsep=2pt,itemsep=0pt]
    \item $ A=B $ means they are the same event
    \item $ A\cup B $ means one of them happens
    \item $ A\cap B $ or $ AB $ means both happen 
\end{itemize}

And some more complex ones
\begin{itemize}[topsep=2pt,itemsep=0pt]
    \item $ A\cup B=B\cup A $, $ A\cap B=B\cap A $
    \item $ A\cup (B\cup C)=A\cup B\cup C $, $ A\cap (B\cap C)=A\cap B\cap C $
    \item $ A\cap (B\cup C)=(A\cap B)\cup (B\cap C) $, $ A\cup (B \cap C)=(A\cup B)\cap (A\cup C) $
    \item $ A\cup B=A+ A^\complement\cap B $, $ A=A\cap B+A\cap B^\complement $
    \item[$ \Delta  $] $ (A\cup B)^\complement =A^\complement\cap B^\complement $, $ (A\cap B)^\complement = A^\complement \cup B^\complement  $
    \item $ (\bigcup_{j=1}^\infty A_j)^\complement =\bigcap_{j=1}^\infty A_j^\complement $, $ (\bigcap{j=1}^\infty A_j)^\complement =\bigcup{j=1}^\infty A_j^\complement $
\end{itemize}
    
\subsubsection{Axioms of Probability}

    $\mathbb{P}(\,\cdot\,):\,\mathscr{F}\mapsto [0,1]$ is the probability measure (or probability function) defined on $(\Omega,\mathscr{F})$ describing the possibility that some event $ A\in\mathscr{F}  $ happens. Definition of probability $ \mathbb{P}(A) $ in useful models:
    \begin{align}
       \mathbb{P}\left( A \right) :=\begin{cases}
            \dfrac{\#A}{\#\Omega }&\text{Classical Model}\\
            \dfrac{m(A)}{m(\Omega )}&\text{Geometric Model}
        \end{cases}   
    \end{align}
    
    Where $ m(\, \cdot \, ) $ is some measure of events in continuous space, say integral in Euclidean Space $ \mathbb{R}^r $
    \begin{align}
        m_\mathrm{\mathbb{R}^r}(A)=\int_A \,\mathrm{d}x_1\,\mathrm{d}x_2\ldots\,\mathrm{d}x_r  
    \end{align}
    
    
    
    

    
    
    
\begin{point}
    Basic Axioms of Probability Mearure $ \mathbb{P}(\,\cdot\,) $
\end{point}

\begin{itemize}[itemsep=2pt,topsep=-2pt]
\item Non-negativity
\begin{equation}    \mathbb{P}(A)\geq 0\qquad \forall A\in\Omega    
\end{equation}
\item Normalization\footnote{Note: In other sections when dealing with not-yet-normalized distribution (say in Bayesian statistics), I usually use $ Z $ as the normalize constant, following the tradition in statistical physics where $ Z $ is the partition function.
\begin{align}
    \mathbb{P}=\dfrac{1}{Z}\tilde{\mathbb{P}},\quad Z=\int \mathbb{\tilde{P}} 
\end{align}}
\begin{equation}    \mathbb{P}(\Omega)=1    
\end{equation}




\item Countable Subadditivity\index{Countable Additivity}
\begin{equation}    \mathbb{P}(A_1\cup A_2\cup\cdots)=\mathbb{P}(A_1)+\mathbb{P}(A_2)+\cdots\quad ,\, (A_i\bot\!\!\!\bot  A_j\quad \forall i\neq j)
\end{equation}

where `countable subadditivity' means the events can be sequentially listed. e.g. $ [0,1]=\bigcup _{x\in [0,1]}\{x\} $ is not countable, thus
\begin{align}
    1=\mathbb{P}\left( [0,1] \right)=\mathbb{P}\left( \bigcup _{x\in [0,1]}\{x\} \right)   {\color{red}\neq} \sum_{x\in[0,1]}\mathbb{P}\left( x \right) =0
\end{align}


\end{itemize}

    Then $(\Omega,\mathscr{F},\mathbb{P})$ is probability space\index{Probability Space}, where $ \Omega  $ for experiment outcomes and randomness, $ \mathscr{F} $ for events and their algebra, $ \mathbb{P} $ for probability measure.

\begin{point}
        Properties of Probability:
\end{point}

    \begin{itemize}
        \item Addition Formula
        \begin{align}
            \mathbb{P}\left( A\cup B \right) =\mathbb{P}\left( A \right) +\mathbb{P}\left( B \right) -\mathbb{P}\left( A\cap B \right)  
        \end{align}
        \item Monotonicity
        \begin{equation}    
            \mathbb{P}(A)\leq \mathbb{P}(B)\quad \text{for}\, A\subset B
        \end{equation}
        \item Finite Subadditivity (Boole Inequality)\index{Inequality!Boole Inequality}
        \begin{equation}    
            \mathbb{P}(\bigcup_{i=1}^nA_i)\leq\sum_{i=1}^n \mathbb{P}(A_i)    
        \end{equation}
        \item Countable Subadditivity ($ \sigma  $-Subadditivity)\index{Sigma-Subadditivity@$ \sigma  $-Subadditivity}
        \begin{align}
            \mathbb{P}(\bigcup_{i=1}^\infty A_i)\leq\sum_{i=1}^\infty \mathbb{P}(A_i)  
        \end{align}
        
        
        \item Inclusion-Exclusion Formula (Jordan Formula)\index{Inclusion-Exclusion Formula}\index{Jordan Formula}
        \begin{align}
            \mathbb{P}(\bigcup_{i=1}^nA_i)&=\sum_{1\leq i\leq n}\mathbb{P}(A_i)-\sum_{1\leq i<j\leq n}\mathbb{P}(A_i\cap A_j)\\
            &+\sum_{1\leq i<j<k\leq n}\mathbb{P}(A_i\cap A_j\cap A_k)-\cdots\\
            &+(-1)^{n-1}\mathbb{P}(A_1 \cap A_2\cap\cdots \cap A_n)
        \end{align}

        Or in condensed notation:
        \begin{align}
            \mathbb{P}( \bigcup_{i=1}^n A_i)=&\sum_{k=1}^n (-1)^{k-1}\sum_{1\leq j_1<j_2<\ldots<j_k\leq n}\mathbb{P}\left( A_{j_1}\cap A_{j_2}\cap\ldots\cap A_{j_k} \right)   
        \end{align}
        
        
        \item Borel-Cantelli Lemma\index{Borel-Cantelli Lemma}
        \begin{align}
            &\sum_{n=1}^\infty \mathbb{P}(A_n)<\infty\Rightarrow \mathbb{P}(\lim_{n\to\infty}\sup A_n)=0\\
            &\sum_{n=1}^\infty \mathbb{P}(A_n)=\infty\Rightarrow \mathbb{P}(\lim_{n\to\infty}\sup A_n)=1\quad \text{if }A_i\text{ independent}
        \end{align}
            
    \end{itemize}


\begin{point}
    An Example 
\end{point}

    We have $ n $ different balls. Draw $ m $ times with replacement. What is the number of results regardless of order the balls drawn (e.g. $ \{\mathrm{red,red,black} \} $ is the same as $ \{\mathrm{red,black,red} \} $)? 

    The model is the same as we are `voting' for $ n $ different balls, with total ballot ticket $ m $. The $ m $ tickets are divided by $ n-1 $ plates (making them similar to ballot boxes), e.g. here's a $ n=4,m=6 $ vote corresponding to a result $ \omega \in\Omega  $:
    \begin{align}
         \bullet  \big|  \big| \bullet \bullet \bullet \big|  \bullet \bullet 
    \end{align}

    which the same as inserting plates sequentially and then cancel the order of plates:
    \begin{align}
        \# \Omega  = (m+1)*(m+2)\ldots (m+n-1)\bigg/ (n-1)! = \dfrac{(n+m-1)!}{m!(n-1)!}=\binom{n+m-1}{m}
    \end{align}

    (The idea of spacer plate is quite useful in dealing with some troublesome discrete cases, I think.)
    
    \begin{table}[H]
        \centering
        \renewcommand\arraystretch{1}
        \caption{$ \# \Omega  $ of Sampling $ n $ balls $ m $ draw}
        \begin{tabular}{ccc}
            \hline
            \hline
            &\multicolumn{2}{c}{Replacement}\\
            \cline{2-3}
            &With&Without\\
            \hline
            Ordered&$ n^m $&$ A_{n}^m $\\
            Unordered&$ \binom{n+m-1}{m} $&$ \binom{n}{m} $\\
            \hline
            \hline
        \end{tabular}
        \label{}
    \end{table}


    
    
    


\subsubsection{Conditional Probability}
    Motivation: To update the knowledge of probability measure.

    Def. \textbf{Conditional Probability} of $B$ given $A$:
    \begin{equation}    
        \mathbb{P}(B|A)=\frac{\mathbb{P}(A\cap B)}{\mathbb{P}(A)}    
    \end{equation}

    Actually it's a change of $\sigma$-field: $\Omega$ $ \to $ $B$
    \begin{align}
        \mathbb{P}\left( B|A \right) = \dfrac{m(B)}{m(A)} 
    \end{align}


\begin{point}
    Application of conditional probability:
\end{point}

        \begin{itemize}
        \item Multiplication Formula
        \begin{equation}    
            \mathbb{P}(\bigcap_{i=1}^n A_i)=\mathbb{P}(A_1)\prod_{i=2}^n \mathbb{P}(A_i|A_1\cap A_2\cap \cdots\cap A_{i-1})    
        \end{equation}
        \item Total Probability Thm.\index{Total Probability Thm.}
        \begin{equation}    
            \mathbb{P}(B)=\sum_{i=1}^n \mathbb{P}(A_i)\mathbb{P}(B|A_i)  
        \end{equation}
        where $\{A_i\}$ is a partition of $\Omega$: $ \Omega =\bigcup_{i}A_i ,\, A_i\cap A_j=\delta _{ij}\emptyset$

        (Actually just $ B\subset \bigcup_{i}A_i $ is enough, similar for Bayes's rule)
        \item Bayes's Rule\index{Bayes's Rule}
        \begin{equation}    
            \mathbb{P}(A_i|B)=\dfrac{\mathbb{P}(A_i)\mathbb{P}(B|A_i)}{\sum_{j=1}^\mathbb{P}(A_j)\mathbb{P}(B|A_j)}    ,\quad 1\leq i\leq n
        \end{equation}
        where $\{A_i\}$ is a partition of $\Omega$: $ \Omega =\bigcup_{i}A_i,\, A_i\cap A_j=\delta _{ij}\emptyset $
    \end{itemize}

\subsubsection{Independency}
    Statistical Independency is defined as:
    \begin{equation}    
        A\independent B:\,\mathbb{P}(A\cap B) =\mathbb{P}(A)\mathbb{P}(B)
    \end{equation}

    Properties
    \begin{itemize}[topsep=2pt,itemsep=0pt]
        \item Complement set and indepency
        \begin{align}
            A\independent B\Leftrightarrow A^\complement \independent B 
        \end{align}
        \item Independency of multiple events
        \begin{align}
            A_1\independent A_2\independent\ldots \independent A_n \Leftrightarrow &\mathbb{P}\left( A_{j_1}\cap A_{j_2}\cap\ldots\cap A_{j_k} \right)=\mathbb{P}\left( A_{j_1} \right) \mathbb{P}\left( A_{j_2} \right) \ldots \mathbb{P}\left( A_{j_k} \right) \\
            &  \,\forall 1\leq j_1\leq j_2\leq \ldots \leq j_k\leq n\quad \forall k\leq n
        \end{align}
    \end{itemize}
    
        

\subsection{Random Variable and Distribution}\label{SectionPropertiesOfRandomVariableAndVector}

Motivation: defining events is troublesome, and unhelpful to extract the key feature of events. A wise approach is to map samples \& events to numbers $ \Omega \mapsto \mathbb{R}^r $.

\subsubsection{Random Variable}
    Def. \text{Random Variable}: a \textbf{function}/mapping $X$ defined on sample space $\Omega$,  from $\Omega$ to some $\mathscr{X}\in\mathbb{R} $.
    \begin{align}
        X(\omega ):\, \Omega \mapsto \mathscr{X}\in\mathbb{R} 
    \end{align}

    \textbf{Note}: The mapping itself is non-random, the heart of randomness is still sample $ \omega  $ experimented. 

    Naturally $ X $ induces a mapping of probability measure
    \begin{align}
        F_X: \mathscr{X} \mapsto \Omega \mapsto \mathbb{P} 
    \end{align}

    To describe the mapping of probability, def. Cumulative Distribution Function (CDF)\index{CDF (Cumulative Distribution Function)}. (Here $ X(\omega ) $ is still used to remind the origin of randomness, in most case we simply use $ X $. )
    \begin{equation}
        F_X(x)=\mathbb{P} (X(\omega )\leq x)
    \end{equation}



    \begin{itemize}
        \item
        \begin{center}
            \parbox[t]{8.65cm}{PMF:\index{PMF (Probability Mass Function)}\begin{equation}        p_X(x)=F_X(x^+)-F_X(x^-)\end{equation}}
            \parbox[t]{8.65cm}{PDF:\index{PDF (Probability Density Function)}
            \begin{equation}        
                f_X(x)=\frac{\mathrm{d}F_X(x)}{\mathrm{d}x}
            \end{equation}}
        \end{center}
        \item Right-Continuity of CDF: A physical perspective is that PMF could be written as\footnote{Definition of Dirac $ \delta  $ function see \autoref{SubSubSectionFourierAndConvolution}.} 
        \begin{align}
            p_X(x)=\sum_{\tilde{x}\in\mathcal{X}}\mathbb{P}\left( X=\tilde{x} \right) \delta (x-\tilde{x})
        \end{align}
        where discrete $ X $ take values in $ \mathcal{X} $. In this way for any infinitesimal interval containing $ x $: $ \mathbb{I}_{x}\ni x$, we have 
        \begin{align}
            F_X(x^+)-F_X(x^-)=\int_{\mathbb{I}_{x}} p_X(x)\,\mathrm{d}x= \int_{\mathbb{I}_{x}}\sum_{\tilde{x}\in\mathcal{X}}\mathbb{P}\left( X=\tilde{x} \right) \delta (x-\tilde{x})\,\mathrm{d}x=\begin{cases}
                F_X(x^+)-F_X(x^-),&x\in \mathcal{X}\\
                0,&\text{others}
            \end{cases}
        \end{align}

        With such notation, in this note I sometimes ignore the difference between discrete cases / continuous cases.
        \item Representation of events: We could use random variable to express, say event $ A $ defined as        
        \begin{align}
            A:=\{\omega :X(\omega )\leq x\} 
        \end{align}
        
        
        \item Indicator function:\index{Indicator Function}
        \begin{equation}    
            \mathbb{I}_{x\in A}(x)=\begin{cases}
                1& x\in  A\\
                0& x\notin A
            \end{cases}
        \end{equation}
        \item Convolution\index{Convolution}
        \begin{itemize}
            \item $W=X+Y$
            \begin{equation}        
                f_W(w)=\int_{-\infty}^\infty f_X(x)f_Y(w-x)\mathrm{d}x    
            \end{equation}
            \item $V=X-Y$
            \begin{equation}        
                f_V(v)=\int_{-\infty}^\infty f_X(x)f_Y(x-v)\mathrm{d}x    
            \end{equation}
            \item $Z=XY$
            \begin{equation}        
                f_Z(z)=\int_{-\infty}^\infty \frac{1}{|x|}f_X(x)f_Y(\frac{z}{x})\mathrm{d}x
            \end{equation}
        \end{itemize}

            Examples:        
        \begin{itemize}[topsep=2pt,itemsep=0pt]
            \item Poisson\footnote{More about Poisson Distribution / Poisson Process see \autoref{SubSubSectionIndepedentProcess}}
            \begin{align}
                P(\lambda _1)+P(\lambda _2)\sim P(\lambda _1)+P(\lambda _2) 
            \end{align}
            \item Binomial
            \begin{align}
                B(n_1,p)+B(n_2,p)\sim B(n_1+n_2,p) 
            \end{align}
            \item Gamma / Exponential
            \begin{align}
                \Gamma (\alpha _1,\lambda )+\Gamma (\alpha _2,\lambda )\sim \Gamma (\alpha _1+\alpha _2,\lambda ) 
            \end{align}
            with 
            \begin{align}
                \varepsilon (\lambda )=\Gamma (1,\lambda ) 
            \end{align}
            
            
            
            
            \item More relations of distributions see \url{http://www.math.wm.edu/~leemis/chart/UDR/UDR.html}
            
            
        \end{itemize}
        
        \item Order Statistics\index{Order Statistics}\footnote{A relative object is Rank statistics, see \autoref{SubSectionIntroToNonParametricHypothesisTesting}.}
        
        Def $X_{(1)},X_{(2)},\cdots,X_{(n)}$ as order statistics of $\vec{X}$
        \begin{equation}    
            g_{X_{(i)}}=n!\prod_i f(x_i)\qquad \mathrm{for}\, x_1<x_2\cdots <x_n    
        \end{equation}
        PDF of $X_{(k)}$
        \begin{equation}\label{EqaDistributionOfOrderStatistics} 
            g_k(x_k)=nC_{n-1}^{k-1}[F(x_k)]^{k-1}[1-F(x_k)]^{n-k}f(x_k)
        \end{equation}
        \item $p$-fractile\index{Fractile!$ p $-fractile}
        \begin{equation}    \xi_p=F^{-1}(p)=\inf\{x|F(x)\geq p\}\end{equation}
    \end{itemize}






\subsubsection{Random Vector}
    A general case of random variable.\index{r.v. (Random Variable or Random Vector)} Its definition is similar    
    \begin{align}
        \vec{X}(\omega ):\, \Omega \mapsto \mathscr{X}\in\mathbb{R}^n 
    \end{align}
    a $n$-dimension Random Vector $\vec{X}=(X_1,X_2,\ldots,X_n)$ defined on $(\Omega,\mathscr{F},\mathbb{P})$.

    CDF $F(x_1,\ldots,x_n)$ defined on $\mathbb{R}^n$:
    \begin{equation}F(x_1,\ldots,x_n)=\mathbb{P}(X_1\leq x_1,\ldots,X_n\leq x_n)\end{equation}

    Joint PDF of random vector: 
    \begin{equation}
        f(x_1,\ldots,x_n)=\dfrac{\partial^n F(x_1,\ldots,x_n)}{\partial x_1\ldots\partial x_n}
    \end{equation}

    $k$-dimensional Marginal Distribution: For $1\leq k<n$ and index set $S_k=\{i_1.\ldots,i_k\}$, distribution of $\vec{X}=(X_{i_1},X_{i_2},\ldots,X_{i_k})$
    \begin{equation}F_{S_k}(X_{i_1}\leq x_{i_1},X_{i_2}\leq x_{i_2}\ldots,X_{i_k}\leq x_{i_k})=\mathbb{P}(X_{i_1}\leq x_{i_1},\ldots,X_{i_k}\leq x_{i_k};X_{i_{k+1}},\ldots,X_{i_n}\leq\infty)\end{equation}

    Marginal distribution: 
    \begin{equation}
        g_{S_k}(x_{i_1},\ldots,x_{i_k})=\int_{\mathbb{R}^{n-k}}f(x_1,\ldots,x_n)\mathrm{d}x_{i_{k+1}}\ldots\mathrm{d}x_{j_n}=\dfrac{\partial^{n-k}F(x_1,\ldots,x_n)}{\partial x_{i_{k+1}}\ldots\partial x_{i_n}}
    \end{equation}


    \begin{itemize}
        \item[$\Delta$] \textbf{Function of r.v.}
        
        For $\vec{X}=(X_1,X_2,\cdots,X_n)$ with PDF $f(\vec{X})$ and define 
        \begin{equation}    
            \vec{Y}=(Y_1,Y_2,\cdots,Y_n)=\left(y_1(\vec{X}),y_2(\vec{X}),\cdots,y_n(\vec{X})\right)
        \end{equation}
        with inverse mapping
        \begin{equation}    
            \vec{X}=(X_1,X_2,\cdots,X_n)=\left(x_1(\vec{Y}),x_2(\vec{Y}),\cdots,x_n(\vec{Y})\right)
        \end{equation}
        then
        \begin{equation}    
            g(\vec{Y})= f\left(x_1(\vec{Y}),x_2(\vec{Y}),\cdots,x_n(\vec{Y})\right)\left|\frac{\partial \vec{X}}{\partial\vec{Y}}\right|\mathbb{I}_{D_Y}
        \end{equation}

        (Intuitively: $g(\vec{Y} )\mathrm{d}\vec{Y}=\mathrm{d}\mathbb{P}=f(\vec{X})\mathrm{d}\vec{X}$)
    \end{itemize}



    




\subsection{Expectation $\mathbb{E}$, Variance $var$ and Covariance $cov$}
Motivation: what would happen `on average'?

Expectation and Variance of common distributions see \autoref{SectionImportantDistributions}.

\subsubsection{Expection $ \mathbb{E}(\,\cdot\,) $}
    Expectation of r.v. $g(X)$ def.:
    \begin{equation}
    \mathbb{E} [g(X)]=\begin{cases}
        {\displaystyle\int_\Omega g(x) f_X(x)\mathrm{d}x=\int_\Omega g(x)\mathrm{d}F(x)}\\
        {\displaystyle\sum_{\Omega}g(x)f_X(x)}
    \end{cases}
\end{equation}

    Sometimes when there are more than 1 variables, say $ x,y $, we would use notation $ \mathbb{E}_X\left( g(X,Y) \right)  $ to specify the variable to avoid confusion.

    \textbf{Note}: For discrete r.v. the expectation always exists, but for continuous \& unbounded r.v. the expectation might diverge, rigorously speaking:
    \begin{align}
        \mathbb{E}\left[ X \right]\exists:\, \int_{\mathbb{R}}|x|f(x)\,\mathrm{d}x<\infty  
    \end{align}
    
    
 
\begin{point}
    Properties of Expectation $E(\,\cdot\,)$:
\end{point}

\begin{itemize}
    \item Linearity of Expectation\begin{equation}
        \mathbb{E}(aX+bY)=a \mathbb{E}(X)+b\mathbb{E}(Y)
    \end{equation}
    \item Conditional Expectation\begin{equation}
        \mathbb{E}(X|A)=\frac{\mathbb{E}(X\mathbb{I}_A)}{\mathbb{P}(A)}
    \end{equation}
    
    Note: if take $A$ as $Y$ is also a r.v. then conditional expectation is actually a function of $Y$
    \begin{equation}\xi (Y)=\mathbb{E}(X|Y)=\int xf_{X|Y}(x)\mathrm{d}x\end{equation}

    

    \item Law of Total Expectation\begin{equation}
        \mathbb{E}_Y\big\{\mathbb{E}_X[g(X)|Y]\big\}=\mathbb{E}_X[g(X)]
    \end{equation}
    \item r.v.\& Event
    \begin{equation}
        \mathbb{P}(A|X)=\mathbb{E}(\mathbb{I}_A|X)\Rightarrow \mathbb{E}[P(A|X)]=\mathbb{E}(\mathbb{I}_A)=\mathbb{P}(A)
    \end{equation}
    \item Conditional Expectation
    \begin{equation}
        \mathbb{E}\big[h(Y)g(X)|Y\big]=h(Y)\mathbb{E}[g(X)|Y]
    \end{equation}
\end{itemize}


\subsubsection{Variance $ var(\, \cdot \, ) $}
    Variance of r.v. $X$: 
    \begin{equation}
        var(X)=\mathbb{E}\big[(X-\mathbb{E}(X))^2\big]=\mathbb{E}(X^2)-(\mathbb{E}(X))^2
    \end{equation}
    (sometimes denoted as $\sigma^2_X$.)

    Another definition comes from the MMSE estimation, 
    \begin{align}
        var(X)=\mathop{\min}\limits_{c}\mathbb{E}\left[ (X-c)^2 \right]   
    \end{align}
    
    its solution is $ c=\mathbb{E}\left[ X \right]  $. See \autoref{SubSecMMSE} for more.

\begin{point}
    Properties:
\end{point}

\begin{itemize} 
    \item Linear combination of Variance\begin{equation}
        var(aX+b)=a^2var(X)
    \end{equation}
    \item Conditional Variance
    \begin{equation}
        var(X|Y)=\mathbb{E}{[X-\mathbb{E}(X|Y)]^2|Y}
    \end{equation}
    \item Law of Total Variance\begin{equation}
        var(X)=\mathbb{E}[var(X|Y)]+var[\mathbb{E}(X|Y)]
    \end{equation}
\end{itemize}

    Standard Deviation def. as :
    \begin{equation}\sigma_X=\sqrt{var(X)}\end{equation}

    Then can construct \textbf{Standardization}\index{Standardization} of r.v.
    \begin{equation}X_\mathrm{sd} =\frac{X-\mathbb{E}(X)}{\sqrt{var(X)}}\end{equation}


\subsubsection{Covariance $ cov(\, \cdot \, ) $ and Correlation $ corr(\, \cdot \, ) $}\label{SubSubSectionCovarianceAndCorrelation}
    Covariance of r.v. $X$ and $Y$:\begin{equation}
        cov(X,Y)=\mathbb{E}\big[(X-\mu_X)(Y-\mu_Y)\big]=\mathbb{E}(XY)-\mathbb{E}(X)\mathbb{E}(Y)
    \end{equation}

    And Correlation Coefficient\index{Correlation Coefficient}
    \begin{equation}
        \rho_{X,Y}=corr(X,Y)=\frac{cov(X,Y)}{\sqrt{var(X)var(Y)}}
    \end{equation}

    Remark: correlation $\nRightarrow$ cause and effect. 
    Detail on causal effect topic see \autoref{SecCausalInference}.

    Properties:
\begin{itemize}
\item Bilinear of Covariance\begin{align}
    cov(X+Y,Z)&=cov(X,Z)+cov(Y,Z)\\
    cov(X,Y+Z)&=cov(X,Y)+cov(X,Z)
\end{align}
    
\item Variance and Covariance
\begin{equation}\label{EqaVarOfSumOfRV}
    var(X+Y)=var(X)+var(Y)+2cov(X,Y)
\end{equation}
\item Covariance Matrix\index{Covariance Matrix}

    Def $\Sigma=\mathbb{E}\big[(X-\mu)(X-\mu)^T\big]=\{\sigma_{ij}\}$ (where $X$ should be considered as a column vector)
\begin{equation}\label{covariancematrix}
    \Sigma=
        \begin{pmatrix}
        var(X_1) & cov(X_1,X_2) & \ldots & cov(X_1,X_n)\\
        cov(X_2,X_1) & var(X_2) & \ldots & cov(X_2,X_n)\\
        \vdots & \vdots & \ddots & \vdots\\
        cov(X_n,X_1) & cov(X_n,X_2) & \ldots & var(X_n)\\
        \end{pmatrix}    
    \end{equation}
\end{itemize}

Attachment: Independence:\begin{equation}    X_i \independent X_j\Rightarrow \begin{cases}
        f(x_1,x_2,\cdots,x_n)=\prod f(x_i)\\
        F(x_1,x_2,\cdots,x_n)=\prod F(x_i)\\
        E(\prod X_i)=\prod E(X_i)\\
        var(\sum X_i)=\sum var(X_i)
    \end{cases}
\end{equation}


\subsection{PGF, MGF and C.F}\label{SectionPGFMGFCF}

    Generating Function: Representation of $\mathbb{P}$ in function space. $\mathbb{P}\Leftrightarrow$ Generating Function.

\subsubsection{Probability Generating Function}
    PGF\index{PGF (Probability Generating Function)}: used for non-negative, integer $X$, which is the $ z $-transform of $ p_X $
    \begin{equation}
        g(s)=\mathbb{E}(s^X)=\sum_{j=0}^\infty s^j\mathbb{P}(X=j)    ,s\in[-1,1]
    \end{equation}

\begin{point}
        Properties
\end{point}
    \begin{itemize}[topsep=2pt,itemsep=0pt]
        \item $\mathbb{P}(X=k)=\dfrac{g^{(k)}(0)}{k!}$
        \item $\mathbb{E}(X)=g^{(1)}(1)$
        \item $var(X)=g^{(2)}(1)+g^{(1)}(1)-[g^{(1)}(1)]^2 $
        \item For $X_1,X_2,\cdots,X_n$ independent with $g_i(s)=\mathbb{E}(s^{X_i})$, $Y={\displaystyle \sum_{i=1}^n} X_i$, then
        \begin{equation}    
            g_Y(s)=\prod_{i=1}^n g_i(s),s\in[-1,1]
        \end{equation}
        \item For ${X_i}$ i.i.d with $\psi_i(s)=\psi(s)\equiv \mathbb{E}(s^{X_i})$, $Y$ with $G(s)\equiv\mathbb{E}(s^{Y})$, $W=X_1+X_2+\cdots +X_Y$,then
        \begin{equation}    
            g_W(s)=G[\psi(s)]    
        \end{equation}
        \item 2-Dimensional PGF of $(X,Y)$
        \begin{equation}    
            g(s,t)=\mathbb{E}(s^Xt^Y)=\sum_{i=o}^\infty\sum_{j=0}^\infty \mathbb{P}_{(X,Y)}(X=i,Y=j)s^it^j,\quad s,t\in[-1,1]
        \end{equation}
    \end{itemize}
\subsubsection{Moment Generating Function}
    MGF\index{MGF (Moment Generating Function)}: used for non-negative $ X $, which is the Laplacian transformation of $ f_X $.
    \begin{equation}
        M_X(s)=\mathbb{E}(e^{sX})=\begin{cases}
            \sum_je^{sx}\mathbb{P}(X=x_j)\\
            \int_{-\infty}^\infty e^{sx}f_X(x)\mathrm{d}x
        \end{cases}
    \end{equation}

    Properties
    \begin{itemize}
        \item MGF of $Y=aX+b$: $
            M_Y(s)=e^{sb}M(sa)    $
        \item $\mathbb{E}(X^k)=M^{(k)}(0)$
        \item $\mathbb{P}(X=0)={\displaystyle\lim_{s\to -\infty}}M(s)$
        \item For $X_1,X_2,\cdots,X_n$ independent with $M_{X_i}(s)=\mathbb{E}(e^{sX_i})$, $Y={\displaystyle \sum_{i=1}^n} X_i$, then
        \begin{equation}    
            M_Y(s)=\prod_{i=1}^n M_{X_i}(s)
        \end{equation}
    \end{itemize}
\subsubsection{Characteristic Function}
    C.F \index{C.F. (Characteristic Function)}is actually the Fourier Transform of $f_X$.
    \begin{equation}
        \phi(t)=\mathbb{E}(e^{itX}) = \int_{-\infty}^\infty e^{itx}f_X(x)\mathrm{d}x
    \end{equation}

    Properties
    \begin{itemize}
    \item if $E(|X|^k)<\infty$,then
    \begin{equation}
        \phi^{(k)}(t)=i^k\mathbb{E}(X^ke^{itX})\qquad \phi^{(k)}(0)=i^k\mathbb{E}(X^k)    
    \end{equation}
    \item For $X_1,X_2,\cdots,X_n$ independent with $\phi_{X_i}(t)=\mathbb{E}(e^{itX_i})$, $Y={\displaystyle \sum_{i=1}^n} X_i$, then
    \begin{equation}
        \phi_Y(t)=\prod_{i=1}^n \phi_{X_i}(t)
    \end{equation}
    \item Inverse (Fourier) Transform
    \begin{equation}
        f(x)=\frac{1}{2\pi}\int_{-\infty}^\infty e^{-itx}\phi(t)\mathrm{d}t    
    \end{equation}
\end{itemize}



\subsection{Convergence and Limit Distribution}
\subsubsection{Convergence Mode}
    \index{Convergence}
    \begin{equation}
        \begin{cases}
            \text{Convergence in Distribution }&{\displaystyle X_n\xrightarrow[]{\mathrm{d}}X:\lim_{n\to\infty}F_n(x)=F(x)}\\
            \text{Convergence in Probability }&{\displaystyle X_n\xrightarrow[]{\mathrm{p}}X:\lim_{n\to\infty}\mathbb{P}(|X_n-X)\geq\varepsilon)=0\, ,\forall\varepsilon>0}\\
            \text{Almost Sure Convergence }&{\displaystyle X_n\xrightarrow[]{\text{a.s.}}X:\mathbb{P}(\lim_{n\to\infty}X_n=X)=1}\\
            L_p\text{ Convergence }&{\displaystyle X_n\xrightarrow[]{L_p}X:\lim_{n\to\infty}\mathbb{E}(|X_n-X|^p)=0}
        \end{cases}
    \end{equation}

        Relations between convergence:
        \begin{center}
            \begin{tikzpicture}
                \draw(-1,-1)rectangle(1,1);
                \draw(-3,-0.5)rectangle(-5,-2.5);
                \draw(-3,0.5)rectangle(-5,2.5);
                \draw(3,-1)rectangle(5,1);
                \draw[-latex](-3,-1.5)--(-1,-0.5);
                \draw[-latex](-1,-0.75)--(-3,-1.75);
                \draw[-latex](-3,1.5)--(-1,0.5);
                \draw[-latex](1,0)--(3,0);
                \node at (0,0){$X_n\xrightarrow[]{\mathrm{p}}X$};
                \node at (-4,-1.5){$X_n\xrightarrow[]{L_p}X$};
                \node at (-4,1.5){$X_n\xrightarrow[]{\text{a.s.}}X$};
                \node at (4,0){$X_n\xrightarrow[]{\mathrm{d}}X$};
                \node at (-1.5,-1.5){$L_p<\infty $};
            \end{tikzpicture}
        \end{center}

        Note: $ L_2 $ convergence is also denoted m.s. (mean squared) convergence $ \xrightarrow[]{\mathrm{m.s.}}  $.

        Useful Thm.:
        \begin{itemize}
            \item Continuous Mapping Thm.: For continuous function $g(\cdot)$\index{Continuous Mapping Thm.}\hypertarget{ContinuousMapping}{}
            \begin{enumerate}
                \item $X_n\xrightarrow[]{\text{a.s.}}X\Rightarrow g(X_n)\xrightarrow[]{\text{a.s.}}g(X)$
                \item $X_n\xrightarrow[]{\mathrm{p}}X\Rightarrow g(X_n)\xrightarrow[]{\mathrm{p} }g(X)$
                \item $X_n\xrightarrow[]{\mathrm{d}}X\Rightarrow g(X_n)\xrightarrow[]{\mathrm{d}}g(X)$
            \end{enumerate}
            \item Slutsky's Thm.\index{Slutsky's Thm.}: For $X_n\xrightarrow[]{\mathrm{d}}X,Y_n\xrightarrow[]{\mathrm{p}}c$
            \begin{enumerate}
                \item $X_n+Y_n\xrightarrow[]{\mathrm{d}}X+c$
                \item $X_nY_n\xrightarrow[]{\mathrm{d}}cX$
                \item $X_n/Y_n\xrightarrow[]{\mathrm{d}}X/c$
            \end{enumerate}
            \item Continuity Thm. for characteristic function:
            \begin{equation}        \lim_{n\to\infty}\phi_n(t)=\varphi(t)\Leftrightarrow X_n\xrightarrow[]{\mathrm{d}}X\end{equation}
        \end{itemize} 


\subsubsection{Law of Large Number \& Central Limit Theorem}

\begin{itemize}
\item m.s. LLN\index{m.s. LLN (Mean-Squared Law of Large Number)}: For $ X_i $ with $ cov(X_i,X_j)=0$, if $ i\neq j $, and $ \mathbb{E}\left[ X_i \right] =\mu <\infty $
\begin{align}
    \frac{1}{n}\sum X_i\xrightarrow[]{L_2} \mathbb{E}\left[ X_1 \right]
\end{align}


\item WLLN\index{LLN (Law of Large Number)}\index{WLLN (Weak Law of Large Number)}: For $ X_i $ i.i.d. $ \sim f_X $, with $ \mathbb{E}\left[ X_i \right]=\mu <\infty $
\begin{equation}\label{EqaWLLN}    \frac{1}{n}\sum X_i\xrightarrow[]{\mathrm{p}}\mu 
\end{equation}
\item SLLN\index{SLLN (Strong Law of Large Number)}: For $ X_i $ i.i.d. $ \sim f_X $, with $ \mathbb{E}\left[ X_i \right] =\mu <\infty $
\begin{equation}    \frac{1}{n}\sum X_i\xrightarrow[]{\text{a.s.}}  \mu 
\end{equation}
\item CLT\index{CLT (Central Limit Thm.)}: For $ X_i $ i.i.d. $ \sim f_X $, with $ \mathbb{E}\left[ X_i \right] =\mu <\infty $, $ var(X_i)=\sigma ^2<\infty $
\begin{align}
     \dfrac{\sqrt{n}\left(\bar{X}-\mu \right)}{\sigma }\xrightarrow[]{\mathrm{d}} N(0,1)
\end{align}
or in equivalent form
\begin{align}    
    &\frac{1}{\sigma\sqrt{n}}\sum(X_k-\mu)\xrightarrow[]{\mathrm{d}} N(0,1)\\
    &\bar{X}\xrightarrow[]{\mathrm{d}} N(\mu ,\dfrac{\sigma ^2}{n})
\end{align}

\begin{proof}
    Denote the characteristic function of $ X\sim f_X(x) $ as $ \phi _X(t):=\mathbb{E}\left[ e^{itX} \right] $, with expectation $ \mu :=\mathbb{E}\left[ X \right]  $ and variance $ \sigma^2:=var(X)=\mathbb{E}\left[ X^2\right] -\mu  ^2 $.
    
    Define $ Z=\dfrac{X-\mu }{\sigma } $ The taylor series of $ \phi _Z(t) $ at $ t=0 $ yields:
    \begin{align*}
        \phi _Z(t)=1-\dfrac{t^2}{2}+o(t^2) 
    \end{align*}
    The characteristic function of mean $ \displaystyle\bar{Z}:=\dfrac{1}{n}\sum_{i=1}^nZ_i=\dfrac{1}{n}\sum_{i=1}^n\dfrac{X_i-\mu }{\sigma } $ w.r.t. $ X_i $ i.i.d. $ \sim f_X(x) $
    \begin{align*}
        \phi _{\bar{Z}}(t)=\mathbb{E}\left[ e^{it\bar{Z}} \right]=&\left[\phi _{Z}(\dfrac{t}{n})\right]^n=\left[1-\dfrac{t^2}{2n^2}\right] ^n 
    \end{align*}
    with $ n\to\infty $ limit:\footnote{Note: if use characteristic function of $ X_i $ directly, notice that
    \begin{align*}
        n\log\left(1+ \dfrac{at}{n}-\dfrac{bt^2}{2n^2}\right)= at -\left(b+a^2\right)\dfrac{t^2}{2n}+\mathcal{O}(\dfrac{1}{n^2})
    \end{align*}
    using the taylor series of $ \log(1+\xi ) $ at $ \xi =0 $.
    }
    \begin{align*}
        \lim_{n\to\infty}\phi _{\bar{Z}}(t)=&\lim_{n\to\infty} \left[1-\dfrac{1}{n}\dfrac{t^2}{2n}\right] ^n=e^{-\frac{t^2}{2n}} \Rightarrow \bar{Z}=\dfrac{\bar{X}-\mu }{\sigma }\xrightarrow[]{\mathrm{d}} N(0,\dfrac{1}{n})
    \end{align*}

\end{proof}

\item de Moivre-Laplace Thm.\index{de Moivre-Laplace Thm.} is a special case of CLT at $ S_n\sim B (n,p) $
\begin{equation}    \mathbb{P}(k\leq S_n\leq m)\approx \Phi(\frac{m+0.5-np}{\sqrt{npq}})-\Phi(\frac{k-0.5-np}{\sqrt{npq}})
\end{equation}
\item Stirling Eqa. derived from CLT
\begin{equation}    \frac{\lambda^k}{k!}e^{-\lambda}\approx \frac{1}{\sqrt{\lambda}\sqrt{2\pi}}e^{-\frac{(k-\lambda)^2}{2\lambda}}\xrightarrow[\lambda=n]{k=n}n!\approx\sqrt{2\pi n}(\frac{n}{e})^n\sim O\left((\dfrac{n}{e})^n\right)
\end{equation}

\end{itemize}


\subsection{Inequalities}\label{SubSectionUsefulInequality}
    
\begin{itemize}
    \item Cauchy-Schwarz Inequality\index{Inequality!Cauchy-Schwarz Inequality}
    \begin{equation}
        \left\vert \mathbb{E}(XY) \right\vert\leq \sqrt{\mathbb{E}(X^2)\mathbb{E}(Y^2)}
    \end{equation}

    \item Bonferroni Inequality\index{Inequality!Bonferroni Inequality}
\begin{equation}    \mathbb{P}(\bigcup_{i=1}^n A_i)\geq \sum_{1\leq i\leq n}  \mathbb{P}(A_i)+\sum_{1\leq i <j\leq n}  \mathbb{P}(A_i\cap A_j)
\end{equation}
    \item Markov Inequality\index{Inequality!Markov Inequality}
\begin{equation}     \mathbb{P}(|X|\geq \epsilon)\leq\frac{\mathbb{E}(|X|^\alpha)}{\epsilon^\alpha}
\end{equation}

    with $ \alpha =1 $, and $ \varepsilon  $ selected as a multiple of $ \mathbb{E}\left[ |X| \right]  $:
    \begin{align}
        \mathbb{P}\left( |X|\geq m\mathbb{E}\left[ |X| \right]  \right) \leq \dfrac{1}{m} 
    \end{align}
    
    
    \item Chebyshev Inequality\index{Inequality!Chebyshev Inequality}
\begin{equation}     \mathbb{P}(|X-\mathbb{E}(X)|\geq\epsilon)\leq\frac{var(X)}{\epsilon^2}
\end{equation}

    Chebyshev inequality is used to proof WLLN \autoref{EqaWLLN}
    \item Jensen Inequality\index{Jensen Inequality}: For convex function $h(x)$:\footnote{Or equivalently for concave function $ \tilde{h}(x) $:
    \begin{align}
         \mathbb{E}[\tilde{h}(X)]\leq \tilde{h}(\mathbb{E}(X))
    \end{align}
    }
\begin{equation}    \mathbb{E}[h(X)]\geq h(\mathbb{E}(X))
\end{equation}
    
    Example of using Jensen Eqa. to proof some other inequalities:
    \begin{itemize}[topsep=2pt,itemsep=0pt]
        \item Non-negativity of Kullback-Leibler Divergence: For two distributions $ f(\, \cdot \, ) $ and $ g(\, \cdot \, ) $, the K-L Divergence is defined as 
        \begin{align}
            \mathrm{KL}(f\Vert g):=-\int f(x)\log \dfrac{g(x)}{f(x)}  \,\mathrm{d}x
        \end{align}
        
        Take $ h(\xi ):=\log \xi  $ a concave function for $ \xi \in(0,\infty) $ and $ Z:=\dfrac{g(X)}{f(X)} $ with $ X\sim f(x) $, then
        \begin{align}
            \mathbb{E}\left( h(Z) \right) =&\int _A \left(\log z\right) f_Z(z) \,\mathrm{d}z=\int _A \left(\log \dfrac{g(x)}{f(x)}\right) f(x) \,\mathrm{d}x\\
            \leq &h(\mathbb{E}\left( Z \right) )=\log \int _A zf_Z(z) \,\mathrm{d}z=\log \int _A \dfrac{g(x)}{f(x)}f(x) \,\mathrm{d}x=0\\
            \Rightarrow& -\int _A \log f(x)\dfrac{g(x)}{f(x)} \,\mathrm{d}x\geq 0
        \end{align}
        % \item Non-negativity of \hyperlink{NormDefinition}{$ \ell_p $ Norm}:
        % \begin{align}
        %     \left\Vert x \right\Vert _p= \left(\sum_{i=1}^n |x_i|^p \right)^{1/p},\quad p\geq 1
        % \end{align}
        
        % Take $ h(\xi ):=|\xi |^p  $ a convex function for $ p\geq 1 $ and r.v. $ Z_n $ is defined a discrete one with distribution
        % \begin{align}
        %     Z_n=x_1,x_2\,\ldots, x_n ,\,\mathrm{w.p.} \dfrac{1}{n}
        % \end{align}
        
        % i.e. $ Z_n $ has equal probability to be assigned value in $ \{x_1,x_2\,\ldots, x_n\} $. Then
        % \begin{align}
        %     \mathbb{E}\left( h(Z) \right) =&\sum_{i=1}^n\dfrac{1}{n}|x_i|^p\\
        %     \geq&h(\mathbb{E}\left( Z \right) )=\left|\sum_{i=1}^n\dfrac{1}{n}x_i\right|^p\\
        %     \Rightarrow & \left|\sum_{i=1}^nx_i\right|^p\leq n^{p-1}\sum_{i=1}^n\dfrac{1}{n}|x_i|^p,\quad p\geq 1
        % \end{align}
    \end{itemize}
    \item Cantelli Inequality\index{Cantelli Inequality}
    \begin{align}
        \mathbb{P}\left( X-\mathbb{E}\left[ X \right] \geq \lambda  \right) \leq \dfrac{var(X)}{var(X)+\lambda ^2} 
    \end{align}
    with $ \lambda =\sqrt{var(X)}:=\sigma  $, we have
    \begin{align}
        \begin{cases}
            \mathbb{P}\left( X\geq \mathbb{E}\left[ X \right] + \sigma \right) \leq \dfrac{1}{2}\\
            \mathbb{P}\left( X\leq \mathbb{E}\left[ X \right] - \sigma \right) \leq \dfrac{1}{2}
        \end{cases}
    \end{align}
    i.e. difference between mean and median is upperly bounded by standard deviation
    \begin{align}
        |\mathbb{E}\left[ X \right] - \mathrm{med}(X ) |\leq \sigma  
    \end{align}
    \item Hoeffding Inequality\index{Hoeffding Inequality}: with independent r.v. sequence $ X_i\in [a_i,b_i] $, and $ S_n:=\sum_{i=1}^nX_i $
    \begin{align}
        \mathbb{P}\left( \left\vert S_n-\mathbb{E}\left[ S_n \right]  \right\vert \geq \varepsilon   \right)  \leq 2\exp\left[ -\dfrac{2\varepsilon ^2}{\sum_{i=1}^n (b_i-a_i)^2} \right]
    \end{align}

    Or in equivalent form $ \varepsilon =nt $
    \begin{align*}
        \mathbb{P}\left( \dfrac{1}{n}\left\vert \sum_{i=1}^n\left(X_i-\mathbb{E}\left[ X_i \right] \right) \right\vert \geq t   \right)  \leq 2\exp\left[ -\dfrac{2n^2t ^2}{\sum_{i=1}^n (b_i-a_i)^2} \right]
    \end{align*}
    
    For special case of $ [a_i,b_i]=[a,b] $, $ \forall i $, $ |[a,b]|:=L $,
    \begin{align*}
        \mathbb{P}\left( \dfrac{1}{n}\left\vert \sum_{i=1}^n\left(X_i-\mathbb{E}\left[ X_i \right] \right)  \right\vert \geq t   \right)  \leq 2\exp\left[ -\dfrac{2nt^2}{L^2} \right]
    \end{align*}
    
    

    The proof needs the Hoeffding Lemma: for $ \mathbb{E}\left[ Z \right]=0  $ and $ Z\in[a,b] $
    \begin{align}
        \mathbb{E}\left[ e^{tZ} \right]\leq \exp\left[ \dfrac{t^2(b-a)^2}{8} \right],\quad \forall t  
    \end{align}
    
    \item McDiarmid Inequality\index{McDiarmid Inequality}: with independent r.v. sequence $ X_i $, and a function $ f(\, \cdot \, ) $ with bounded difference $ c_i $:
    \begin{align*}
        \left| f\left(X_1,X_2,\ldots,X_{n+1}\right)- f\left(X_1,X_2,\ldots,X_{n}\right) \right| \leq c_i
    \end{align*}

    we have McDiarmid inequality
    \begin{align*}
        \mathbb{P}\left( \left| f(X_1,\ldots,X_n)-\mathbb{E}\left[ f(X_1,\ldots,X_n) \right]  \right| \geq nt\right) \leq 2\exp\left[ -\dfrac{2n^2t^2}{\sum_{i=1}^nc_i^2} \right] 
    \end{align*}
    
\end{itemize}


\subsection{Multivariate Normal Distribution}\label{SubsectionDerivationMultivariateNormal}
    General Case and more discussion see \autoref{SubSubSectionMultivariateNormalDistribution}.

    Distribution of Normal $ X\sim N(\mu ,\sigma ^2) $:\index{Distribution!Normal Distribution}
    \[
        f(x)=\dfrac{1}{\sqrt{2\pi}\sigma }e^{-\frac{(x-\mu )^2}{2\sigma ^2}} 
    \]
    
    

    For $X_1,X_2,\cdots,X_n$ independent and $X_k\sim N(\mu_k,\sigma^2_k),\, k=1,\cdots,n$, $T={\displaystyle\sum_{k=1}^n c_kX_k}, (c_k$ const), then
    \begin{equation}
        T\sim N(\sum_{k=1}^nc_k\mu_k,\sum_{k=1}^n c_k^2\sigma^2_k)    
    \end{equation}

    Deduction in some special cases:
    \begin{itemize}
        \item Given $\mu_1=\mu_2=\cdots=\mu_n=\mu,\, \sigma^2_1=\sigma^2_2=\cdots=\sigma^2_n=\sigma^2$, i.e. $X_k$ i.i.d., then
        \begin{equation}\label{EqaDistributionOfSumOfiidNormal}
            T\sim   N(\mu\sum_{k=1}^n c_k,\sigma^2\sum_{k=1}^n c_k^2) 
        \end{equation}
        \item Further take $c_1=c_2=\cdots=c_n=\dfrac{1}{n}$, i.e. $T={\displaystyle \sum_{k=1}^n X_k /n}=\bar{X}$, then
        \begin{equation}    
            T=\bar{X}\sim N(\mu,\frac{\sigma^2}{n})    
        \end{equation}
    \end{itemize}





\subsubsection{Linear Transform}
    First consider $\epsilon_1,\epsilon_2,\cdots,\epsilon_m$ i.i.d. $\sim N(0,1)$, $n\times 1$ const column vector $\vec{\mu}$, $n\times m$ const matrix $\bm{B}=\{b_{ij}\}$, def.$X_i={\displaystyle\sum_{j=1}^m b_{ij}\epsilon_j}$, i.e.
    \begin{equation}
        \vec{X}=
        \begin{pmatrix}
            X_1\\X_2\\ \vdots\\X_n
        \end{pmatrix}
        =
        \begin{pmatrix}
            b_{11}&b_{12}&\ldots&b_{1m}\\
            b_{21}&b_{22}&\ldots&b_{2m}\\
            \vdots&\vdots&\ddots&\vdots\\
            b_{n1}&b_{n2}&\ldots&b_{nm}
        \end{pmatrix}
        \begin{pmatrix}
            \epsilon_1\\
            \epsilon_2\\
            \vdots\\
            \epsilon_m
        \end{pmatrix}
        +\begin{pmatrix}
            \mu _1\\\mu _2\\ \vdots\\\mu _n
        \end{pmatrix}=\bm{B}\vec{\varepsilon }+\vec{\mu }
    \end{equation}

    
    We have: $\vec{X}\sim N(\vec{\mu},\Sigma)$, where $\Sigma$, as defined in \autoref{covariancematrix} is
    \begin{equation}
        \Sigma=\mathbb{E}[(\vec{X}-\vec{\mu})(\vec{X}-\vec{\mu})^T]=\bm{BB}^T=
        \begin{pmatrix}
        var(X_1) & cov(X_1,X_2) & \ldots & cov(X_1,X_n)\\
        cov(X_2,X_1) & var(X_2) & \ldots & cov(X_2,X_n)\\
        \vdots & \vdots & \ddots & \vdots\\
        cov(X_n,X_1) & cov(X_n,X_2) & \ldots & var(X_n)\\
        \end{pmatrix}  
        =\{\sigma_{ij}\}  
    \end{equation}

    Furthur Consider $\vec{Y}=(Y_1,\cdots,Y_n)^T$, $n\times n$ const square matrix $\bf{A}=\{a_{ij}\}$ and def. $\vec{Y}=\bf{A}\vec{X}$ i.e.
    \begin{equation}
        \begin{pmatrix}
            Y_1\\
            Y_2\\
            \vdots\\
            Y_n
        \end{pmatrix}
        =
        \begin{pmatrix}
            a_{11}&a_{12}&\ldots&a_{1n}\\
            a_{21}&a_{22}&\ldots&a_{2n}\\
            \vdots&\vdots&\ddots&\vdots\\
            a_{n1}&a_{n2}&\ldots&a_{nn}
        \end{pmatrix}
        \begin{pmatrix}
            X_1\\
            X_2\\
            \vdots\\
            X_n
        \end{pmatrix}
    \end{equation}

    Then $\vec{Y}\sim N(\bm{A}\vec{\mu},\bm{A}\Sigma\bm{A}^T)$
%    Then $Y_1,\cdots,Y_n\sim N$ 

    Special case: $X_1,\cdots,X_n$ i.i.d. $\sim N(\mu,\sigma^2)$, $\vec{X}=(X_1,\cdots,X_n)^T$, 
    \begin{align}
        \mathbb{E}(Y_i)=&\mu\sum_{k=1}^n a_{ik}\\
        var(Y_i)=&\sigma^2\sum_{k=1}^n a_{ik}^2\\
        cov(Y_i,Y_j)=&\sigma^2\sum_{k=1}^n a_{ik} a_{jk}
    \end{align}

    Specially when 
    $\bm{A}=\{a_{ij}\}$ orthonormal, we have $Y_1,\cdots,Y_n$ independent
    \begin{equation}
        Y_i\sim N(\mu\sum_{k=1}^n a_{ik},\sigma^2)    
    \end{equation}

    \begin{point}
        Definition of Jointly Gaussian/Normal\index{Jointly Gaussian Variable}
    \end{point}

    A random vector $ \vec{X} $ is called jointly Gaussian if and only if any (finite) linear combination of $ \vec{X} $ is still Gaussian (Normal)
    \begin{align}
        \sum_{k=1}^m\alpha _{k}X_{i_k}\sim N(\, \cdot \, , \, \cdot \, ),\,\forall \{\alpha _k\}_{k=1}^m,\,\forall \{i_k\}_{k=1}^m,\,\forall m\leq n
    \end{align}

    Counter Example: $ [X,Y] $ in which $ X\sim N(0,1) $, $ Y=-X $ is not jointly Gaussian.
    
    
    

    \subsubsection{Distributions of Function of Normal Variable: $\chi^2,$ $t\,\& \,F$}\label{chi2_t_F_properties}
        Consider $X_1,X_2,\ldots,X_n$ i.i.d. $\sim N(0,1)$; $Y,Y_1,Y_2,\ldots,Y_m$ i.i.d. $\sim N(0,1)$
        \begin{itemize}
            \item $\chi^2$ Distribution: Def. $\chi^2$ distribution with degree of freedom $n$:\index{Distribution!$ \chi^2 $ Distribution}
            \begin{equation}        
                \xi =\sum_{i=1}^n X_i^2\sim \chi^2_n
            \end{equation}

            PDF of $\chi^2_n$:
            \begin{equation}        
                g_n(x)=\dfrac{1}{2^{n/2}\Gamma(n/2)}x^{\frac{n}{2}-1}e^{-x/2}\mathbb{I}_{x>0}  
            \end{equation}

            Properties
            \begin{itemize}
                \item $\mathbb{E}$ and $var$ of $\xi\sim\chi^2_n$
                \begin{equation}            \mathbb{E}(\xi)=n\qquad var(\xi)=2n\end{equation}
                \item For independent $\xi_i\sim\chi^2_{n_i},\, i=1,2,\ldots,k$:\begin{equation}            
                    \xi_0=\sum_{i=1}^k\xi_i\sim\chi^2_{n_1+\ldots+n_k}\end{equation}
                \item Denoted as $\Gamma(\alpha,\lambda)$: \begin{equation}            \xi=\sum_{i=1}^nX_i^2\sim\Gamma(\frac{n}{2},\frac{1}{2})=\chi^2_n\end{equation}
            \end{itemize}
            \item $t$ Distribution: Def. $t$ distribution with degree of freedom $n$:\index{Distribution!$ t $ Distribution}
            \begin{equation}        
                T=\frac{Y}{\sqrt{\dfrac{\sum_{i=1}^nX_i^2}{n}}}=\frac{Y}{\sqrt{\xi \big/ n}}\sim t_n
            \end{equation}

            (Usually take $\nu$ instead of $n$ as degree of freedom for $ t $ distribution)

            PDF of $t_\nu$:
            \begin{equation}        
                t_\nu(x)=\dfrac{\Gamma(\frac{\nu+1}{2})}{\Gamma(\frac{\nu}{2})\sqrt{\nu\pi}}\left(1+\frac{x^2}{\nu}\right)^{-\frac{\nu+1}{2}}
            \end{equation}

            Denote: Upper $\alpha$-fractile\index{Fractile!Upper $ \alpha $-fractile} of $t_\nu$, satisfies $\mathbb{P}(T\geq c)=\alpha$:
            \begin{equation}        
                t_{\nu,\alpha}=\mathop{\arg}\limits_{c}\mathbb{P}(T\geq c)=\alpha,\quad T\sim t_\nu 
            \end{equation}
            
            (Similar for $ N $, $\chi^2_n$ and $F_{m,n}$ etc.)
            \item $F$ Distribution: Def. $F$ distribution with degree of freedom $m$ and $n$:\index{Distribution!$ F $ Distribution}
            \begin{equation}        
                F=\frac{\sum_{i=1}^mY_i^2\big/ m}{\sum_{i=1}^nX_i^2\big/ n}\sim F_{m,n}
            \end{equation}

            PDF of $F_{m,n}$:
            \begin{equation}        
                f_{m,n}(x)=\frac{\Gamma(\frac{m+n}{2})m^\frac{m}{2}n^{\frac{n}{2}}}{\Gamma(\frac{m}{2})\Gamma(\frac{n}{2})}x^{\frac{m}{2}-1}(mx+n)^{-\frac{m+n}{2}} \mathbb{I}_{x>0}
            \end{equation}

            Properties
            \begin{itemize}
                \item If $Z\sim F_{m,n}$, then $\dfrac{1}{Z}\sim F_{n,m}$.
                \item If $T\sim t_n$, then $T^2\sim F_{1,n}$
                \item $F_{m,n,1-\alpha}=\dfrac{1}{F_{n,m,\alpha}}$
            \end{itemize}
        \end{itemize}

        \begin{point}
            Some useful Lemma (uesd in statistic inference, see \autoref{SubSectionConfidenceIntervalForDistributions}):
        \end{point}
        
            
        \begin{itemize}
            \item For $X_1,X_2,\ldots,X_n$ independent with $X_i\sim N(\mu_i,\sigma^2_i)$, then
            \begin{equation}        
                \sum_{i=1}^n\left(\frac{X_i-\mu_i}{\sigma_i}\right)^2\sim \chi^2_n
            \end{equation}  
            \item For $X_1,X_2,\ldots,X_n$ i.i.d.$\sim N(\mu,\sigma^2)$, then
            \begin{equation}        
                T=\frac{\sqrt{n}(\bar{X}-\mu)}{S}\sim t_{n-1}   
            \end{equation}
            
            For $X_1,X_2,\ldots,X_m$ i.i.d.$\sim N(\mu_1,\sigma^2)$, $Y_1,Y_2,\ldots,Y_n$ i.i.d.$\sim N(\mu_2,\sigma^2)$, \\ denote sample pooled variance $S_{\omega}^2=\dfrac{(m-1)S^2_1+(n-1)S^2_2}{m+n-2}$, then
            \begin{equation}        
                T=\frac{(\bar{X}-\bar{Y})-(\mu_1-\mu_2)}{S_{\omega}}\cdot \sqrt{\frac{mn}{m+n}}\sim t_{m+n-2}
            \end{equation}
            \item For $X_1,X_2,\ldots,X_m$ i.i.d.$\sim N(\mu,\sigma^2)$, $Y_1,Y_2,\ldots,Y_n$ i.i.d.$\sim N(\mu_2,\sigma^2)$, then
            \begin{equation}        
                T=\frac{S_1^2}{S_2^2}\frac{\sigma^2_2}{\sigma^2_1}\sim F_{m-1,n-1}   
            \end{equation}
            \item For $X_1,X_2,\ldots,X_n$ i.i.d. $\sim \epsilon(\lambda)$, then
            \begin{equation}        
                2\lambda n\bar{X}=2\lambda\sum_{i=1}^nX_i \sim\chi^2_{2n} 
            \end{equation}

            Remark: for $X_i\sim\epsilon(\lambda)=\Gamma(1,\lambda)\Rightarrow 2\lambda\sum_{i=1}^nX_i\sim\Gamma(n,1/2)=\chi^2_{2n}$. 
        \end{itemize}


\newpage

\section{统计推断部分}
\begin{center}
    Instructor: Jiangdian Wang
\end{center}

    \textbf{Statistical Inference}: use sample to estimate population.
    
    Two main tasks of Statistical Inference:
    \begin{itemize}[topsep= -5 pt,itemsep= -5 pt,parsep= 0 pt]
        \item Parameter Estimation
        \begin{itemize}
            \item Point Estimation: \hyperref[SectionPointEstimation]{\ref{SectionPointEstimation}}
            \item Interval Estimation: \hyperref[SectionIntervalEstimation]{\ref{SectionIntervalEstimation}}
        \end{itemize}
        \item Hypothesis Testing: \hyperref[SectionHypothesisTesting]{\ref{SectionHypothesisTesting}}
    \end{itemize}

\subsection{Statistical Model and Statistics}\label{SectionStatisticalModelandStatistics}
    Random sample comes from population $X$. In parametric model case, we have population distribution family:
    \begin{equation}\mathscr{F}=\{f(x;\vec{\theta)}|\vec{\theta}\in\Theta\}\end{equation}

    where \text{parameter} $\vec{\theta}$ reflect some quantities of population (e.g. mean, variance, etc.), each $\vec{\theta}$ corresponds to a distribution of population $X$.
    
    Sample space\index{Sample Space}: Def. as $\mathscr{X}=\{\{x_1,x_2,\ldots,x_n\},\forall x_i\}$, then $\{X_i\}\in\mathscr{X}$ is random sample from population $X\sim f(x;\vec{\theta})$.

    
\subsubsection{Statistics}\label{SubSectionStatistics}
    Statistic(s): function of random sample $\vec{T}(X_1,X_2,\ldots,X_n)$, \textbf{but not a function of parameter}.
    
    Some useful statistics, e.g.
    \begin{itemize}
        \item Sample mean (Consider $X_i$ i.i.d.)
        \begin{equation}
            \bar{X}=\frac{1}{n}\sum_{i=1}^n X_i
        \end{equation}
        \item Sample variance
        \begin{equation}
            S^2=\frac{1}{n-1}\sum_{i=1}^n(X_i-\bar{X})^2  
        \end{equation}
        \item Sample moments
        \begin{itemize}
            \item Origin moment
            \begin{equation}
                a_{n,k}=\frac{1}{n}\sum_{i=1}^k X_i^k\qquad k=1,2,3,\ldots    
            \end{equation}
            \item Center moment
            \begin{equation}
                m_{n,k}=\frac{1}{n}\sum_{i=1}^n (X_i-\bar{X})^k\qquad k=2,3,4,\ldots    
            \end{equation}
        \end{itemize}
        \item Order statistics
        \begin{equation}
            (X_{(1)},X_{(2)},\ldots,X_{(n)}),\,\text{for }X_{(1)}\leq X_{(2)} \leq \ldots\leq X_{(n)}    
        \end{equation}
        \item Sample $p$-fractile
        \begin{equation}
            m_p=X_{(m)},\quad m=[(n+1)p]   
        \end{equation}
        \item Sample coefficient of variation
        \begin{equation}
            \hat{\nu}=\frac{S}{\bar{X}}    
        \end{equation}
        \item Skewness and Kurtosis
        \begin{equation}
            \hat{g}_1=\frac{m_{n,3}}{m_{n,2}^{3/2}}\qquad \hat{g}_2=\frac{m_{n,4}}{m_{n,2}^2}    -3
        \end{equation}
    \end{itemize}

    \begin{point}
        Properties
    \end{point}
    
        

    Statistic $T$ is a function of random sample $\{X_i\}$, thus has distribution (say $g_T(t)$) called \textbf{Sampling Distribution}.

        For $X_i$ i.i.d. from $X\sim f(x)$ with population mean $\mu$ and variance $\sigma^2$
    \begin{itemize}
        \item Calculation of sample variance $S^2$
        \begin{equation}(n-1)S^2=\sum_{i=1}^n x_i^2-n\bar{x}^2\end{equation}
        \item $E$ and $var$ of $\bar{X}$ and $S^2$
        \begin{equation}E(\bar{X})=\mu\qquad var(\bar{X})=\frac{\sigma^2}{n}\qquad E(S^2)=\sigma^2\end{equation}
    \end{itemize}

    Further if $X_i$ i.i.d. from $X\sim N(\mu,\sigma^2)$ where $\mu$ and $\sigma^2$ unknown.
    \begin{itemize}
        \item Independence of $\bar{X}$ and $S^2$ 
            \begin{equation}\bar{X}\text{ and }S^2 \text{ are independent}\end{equation}
        \begin{itemize}[topsep=6pt,itemsep=4pt]
        \item Distribution of $\bar{X}={\displaystyle\frac{1}{n}\sum_{i=1}^n X_i}$
        \begin{equation}\bar{X}\sim N(\mu,\frac{\sigma^2}{n})\end{equation}
        \item Distribution of $S^2={\displaystyle\frac{1}{n-1}\sum_{i=1}^n(X_i-\bar{X})^2}$
        \begin{equation}\frac{(n-1)S^2}{\sigma^2}\sim\chi^2_{n-1}\end{equation}
        \end{itemize}        
    \end{itemize}

    \subsubsection{Exponential Family}\label{SubSectionExponentialFamily}
        Def. $\mathscr{F}=\{f(x;\vec{\theta}|\vec{\theta}\in\Theta)\}$ is \textbf{Exponential Family} if $f(x;\vec{\theta})$ has the form as\index{EF (Exponential Family)}
\begin{equation}
    f(x;\vec{\theta})=C(\vec{\theta})h(x)\exp \left[  \sum_{i=1}^k Q_i(\vec{\theta})T_i(x) \right]\quad\vec{\theta}\in\Theta
\end{equation}    

    Canonical Form: Take $Q_i(\vec{\theta})=\varphi_i$, then $\vec{\varphi}=(\varphi_1,\varphi_2,\ldots,\varphi_k)=$$(Q_1(\vec{\theta}),Q_2(\vec{\theta}),\ldots,Q_k(\vec{\theta}))$ is a transform from $\Theta$ to $\Theta^*$, s.t. $\mathscr{F}$ has canonical form, i.e.
    \begin{equation}
        f(x;\vec{\varphi})=C^*(\vec{\varphi})h(x)   \exp\left[  \sum_{i=1}^k \varphi_i T_i(x) \right] \quad \vec{\varphi}\in\Theta^*
    \end{equation}

    $\Theta^*$ is canonical parameter space.

\begin{point}
    Why we need exponential family? Have some nice properties.
\end{point}




\subsubsection{Sufficient and Complete Statistics}\label{SubSectionSufficient_CompleteStatistics}
    Note: For simplification, the following parts denote $\vec{\theta},\vec{T},\ldots$  as $\theta,T,\ldots$ etc.
    \begin{itemize}
        \item[$\blacktriangleright$] \index{Statistics!Sufficient Statistic}A \textbf{Sufficient Statistic} $T(\vec{X})$ for $\theta$ contains all the information of sample when infer $\theta$, i.e.
        \begin{equation}
            f(\vec{X};T(\vec{X}))=f(\vec{X};T(\vec{X}),\theta)
        \end{equation}

        Properties
        \begin{itemize}
            \item \textbf{Factorization Thm.}\index{Factorization Thm.} $T(\vec{X})$ is sufficient \textbf{if and only if} $f_{\vec{X}}(\vec{x};\theta)=f(\vec{x};\theta)$ can be written as 
            \begin{equation}
                f(\vec{x};\theta)=g[t(\vec{x});\theta]h(\vec{x})
            \end{equation}            
            \item If $T(\vec{X})$ sufficient, then $T'(\vec{X})=g[T(\vec{X})]$ also.(require $g$ single-valued and invertible)
            \item If $T(\vec{X})$ sufficient, then $(T,T_1)$ also.
            \item Minimal sufficient statistic $T_\theta(\vec{X})$ satisfies 
            \begin{equation}
                \forall\,\text{sufficient statistic }S,\,\exists\, q_S(\cdot),\, \text{s.t.} T_\theta=q_S(S)
            \end{equation}

            A minimal sufficient statistic not always exists.

            Sufficient \& Complete $\Rightarrow $ Minimal sufficient.
            \item Usually dimension of $\vec{T}_\theta$ and $\theta$ equals.
        \end{itemize}
        
        Sufficient statistic is \textbf{not} unique.



        \item[$\blacktriangleright$] A \textbf{Complete Statistic}\index{Statistics!Complete Statistic} $T(\vec{X})$ for $\theta$ satisfies
        \begin{equation}
            \forall\theta\in\Theta\, ;\,\forall\varphi\text{ satisfies }E[\varphi(T(\vec{X}))]=0\text{, we have }P[\varphi(T)=0;\theta]=1
        \end{equation}

        Explanation: $T\sim g_T(t)$. Rewrite as
        \begin{equation}
            \int\varphi (t) g_T(t)\,\mathrm{d} t=0  \,\,\forall\, \theta\Rightarrow\varphi(T)=0 \text{  a.s. }
        \end{equation}

        i.e. \underline{$\mathrm{span}\{g_T(t);\forall\theta\}$ is a complete space}. Or to say that $\nexists$ none-zero $\varphi(t)$ so that $E(\varphi(T))=0$ (unbiased estimation)

        \begin{equation}
            \varphi(T)\neq 0 \,\,\forall \theta\Rightarrow E[\varphi(T(\vec{X}))]\neq 0  
        \end{equation}

        So make sure the uniqueness of unbiased estimation of $\hat{\theta}$ using $T$.

        Properties
        \begin{itemize}
            \item If $T(\vec{X})$ complete, then $T^\prime(\vec{X})=g[T(\vec{X})]$ also.(require $g$ measurable)

            \item A complete statistic not always exists.
        \end{itemize}
        \item[$\blacktriangleright$]\index{Statistics!Ancillary Statistic}  An \textbf{Ancillary Statistic} $S(\vec{X})$ is a statistic whose distribution does not depend on $\theta$
        
        \textbf{Basu Thm.}\index{Basu Thm.}: $\vec{X}=(X_1,X_2,\ldots,X_n)$ is sample from $\mathscr{F}=\{f(x;\theta),\theta\in\Theta\}$. $T(\vec{X})$ is a complete and minimal sufficient statistic, $S(\vec{X})$ is ancillary statistic, then $S(\vec{X})\parallel T(\vec{X})$.

        \item[$\blacktriangleright$] Exponential family: For $\vec{X}=(X_1,X_2,\ldots,X_n)$ from exponential family with canonical form, i.e.
    \begin{equation}
        f(\vec{x};\theta)=C(\theta)h(\vec{x})\exp\left[\sum_{i=1}^k \theta_i T_i(\vec{x})\right] ,\quad \theta\in\Theta
    \end{equation}

    Then if $\Theta\in\mathbb{R}^k$ interior point exists, then $T(\vec{X})=(T_1(\vec{X}),T_2(\vec{X}),\ldots,T_k(\vec{X}))$ is sufficient \& complete statistic.


\end{itemize} 




\subsection{Point Estimation}\label{SectionPointEstimation}
    For parametric distribution family $\mathscr{F}=\{f(x,\theta),\theta\in\Theta\}$, random sample $\vec{X}=(X_1,X_2,\ldots,X_n)$ from $\mathscr{F}$. $g(\theta)$ is a function defined on $\Theta$. 

    Mission: use sample $\{X_i\}$ to estimate $g(\theta)$, called \textbf{Parameter Estimation}.

    \begin{equation}
        \text{Parameter Estimation}\begin{cases}
            \text{Point Estimation}& \surd\\
            \text{Interval Estimation}&
        \end{cases}    
    \end{equation}

    Point estimation: when estimating $\theta$ or $g(\theta)$, denote the estimator (defined on sample space $\mathscr{X}$) as
    \begin{equation}
        \hat{\theta}(\vec{X})\qquad \hat{g}(\vec{X})    
    \end{equation}

    Estimator is a statistic, with sampling distribution.
\subsubsection{Optimal Criterion}\label{SubSectionOptimalCriterion}
        Some nice properties of estimators (that we expect)
    \begin{itemize}
        \item Unbiasedness
        \begin{equation}
            E(\hat{\theta})=\theta   \quad \text{or}\quad E(\hat{g}(\vec{X})) =g(\theta)
        \end{equation}

        Otherwise, say $\hat{\theta}$ or $\hat{g}$ biased. Def. \textbf{Bias}: $E(\hat{\theta})-\theta$

        Asymptotically unbiasedness
        \begin{equation}
            \lim_{n\to\infty}  E(\hat{g}_n(\vec{X})) =g(\theta)  
        \end{equation}
        \item Efficiency: say $\hat{g}_1(\vec{X})$ is more efficient than $\hat{g}_2(\vec{X})$, if
        \begin{equation}
            var(\hat{g}_1)\leq var(\hat{g}_2)  \quad\forall\theta\in\Theta  
        \end{equation}
        \item Mean Squared Error (MSE)\index{MSE (Mean Squared Error)}
        \begin{equation}
            \text{MSE}=E[(\hat{\theta}-\theta)^2]=var(\hat{\theta})+[Bias(\hat{\theta})]^2
        \end{equation}

        For unbiased estimator, i.e. $Bias(\hat{\theta})=0$, we have
        \begin{equation}
            \text{MSE}=E[(\hat{\theta}-\theta)^2]=var(\hat{\theta})
        \end{equation}
        \item (Weak) Consistency
        \begin{equation}
            \lim_{n\to\infty}P(|\hat{g}_n(\vec{X})-g(\theta)|\geq \varepsilon)=0\quad\forall\varepsilon>0    
        \end{equation}
        \item Asymptotic Normality
    \end{itemize}


\subsubsection{Method of Moments}\label{SubSectionMoM}
    Review: Population moments \& Sample moments\index{MoM (Method of Moments)}
    \begin{align*}
        \alpha_k&=E(X^k)&\mu_k&=E[(X-E(X))^k]\\
        a_{n,k}&=\frac{1}{n}\sum_{i=1}^nX_i^k&m_{n,k}&=\frac{1}{n}\sum_{i=1}^n(X_i-\bar{X})^k
    \end{align*}

    Property: $a_{n,k}$ is the unbiased estimator of $\alpha_k$.(while $m_{n,k}$ unually biased for $\mu_k$)

    For sample $\vec{X}=(X_1,X_2,\ldots,X_n)$ from $\mathscr{F}=\{f(x;\theta,\theta\in\Theta)\}$, unknown parameter (or its function) $g(\theta)$ can be written as
    \begin{equation}
        g(\theta)=G(\alpha_1,\alpha_2,\ldots,\alpha_k;\mu_2,\mu_3,\ldots,\mu_l)    
    \end{equation}

    Then its \textbf{Moment Estimate} $\hat{g}(\vec{X})$ is
\begin{equation}
    \hat{g}(\vec{X})=G(a_{n,1},a_{n,2},\ldots,a_{n,k};m_{n,2},m_{n,3},\ldots,m_{n,l}) 
\end{equation}

    Example: coefficient of variance \& skewness 
    \begin{equation}\hat{\nu}=\dfrac{S}{\bar{X}}\quad\hat{\beta}_1=\dfrac{m_{n,3}}{m_{n,2^{3/2}}}=\sqrt{n}{\displaystyle\frac{\displaystyle{\sum_{i=1}^n(X_i-\bar{X})^3}}{\displaystyle{[\sum_{i=1}^n(X_i-\bar{X})^2]^{\frac{3}{2}}}  }}\end{equation}

    \begin{point}
        Note:
    \end{point}
    
        
    \begin{itemize}
        \item $G$ may not have explicit expression.
        \item Moment estimate may not be unique.
        \item If $G={\displaystyle\sum_{i=1}^kc_i\alpha_i}$ (linear combination of $\alpha$, without $\mu$), then $\hat{g}(\vec{X})={\displaystyle\sum_{i=1}^kc_ia_{n,i}}$ unbiased.
        
        \qquad Usually $\hat{g}(\vec{X})$ is asymptotically unbiased.
        \item For small sample, not so accurate.
        \item May not contain all the information about $\theta$, i.e. may not be sufficient statistic.
        \item Do not require a statistic model.
    \end{itemize}


\subsubsection{Maximum Likelihood Estimation}\label{SubSectionMLE}
    \index{MLE (Maximum Likelihood Estimation)}For sample $\vec{X}=(X_1,X_2,\ldots,X_n)$ with distribution $f(\vec{x};\theta)$ from $\mathscr{F}=\{f(x;\theta),\theta\in\Theta\}$ , def. \textbf{Likelihood Function} $L(\theta;\vec{x})$, defined on $\Theta$ (as a function of $\theta)$
    \begin{equation}
        L(\theta;\vec{x})=f(\vec{x};\theta)\qquad \theta\in\Theta,\,\vec{x}\in\mathscr{X}    
    \end{equation}

    Also def. log-likelihood function $l(\theta;\vec{x})=\ln L(\theta;\vec{x})$.

    If estimator $\hat{\theta}=\hat{\theta}(\vec{X})$ satisfies
    \begin{equation}
        L(\hat{\theta};\vec{x})=\sup_{\theta\in\Theta}L(\theta;\vec{x}),\quad \vec{x}\in\mathscr{X}
    \end{equation}

    Or equivalently take $l(\theta;\vec{x})$ instead of $L(\theta;\vec{x})$.

    Then $\hat{\theta}=\hat{\theta}(\vec{X})$ is a \textbf{Maximum Likelihood Estimate}(MLE) of $\theta=(\theta_1,\theta_2,\ldots,\theta_k)$

    How to identify MLE?
    \begin{itemize}
        \item Differentiation: Fermat Lemma
        \begin{equation}
            \frac{\partial L}{\partial \theta_i}\bigg|_{\theta=\hat{\theta}}=0\qquad \frac{\partial^2 L}{\partial \theta_i \partial \theta_j}\bigg|_{\theta=\hat{\theta}}\text{negative definite}\qquad \forall i,j=1,2,\ldots,k
        \end{equation}
        \item Graphing method.
        \item Numerically compute maximum.
    \end{itemize}

    \begin{point}
        Some properties of MLE
    \end{point}
    
        
    \begin{itemize}
        \item (Depend on the case, not always) unbiased.
        \item Invariance of MLE\index{Invariance of MLE}: If $\hat{\theta}$ is MLE of $\theta$, invertible function $g(\theta)$, then $g(\hat{\theta})$ is MLE of $g(\theta)$.
        \item MLE and Sufficiency: $T=T(X_1,X_2,\ldots,X_n)$ is a sufficient statistic of $\theta$, if MLE of $\theta$ exists, say $\hat{\theta}$, then $\hat{\theta}$ is a function of $T$, i.e.
        \begin{equation}  
            \hat{\theta}=\hat{\theta}(\vec{X})=\hat{\theta}^*(T(\vec{X}))    
        \end{equation}
        \item Asymptotic Normality: 
        \begin{equation}
            \sqrt{n}(\hat{\theta}_n-\theta) \xrightarrow[]{d}N(0,\sigma^2_\theta),\quad \sigma^2_\theta=\frac{1}{E_\theta[\frac{\partial}{\partial\theta}\ln f(\vec{X};\theta)]^2}   
        \end{equation}

        i.e.
        \begin{equation}
            \hat{\theta}_n\xrightarrow[]{d}N(\theta,\frac{\sigma^2_\theta}{n})    
        \end{equation}
        
    \end{itemize}

    \begin{point}
        Comparison: MoM and MLE
    \end{point}
    
        
    \begin{itemize}
        \item MoM do not require statistic model; MLE need to know PDF.
        \item MoM is more robust than MLE.
    \end{itemize}


    MLE in Exponential Family:

        For sample $\vec{X}=(X_1,X_2,\ldots,X_n)$ from canonical exponential family $\mathscr{F}=\{f(x;\theta),\theta\in\Theta\}$
        \begin{equation}
            f(x;\theta)=C(\theta)h(x)\exp\left[\sum_{i=1}^k\theta_iT_i(x)\right]\quad \theta=(\theta_1,\ldots,\theta_k)\in\Theta
        \end{equation}

        Likelihood function $L(\theta,\vec{x})=\prod_{j=i}^nf(x_j;\theta)$ and log-likelihood function $l(\theta,\vec{x})$
        \begin{align*}
            L(\theta,\vec{x})&=C^n(\theta)\prod_{j=1}^nh(x_j)\exp\left[\sum_{i=1}^k\theta_i\sum_{j=1}^n T_i(x_j)\right]\\
            l(\theta,\vec{x})&=n\ln C(\theta)+\sum_{j=1}^n\ln h(x_j)+\sum_{i=1}^k\theta_i\sum_{j=1}^nT_i(x_j)
        \end{align*}

        Solution of MLE: (Require $\hat{\theta}\in\Theta$)
        \begin{equation}
            \frac{n}{C(\theta)}\frac{\partial C(\theta)}{\partial \theta_i}\bigg|_{\theta=\hat{\theta}}=-\sum_{j=1}^nT_i(x_j),\quad i=1,2,\ldots,k    
        \end{equation}


\subsubsection{Uniformly Minimum Variance Unbiased Estimator}\label{SubSectionUMVUE}
        \index{UMVUE (Uniformly Minimum Variance Unbiased Estimator)}MSE: For $\hat{g}(\vec{X})$ is estimate of $g(\theta)$ ,then MSE
        \begin{equation}
            \mathrm{MSE}(\hat{g}(\vec{X}))=E[(\hat{g}(\vec{X})-g(\theta))^2]=var(\hat{g})+[Bias(\hat{g})]^2
        \end{equation}

        Note:
    Unbiased estimator (i.e. $Bias(\hat{g})=0$) not unique; not always exist.


    


        Now only consider unbiased estimators of $g(\theta)$ exists, say $\hat{g}(\vec{X})$, then
        \begin{equation} \mathrm{MSE}(\hat{g}(\vec{X}))=var(\hat{g}(\vec{X})) \end{equation}

        If $\forall$ unbiased estimate $\hat{g}\prime(\vec{X})$, $\hat{g}$ satisfies
        \begin{equation}
            var[\hat{g}(\vec{X})]\leq var[\hat{g}\prime(\vec{X})]    
        \end{equation}

\begin{point}
        Then $\hat{g}(\vec{X})$ is \textbf{Uniformly Minimum Variance Unbiased Estimator(UMVUE)} of $g(\theta)$
\end{point}


        How to determine UMVUE? (Not an easy task)
        \begin{itemize}
            \item Zero Unbiased Estimate Method
            \item Sufficient and Complete Statistic Method
            \item Cramer-Rao Inequality
        \end{itemize}

\begin{enumerate}
\item \textbf{Zero Unbiased Estimate Method}
            
    Let $\hat{g}(\vec{X})$ be an unbiased estimate with $var(\hat{g})<\infty$. If $\forall$ $E(\hat{l}(\vec{X}))=0$ , $\hat{g}$ holds that
    \begin{equation}
        cov(\hat{g},\hat{l})=E(\hat{g}\cdot\hat{l})=0,\quad\forall\theta\in\Theta    
    \end{equation}

    Then $\hat{g}$ is a UMVUE of $g(\theta)$ (sufficient \& necessary).





\item \textbf{Sufficient and Complete Statistic Method}

    For $T(\vec{X})$ sufficient statistic, $\hat{g}(\vec{X})$ unbiased estimate of $g(\theta)$, then 
\begin{equation}
    h(T)=E(\hat{g}(\vec{X})| T)    
\end{equation}

    is an unbiased estimate of $g(\theta)$ and $var(h(T))\leq var(\hat{g})$.

    Remark:
    \begin{itemize}
        \item A method to improve estimator.
        \item A UMVUE has to be a function of sufficient statistic.
    \end{itemize}

    \textbf{Lehmann-Scheffé Thm.}\index{LS Thm. (Lehmann-Scheffé Thm.)}: For $\vec{X}=(X_1,X_2,\ldots,X_n)$ from population $X\sim\mathscr{F}=\{f(x,\theta,\vec{\theta\in\Theta})\}$. $T(\vec{X})$ sufficient and complete, and $\hat{g}(T(\vec{X}))$ be an unbiased estimator, then $\hat{g}(T(\vec{X}))$ is the unique UMVUE.

    Can be used to construct UMVUE: given $T(\vec{X})$ sufficient and complete and some unbiased estimator $\hat{g}\prime(\theta)$ then 
    \begin{equation}
        \hat{g}(T)=E(\hat{g}\prime|T)    
    \end{equation}

    is the unique UMVUE.



\item \textbf{Cramer-Rao Inequality}\index{CR Inequality (Cramer-Rao Inequality)}

    Core idea: determine a lower bound of $var(\hat{g})$.

    Consider $\theta=\theta$ (One dimension parameter); For $\{X_i\}$ i.i.d. $f(x,\theta)$: def.
    \begin{itemize}
        \item \textbf{Score function}\index{Score Function}: Reflects the steepness/slope of likelihood function $f$.
        \begin{equation}
            S(\vec{x};\theta)=\frac{\partial\ln f(\vec{x};\theta)}{\partial\theta}=\sum_{i=1}^n\frac{\partial\ln f(x_i;\theta)}{\partial\theta}
        \end{equation}
        \begin{equation}E[S(\vec{X};\theta)]=0\end{equation}
        \item \textbf{Fisher Information}\index{Fisher Information}: Variance of $S(\vec{x};\theta)$, reflects the accuracy to conduct estimation, i.e. reflects information of statistic model that sample brings.
        \begin{equation}
            I(\theta)=E\left[\left(\frac{\partial \ln f(\vec{x};\theta)}{\partial\theta}\right)^2\right]=-E\left[\frac{\partial^2\ln f(\vec{x};\theta)}{\partial \theta^2}\right]
        \end{equation}
    \end{itemize}

    Consider $\mathscr{F}$ satisfies some regularity conditions(in most cases, regularity conditions do  hold), then the lower bound of $var(\hat{g})$ satisfies \textbf{Cramer-Rao Inequality}:
    \begin{equation}
        var(\hat{g}(\vec{X}))\geq\frac{[g'(\theta)]^2}{nI(\theta)}
    \end{equation}

    Special case: $g(\theta)=\theta$ then
    \begin{equation}
        var(\hat{\theta})\geq\frac{1}{nI(\theta)}    
    \end{equation}

    note:
    \begin{itemize}
        \item C-R Inequality determine a lower bound, not the infimum(i.e. UMVUE$\nRightarrow var(\hat{g}(\vec{X}))=\dfrac{[g'(\theta)]^2}{nI(\theta)}$).
        \item Take '=': Only some cases in Exponential family.
        \item \textbf{Efficiency}: How good the estimator is.
        \begin{equation}
            e_{\hat{g}(\vec{X})}(\theta)=   \frac{[g'(\theta)]^2/(nI(\theta))}{var(\hat{g}(\vec{X}))} 
        \end{equation} 
    \end{itemize}


\item \textbf{Multi-Dimensional Cramer-Rao Inequality}

    ReDef. Fisher Information:
    \begin{equation}
        \mathbf{I}(\theta)=\{I_{ij}(\theta)\}=\{E\left[\left(\frac{\partial\ln f(\vec{x};\theta)}{\partial\theta_i}\right)\left(\frac{\partial\ln f(\vec{x};\theta)}{\partial\theta_j}\right)\right]\}  
    \end{equation}

    Then covariance matrix $\Sigma(\theta)$ satisfies \textbf{Cramer-Rao Inequality}
    \begin{equation}
        \Sigma(\theta)\geq (n\mathbf{I}(\theta))^{-1}
    \end{equation}

    Note: '$\geq$' holds for all diagonal elements, i.e.
\begin{equation}
    var(\hat{\theta}_i)\geq \frac{I^*_{ii}(\theta)}{n},\quad \forall\,i=1,2,\ldots,k  
\end{equation}


    
\end{enumerate}






\subsubsection{MoM and MLE in Linear Regression}\label{SubSectionMoM_MLE_LinearRegression}
    \textbf{Note:} More detailed knowledge see sec.\ref{SecLinearRegressionAnalysis} Linear Regression Analysis.

\begin{point}
    Linear Regression Model(1-dimension case):
\end{point}

    \begin{equation}
        y_i=\beta_0+\beta_1x_0+\epsilon_i    
    \end{equation}

    where $\beta_0,\beta_1$ are regression coefficient, and $\epsilon_i$ are unknown random \textbf{error}. 
    
    Basic Assumptions (Guass-Markov Assumption):
    \begin{align*}
        \text{Zero-Mean: }&\epsilon_i\text{ are i.i.d.}\\
        \text{Homogeneity of Variance: }&E(\epsilon_i|x_i)=0\\
        \text{Independent: }&var(\epsilon_i)=\sigma^2
    \end{align*}

    Mission: use data $\{(x_i,y_i)\}$ to estimate $\beta_0,\beta_1$(i.e. regression line), and error $\epsilon_i$.

    \begin{enumerate}
        \item OLS (Ordinary Least Squares)\index{OLS (Ordinary Least Squares)}: Take $\beta_0,\beta_1$ so that MSE min, i.e. SSE min
        \begin{equation}
            (\hat{\beta_0},\hat{\beta_1})=\arg\min\sum_{i=1}^n(y_i-\beta_0-\beta_1 x_i)^2    
        \end{equation}

        (Express in Matrix Notation (eqa.\ref{EqaMatrixRepreOfSSE}), so that it can be generalized to multidimensional case) SSE can be expressed as the \textbf{Excliean Distance }between $ \{y_i\} $ and $ \{\hat{\beta}_0+\hat{\beta}_1x_i\}  $, i.e.
        \begin{equation}
            \arg\min d(y
            ,X \hat{\beta})
        \end{equation}

         i.e. $ \hat{\beta} $ is the Projection of $y $ onto hyperplane $ X $, then
         \begin{equation}
            ( X\hat{\beta} )^T(y- X\hat{\beta})=0 \Rightarrow \hat{\beta}=(X^TX)^{-1}X^Ty 
         \end{equation}
        

        Solution for 2-D case:
        \begin{equation}
            \hat{\beta}=\begin{bmatrix}
                \hat{\beta}_0\\ \hat{\beta}_1
            \end{bmatrix}
            =
            \begin{bmatrix}
            \bar{y}- \hat{\beta}_1\bar{x}\\
            \dfrac{\sum\limits_{i=1}^n(x_i-\bar{x})(y_i-\bar{y})}{\sum\limits_{i=1}^n(x_i-\bar{x})^2}
            \end{bmatrix}
        \end{equation}

        % =\{\sum\limits_{i=1}^n(x_i-\bar{x})^2\}^{-1}\{\sum\limits_{i=1}^n(x_i-\bar{x})(y_i-\bar{y})\}

        So get regression line:$y=\hat{\beta_0}+\hat{\beta_1}x$

        Def. Residuals
        \begin{equation}e_i=\hat{\epsilon}_i=y_i-\hat{y_i}=y_i-(\hat{\beta_0}+\hat{\beta_1}x_i)\end{equation}


        Residuals can be used to estimate $\epsilon_i$: $E[(\epsilon_i)^2]=\sigma^2$
        \begin{equation}\hat{\sigma^2}=\frac{1}{n-2}\sum_{i=1}^n(y_i-\hat{\beta_0}-\hat{\beta_1}x_i)\end{equation}

    \item MoM: Consider r.v. $\epsilon\sim f(\varepsilon;x,y,\beta_0,\beta_1)$, sample $\{\epsilon_i|\epsilon_i=y_i-\beta_0-\beta_1x_i\}$, then obviously
        \begin{equation}\bar{\epsilon}=\bar{y}-\beta_0-\beta_1\bar{x}\end{equation}

        Take moment estimate of $\epsilon$, we have 
        \begin{equation}E(\epsilon_i)=0\qquad E(\epsilon_i x_i)=0\text{ (note that}E(\epsilon|x)=0)\end{equation}
        \begin{equation}\text{i.e.}\begin{cases}
            
            \dfrac{1}{n}\sum_{i=1}^n(y_i-\beta_0-\beta_1x_i)=0\\
            \dfrac{1}{n}\sum_{i=1}^nx_i(y_i-\beta_0-\beta_1x_i)=0
        \end{cases}\end{equation}

        Solution:
        \begin{equation}\begin{cases}
            \hat{\beta_0}&=\bar{y}-\beta_1\bar{x}\\
            \hat{\beta_1}&=\dfrac{\sum_{i=1}^n(x_i-\bar{x})(y_i-\bar{y})}{\sum_{i=1}^n(x_i-\bar{x})^2}
        \end{cases}\end{equation}

        (Same as OLS)

        Moment estimate of $\sigma^2$
        \begin{equation}\hat{\sigma}^2_n=\frac{1}{n}\sum_{i=1}^n(y_i-\hat{\beta}_0-\hat{\beta}_1x_i)\end{equation}

    \item MLE: Assume $\epsilon_i\sim N(0,\sigma^2)$, then $y_i|x_i\sim N(\beta_0+\beta_1x_i,\sigma^2)$. Get likelihood function:
        \begin{equation}
            L(\beta_0,\beta_1,\sigma^2;x_1,\ldots,x_n,y_1,\ldots,y_n)=(2\pi\sigma^2)^{-\frac{n}{2}}\exp\left[-\frac{\sum_{i=1}^n(y_i-\beta_0-\beta_1x_i)}{2\sigma^2}\right]  
        \end{equation}

    Log-likelihood:
    \begin{equation}
        l(\beta _0,\beta _1,\sigma ^2;x_1,\ldots,x_n,y_1,\ldots,y_n)=-\dfrac{n}{2} \ln(2\pi\sigma ^2)-\dfrac{1}{2\sigma ^2}\sum_{i=1}^n(y_i-\beta _0-\beta _1x_i)^2
    \end{equation}

    MLE, use Fermat Lemma:

\begin{equation}
    \begin{cases}
        \dfrac{\partial^{} l}{\partial \beta _0^{}}=0&\Rightarrow -\dfrac{1}{\sigma ^2}{\displaystyle\sum_{i=1}^n(y_i-\beta _0-\beta _1x_i)}=0\\
        \dfrac{\partial^{} l}{\partial \beta _1^{}}=0&\Rightarrow -\dfrac{1}{\sigma ^2}{\displaystyle\sum_{i=1}^nx_i(y_i-\beta _0-\beta _1x_i)}=0\\
        \dfrac{\partial^{} l}{\partial \sigma^2}=0&\Rightarrow -\dfrac{n}{2}\dfrac{1}{\sigma ^2}+\dfrac{1}{2(\sigma ^2)^2} {\displaystyle\sum_{i=1}^n(y_i-\beta _0-\beta _1x_i)}=0
    \end{cases} 
\end{equation}

    Solution:
    \begin{align*}
        \hat{\beta_0}&=\bar{y}-\beta_1\bar{x}\\
        \hat{\beta_1}&=\dfrac{\sum_{i=1}^n(x_i-\bar{x})(y_i-\bar{y})}{\sum_{i=1}^n(x_i-\bar{x})^2}\\
        \hat{\sigma}^2_n&=\frac{1}{n}\sum_{i=1}^n(y_i-\hat{\beta}_0-\hat{\beta}_1x_i)
    \end{align*}
    
    
    \end{enumerate}

\begin{point}
    Linear Regression Model(Multi-dimension case):
\end{point}

    
\begin{equation}
    y_i=\beta_0+\beta_1x_{i1}+\beta_2x_{i2}+\cdots+\beta_px_{ip}+\epsilon_i    
\end{equation}

    Denote: $\vec{\beta}=(\beta_0,\beta_1,\ldots,\beta_p),\, \vec{x}_i=(1,x_{i1},x_{i2},\ldots,x_{ip})$, then for each $i$: $y_i=x_i^T\beta+\epsilon_i$

    Further denote: Matrix form:
    \begin{equation}\label{EqaMatrixRepreOfSSE}
        y=\begin{pmatrix}
            y_1\\
            y_2\\
            \vdots\\
            y_n
        \end{pmatrix}  
        =
        \begin{pmatrix}
            1&x_{11}&\ldots&x_{1p}\\
            1&x_{21}&\ldots&x_{2p}\\
            \vdots&\vdots&\ddots&\vdots\\
            1&x_{n1}&\ldots&x_{np}
        \end{pmatrix}
        \begin{pmatrix}
            \beta_0\\
            \beta_1\\
            \vdots\\
            \beta_p
        \end{pmatrix}
        +
        \begin{pmatrix}
            \epsilon_1\\
            \epsilon_2\\
            \vdots\\
            \epsilon_n
        \end{pmatrix}
        =X\vec{\beta}+\vec{\epsilon}
    \end{equation}

    Basic Assumptions: Gauss-Markov Assumptions
    \begin{itemize}
        \item OLS unbiased\begin{equation}E(\epsilon_i|x_i)=0\qquad E(y_i|x_i)=x_i^T\beta\end{equation}
        \item Homogeneity of $\epsilon_i$\begin{equation}var(\epsilon_i)=\sigma^2\end{equation}
        \item Independent of $\epsilon$
        \item (For MLE) $\epsilon_i\text{ i.i.d.}\sim N(0,\sigma^2)$
    \end{itemize}

    Residuals:
    \begin{equation}e_i=\hat{\epsilon}_i=y_i-\hat{y}_i=y_i-x_i^T\beta\end{equation}

    Def. Error Sum of Squares (SSE)
    \begin{equation}\mathrm{RSS}=\sum_{i=1}^ne_i^2=\sum_{i=1}^n(y_i-x_i^T\beta)^2\end{equation}

    Estimator exists and unique:($\hat{\sigma}^2$ is after bias correction)
    \begin{align}
        \hat{\beta}&=(X^TX)^{-1}X^Ty\notag \\
        \hat{\sigma}_n^2&=\frac{1}{n}\sum_{i=1}^n(y_i-x^T_i\hat{\beta})^2 \notag\\ 
        \hat{\sigma}^2&=\frac{1}{n-p-1}\sum_{i=1}^n(y_i-x_i^T\hat{\beta})^2\label{EqaEstimatorSigmaWithDoF}
    \end{align}

    

\subsubsection{Kernel Density Estimation}\label{SubSectionKernelDensityEstimation}
    \index{KDE (Kernel Density Estimation)}Given random sample $\{X_i\}$. Def. Empirical CDF.\index{ECDF (Empirical CDF)}
    \begin{equation}\label{empiricaldisreibutionfunction}
        \hat{F}_n(x)=\frac{1}{n}\sum_{i=1}^nI_{(-\infty,x]}(X_i) 
    \end{equation}
        

    Problem: Overfitting when getting $\hat{f}$. Solution: Using \textbf{Kernel Estimate}, replace $I_{(-\infty,x]}(\cdot)$ with Kernel function $K(\cdot)$, then
    \begin{equation}
        \hat{f}_n(x)=\dfrac{F_n(x+h_n)-F-n(x-h_n)}{2h_n}=\frac{1}{nh_n}\sum_{i=1}^nK(\frac{x-X_i}{h_n})
    \end{equation}

    where $h_n$ is \textbf{bandwidth}. Take proper kernel function $K$ to get estimate of $f$.

    Can be considered as a convolution of sample $\{X_i\}$ and kernel function $K$.

    Useful Kernel Functions:
    \begin{itemize}[itemsep= -6 pt,parsep= 0 pt]
        \item $K(x)=\dfrac{1}{2}I_{[-\frac{1}{2},\frac{1}{2}]}$\\
        \item $K(x)=(1-|x|)I_{[-1,1]}$\\
        \item $K(x)=\dfrac{1}{2\pi}e^{-\frac{x^2}{2}}$\\
        \item $K(x)=\dfrac{1}{\pi(1+x^2)}$\\
        \item $K(x)=\dfrac{1}{2\pi}\mathrm{sinc}^2(\dfrac{x}{2})$
    \end{itemize}
    










\subsection{Interval Estimation}\label{SectionIntervalEstimation}
\begin{equation}
    \text{Parameter Estimation}\begin{cases}
        \text{Point Estimation}& \\
        \text{Interval Estimation}& \surd
    \end{cases}    
\end{equation}

    Interval Estimation: to estimate $g(\theta)$, give \textbf{two} estimators $\hat{g}_1(\vec{X}),\, \hat{g}_2(\vec{X})$ defined on $\mathscr{X}$ as the two ends of interval (i.e. give an interval $[\hat{g}_1(\vec{X}),\, \hat{g}_2(\vec{X})]$), then random interval $[\hat{g}_1(\vec{X}),\, \hat{g}_2(\vec{X})]$ is an \textbf{Interval Estimation} of $g(\theta)$.

    \subsubsection{Confidence Interval}\label{SubSectionConfidenceInterval}
    \index{CI (Confidence Interval)}How to judge an interval estimation?
    \begin{itemize}
        \item Reliability
        \begin{equation}P(g(\theta)\in[\hat{g}_1,\hat{g}_2])\end{equation}
        \item Precision
        \begin{equation}E(\hat{g}_2-\hat{g}_1)\end{equation}
    \end{itemize}

    Trade off: (in most cases)
    \begin{quote}
        Given a level of reliability, find an interval with the highest precision above the level
    \end{quote}

\begin{point}
    For a given $0<\alpha<1$, if 
\end{point}

    \begin{equation}
        P(\hat{g}_1\leq g(\theta)\leq \hat{g}_2)\geq 1-\alpha
    \end{equation}

    then $[\hat{g}_1,\hat{g}_2]$ is a \textbf{Confidence Interval} for $g(\theta)$, with \textbf{Confidence Level} $1-\alpha$. 
    
    \textbf{Confidence Coefficient}\index{Confidence Coefficient}:
    \begin{equation}\inf_{\forall\theta\in\Theta}P(
        \theta\in\mathrm{CI}
    )\end{equation}

    Other cases:
    \begin{itemize}[topsep=-4pt]
        \item \textbf{Confidence Limit}:Upper/Lower Confidence Limit
    \begin{align*}
        P(g\leq \hat{g}_U)\geq 1-\alpha\\
        P(\hat{g}_L\leq \theta)\geq 1-\alpha
    \end{align*}
        \item \textbf{Confidence Region}: For high dimensional parameters $\vec{g}=(g_1,g_2,\ldots,g_k)$
        \begin{equation}P(\vec{g}\in S(\vec{X}))\geq 1-\alpha\quad \forall \theta\in\Theta \end{equation}
        
    \end{itemize}

    Mission: Determine $\hat{g}_1,\hat{g}_2$.


\subsubsection{Pivot Variable Method}\label{SubSectionPivotVariableMethod}
    Idea: Based on point estimation, construct a new variable and thus find the interval estimation.

    Def. \textbf{Pivot Variable}\index{Pivot Variable} $T$, satisfies: 
    \begin{itemize}[itemsep= -2 pt,parsep= -2 pt]
        \item Expression of $T$ contains $\theta$ (thus $T$ is not a statistic).
        \item Distribution of $T$ independent of $\theta$.
    \end{itemize}

    In different cases, construct different pivot variable, usually base on sufficient statistics and transform.
    
    Knowing a proper pivot variable $T=T(\hat{\varphi},g(\theta))\sim f$, ($f$ is some distribution independent of $\theta$), $\hat{\varphi}$ is a sufficient statistic), then we can take $T$ satisfies:
    \begin{equation}
        P(f_{1-\frac{\alpha}{2}}\leq T\leq f_{\frac{\alpha}{2}})=1-\alpha
    \end{equation}

    Construct the inverse mapping of $T=T(\hat{\varphi},g(\theta))\rightleftarrows g(\theta)=T^{-1}(T,\hat{\varphi})$, we get
    \begin{equation}
        P[T^{-1}(f_{1-\frac{\alpha}{2}},\hat{\varphi})\leq\hat{g}\leq T^{-1}(f_{\frac{\alpha}{2}},\hat{\varphi})]=1-\alpha
    \end{equation}
    
    Thus get a confidence interval for $\theta$ with confidence coefficient $1-\alpha$.\\


\newpage
\subsubsection{Confidence Interval for Common Distributions}\label{SubSectionConfidenceIntervalForDistributions}

    Some important properties of $\chi^2$, $t$ and $F$ see section \hyperref[chi2_t_F_properties]{\ref{chi2_t_F_properties}}.
    \begin{enumerate}
        \item Single normal population: $\vec{X}=\{X_1,X_2,\ldots,X_n\}\in\mathscr{X}$ i.i.d from Normal Distribution population $N(\mu,\sigma^2)$. Denote sample mean and sample variance: 
        \begin{equation}\bar{X}=\frac{1}{n}\sum_{i=1}^nX_i\qquad S^2=\frac{1}{n-1}\sum_{i=1}^n(X_i-\bar{X})^2\qquad S_\mu=\dfrac{1}{n}\sum_{i=1}^n(X_i-\mu)^2\text{,(for }\mu\text{ known)}\end{equation}

        \hypertarget{OneSampletTest}{Estimating} $\mu\,\&\,\sigma^2$: construction of pivot variable under different circumstances:


        \begin{table}[htbp]
            \centering
            \renewcommand\arraystretch{1.9}
            \begin{tabularx}{\linewidth}{|c|c|Y|}
                \hline 
                Estimation& Pivot Variable & Confidence Interval\\
                \hline
                $\sigma^2$ known, estimate $\mu$    &   $T=\dfrac{\sqrt{n}(\bar{X}-\mu)}{\sigma}\sim N(0,1)$ & $\left[ \bar{X}-\dfrac{\sigma}{\sqrt{n}}N_\frac{\alpha}{2},\bar{X}+\dfrac{\sigma}{\sqrt{n}}N_\frac{\alpha}{2} \right]$\\
                \hline
                $\sigma^2$ unknown, estimate $\mu$&$T=\dfrac{\sqrt{n}(\bar{X}-\mu)}{S}\sim t_{n-1}$&$\left[\bar{X}-\dfrac{S}{\sqrt{n}}t_{n-1,\frac{\alpha}{2}},\bar{X}+\dfrac{S}{\sqrt{n}}t_{n-1,\frac{\alpha}{2}}\right]$\\
                \hline
                $\mu$ known, estimate $\sigma^2$&$T=\dfrac{nS_\mu^2}{\sigma^2}\sim\chi_n^2$&$\left[\dfrac{nS^2_\mu}{\chi^2_{n,\frac{\alpha}{2}}},\dfrac{nS^2_\mu}{\chi^2_{n,1-\frac{\alpha}{2}}}\right]$\\
                \hline
                $\mu$ unknown, estimate $\sigma^2$&$T=\dfrac{(n-1)S^2}{\sigma^2}\sim\chi^2_{n-1} $&$\left[\dfrac{(n-1)S^2}{\chi^2_{n-1,\frac{\alpha}{2}}},\dfrac{(n-1)S^2}{\chi^2_{n-1,1-\frac{\alpha}{2}}}\right]$\\
                \hline
            \end{tabularx}
        \end{table}
    % \begin{itemize}
    %     \item $\sigma^2$ known, estimate $\mu$:
    %     \begin{equation}
    %         T=\dfrac{\sqrt{n}(\bar{X}-\mu)}{\sigma}\sim N(0,1)
    %     \end{equation}

    %     confidence interval:
    %     \begin{equation}
    %         \left[ \bar{X}-\frac{\sigma}{\sqrt{n}}N_\frac{\alpha}{2},\bar{X}+\frac{\sigma}{\sqrt{n}}N_\frac{\alpha}{2} \right]
    %     \end{equation}
    %     \item $\sigma^2$ unknown, estimate $\mu$:
    %     \begin{equation}
    %         T=\frac{\sqrt{n}(\bar{X}-\mu)}{S}\sim t_{n-1}
    %     \end{equation}

    %     confidence interval:
    %     \begin{equation}
    %         \left[\bar{X}-\frac{S}{\sqrt{n}}t_{n-1,\frac{\alpha}{2}},\bar{X}+\frac{S}{\sqrt{n}}t_{n-1,\frac{\alpha}{2}}\right]
    %     \end{equation}
    %     \item $\mu$ known, estimate $\sigma^2$: denote $S_\mu=\dfrac{1}{n}\sum_{i=1}^n(X_i-\mu)^2$
    %     \begin{equation}
    %         T=\frac{nS_\mu^2}{\sigma^2}\sim\chi_n^2
    %     \end{equation}

    %     confidence interval:
    %     \begin{equation}
    %         \left[\frac{nS^2_\mu}{\chi^2_{n,\frac{\alpha}{2}}},\frac{nS^2_\mu}{\chi^2_{n,1-\frac{\alpha}{2}}}\right]
    %     \end{equation}
    %     \item $\mu$ unknown, estimate $\sigma^2$:
    %     \begin{equation}
    %         T=\frac{(n-1)S^2}{\sigma^2}\sim\chi^2_{n-1}  
    %     \end{equation}

    %     confidence interval:
    %     \begin{equation}
    %         \left[\frac{(n-1)S^2}{\chi^2_{n-1,\frac{\alpha}{2}}},\frac{(n-1)S^2}{\chi^2_{n-1,1-\frac{\alpha}{2}}}\right]
    %     \end{equation}
    % \end{itemize}

    \item Double normal population: $\vec{X}=\{X_1,X_2,\ldots,X_m\}$ i.i.d. from $N(\mu_1,\sigma_1^2)$; $\vec{Y}=\{Y_1,Y_2,\ldots,Y_n\}$ i.i.d. from $N(\mu_2,\sigma^2_2)$

    
    Denote sample mean, sample variance and pooled sample variance:
\begin{align*}
    \bar{X}&=\frac{1}{m}\sum_{i=1}^nX_i &S_X^2&=\frac{1}{m-1}\sum_{i=1}^m(X_i-\bar{X})^2& S_{\mu_1}^2 &=\dfrac{1}{m}\sum_{i=1}^m(X_i-\mu_1)^2,(\mu_1\text{ known}) \\\bar{Y}&=\frac{1}{n}\sum_{i=1}^n Y_i&S^2_Y&=\frac{1}{n-1}\sum_{i=1}^n(Y_i-\bar{Y})^2& S_{\mu_2}^2 &=\dfrac{1}{n}\sum_{i=1}^n(Y_i-\mu_2)^2,(\mu_2\text{ known})\\
    &&S_\omega^2&=\dfrac{(m-1)S_X^2+(n-1)S_Y^2}{m+n-2}&&
\end{align*}



    % \begin{align*}
    %     \bar{X}&=\frac{1}{m}\sum_{i=1}^nX_i &\bar{Y}&=\frac{1}{n}\sum_{i=1}^n Y_i  \\
    %     S_X^2&=\frac{1}{m-1}\sum_{i=1}^m(X_i-\bar{X})^2& S^2_Y&=\frac{1}{n-1}\sum_{i=1}^n(Y_i-\bar{Y})^2\\
    %     S_{\mu_1}^2 &=\dfrac{1}{m}\sum_{i=1}^m(X_i-\mu_1)^2,(\mu_1\text{ known}) & S_{\mu_2}^2 &=\dfrac{1}{n}\sum_{i=1}^n(Y_i-\mu_2)^2,(\mu_2\text{ known})\\
    %     S_\omega^2&=\dfrac{(m-1)S_X^2+(n-1)S_Y^2}{m+n-2}&&
    % \end{align*}
    

    \hypertarget{TwoSampletTest}{Estimating} $\mu_1-\mu_2$:
    
    When $\sigma_1^2\neq\sigma^2_2$ unknown, estimate $\mu_1-\mu_2$: Behrens-Fisher Problem, remain unsolved, but can deal with simplified cases. 

    \begin{table}[H]
        \centering
        \renewcommand\arraystretch{2.2}
        \begin{tabularx}{\linewidth}{|c|c|Y|}
            \hline
            Estimation& Pivot Variable & Confidence Interval\\
            \hline
            \makecell{$\sigma_1^2\,\&\,\sigma_2^2$ known,\\estimate $\mu_1-\mu_2$}&$ T=\dfrac{\bar{X}-\bar{Y}-(\mu_1-\mu_2)}{\sqrt{\dfrac{\sigma_1^2}{m}+\dfrac{\sigma^2_2}{n}}}\sim N(0,1)$&\makecell{$\left[\bar{X}-\bar{Y}-N_{\frac{\alpha}{2}}\sqrt{\dfrac{\sigma_1^2}{m}+\dfrac{\sigma_2^2}{n}},\right.\quad $\\$\quad\left.\bar{X}-\bar{Y}+N_{\frac{\alpha}{2}}\sqrt{\dfrac{\sigma_1^2}{m}+\dfrac{\sigma_2^2}{n}}   \right]$}\\
            \hline
            \makecell{$\sigma_1^2=\sigma^2_2$ unknown,\\estimate $\mu_1-\mu_2$}&$T=\dfrac{\bar{X}-\bar{Y}-(\mu_1-\mu_2)}{S_\omega\sqrt{\dfrac{1}{m}+\dfrac{1}{n}}}\sim t_{m+n-2}$&\makecell{$\left[\bar{X}-\bar{Y}-S_\omega t_{m+n-2,\frac{\alpha}{2}}\sqrt{\dfrac{1}{m}+\dfrac{1}{n}}\right.,\quad $\\$\left.\quad\bar{X}-\bar{Y}+S_\omega t_{m+n-2,\frac{\alpha}{2}}\sqrt{\dfrac{1}{m}+\dfrac{1}{n}} \right]$}\\
            \hline
            \makecell{Welch's $t$-Interval\\(when $m$, $n$ large enough)}&$T=\dfrac{\bar{X}-\bar{Y}-(\mu_1-\mu_2)}{\sqrt{\dfrac{S_X^2}{m}+\dfrac{S^2_Y}{n}}}\xrightarrow[]{\mathscr{L}} N(0,1)$&\makecell{$\left[\bar{X}-\bar{Y}-N_{\frac{\alpha}{2}}\sqrt{\dfrac{S_1^2}{m}+\dfrac{S_2^2}{n}}\right.,\quad$\\$\quad\left.\bar{X}-\bar{Y}+N_{\frac{\alpha}{2}}\sqrt{\dfrac{S_1^2}{m}+\dfrac{S_2^2}{n}}\right]$}\\
            \hline
        \end{tabularx}
    \end{table}

    Estimating $\dfrac{\sigma^2_1}{\sigma_2^2}$:
    \begin{table}[H]
        \centering
        \renewcommand\arraystretch{2.2}
        \begin{tabularx}{\linewidth}{|c|c|Y|}
            \hline
            Estimation& Pivot Variable & Confidence Interval\\
            \hline
            $\mu_1,\mu_2$ known, estimate $\dfrac{\sigma^2_1}{\sigma_2^2}$&$T=\dfrac{S_{\mu_2}^2}{S_{\mu_1}^2}\dfrac{\sigma_1^2}{\sigma^2_2}\sim F_{n,m}$&\makecell{$\left[\dfrac{S_{\mu_1}^2}{S_{\mu_2}^2}\dfrac{1}{F_{m,n,\frac{\alpha}{2}}},\dfrac{S_{\mu_1}^2}{S_{\mu_2}^2}\dfrac{1}{F_{m,n,1-\frac{\alpha}{2}}}\right]$\\or $\left[\dfrac{S_{\mu_1}^2}{S_{\mu_2}^2}{F_{m,n,\frac{\alpha}{2}}},\dfrac{S_{\mu_1}^2}{S_{\mu_2}^2}F_{n,m,\frac{\alpha}{2}}\right]$}\\
            \hline
            $\mu_1,\mu_2$ unknown, estimate $\dfrac{\sigma^2_1}{\sigma_2^2}$&$T=\dfrac{S_Y^2}{S_X^2}\dfrac{\sigma_1^2}{\sigma^2_2}\sim F_{n-1,m-1}$&\makecell{$\left[\dfrac{S_X^2}{S_Y^2}\dfrac{1}{F_{m-1,n-1,\frac{\alpha}{2}}},\dfrac{S_X^2}{S_Y^2}\dfrac{1}{F_{m-1,n-1,1-\frac{\alpha}{2}}}\right]$\\or $\left[\dfrac{S_X^2}{S_Y^2}\dfrac{1}{F_{m-1,n-1,\frac{\alpha}{2}}},\dfrac{S_X^2}{S_Y^2}F_{n-1,m-1,\frac{\alpha}{2}}\right]$}\\
            \hline
        \end{tabularx}
    \end{table}
    
        \item Non-normal population:% $U(0,\theta)$, 
        \begin{table}[H]
            \centering
            \renewcommand\arraystretch{2.2}
            \begin{tabularx}{\linewidth}{|c|c|Y|}
                \hline
                Estimation& Pivot Variable & Confidence Interval\\
                \hline
                \makecell{Uniform Distribution: \\$\vec{X}$ i.i.d. from $U(0,\theta)$}&$T=\dfrac{X_{(n)}}{\theta}\sim U(0,1)$&$\left[X_{(n)},\dfrac{X_{(n)}}{\sqrt[n]{\alpha}}\right]$\\
                \hline
                \makecell{Exponential Distribution: \\$\vec{X}$ i.i.d. from $\epsilon(\lambda)$}&$T=2n\lambda\bar{X}\sim\chi^2_{2n}$&$\left[\dfrac{\chi_{2n,1-\frac{\alpha}{2}}^2}{2n\bar{X}},\dfrac{\chi_{2n,\frac{\alpha}{2}}^2}{2n\bar{X}}\right]$\\
                \hline
                \makecell{Bernoulli Distribution: \\$\vec{X}$ i.i.d. from $B(1,\theta)$}&$T=\dfrac{\sqrt{n}(\bar{X}-\theta)}{\sqrt{\bar{X}(1-\bar{X})}}\xrightarrow[]{\mathscr{L}}N(0,1)$&$\left[\bar{X}-N_{\frac{\alpha}{2}}\sqrt{\dfrac{\bar{X}(1-\bar{X})}{n}},\bar{X}+N_{\frac{\alpha}{2}}\sqrt{\dfrac{\bar{X}(1-\bar{X})}{n}}\right]$\\
                \hline
                \makecell{Poisson Distribution: \\$\vec{X}$ i.i.d. from $P(\lambda)$}&$T=\dfrac{\sqrt{n}(\bar
                X-\lambda)}{\sqrt{\bar{X}}}\xrightarrow[]{\mathscr{L}}N(0,1)$&$\left[\bar{X}-N_{\frac{\alpha}{2}}\sqrt{\dfrac{\bar{X}}{n}},\bar{X}+N_{\frac{\alpha}{2}}\sqrt{\dfrac{\bar{X}}{n}}\right]    $\\
                \hline
            \end{tabularx}
        \end{table}
        
    %     $\epsilon(\lambda)$
    %     \begin{itemize}
    %         \item Uniform Distribution: $\vec{X}=(X_1,X_2,\ldots,X_n)$ i.i.d. from $U(0,\theta)$
    %         \begin{equation}
    %             T=\frac{X_{(n)}}{\theta}\sim U(0,1)
    %         \end{equation}
    
    %         confidence interval:
    %         \begin{equation}
    %             \left[X_{(n)},\frac{X_{(n)}}{\sqrt[n]{\alpha}}\right]
    %         \end{equation}
    %         \item Exponential Distribution: $\vec{X}=(X_1,X_2,\ldots,X_n)$ i.i.d. from $\epsilon(\lambda)$
    %         \begin{equation}
    %             T=2n\lambda\bar{X}\sim\chi^2_{2n}
    %         \end{equation}
    
    %         confidence interval:
    %         \begin{equation}
    %             \left[\frac{\chi_{2n,1-\frac{\alpha}{2}}^2}{2n\bar{X}},\frac{\chi_{2n,\frac{\alpha}{2}}^2}{2n\bar{X}}\right]
    %         \end{equation}
    %         \item Bernoulli Distribution: $\vec{X}=(X_1,X_2,\ldots,X_n)$ i.i.d. from $B(1,\theta)$, take large sample approximation
    %         \begin{equation}
    %             T=\frac{\sqrt{n}(\bar{X}-\theta)}{\sqrt{\theta(1-\theta)}}\xrightarrow[]{\mathscr{L}}N(0,1)
    %         \end{equation}
    
    %         which is not easy to construct inverse mapping. Simplify:
    %         \begin{equation}
    %             T=\frac{\sqrt{n}(\bar{X}-\theta)}{\sqrt{\bar{X}(1-\bar{X})}}\xrightarrow[]{\mathscr{L}}N(0,1)
    %         \end{equation}
    
    %         confidence interval:
    %         \begin{equation}
    %             \left[\bar{X}-N_{\frac{\alpha}{2}}\sqrt{\frac{\bar{X}(1-\bar{X})}{n}},\bar{X}+N_{\frac{\alpha}{2}}\sqrt{\frac{\bar{X}(1-\bar{X})}{n}}\right]
    %         \end{equation}
    
    %         \item Poisson Distribution: $\vec{X}=(X_1,X_2,\ldots,X_n)$ i.i.d. from $P(\lambda)$, take large sample approximation:
    %         \begin{equation}
    %             T=\frac{\sqrt{n}(\bar
    %             X-\lambda)}{\sqrt{\bar{X}}}\xrightarrow[]{\mathscr{L}}N(0,1)  
    %         \end{equation}
    
    %         confidence interval:
    %         \begin{equation}
    %         \left[\bar{X}-N_{\frac{\alpha}{2}}\sqrt{\frac{\bar{X}}{n}},\bar{X}+N_{\frac{\alpha}{2}}\sqrt{\frac{\bar{X}}{n}}\right]    
    %         \end{equation}
    %     \end{itemize}
    % \newpage

    % \begin{itemize}
    %     \item $\sigma_1^2\,\&\,\sigma_2^2$ known, estimate $\mu_1-\mu_2$:
    %     \begin{equation}
    %         T=\frac{\bar{X}-\bar{Y}-(\mu_1-\mu_2)}{\sqrt{\dfrac{\sigma_1^2}{m}+\dfrac{\sigma^2_2}{n}}}\sim N(0,1)
    %     \end{equation}

    %     condidence interval:
    %     \begin{equation}
    %         \left[\bar{X}-\bar{Y}-N_{\frac{\alpha}{2}}\sqrt{\frac{\sigma_1^2}{m}+\frac{\sigma_2^2}{n}},\bar{X}-\bar{Y}+N_{\frac{\alpha}{2}}\sqrt{\frac{\sigma_1^2}{m}+\frac{\sigma_2^2}{n}}  \right]
    %     \end{equation}

    %     \item $\sigma_1^2=\sigma^2_2$ unknown, estimate $\mu_1-\mu_2$:
    %     \begin{equation}
    %         T=\frac{\bar{X}-\bar{Y}-(\mu_1-\mu_2)}{\sqrt{\dfrac{(m-1)S_X^2+(n-1)S_Y^2}{m+n-2}\left(\dfrac{1}{m}+\dfrac{1}{n}\right)}}\sim t_{m+n-2}
    %     \end{equation}


    %     where $S_\omega^2=\dfrac{(m-1)S_X^2+(n-1)S_Y^2}{m+n-2}$ is Pooled Sample Variance.

    %     condidence interval:
    %     \begin{equation}
    %         \left[\bar{X}-\bar{Y}-S_\omega t_{m+n-2,\frac{\alpha}{2}}\sqrt{\frac{1}{m}+\frac{1}{n}},\bar{X}-\bar{Y}+S_\omega t_{m+n-2,\frac{\alpha}{2}}\sqrt{\frac{1}{m}+\frac{1}{n}} \right]
    %     \end{equation}

    %     \item Approximate solution to Behrens-Fisher Problem: Welch's $t$-Interval (used when $m$, $n$ are big enough)
    %     \begin{equation}
    %         T=\frac{\bar{X}-\bar{Y}-(\mu_1-\mu_2)}{\sqrt{\dfrac{S_X^2}{m}+\dfrac{S^2_Y}{n}}}\xrightarrow[]{\mathscr{L}} N(0,1)
    %     \end{equation}

    %     confidence interval:
    %     \begin{equation}
    %         \left[\bar{X}-\bar{Y}-N_{\frac{\alpha}{2}}\sqrt{\frac{S_1^2}{m}+\frac{S_2^2}{n}},\bar{X}-\bar{Y}+N_{\frac{\alpha}{2}}\sqrt{\frac{S_1^2}{m}+\frac{S_2^2}{n}}\right]
    %     \end{equation}
    % \end{itemize}


    % Denote sample mean and sample variance:
    % \begin{align*}
    %     \bar{X}&=\frac{1}{m}\sum_{i=1}^nX_i & S_X^2&=\frac{1}{m-1}\sum_{i=1}^m(X_i-\bar{X})^2\\
    %     \bar{Y}&=\frac{1}{n}\sum_{i=1}^n Y_i & S^2_Y&=\frac{1}{n-1}\sum_{i=1}^n(Y_i-\bar{Y})^2
    % \end{align*}
        


    % \begin{itemize}
    %     \item $\mu_1,\mu_2$ known, estimate $\dfrac{\sigma^2_1}{\sigma_2^2}$: denote $S_{\mu_1}^2=\dfrac{1}{m}\sum_{i=1}^m(X_i-\mu_1)^2$, $S_{\mu_2}^2=\dfrac{1}{n}\sum_{i=1}^n(Y_i-\mu_2)^2$
    %     \begin{equation}
    %         T=\frac{S_{\mu_2}^2}{S_{\mu_1}^2}\frac{\sigma_1^2}{\sigma^2_2}\sim F_{n,m}
    %     \end{equation}

    %     confidence interval:
    %     \begin{equation}
    %         \left[\frac{S_{\mu_1}^2}{S_{\mu_2}^2}\frac{1}{F_{m,n,\frac{\alpha}{2}}},\frac{S_{\mu_1}^2}{S_{\mu_2}^2}\frac{1}{F_{m,n,1-\frac{\alpha}{2}}}\right]
    %     \end{equation}

    %     or use $F_{m,n,1-\frac{\alpha}{2}}=\dfrac{1}{F_{n,m,\frac{\alpha}{2}}}$, transform as:
    %     \begin{equation}
    %         \left[\frac{S_{\mu_1}^2}{S_{\mu_2}^2}{F_{m,n,\frac{\alpha}{2}}},\frac{S_{\mu_1}^2}{S_{\mu_2}^2}F_{n,m,\frac{\alpha}{2}}\right]
    %     \end{equation}

    %     \item $\mu_1,\mu_2$ unknown, estimate $\dfrac{\sigma^2_1}{\sigma_2^2}$:
    %     \begin{equation}
    %         T=\frac{S_Y^2}{S_X^2}\frac{\sigma_1^2}{\sigma^2_2}\sim F_{n-1,m-1}
    %     \end{equation}

    %     confidence interval:
    %     \begin{equation}
    %         \left[\frac{S_X^2}{S_Y^2}\frac{1}{F_{m-1,n-1,\frac{\alpha}{2}}},\frac{S_X^2}{S_Y^2}\frac{1}{F_{m-1,n-1,1-\frac{\alpha}{2}}}\right]
    %     \end{equation}

    %     transform as:
    %     \begin{equation}
    %         \left[\frac{S_X^2}{S_Y^2}\frac{1}{F_{m-1,n-1,\frac{\alpha}{2}}},\frac{S_X^2}{S_Y^2}F_{n-1,m-1,\frac{\alpha}{2}}\right]
    %     \end{equation}
    % \end{itemize}
    
    \item General Case: Use asymptotic normality of MLE to construct CLT for large sample. MLE of $\theta$ satisfies:
    \begin{equation}
        \sqrt{n}(\hat{\theta}^*-\theta)\xrightarrow[]{\mathscr{L}}N(0,\frac{1}{I(\theta)})
    \end{equation}

    where $\hat{\theta}^*$ is MLE of $\theta$. Replace $\dfrac{1}{I(\theta)}$ by $\sigma^2(\hat{\theta}^*)$, then
    \begin{equation}
        T=\frac{\sqrt{n}(\hat{\theta}^*-\theta)}{\sigma(\hat{\theta}^*)}\xrightarrow[]{\mathscr{L}}N(0,1)    
    \end{equation}

    confidence interval:
    \begin{equation}
        \left[\hat{\theta}^*-\frac{N_{\frac{\alpha}{2}}}{\sqrt{n}}\sigma(\hat{\theta}^*),\hat{\theta}^*+\frac{N_{\frac{\alpha}{2}}}{\sqrt{n}}\sigma(\hat{\theta}^*)\right]
    \end{equation}
    \end{enumerate}

\subsubsection{Fisher Fiducial Argument*}\label{SubSectionFisherFiducialArgument}
    Idea: When sample is known, we can get '\textbf{Fiducial Probability}' of $\theta$, thus can find an interval estimation based on fiducial distribution.(Similar to the idea of MLE)

    Remark: Fiducial probability (denoted as $\tilde{P}(\theta)$) is 'probability of parameter', in the case that sample is known. \textbf{Fiducial probability is different from Probability}.

    Thus get
    \begin{equation}
        \tilde{P}(\hat{g}_1\leq g(\theta)\leq \hat{g}_2)=1-\alpha
    \end{equation}







\subsection{Hypothesis Testing}\label{SectionHypothesisTesting}
    \index{HT (Hypothesis Testing)}Hypothesis is a statement about the characteristic of population, e.g. distribution form, parameters, etc. 
    
    Mission: Use sample to test the hypothesis, i.e. judge whether population has some characteristic.

\subsubsection{Basic Concepts}\label{SubSectionHypothesisTestingBasicConcepts}
    Parametric hypothesis testing.

    For random sample $\vec{X}=(X_1,X_2,\ldots,X_n)\in\mathscr{X}$ i.i.d. from $\mathscr{F}=\{f(x;\theta);\theta\in\Theta\}$
    \begin{itemize}[topsep = -3 pt]
        \item Null Hypothesis $H_0$ \& Alternative Hypothesis $H_1$: Wonder whether a statement is true. Def. \textbf{Null Hypothesis}: $H_0:\theta\in\Theta_0\subset\Theta$, \textbf{a statement that we try to reject based on sample}; $H_1:\theta\in\Theta_1=\Theta-\Theta_0$ is \textbf{Alternative Hypothesis}.
        
\begin{point}
    Note: \textbf{Cannot} exchange $ H_0 $ and $ H_1 $, because when the evidence is ambiguity, we have to accept $ H_0 $, regardless of what $ H_0 $ is. So it is \textbf{very important} to pick the proper $ H_0 $.
\end{point}


        Thus Hypothesis Testing:
        \begin{equation}
            H_0:\theta\in\Theta_0\longleftrightarrow H_1:\theta\in\Theta_1
        \end{equation}
        
        \item Rejection Region $R$ \& Acceptance Region $R^C$: Judge whether to reject $H_0$ from sample, Def. \textbf{Rejection Region}:
        \begin{equation}R\subset\mathscr{X}\text{: reject } H_0 \text{ if } \vec{X}\in R\end{equation}

        Acceptance Region: accept $H_0$ if $\vec{X}\in R^C$
        \item Test Function\index{Test Function}: Describe how to make a decision.
        \begin{itemize}
            \item Continuous Case:
        \begin{equation}
            \varphi(\vec{X})=\begin{cases}
                1,&\vec{X}\in R\\
                0,&\vec{X}\in R^C
            \end{cases}
        \end{equation}

        i.e. $R=\{\vec{X}:\varphi(\vec{X})=1\}$. Where $R$ to be determined.

        \item Discrete Case: Randomized Test Function
        \begin{equation}
        \varphi(\vec{X})=\begin{cases}
            1,&\vec{X}\in R-\partial R\\
            r,&\vec{X}\in \partial R\\
            0,&\vec{X}\in R^C
        \end{cases}    
        \end{equation}

        Where $R$ and $r$  to be determined.
    \end{itemize}
        \item Type I Error \& Type II Error: Sample is random, possible to make a wrong judge.
            
        \begin{itemize}[topsep = -4 pt]
            \item Type I Error (弃真): $H_0$ is true but sample falls in $R$, thus $H_0$ is rejected.
            \begin{equation}P(\text{type I error})=P(\vec{X}\in R|H_0)=\alpha(\theta)\end{equation}
            \item Type II Error (取伪): $H_0$ is wrong but sample falls in $R^C$, thus $H_0$ is accepted.
            \begin{equation}P(\text{type II error})=P(\vec{X}\notin R|H_1)=\beta(\theta)\end{equation}
        \end{itemize}

    \begin{table}[htbp]
        \centering
        \begin{tabular}{c|ccc}
            \hline
            &\multicolumn{3}{c}{Judgement}\\
            \hline
            \multirow{3}{*}{Real Case}&&Accept $H_0$&Reject $H_0$\\ 
            &$H_0$&$\surd$&Type I Error\\ 
            &$H_1$&Type II Error&$\surd$\\ 
            \hline
        \end{tabular}
        \caption{'Confusion Matrix'}
    \end{table}


        Impossible to make probability of Type I \& II Error small simultaneously, how to pick a proper test $\varphi(\vec{x})$? 
        
\begin{point}
            \textbf{Neyman-Pearson Principle}: First control $\alpha\leq\alpha_0$, then take $\min \beta$.
\end{point}


        How to determine $\alpha_0$? Depend on specific problem.\footnote{In most cases, take $\alpha_0=0.05$.}

        \item $p$-value: probability to get larger bias than observed $\vec{x}_0$ \uline{under $H_0$.}
        
        e.g. For reject region $R=\{\vec{X}|T(\vec{X})\geq C\},$ $p$-value:
        \begin{equation}
            p(\vec{x})=P[T(\vec{X})\geq t(\vec{x}_0)|H_0]
        \end{equation}


        Remark: Under $H_0$, the probability to get a \textbf{worse} result than $\vec{x}_0$.
        
        Rule: Reject $H_0$ if $p(\vec{x}_0)\leq\alpha_0$

        \item Power Function\index{Power Function}: (when $H_0$ is given), probability to reject $H_0$ by sampling.
        \begin{equation}
            \pi(\theta)=\begin{cases}
                P(\text{type I error}),& \theta\in\Theta_0\\
                1-P(\text{type II error}),& \theta\in\Theta_1
            \end{cases}
            =
            \begin{cases}
                \alpha(\theta),&\theta\in\Theta_0\\
                1-\beta(\theta),&\theta\in\Theta_1
            \end{cases}
        \end{equation}

        Express as test function:
        \begin{equation}
            \pi(\theta)=E[\varphi(\vec{X})|\theta]
        \end{equation}

        A nice test: $\pi(\theta)$ small under $H_0$, large under $H_1$ (and grows very fast at the boundary of $ H_0 $ and $ H_1 $).
        \end{itemize}

        \begin{point}
            \textbf{General Steps of Hypothesis Testing:}
        \end{point}
        
        

        \begin{enumerate}[topsep=0pt]
            \item Propose $H_0\,\&\, H_1$.
            \item Determine $R$ (usually in the form of a statistic, e.g. $R=\{\vec{X}:T(\vec{X})\geq c\}$).
            \item Select a proper $\alpha$ (to determine $c$).
            \item Sampling, get sample (as well as $t(\vec{x})$), then 
            \begin{itemize}[topsep=-1pt,itemsep=-2pt]
                \item compare with $R$ and determine whether to reject/accept $H_0$, or
                \item calculate $ p $-value and determine whether to reject/accept$ H_0 $
            \end{itemize}
            
                
            

        \end{enumerate}

% \subsubsection{Hypotheses Testing for Common Distributions}
%     \begin{itemize}
%         \item Single normal distribution: $\vec{X}=(X_1,X_2,\ldots,X_n)$ i.i.d. from $N(\mu,\sigma^2)$
        
%         Testing $\mu$:


%     \end{itemize}

\subsubsection{Hypothesis Testing of Common Distributions}\label{SubSectionHypothesisTestingOfCommonDistributions}
    For some common distribution populations, determine rejection region $R$ under certain $H_0$ with confidence coefficient $\alpha$.

    Definition of necessary statistics see section \hyperref[SubSectionConfidenceIntervalForDistributions]{\ref{SubSectionConfidenceIntervalForDistributions}}.

    \begin{enumerate}
        \item Single normal population:\index{$ t $-test}

        \begin{table}[H]
            \centering
            \renewcommand\arraystretch{1.2}
            \begin{tabularx}{\linewidth}{|c|c|c|Y|c|}
                \hline
                Condition&$H_0$&$H_1$&Testing Statistic $T$&Rejection Region $R$\\
                \hline
                \multirow{3}{*}{$\sigma^2$ known, test $\mu$}&$\mu=\mu_0$&$\mu\neq\mu_0$&\multirow{3}{*}{$T=\dfrac{\sqrt{n}(\bar{X}-\mu_0)}{\sigma}\sim N(0,1)$}&$|T|>N_\frac{\alpha}{2}$\\
                &$\mu\leq\mu_0$&$\mu>\mu_0$&&$T>N_\alpha$\\
                &$\mu\geq\mu_0$&$\mu<\mu_0$&&$T<-N_\alpha$\\
                \hline
                \multirow{3}{*}{$\sigma^2$ unknown, test $\mu$}&$\mu=\mu_0$&$\mu\neq\mu_0$&\multirow{3}{*}{$T=\dfrac{\sqrt{n}(\bar{X}-\mu_0)}{S}\sim t_{n-1}$}&$|T|>t_{n-1,\frac{\alpha}{2}}$\\
                &$\mu\leq\mu_0$&$\mu>\mu_0$&&$T>t_{n-1,\alpha}$\\
                &$\mu\geq\mu_0$&$\mu<\mu_0$&&$T<-t_{n-1,\alpha}$\\
                \hline
                \multirow{3}{*}{$\mu$ known, test $\sigma^2$}&$\sigma^2=\sigma_0^2$&$\sigma^2\neq\sigma_0^2$&\multirow{3}{*}{$T=\dfrac{nS_\mu^2}{\sigma_0^2}\sim \chi_n^2$}&$T<\chi^2_{n,1-\frac{\alpha}{2}}\cup T>\chi^2_{n,\frac{\alpha}{2}}$\\
                &$\sigma^2\leq\sigma_0^2$&$\sigma^2>\sigma_0^2$&&$T>\chi^2_{n,\alpha}$\\
                &$\sigma^2\geq\sigma_0^2$&$\sigma^2<\sigma_0^2$&&$T<\chi^2_{n,1-\alpha}$\\
                \hline
                \multirow{3}{*}{$\mu$ unknown, test $\sigma^2$}&$\sigma^2=\sigma_0^2$&$\sigma^2\neq\sigma_0^2$&\multirow{3}{*}{$T=\dfrac{(n-1)S^2}{\sigma_0^2}\sim \chi_{n-1}^2$}&$T<\chi^2_{n-1,1-\frac{\alpha}{2}}\cup T>\chi^2_{n-1,\frac{\alpha}{2}}$\\
                &$\sigma^2\leq\sigma_0^2$&$\sigma^2>\sigma_0^2$&&$T>\chi^2_{n-1,\alpha}$\\
                &$\sigma^2\geq\sigma_0^2$&$\sigma^2<\sigma_0^2$&&$T<\chi^2_{n-1,1-\alpha}$\\
                \hline
            \end{tabularx}
        \end{table}


    \item Double normal population:
    
    \begin{table}[htbp]
        \centering
        \renewcommand\arraystretch{1.2}
        \begin{tabularx}{\linewidth}{|c|c|c|Y|c|}
            \hline
            Condition&$H_0$&$H_1$&Testing Statistic $T$&Rejection Region $R$\\
            \hline
            \multirow{3}{*}{\makecell{$\sigma_1^2,\sigma_2^2$ known,\\test $\mu_1-\mu_2$}}&$\mu_1-\mu_2=\mu_0$&$\mu_1-\mu_2\neq\mu_0$&\multirow{3}{*}{$T=\dfrac{\bar{X}-\bar{Y}-\mu_0}{\sqrt{\dfrac{\sigma_1^2}{m}+\dfrac{\sigma_2^2}{n}}}\sim N(0,1)$}&$|T|>N_\frac{\alpha}{2}$\\
                &$\mu_1-\mu_2\leq\mu_0$&$\mu_1-\mu_2>\mu_0$&&$T>N_\alpha$\\
                &$\mu_1-\mu_2\geq\mu_0$&$\mu_1-\mu_2<\mu_0$&&$T<-N_\alpha$\\
                \hline
                \multirow{3}{*}{\makecell{$\sigma_1^2,\sigma_2^2$ unknown,\\test $\mu_1-\mu_2$}}&$\mu_1-\mu_2=\mu_0$&$\mu_1-\mu_2\neq\mu_0$&\multirow{3}{*}{\makecell{$T=\dfrac{\bar{X}-\bar{Y}-\mu_0}{S_\omega}\sqrt{\dfrac{mn}{m+n}}$\\$\sim t_{m+n-2}$}}&$|T|>t_{m+n-2,\frac{\alpha}{2}}$\\
                &$\mu_1-\mu_2\leq\mu_0$&$\mu_1-\mu_2>\mu_0$&&$T>t_{m+n-2,\alpha}$\\
                &$\mu_1-\mu_2\geq\mu_0$&$\mu_1-\mu_2<\mu_0$&&$T<-t_{m+n-2,\alpha}$\\
                \hline
                \multirow{3}{*}{\makecell{$\mu_1,\mu_2$ known,\\test $\dfrac{\sigma^2_1}{\sigma_2^2}$}}&$\sigma_1^2=\sigma_2^2$&$\sigma_1^2\neq\sigma_2^2$&\multirow{3}{*}{$T=\dfrac{S_{\mu_2}^2}{S_{\mu_1}^2}\sim F_{n,m}$}&\makecell{$T<F_{n,m,1-\frac{\alpha}{2}}$\\$\cup \,T>F_{n,m,\frac{\alpha}{2}}$}\\
                &$\sigma_1^2\geq\sigma_2^2$&$\sigma_1^2<\sigma_2^2$&&$T>F_{n,m,\alpha}$\\
                &$\sigma_1^2\leq\sigma_2^2$&$\sigma_1^2>\sigma_2^2$&&$T<F_{n,m,1-\alpha}$\\
                \hline
                \multirow{3}{*}{\makecell{$\mu_1,\mu_2$ unknown,\\test $\dfrac{\sigma^2_1}{\sigma_2^2}$}}&$\sigma_1^2=\sigma_2^2$&$\sigma_1^2\neq\sigma_2^2$&\multirow{3}{*}{$T=\dfrac{S_{2}^2}{S_{2}^2}\sim F_{n-1,m-1}$}&\makecell{$T<F_{n-1,m-1,1-\frac{\alpha}{2}}$\\$\cup\, T>F_{n-1,m-1,\frac{\alpha}{2}}$}\\
                &$\sigma_1^2\geq\sigma_2^2$&$\sigma_1^2<\sigma_2^2$&&$T>F_{n-1,m-1,\alpha}$\\
                &$\sigma_1^2\leq\sigma_2^2$&$\sigma_1^2>\sigma_2^2$&&$T<F_{n-1,m-1,1-\alpha}$\\
                \hline
        \end{tabularx}
    \end{table}

    \item None normal population:
    
    \begin{table}[htbp]
        \centering
        \renewcommand\arraystretch{1.7}
        \begin{tabularx}{\linewidth}{|c|c|c|Y|c|}
            \hline
            Condition&$H_0$&$H_1$&Testing Statistic $T$&Rejection Region $R$\\
            \hline
            \makecell{$\vec{X}$ from $B(1,p)$, test $p$}&$p=p_0$&$p\neq p_0$&$T=\dfrac{\sqrt{n}(\bar{X}-p_0)}{\sqrt{p_0(1-p_0)}}\xrightarrow[]{\mathscr{L}}N(0,1)$&$|T|>N_\frac{\alpha}{2}$\\
            \hline
            \makecell{$\vec{X}$ from $P(\lambda)$, test $\lambda$}&$\lambda=\lambda_0$&$\lambda\neq \lambda_0$&$T=\dfrac{\sqrt{n}(\bar{X}-\lambda_0)}{\sqrt{\lambda_0}}\xrightarrow[]{\mathscr{L}}N(0,1)$&$|T|>N_\frac{\alpha}{2}$\\
            \hline
        \end{tabularx}
    \end{table}
\end{enumerate}

\subsubsection{Likelihood Ratio Test}\label{SubSectionLRT}
    \index{LRT (Likelihood Ratio Test)}
    Idea: To test $H_0:\theta\in\Theta_0\longleftrightarrow H_1:\theta\in\Theta_1$ known $\vec{x}$, examine the likelihood function $L(\theta;\vec{x})$ and \textbf{compare} $L_{\theta\in\Theta_0}$ and $L_{\theta\in\Theta}$ to see the likelihood that $H_0$ is true.

    Def. \textbf{Likelihood Ratio} (LR):
    \begin{equation}
    \Lambda (\vec{x})=\dfrac{{\displaystyle\sup_{\theta\in\Theta_0}L(\theta;\vec{x})}}{{\displaystyle\sup_{\theta\in\Theta}L(\theta;\vec{x})}}
    \end{equation}

    Reject $H_0$ if $\Lambda(\vec{x})<\Lambda_0$. Or equivalently: Reject $H_0$ if $-2\ln\Lambda(\vec{x})>C(=-2\ln\Lambda_0)$.

    where $\Lambda_0$ (or equivalently $C=-2\ln\Lambda_0$) satisfies:
    \begin{equation}E_{\Theta_0}[\varphi(\vec{X})]\leq\alpha,\quad\forall\theta\in\Theta_0\end{equation}

    LR and sufficient statistic: $\Lambda(\vec{x})$ can be expressed as $\Lambda(\vec{x})=\Lambda^*(T(\vec{x}))$, where $T(\vec{X})$ is sufficient statistic.


\begin{point}
    LRT for one-sample $ t $-test: For $ X_1,X_2,\ldots,X_n $ i.i.d. $ \sim N(\mu,\sigma ^2) $, test

\[
    H_0: \mu=\mu_0\longleftrightarrow H_1:\mu\neq\mu_0\quad\text{when }\sigma ^2\text{ unknown}
\]

    Can prove:
    \[
        \Lambda^{2/n}=\dfrac{\sum\limits_{i=1}^n(x_i-\bar{x})^2}{\sum\limits_{i=1}^n(x_i-\mu_0)^2} 
    \]
    
    Denote $ T=\dfrac{\sqrt{n}(\bar{x}-\mu_0)}{S}$, then LRT is
    \[
        \Lambda = \left( 1+\dfrac{T^2}{n-1} \right)^{-n/2}
    \]
    
    The Multivariate case see sec. \hyperlink{PartHotellingT2Test}{\ref{SubSectionMultivariateHypothesisTesting}}, where $ T^2 $ itself is the Hotelling's $ T^2 $ statistic.
    
    

\end{point}



\begin{point}
    Limiting Distribution of LR: Wilks' Thm.\index{Wilk's Thm.}
\end{point}

    

    
    If $\dim\Theta=k>\dim\mathrm{span}\{\Theta_0\}=s$\footnote{Here 'dimension' refers to 'degree of freedom'.}, then under $H_0:\theta\in\Theta_0$:
    \begin{equation}
        \Lambda_{\theta\in\Theta_0}(\vec{x})=-2\ln \lambda(\vec{x})\xrightarrow[]{\mathscr{L}}\chi_{k-s}^2
    \end{equation}

\subsubsection{Uniformly Most Powerful Test}\label{SUbSectionUMP}
    \index{UMPT (Uniformly Most Powerful Test)}Idea: Neyman-Pearson Principle: control $\alpha$, find $\min\beta$. i.e. control $\alpha$, find $\max\pi(\theta)$

    Def. \textbf{Uniformly Most Powerful Test} (UMP) $\varphi_{\mathrm{UMP}}$ with level of significance $\alpha$ satisfies
    \begin{equation}
        \pi_{\mathrm{UMP}}(\theta)\geq\pi(\theta),\,\forall\theta\in\Theta_1
    \end{equation}

    \textbf{Neyman-Pearson Lemma}\index{NP-Lemma (Neyman-Pearson Lemma)}: For $\vec{X}=(X_1,X_2,\ldots,X_n)$ i.i.d. from $f(\vec{x};\theta)$. 
    
    Test hypothesis $H_0:\theta=\theta_0\longleftrightarrow H_1:\theta=\theta_1$. Def. test function $\varphi$ as:
    \begin{equation}\label{UMPtestfunction}
        \varphi(\vec{x})=\begin{cases}
            1,&\dfrac{f(\vec{x};\theta_1)}{f(\vec{x};\theta_0)}>C\\
            r,&\dfrac{f(\vec{x};\theta_1)}{f(\vec{x};\theta_0)}=C\\
            0,&\dfrac{f(\vec{x};\theta_1)}{f(\vec{x};\theta_0)}<C
        \end{cases}
    \end{equation}

    Then there exists $C$ and $r$ such that
    \begin{itemize}
        \item $E[\varphi(\vec{x})|\theta_0]=P(\dfrac{f(\vec{x};\theta_1)}{f(\vec{x};\theta_0)}>C)+rP(\dfrac{f(\vec{x};\theta_1)}{f(\vec{x};\theta_0)}=C)=\alpha$
        \item This $\varphi$ is UMP of level of significance $\alpha$
    \end{itemize}

    Actually kind of $1$-dimensional case of LRT.

    Note: UMT exist for\textbf{ simple }$H_0,H_1$, otherwise may not exist.

    UMP and sufficient statistics: Test function $\varphi(\vec{X}$ given by eqa.\ref{UMPtestfunction} is function of sufficient statistics $T(\vec{X})$, i.e. $\varphi(\vec{X})=\varphi^*(T(\vec{X}))$.

    UMP and Exponential Family: For sample $\vec{X}=(X_1,X_2,\dots,X_n)$ from exponential family:
    \begin{equation}
    f(\vec{x};\theta)=C(\theta)h(\vec{x})\exp\{Q(\theta)T(\vec{x})\}    
    \end{equation}

    Test single hypothesis $H_0:\theta=\theta_0\longleftrightarrow H_1:\theta=\theta_1$, (where $ \theta_0<\theta_1 $ ).
    If 
    \begin{itemize}[topsep=0.5pt,itemsep=0pt]
        \item $\theta_0$ is inner point of $\Theta$
        \item $Q(\theta)$  monotone increase with $\theta$
    \end{itemize}

    Then UMP exists, in the form of:
    \begin{equation}\label{UMPtestfunctioninExponentialFamily}
            \varphi(\vec{x})=\begin{cases}
        1,&T(\vec{x})>C\\
        r,&T(\vec{x})=C\\
        0,&T(\vec{x})<C
    \end{cases} 
    \end{equation}
   
    

    where $C$ and $r$ satisfies $E[\varphi(\vec{x})|\theta_0]=\alpha$.

    Note: or take $Q(\theta)$ mono decreased, then in eqa.\ref{UMPtestfunctioninExponentialFamily}, take opposite inequality operators.
    
\begin{point}
    \textbf{General Steps of UMP}:
\end{point}

    
    \begin{enumerate}
        \item Find a point $\theta_0\in\Theta_0$ and a point $\theta_1\in\Theta_1$. (Note: \textbf{one} point)
        \item Construct test function in the form of eqa.\hyperref[UMPtestfunction]{\ref{UMPtestfunction}}, use $E[\varphi(\vec{x})|\theta_0]=\alpha$ to determine $C$ and $r$.
        \item Get $R$ and $\varphi(\vec{x})$.
        \item If $\varphi$ does \textbf{not} depend on $\theta_1$, then $H_1$ can be generalized to $H_1:\theta\in\Theta_1$.
        \item If $\varphi$ satisfies $E_{\theta\in\Theta_0}(\varphi)\leq\alpha$, then $H_0$ an be generalized to $H_0:\theta\in\Theta_0$.
    \end{enumerate}

\subsubsection{Duality of Hypothesis Testing and Interval Estimation}

\begin{itemize}
    \item Thm.: $\forall\theta_0\in\Theta$ there exists hypothesis testing $H_0:\theta=\theta_0\longleftrightarrow H_1:\theta\neq\theta_0$ of level $\alpha$ with rejection region $R_{\theta_0}$. Then
    \begin{equation}
        C(\vec{X})=\{\theta:\vec{X}\in R^C_{\theta}\}
    \end{equation}

    is a $1-\alpha$ confidence region for $\theta$

    \item Thm.: $C(\vec{X})$ is a $1-\alpha$ confidence region for $\theta$. Then $\forall\theta_0\in C(\vec{X})$, the rejection region of hypothesis testing $H_0:\theta=\theta_0\longleftrightarrow H_1:\theta\neq\theta_0$ of level $\alpha$ satisfies
    \begin{equation}
    R^C_{\theta_0}=\{\vec{X}:\theta_0\in C(\vec{X})\}
    \end{equation}
\end{itemize}
    
    \begin{point}
        Idea:
    \end{point}
    
        
\begin{itemize}[itemsep=-3pt]
    \item[] \centering $H_0:\theta=\theta_0\longleftrightarrow H_1:\theta\neq\theta_0$
    \begin{equation}\updownarrow\end{equation}
    \item[] \centering $P(R^C(\vec{X})|H_0)=P(R^C(\vec{X})|\theta_0)=1-\alpha$
    \begin{equation}\updownarrow\end{equation}
    \item[] Confidence Interval: $\theta_0\in R^C(\vec{X})$
\end{itemize}

    Similar for Confidence Limit and One-Sided Testing.

\subsubsection{Introduction to Non-Parametric Hypothesis Testing}\label{SubSectionIntroToNonParametricHypothesisTesting}

    Motivation: Usually distribution form unknown, cannot use parametric hypothesis testing.

    Useful Method:
    \begin{itemize}
        \item Sign Test: Used for paired comparison $\vec{X}=(X_1,X_2,\ldots,X_n$, $\vec{Y}=(Y_1,Y_2,\ldots,Y_n)$.
        
        Take $Z_i=Y_i-X_i$ i.i.d., denote $E(Z)=\mu$. Test $H_0:\mu=0\longleftrightarrow H_1:\mu\neq 0$.

        Denote $n_+=\#(\text{positive } Z_i)$ and $n_-=\#(\text{negative }Z_i)$, $n_0=n_++n_-$. Then $n_+\sim B(n_0,\theta)$, test $H_0:\theta=\dfrac{1}{2}\longleftrightarrow H_1:\theta\neq\dfrac{1}{2}$
        
        Then use Binomial Testing or large sample CLT Normal Testing.

        Remark:
        \begin{itemize}
            \item Also can test $H_0:\theta\leq\dfrac{1}{2}\longleftrightarrow H_1:\theta>\dfrac{1}{2}$
            \item Drawback: ignores magnitudes.
        \end{itemize}
        
        \item \index{WSRT (Wilcoxon Signed Rank Sum Test)}Wilcoxon Signed Rank Sum Test: Improvement of Sign Test. Base on order statistics.
        
        Order Statistics of $Z_i$: $Z_{(1)}<Z_{(2)}<\ldots<Z_{(n)}$, where each $Z_{(j)}$ corresponds to some $Z_i$, denote as $Z_i=Z_{(R_i)}$, then $R_i$ is the rank of $Z_i$.\footnote{If some $X_i,X_j,\ldots$ equal, then take same rank $R=\mathrm{mean}\{R_i,R_j,\ldots\}$.}
        
        Def. $\vec{R}=(R_1,R_2,\ldots,R_n)$ is \textbf{Rank Statistics} of $(Z_1,Z_2,\ldots,Z_n)$

        Def. \textbf{Sum of Wilcoxon Signed Rank}: 
        \begin{equation}
        W^+=\sum_{i=1}^{n_0}R_iI_{Z_i>0} 
        \end{equation}

        Distribution of $W^+$ is complex. $E$ and $var$ of $W^+$ under $H_0$:
        \begin{equation}
        E(W^+)=\frac{n_0(n_0+1)}{4}\qquad var(W^+)=\frac{n_0(n_0+1)(2n_0+1)}{24}    
        \end{equation}

        Usually consider large sample CLT, construct normal approximation:
        \begin{equation}
            T=\frac{W^+-E(W^+)}{\sqrt{var(W^+)}}\xrightarrow[]{\mathscr{L}}N(0,1)
        \end{equation}

        Rejection Region: $R=\{|T|>N_\frac{\alpha}{2}\}$

        \item Wilcoxon Two-Sample Rank Sum Test: Used for two independent sample comparison.
        
        Assume $\vec{X}=(X_1,\ldots,X_m)$ i.i.d. $\sim f(x)$; $\vec{Y}=(Y_1,\ldots,Y_n)$ i.i.d. $\sim f(x-\theta)$, test $H_0:\theta=0\longleftrightarrow H_1:\theta\neq 0$.

        Rank $X_i$ and $Y_i$ as:
        \begin{equation}
            Z_1\leq Z_2\leq\ldots\leq Z_{m+n}
        \end{equation}

        in which denote rank of $Y_i$ as $R_i$, and def. \textbf{Wilcoxon two-sample rank sum}:
        \begin{equation}W=\sum_{i=1}^n R_i\end{equation}

        $E$ and $var$ of $W$ under $H_0$:
\begin{equation}E(W)=\frac{n(m+n+1)}{2}\qquad var(W)=\frac{mn(n+m+1)}{12}\end{equation}

        Use large sample approximation, construct CLT:
        \begin{equation}
            T=\frac{W-E(W)}{\sqrt{var(W)}}\xrightarrow[]{\mathscr{L}}N(0,1)
        \end{equation}







        \item Goodness-of-Fit Test: For $\vec{X}=(X_1,X_2,\ldots,X_n)$ i.i.d. from some certain population $X$. Test $H_0:X\sim F(x)$.
        
        where $F$ is theoretical distribution, can be either parametric or non-parametric.

        Idea: Define some \textit{quantity} $D=D(X_1,\ldots,X_n;F)$ to measure the difference between $F$ and sample. And def. \textit{Goodness-of-fit} when observed value of $D$ (say $d_0$) is given:
        \begin{equation}p(d_0)=P(D\geq d_0|H_0)\end{equation}

        \textbf{Goodness-of-Fit Test}: Reject $H_0$ if $p(d_0)<\alpha$.


            Pearson $\chi^2$ Test: Usually used for discrete case. 
            
            Test $H_0:P(X_i=a_i)=p_i,\, i=1,2,\ldots,r$. Denote $\#(X_j=a_i)=\nu_i$, take $D$ as:
            \begin{equation}\label{Pearson_chi_test_differenceKn}
                K_n=K_n(X_1,\ldots,X_n;F)=\sum_{i=1}^r\frac{(\nu_i-np_i)^2}{np_i}
            \end{equation}

            Pearson Thm.: For $K_n$ defined as eqa.\ref{Pearson_chi_test_differenceKn}, then under $H_0$:
            \begin{equation}
                K_n\xrightarrow[]{\mathscr{L}}\chi^2_{r-1-s}
            \end{equation} 

            Here $s$ is number of unknown parameter, $r-1-s$ is the degree of freedom.

            Note:
            \begin{itemize}
                \item $a_i$ must \textbf{not} depend on sample.
                \item For continuous case, construct division:
                \begin{equation}\mathbb{R}\rightarrow(-\infty,a_1,a_2,\ldots,a_{r-1},\infty=a_r) \end{equation}

                and test $H_0:P(X\in I_j)=p_j$

                Criterion: Pick proper interval so that $np_i$ and $\nu_i$ both $\geq 5$.
            \end{itemize}
 


        \item Contingency Table Independence \& Homogeneity Test
        \index{Contingency Table}
 
\begin{itemize}
    \item Independence Test:
    
    Test a two-parameter sample and to see whether these two parameters(features) are independent. Denote $Z=(X,Y)$ are some 'level' of sample, $n_{ij}$ is number of sample with level $(i,j)$

    Contingency Table:
    \begin{table}[H]
        \centering
        \begin{tabular}{|c|ccccc|c|}
            \hline
            \diagbox{X}{Y}&1&$\ldots$&$j$&$\ldots$&$s$&$\sum$\\
            \hline
            1&$n_{11}$&$\ldots$&$n_{1j}$&$\ldots$&$n_{1s}$&$n_{1\cdot}$\\
            $\vdots$&$\vdots$&$\ddots$&$\vdots$&$\ddots$&$\vdots$&$\vdots$\\
            $i$&$n_{i1}$&$\ldots$&$n_{ij}$&$\ldots$&$n_{is}$&$n_{i\cdot}$\\
            $\vdots$&$\vdots$&$\ddots$&$\vdots$&$\ddots$&$\vdots$&$\vdots$\\
            $r$&$n_{r1}$&$\ldots$&$n_{rj}$&$\ldots$&$n_{rs}$&$n_{r\cdot}$\\
            \hline
            $\sum$&$n_{\cdot 1}$&$\ldots$&$n_{\cdot j}$&$\ldots$&$n_{\cdot s}$&$n$\\
            \hline
        \end{tabular}
    \end{table}

        Test $H_0:X\,\&\, Y$ are independent. i.e. $H_0:P(X=i,Y=j)=P(X=i)P(Y=j)=p_{i\cdot}p_{\cdot j}$.

        Construct $\chi^2$ test statistic:
        \begin{equation}
            K_n=\sum_{i=1}^r\sum_{j=1}^s\frac{[n_{ij}-n(\frac{n_{i\cdot}}{n})(\frac{n_{\cdot j}}{n})]^2}{n(\frac{n_{i\cdot}}{n})(\frac{n_{\cdot j}}{n})}=n\left(\sum_{i=1}^r\sum_{j=1}^s\frac{n_{ij}^2}{n_{i\cdot}n_{\cdot j}}-1\right)
        \end{equation}

        Then under $H_0$, $K_n\xrightarrow[]{\mathscr{L}}\chi^2_{rs-1-(r+s-2)}=\chi^2_{(r-1)(s-1)}$

        Reject $H_0$ if $p(k_0)=P(K_n\geq k_0)<\alpha$


        \item Homogeneity Test:
        
        Test $R$ groups of sample with category rank, to see whether these groups has similar rank distribution.

        \begin{table}[H]
            \centering
            \begin{tabular}{|c|ccccc|c|}
                \hline
                \diagbox{Group}{Category}&Category 1&$\ldots$&Category $j$&$\ldots$&Category $C$&$\sum$\\
                \hline
                Group 1&$n_{11}$&$\ldots$&$n_{1j}$&$\ldots$&$n_{1C}$&$n_{1\cdot}$\\
                $\vdots$&$\vdots$&$\ddots$&$\vdots$&$\ddots$&$\vdots$&$\vdots$\\
                Group $i$&$n_{i1}$&$\ldots$&$n_{ij}$&$\ldots$&$n_{iC}$&$n_{i\cdot}$\\
                $\vdots$&$\vdots$&$\ddots$&$\vdots$&$\ddots$&$\vdots$&$\vdots$\\
                Group $R$&$n_{R1}$&$\ldots$&$n_{Rj}$&$\ldots$&$n_{RC}$&$n_{R\cdot}$\\
                \hline
                $\sum$&$n_{\cdot 1}$&$\ldots$&$n_{\cdot j}$&$\ldots$&$n_{\cdot C}$&$n$\\
                \hline
            \end{tabular}
        \end{table}


    Denote $P(\text{Category }j|\text{Group }i)=p_{ij}$. Test $H_0:p_{ij}=p_j,\,\forall 1\leq i\leq R$.

    Construct $\chi^2$ test statistic:
    \begin{equation}
        D=\sum_{i=1}^R\sum_{j=1}^C\frac{[n_{ij}-n(\frac{n_{i\cdot}}{n})(\frac{n_{\cdot j}}{n})]^2}{n(\frac{n_{i\cdot}}{n})(\frac{n_{\cdot j}}{n})}=n\left(\sum_{i=1}^R\sum_{j=1}^C\frac{n_{ij}^2}{n_{i\cdot}n_{\cdot j}}-1\right)
    \end{equation}

    Then under $H_0$, $D\xrightarrow[]{\mathscr{L}}\chi^2_{R(C-1)-(C-1)}=\chi^2_{(R-1)(C-1)}$
    \end{itemize}

    \item \hypertarget{testofnormality}{Test of Normality}: normality is a good \& useful assumption.
    
    For $\vec{Y}=(Y_1,Y_2,\ldots,Y_n)$,

    Test $H_0:\text{exists }\mu\,\&\, \sigma^2$ such that $Y_i$ i.i.d. $\sim N(\mu,\sigma^2)$.

    \begin{itemize}
        \item Kolmogorov-Smirnov Test\index{K-S Test (Kolmogorov-Smirnov Test)}: Assume $\vec{X}$ form population CDF $F(x)$, test $H_0:F(x)=F_0(x)$(where can take $F_0=\Phi$ or some other known CDF).
        
        use $F_n(x)$ (as defined in eqa.\ref{empiricaldisreibutionfunction}) as approx. to $F(x)$, test
        \begin{equation}
            D_n=\sum_{-\infty< x<+\infty}|F_n(x)-F_0(x)|
        \end{equation}

        Reject $H_0$ if $D_n>c$

        or use goodness-of-fit: denote observed value of $D_n$ as $d_n$. Reject $H_0$ if
        \begin{equation}
            p(d_n)=P(D_n>d_n|H_0)<\alpha
        \end{equation}

        \item Shapiro-Wilk Test:\index{S-W Test (Shapiro-Wilk Test)}
        
        Test $H_0:\text{exists }\mu\,\&\, \sigma^2$ such that $X_i$ i.i.d. $\sim N(\mu,\sigma^2)$.

        Denote $Y_{(i)}=\dfrac{X_{(i)}-\mu}{\sigma}$, $m_i=E(Y_{(i)})$

        Under $H_0$, $(X_{(i)},m_i)$ falls close to straight line. Test Statistic: Correlation
        \begin{equation}
            R^2=\dfrac{\left(\sum_{i=1}^n(X_{(i)}-\bar{X})(m_i-\bar{m})\right)^2}{\sum_{i=1}^n(X_{i}-\bar{X})^2\sum_{i=1}^n(m_i-\bar{m})^2}=corr(X_{(i)},m_i)
        \end{equation}

        Reject $H_0$ if $R^2<c$

        Shapiro-Wilk correction:
        \begin{equation}
            W=\dfrac{\left(\sum_{i=1}^{[n/2]}a_i(X_{(n+1-i)}-X_{(i)})\right)^2}{\sum_{i=1}^n(X_{(i)}-\bar{X})^2}
        \end{equation}
    \end{itemize}
\end{itemize}

\begin{point}
    Summary: Useful Non-Parameter Hypothesis Testing.
\end{point}
\\
\\

\begin{equation*}
    \text{\makecell{Non-Parameter\\Hypothesis Testing}}
    \smash[htbp]{
    \begin{cases}
        \text{One Population Sample}
            \smash[t]{
                \begin{cases}
                    \chi^2\text{ Test}\\
                    \text{Binomial Test}\\
                    \text{One-Sample K-S Test}\\
                    \text{Wilcoxon Sign Test}\\
                    \text{Runs Test}
                \end{cases}
            }\\
            \\
            \\
            \\
            \\
        \text{Two Population Sample}
            \smash[t]{
                \begin{cases}
                    \text{Independent Sample}
                    \smash[t]{
                        \begin{cases}
                            \text{Mann-Whitney Test}\\
                            \text{K-S Test}\\
                            \text{Wald-Wolfowitz Test}\\
                            \text{Moses Test of Extreme Reactions}
                        \end{cases}
                    }\\
                    \\
                    \text{Relative Sample}
                    \smash[b]{
                        \begin{cases}
                            \text{Sign Test}\\
                            \text{McNemar Test}\\
                            \text{Wilcoxon Rank Sum Test}\\
                            \text{Marginal Homogeneity Test}
                        \end{cases}
                    }
                \end{cases}
            }\\
            \\
            \\
            \\
        \text{Multi-Population Sample}
            \smash[b]{
                \begin{cases}
                    \text{Independent Sample}
                    \smash[t]{
                        \begin{cases}
                            \text{Median Test}\\
                            \text{K-W One-Way ANOVA Test}\\
                            \text{Jonckheere-Terpstra Test}
                        \end{cases}
                    }\\
                    \\
                    \text{Relative Sample}
                    \smash[b]{
                        \begin{cases}
                            \text{Friedman Rank Sum Test}\\
                            \text{Kendall's Coefficient of Concordance Test}\\
                            \text{Cochran Q Test}
                        \end{cases}
                    }
                \end{cases}
            }
    \end{cases}  
    }
\end{equation*}

\newpage

 \include{sections/R}
 \newpage

\section{线性回归分析部分}\label{SecLinearRegressionAnalysis}
\begin{center}
    Instructor: Zaiying Zhou
\end{center}
\begin{point}
    Steps in Regression Analysis
\end{point}

\begin{enumerate}[topsep=2pt,itemsep=2pt]
    \item Statement of the problem;
    \item Selection of potentially relevant \textbf{variables};
    \item Data collection;
    \item Exploratory Data Analysis (\textbf{EDA} )\index{EDA (Exploratory Data Analysis)}
    \item \textbf{Model} specification;
    \item Choice of fitting method;
    \item Model fitting;
    \item Model validation and criticism;
    \item Using the chosen model(s) for the solution of the posed problem;
    \item \textbf{Explain} the result.
\end{enumerate}

    \lstinline|R.| Code for EDA
\begin{lstlisting}[language=R]
libaray('GGally')
head(df)
ggpairs(df)
str(df)
summary(df)
\end{lstlisting}



\begin{point}
    Used Packages in \lstinline|R.|
\begin{lstlisting}[language=R]
library('ggplot2')
libaray('GGally')
library('car')
library('moments')
library('lmtest')
library('nortest')
library('MASS')
library('tseries')

source('package.r')
\end{lstlisting}

\end{point}


\subsection{Linear Regression Model}
% \begin{itemize}[topsep=6pt,itemsep=4pt]
%     \item Assume a Model
%     \begin{enumerate}[topsep=6pt,itemsep=4pt]
%         \item Parameter of the model
%         \item Basic Assumptions
%         \item Dsitribution of error
%     \end{enumerate}
%     \item Parametric Estimation
%     \begin{enumerate}[topsep=6pt,itemsep=4pt]
%         \item Ordinary Least Squares Estimation
%         \item Maximun Likelihood Estimation
%     \end{enumerate}
%     \item Statistics Inference
%     \begin{enumerate}[topsep=6pt,itemsep=4pt]
%         \item Hypotheses Testing
%         \item Interval Estimation
%     \end{enumerate}

% \end{itemize}


    


        
\subsubsection{Data and Model for Simple Linear Regression}

    We will observe pairs of variables, called 'cases'(样本点). A sample is $ (X_1,Y_1),\ldots,(X_n,Y_n) $

\begin{rcode}
    Example data import:
\begin{lstlisting}[language=R]
df <- read.table('dataset/CH01PR27.txt',header=FALSE,
    sep=',',col.names = c('y','x'))
\end{lstlisting}
\end{rcode}

    Linear Model: \footnote{Here in linear regression, we consider $ X_i $ only as real number, \textbf{without} randomness. So here $ Y_i $ can be considered as an r.v. with $ X_i $ as parameter, i.e. $ Y_i|_{X_i=x_i} $}
    \footnote{Note: Why we need $ \varepsilon $ as 'random error term'?
    \begin{itemize}[topsep=6pt,itemsep=4pt]
        \item It represents the intrinsic random property of the model.
        \item Based on $ \varepsilon  $, we can take r.v. into our statistic model.
    \end{itemize}
    }
    
%% 关于线性模型的X_i性质的假定
    \begin{equation}
        Y_i=\beta _0+\beta _1X_i+\varepsilon _i 
    \end{equation}

    with Guass-Markov Assumption:
    \begin{equation}\label{EqaGaussMarkovAssumption}
        \begin{aligned}
            \text{Zero-Mean: }&E(\epsilon_i|X_i)=0 \\
            \text{Homogeneity of Variance: }&var(\epsilon_i)=\sigma^2\\
            \text{Independent: }&\epsilon_i\text{ i.i.d. }\sim \varepsilon
        \end{aligned}
    \end{equation}
 
  

    Normal Error Assumption: Further in most cases, we consider $ \varepsilon \sim N(0,\sigma^2) $ ----because of its well-property distribution, $ \varepsilon _1,\varepsilon _2,\ldots,\varepsilon _n $ i.i.d. $ N(0,\sigma ^2) $.\footnote{i.e. $ Y_i $ are independent
    \begin{equation}
        Y_i\sim N(\beta _0+\beta _1X_i,\sigma^2)\quad i=1,2,\ldots ,n 
    \end{equation}
    
    }
        
    What does Linear Regression do? Under Linear Model, try to estimate 
    \begin{itemize}[topsep=0pt,itemsep=-2pt]
        \item $ \beta _0\text{ (intercept) }$;
        \item $\beta _1\text{ (slope) }$;
        \item $\sigma ^2\text{ (variance of error)} $.
    \end{itemize}
    
    
    (Thus Linear Regression is also a Statistics Inference process: deduce properties of model from data)
        
\subsubsection{The Ordinary Least Square Estimation}
    Aim: use $ (x_i,y_i) $  to estimate $ \beta _0,\beta _1,\sigma^2 $. The idea is to define a 'loss function' to reflect the 'distance' from sample point to estimation point.

    Estimate Principle: \footnote{Detailed Definition and Derivation see sec.\ref{SubSectionMoM_MLE_LinearRegression}.}
    \begin{itemize}[topsep=2pt,itemsep=2pt]
        \item Ordinary Least Squares\index{Ordinary Least Squares}:
        \begin{equation}
            (\hat{\beta  }_0,\hat{\beta _1})=\arg\min\sum_{i=1}^n (y-\beta _0-\beta _1x_i)^2
        \end{equation}
        \item MLE or MoM Estimation.
    \end{itemize}
    

    
    And get $ \hat{\beta _1},\hat{\beta _0}$ as well as $ \hat{\sigma^2} $(see eqa(\ref{EqaOLSEstimatorOfSigma}):\footnote{A memory trick: use $ \dfrac{Y}{\sqrt{s_Y}}=r_{XY}\dfrac{X}{\sqrt{s_X}} $ to get formular of $ Y\sim X $:
    \begin{equation}
        \hat{\beta }_1=r_{XY}\dfrac{\sqrt{s_Y}}{\sqrt{s_X}}=\dfrac{{\displaystyle\sum (x_i-\bar{x})(y_i-\bar{y})}}{{\displaystyle\sum (x_i-\bar{x})^2}} 
    \end{equation}}

%LSE beta_0 beta_1
\begin{equation}\label{EqaOLSEstimatorOfBeta}
    \begin{aligned}
        \hat{\beta }_1=&\dfrac{\sum\limits_{i=1}^n (x_i-\bar{x})(y_i-\bar{y})}{\sum\limits_{i=1}^n (x_i-\bar{x})^2}\\
        \hat{\beta }_0=&\bar{y}-\hat{\beta _1}\bar{x}\\
        \hat{\sigma^2}=&\dfrac{1}{n-p-1}\sum_{i=1}^n(y_i-\hat{\beta }_0-\hat{\beta }_1x_i)^2
    \end{aligned}
\end{equation}


    
    Def. \index{Residual}\textbf{Residual}: distance from sample point to estimate point, to reflect how the sample points fit the model.
    \begin{equation}
        e_i=y_i-\hat{y}_i=\text{observed value of }\varepsilon _i 
    \end{equation}
    
    Note: under least square estimation, we have\footnote{Intuitively, they each means '$ E(\varepsilon )=0 $' and '$ X\parallel \varepsilon  $'.}
\begin{equation}\label{Limit_to_Residual}
        \sum e_i=0\qquad \sum x_ie_i=0 
\end{equation}
    

    Then use $ e_i $ to estimate $ \sigma ^2 $ (because it is $ \varepsilon _0 $ that are i.i.d., not $ Y_i $), where $ (n-p-1) $ is Degree of Freedom (df or dof)\footnote{Generally, MLE and LSE are different.

    Comment from R.A.Fisher: $ \sum e_i^2 $ should be divided by 'number of $ e_i^2 $ that contribute to variance'. Here $ (n-p-1) $ corresponds to 'degree of freedom' $ =(n-2) $, $ p=1 $ corresponds to `one' variable (see sec.\ref{SubSectionMoM_MLE_LinearRegression}, eqa(\ref{EqaEstimatorSigmaWithDoF})), and correponds to the two equations of $ e_i $, eqa(\ref{Limit_to_Residual})}
\begin{equation}\label{EqaOLSEstimatorOfSigma}
    \begin{aligned}
        \hat{\sigma _n^2}&=\dfrac{1}{n}\sum e_i^2 \quad\text{(use MLE or MoM)}\\
        \hat{\sigma^2}&=\dfrac{1}{n-p-1}\sum e_i^2=\dfrac{1}{n-2}\sum e_i^2\quad\text{(use OLS, unbiased)}
\end{aligned}
\end{equation}

\textbf{Degree of Freedom}\index{$ dof $/$ df $ (Degree of Freedom)} of a Quadric Form:
\begin{itemize}[topsep=2pt,itemsep=2pt]
    \item Intuitively: the number of independent variable;
    \item Rigorously: for quadric $ \mathrm{SS}=x'Ax $:
    \begin{equation}\label{EqaDefinitionOfDegreeOfFreedom}
        dof_{SS}=\mathrm{rank}(A)
    \end{equation}
    
    
    
\end{itemize}

     

\begin{rcode}
\begin{lstlisting}[language=R]
lmfit <- lm(formula,df)
summary(lmfit,cor=TRUE)
ggcoef(lmfit)
\end{lstlisting}

    \lstinline|lmfit| includes parameters \lstinline|lmfit$coefficient| and \lstinline|lmfit$residuals|

    Example \lstinline|lm()| output:
\begin{lstlisting}[language=R]
    Call:
    lm(formula = y ~ x, data = df)
    
    Residuals:
         Min       1Q   Median       3Q      Max 
    -16.1368  -6.1968  -0.5969   6.7607  23.4731 
    
    Coefficients:
                Estimate Std. Error t value Pr(>|t|)    
    (Intercept) 156.3466     5.5123   28.36   <2e-16 ***
    x            -1.1900     0.0902  -13.19   <2e-16 ***
    ---
    Signif. codes:  0 '***' 0.001 '**' 0.01 '*' 0.05 '.' 0.1 ' ' 1
    
    Residual standard error: 8.173 on 58 degrees of freedom
    Multiple R-squared:  0.7501,    Adjusted R-squared:  0.7458 
    F-statistic: 174.1 on 1 and 58 DF,  p-value: < 2.2e-16
\end{lstlisting}

\end{rcode}

    % MSE SSE dof






    % Review: Statistical Inference
    % \begin{itemize}[topsep=6pt,itemsep=4pt]
    %     \item Basic concepts: HT CI;
    %     \item Inference about $ \beta _1$;
    %     \item Inference about $ \beta _0 $.
    % \end{itemize}

    % Note: the distribution of $ \hat{\beta }_0,\hat{\beta }_1 $ is sampling distribution(抽样分布): distribution of statistics.



    % Power function of testing
    % \begin{itemize}[topsep=6pt,itemsep=4pt]
    %     \item Definition;
    %     \item Calculation;
    %     \item Sample<->Power (Calculation of sampling).
    % \end{itemize}


\subsubsection{Statistical Inference to $ \beta _0 $,$ \beta _1 $,$ e_i $}

\begin{point}
    Sampling Distribution of $ \hat{\beta} _1,\hat{\beta} _0  $
\end{point}

    Consider $ \hat{\beta} _1,\hat{\beta} _0 $ as statistics of sample, then we can examine the sampling distribution of $  \hat{\beta} _1,\hat{\beta} _0 $. Their randomness comes from
    \begin{equation}
        Y_i=\beta_0+\beta_1X_i+\varepsilon _i 
    \end{equation}
    
    

    (The following part treats $\hat{\beta} _1,\hat{\beta} _0 $ as r.v., and note that $ X_i $ are \textbf{not }r.v.. And  for convenience and conciseness, denote $ S_{XX}={\displaystyle\sum_{i=1}^n(X_i-\bar{X})^2} $)

   
\begin{align*}
        \hat{\beta }_1&=\beta _1+\sum_{i=1}^n\dfrac{X_i-\bar{X}}{S_{XX}}\varepsilon _i\\
        \hat{\beta }_0&=\beta _0+\sum_{i=1}^n\left(\dfrac{1}{n}-\dfrac{(X_i-\bar{X})\bar{X}}{S_{XX}}\right)\varepsilon _i
\end{align*}
 
    Denote corresponding variance as $ \sigma^2_{\hat{\beta}_1} $ and $ \sigma^2_{\hat{\beta}_0} $, using eqa(\ref{EqaDistributionOfSumOfiidNormal}) to get:
    \begin{equation}
        \sigma^2_{\hat{\beta}_1}= \dfrac{\sigma^2}{S_{XX}}\qquad \sigma^2_{\hat{\beta}_0}=\sigma^2(\dfrac{1}{n}+\dfrac{\bar{X}^2}{S_{XX}})
    \end{equation} 
    
     And under normal error assumption, distribution of $ \hat{\beta} _1,\hat{\beta} _0  $ are
    \begin{align*}
        \hat{\beta }_1&\sim N(\beta _1,\sigma^2_{\hat{\beta}_1}) =N(\beta_1,\dfrac{\sigma^2}{S_{XX}})\\
        \hat{\beta}_0&\sim N(\beta_0,\sigma^2_{\hat{\beta }_0}) =N(\beta_0,\sigma^2(\dfrac{1}{n}+\dfrac{\bar{X}^2}{S_{XX}}))
    \end{align*}
    
    Based on sampling distribution of $ \hat{\beta} _1,\hat{\beta} _0  $, we can conduct statistical inference, including CI and HT.\footnote{Detail see sec.\ref{SectionHypothesisTesting}, estimating/testing $ \hat{\beta} _1,\hat{\beta} _0  $ usually corresponds to 'estimate $ \mu $, with $ \sigma^2 $ unknown'.}
    
    % \begin{itemize}[topsep=2pt,itemsep=2pt]
    %     \item LSE of $ \beta _1 $ gives 
    %     \begin{equation}
    %         \hat{\beta _1}=\dfrac{\sum (x_i-\bar{x})(y_i-\bar{y})}{\sum (x_i-\bar{x})^2}
    %     \end{equation}
        
    %     and satisfies $ E(\hat{\beta}_1)=\beta_1 $. Can prove that $ \hat{\beta }_1\sim N(\beta _1,\dfrac{\sigma ^2}{\sum (x_i-\bar{x})^2})=N(\beta_1,\sigma^2(\hat{\beta}_1))$
       
    % \end{itemize}
    
    Note: In linear regression model, we usually focus more on $ \beta_1 $. And note that when $ 0 $ is \textbf{not} within the fitting range,$ \beta_0 $ is not so important.\footnote{Two reason:\begin{itemize}[topsep=2pt,itemsep=2pt]
        \item The etimation error of $ Y $ from $ \hat{\beta}_1 $ increases with $ X_h-\bar{X} $;
        \item $ \beta_1==0  $ is important: decides whether linear model can be used. 
    \end{itemize}}


\begin{point}
    Sampling Distribution of $ e_i $ 
\end{point}
    Consider $ e_i $ as r.v. satisfies
    \begin{equation}
        e_i= Y_i-\hat{Y}_i=Y_i-\hat{\beta }_0-\hat{\beta }_1X_i
    \end{equation}

    and get the expression of $ \hat{e}_i $
    \begin{equation}
        \begin{aligned}
            \hat{e}_i=\varepsilon _i-\sum_{k=1}^n\left( \dfrac{1}{n}+\dfrac{(X_i-\bar{X})^2}{S_{XX}} \right)\varepsilon _k
        \end{aligned}
    \end{equation}
    
    
    \begin{equation}
        E(e_i)=0\qquad \sigma ^2_{e_i}=\sigma ^2 \left( 1-\dfrac{1}{n}-\dfrac{(X_i-\bar{X})^2}{S_{XX}} \right)
    \end{equation}

    Under normal assumption:
    \begin{equation}\label{EqaSamplingDistributionOfResiduals}
        e_i\sim N(0,\sigma ^2\left( 1-\dfrac{1}{n}-\dfrac{(X_i-\bar{X})^2}{S_{XX}} \right) ) 
    \end{equation}
    

    Further we can get $ \hat{\sigma }^2=E(\dfrac{1}{n-2}\sum_{i=1}^ne_i^2) $ where $ e_i^2\sim \sigma ^2\left( 1-\dfrac{1}{n}-\dfrac{(X_i-\bar{X})^2}{S_{XX}} \right)\chi^2 $
    \begin{equation}
        \hat{\sigma }^2=\dfrac{1}{n-2}\sigma ^2\sum_{i=1}^n(1-\dfrac{1}{n}-\dfrac{(X_i-\bar{X})^2}{S_{XX}})=\sigma ^2
    \end{equation}
    
    More definition of refined residuals see sec.\ref{SubSecDiagnostics} in page \ref{SubSecDiagnostics}.
    


\begin{point}
    Why we choose OLS to get regression coefficients?


    \index{Gauss-Markov Thm.`'}Gauss–Markov Thm.: the OLS estimator has the lowest sampling variance within the class of linear unbiased estimators, i.e. OLS is the Best Linear Unbiased Estimator(BLUE).\footnote{This Thm. does \textbf{not }require normal error assumption.}
\end{point}
    



\subsubsection{Prediction to $ Y_h $}
    For a new $ X_h $ at which we wish to \textbf{predict }the corresponding $ Y_h $ (based on other known point $ (X_i,Y_i) $), denote the estimator as $ \hat{\mu}_h $:
    \begin{equation}
        \hat{\mu}_h=\hat{\beta}_1X_h+\hat{\beta}_0 =\beta_1X_h+\beta _0+\sum_{i=1}^n\left( \dfrac{1}{n}+\dfrac{(X_i-\bar{X})(X_h-\bar{X})}{S_{XX}} \right)\varepsilon _i
    \end{equation}
    
    Thus we can get\footnote{So $ \sigma ^2(\hat{\mu }_h) $ increases with $ X_h-\bar{X} $. Intuitively it make sense, because $ (\bar{X},\bar{Y})$ must falls on regression line.}
    \begin{equation}
        E(\hat{\mu}_h)= \beta _1X_h+\beta _0\qquad \sigma ^2_{\hat{\mu}_h}=\left( \dfrac{1}{n}+\dfrac{(X_h-\bar{X})^2}{S_{XX}} \right)\sigma^2
    \end{equation}
    
    Under Normal assumption:
    \begin{equation}
        \hat{\mu}_h\sim N(\beta _1X_h+\beta _0,\left( \dfrac{1}{n}+\dfrac{(X_h-\bar{X})^2}{S_{XX}} \right)\sigma^2) 
    \end{equation}
    
    Base on distribution we can give CI and HT.

    Note: We can either consider 
    \begin{itemize}[topsep=2pt,itemsep=2pt]
        \item \textbf{$ Y_h $ itself as an r.v. }: Confidence Interval of $ Y_h $;
        
        And we can just use $  \sigma ^2_{\hat{\mu}_h} $ to construct CI;

        \begin{rcode}
\begin{lstlisting}[language=R]
predict(lmfit,data.frame(x=c(df$x,40)),
    interval="confidence",level=0.95)
\end{lstlisting}
        \end{rcode}
        \item \textbf{predicted $ Y_h $ from other sample points}: Prediction Interval of $ Y_h $
        
        Need to  have the randomness of $ \hat{\beta }_0,\hat{\beta }_1 $ considered(if they are unknown).

        Def. Prediction Error: $ Y_h $ itself is an $ Y $ of the linear model, i.e. $ Y_i=\beta_0+\beta_1X_h+\varepsilon _h $, we can  and define \textbf{Prediction Error}: 
        \begin{equation}
            d_h=Y_h-\hat{\mu}_h 
        \end{equation}
    
        
        \begin{equation}
            E(d_h)=0\qquad \sigma^2_{d_h}=var(Y_h-\hat{\mu }_h)=\sigma^2\left[ 1+\dfrac{1}{n}+\dfrac{(X_h-\bar{X})}{S_{XX}} \right] > \sigma ^2_{\hat{\mu}_h}
        \end{equation}
\begin{rcode}
\begin{lstlisting}[language=R]
predict(lmfit,data.frame(x=c(df$x,40)),
    interval="prediction",level=0.95)
\end{lstlisting}
\end{rcode}
    
    \end{itemize}
    
    % Remember that when we consider the estimator $ \hat{\mu } $, we \textbf{must } have the randomness of $ \hat{\beta }_0,\hat{\beta }_1 $ considered(if they are unknown).
    


    \begin{point}
       Simultaneous Confidence Band(SCB)\index{SCB (Simultaneous Confidence Band)}\index{CB (Confidence Band)}
    \end{point}

    Confidence Band is \textbf{not} the CI at each point, but really a \textbf{band} for the \textbf{entire} regression line.\footnote{Why they are different? We require the confidence band have a \textbf{simultaneous} converage probability. For the same band $ (L(x),U(x)) $, $ P(\text{the whole line})< P(\text{each point})$, so Confidence Band is wider than $ \bigcup $CIs to hold the same $ 1-\alpha $.
    
    Also, we will see that for linear model, the boundary of SCB forms hyperbola, which make sense considering its asymptotic line.}
    
    
    Aim: Find lower and upper function $ L(x) $ and $ U(x) $ such that
    \begin{equation}
        P[L(x)<(\beta _0+\beta _1x)<U(x),\,\forall x\in I_x]=1-\alpha  
    \end{equation}
    
    and get \textbf{Confidence Band}:
    \begin{equation}
        \{(x,y)|L(x)<y<U(x)|\forall x\in I_x\} 
    \end{equation}
    
    % Note: \textbf{Cannot} use CI at each point to form Confidence Band. Band is wider. And we are actually conduce CI \textbf{simoutanesly} to all $ x $.

    Where $ (L(x),U(x)) $ can be derived as
    \begin{equation}
        (L(x),U(x))=\hat{\mu}_x\pm s_{\hat{\mu}_x}W_{2,n-2,1-\alpha}
    \end{equation}

    Where $ W $ correponds to $ W $ distribution: $ W_{m,n}=\sqrt{2F_{m,n}} $
    
    
    
    Small sample case: Bonferroni correction.
    
\begin{rcode}
\begin{lstlisting}[language=R]
library(ggplot2)
ggplot(df,aes(x,y))+geom_point()+geom_smooth(method='lm',formula=y~x)
\end{lstlisting}
\end{rcode}



% 
% 
% 
% 
% 
% 
% 


\subsection{Analysis of Variance}
    \index{ANOVA (Analysis of Variance)}\textbf{AN}alysis \textbf{O}f \textbf{VA}riance (ANOVA): \hyperlink{OneSampletTest}{One-sample $ t $ test} $\rightsquigarrow $ \hyperlink{TwoSampletTest}{Two sample $ t $ test} $\rightsquigarrow $ ANOVA

\begin{point}
    \textbf{Key Point Of ANOVA}: Take Partition of Total Sum of Square To Examine \textbf{Variation}.  

    Because $ Y_i $ are not i.i.d. (different mean), so has different parts of variation from Regression Model/Error Term.
\end{point}

% \begin{itemize}[topsep=2pt,itemsep=2pt]
%     \item Partition of Totla Sum of Squares;
%     \item Partition of Degree of Freedom;
%     \item MSS$ \rightsquigarrow $ F-test;
%     \item ANOVA table;
%     \item General linear test. --to be examined further in later sections.
%     \item (Pearson) Correlation Coefficient $ \leftrightarrow \, R^2$
% \end{itemize}

\subsubsection{Monovariate ANOVA}

    Measure of Variation: Sum of Square (SS) \& Mean Sum of Square (MS).

    MS: Divide each SS by corresponding $ dof $. Definition of $ dof $ see eqa(\ref{EqaDefinitionOfDegreeOfFreedom}).
    \begin{equation}
        \mathrm{MS}=\dfrac{\mathrm{SS}}{dof} 
    \end{equation}

\begin{itemize}[topsep=2pt,itemsep=2pt]
    \item SST: Total Sum of Squares\index{SST (Total Sum of Squares)}
    \begin{equation}
        \mathrm{SST}=\sum_{i=1}^n(Y_i-\bar{Y})^2 \qquad dof_{\mathrm{SST}}=n-1
    \end{equation}
    \item SSRegression: Variation due to Regression Model \index{SSR (Regression Sum of Squares)} (which is explained by regression line);\footnote{$ \mathrm{SSR}=\hat{\beta }_1^2\sum_{i=1}^n(X_i-\bar{X})^2$, so $ dof_R=1 $}
    \begin{equation}
        \mathrm{SSR}= \sum_{i=1}^n(\hat{Y}_i-\bar{Y})^2 \qquad dof_{\mathrm{SSR}}=1
    \end{equation}
    
    \item SSError: Variation attribtes to $ \varepsilon  $ \index{SSE (Error Sum of Squares)} (which is reflected by residuals).
    \begin{equation}
        \mathrm{SSE}= \sum_{i=1}^n(Y_i-\hat{Y_i}) \qquad dof_{\mathrm{SSE}}=n-2
    \end{equation}
\end{itemize}

\fbox{
    \begin{minipage}{0.9\linewidth}
        $ \Delta $ \textbf{IMPORTANT: }In some books \begin{itemize}[topsep=2pt,itemsep=2pt]
        \item SSRegression $ \to $ SSExplained of SSModel;
        \item SSError $ \to $ SSResidual.
    \end{itemize}

    And Cause \textbf{Confusion}! In this summary we take the former.
    \end{minipage}
}\\



    Idea: take partition of SST. i.e.
    \begin{equation}
        Y_i-\bar{Y}=(Y_i-\hat{Y})+(\hat{Y}-\bar{Y})=e_i 
    \end{equation}
    
    And we can prove that
    \begin{equation}
        \mathrm{SST}=\sum_{i=1}^n(Y_i-\bar{Y})^2=\sum_{i=1}^n(\hat{Y}_i-\bar{Y})^2+\sum_{i=1}^n(Y_i-\hat{Y_i})^2=\mathrm{SSR+SSE} 
    \end{equation}

    That is: we \textbf{partition} SST into two parts, so that we can examine them seperately.
    
Properties:
    
\begin{equation}
    E(\mathrm{MSE})=\sigma ^2\qquad E(\mathrm{MSR})=\sigma ^2+\beta _1^2S_{XX} 
\end{equation}


\subsubsection{Multivariate ANOVA}
    Sampling Notation see eqa(\ref{EqaSampleNotationOfMultiLinear}), still consider $ p+1 $ -dim $ (\mathbf{1}_n,X_i) $ v.s. $ 1 $-dim $ Y $, and $ \beta=(\beta _0,\beta _1,\beta _2,\ldots,\beta _p) $

\begin{itemize}[topsep=2pt,itemsep=2pt]
    \item SST:
    \begin{equation}
        \mathrm{SST}=(Y-\bar{Y}\mathbf{1}_n)'(Y-\bar{Y}\mathbf{1}_n)\qquad dof_{\mathrm{SST}}=n-1
    \end{equation}
    \item SSR:
    \begin{equation}
         \mathrm{SSR}=(\hat{Y}-\bar{Y}\mathbf{1}_n)'(\hat{Y}-\bar{Y}\mathbf{1}_n)\qquad dof_{\mathrm{SSR}}=p
    \end{equation}

    Denoted in hat matrix $ H $ and $ \mathcal{J} $ in eqa(\ref{EqaAllOneMatrix})
    
    \begin{equation}\label{EqaSSMInMatrixNotation}
        \mathrm{SSM}=Y'(H-\dfrac{1}{n}\mathcal{J})Y 
    \end{equation}
    
    
    \item SSE:
    \begin{equation}
         \mathrm{SSE}=(Y-\hat{Y})'(Y-\hat{Y})\qquad dof_\mathrm{SSE}=n-p-1
    \end{equation}

    Denoted in residual $ e $ and hat matrix $ H $:
    \begin{equation}
        \mathrm{SSE}=e'e=Y'(I-H)Y 
    \end{equation}
    
    
    
\end{itemize}

\subsubsection{ANOVA Table}
    \begin{table}[H]
        \centering
        \renewcommand\arraystretch{1}
        \begin{tabular}{c|cccc}
            \hline
            Source&$ dof $&SS&MS&$ F $-Statistic\\\hline
            SSRegression&$ p $&$ \sum_{i=1}^n(\hat{Y}_i-\bar{Y})^2  $&SSR/$ dof_R $& $ \mathrm{MSR}/\mathrm{MSE} $\\
            SSError&$ n-p-1 $&$ \sum_{i=1}^n(Y_i-\hat{Y}_i)^2  $&SSE/$ dof_E $& \\
            SSTotal&$ n-1 $&$ \sum_{i=1}^n(Y_i-\bar{Y})^2  $&SST/$ dof_T $& \\
            \hline
        \end{tabular}
    \end{table}
\begin{rcode}
    \begin{lstlisting}[language=R]
anova(lmfit)
    \end{lstlisting}
\end{rcode}    

% \begin{rcode}
% \begin{lstlisting}[language=R]
% anova(lmfit)
% \end{lstlisting}

% Example output:
% \begin{lstlisting}[language=R]
% Analysis of Variance Table

% Response: y
%             Df  Sum Sq Mean Sq F value    Pr(>F)    
% x          1 11627.5 11627.5  174.06 < 2.2e-16 ***
% Residuals 58  3874.4    66.8
% ---
% Signif. codes:  0 '***' 0.001 '**' 0.01 '*' 0.05 '.' 0.1 ' ' 1
% \end{lstlisting}
% \end{rcode}


\subsubsection{Hypotheses Testing to Slope}
    Main focus: whether the linear relation exist:
\begin{equation}
    H_0:\beta _1=\beta _2=\ldots=\beta _p=0\longleftrightarrow H_1:\exists \beta _i\neq 0,\, i=1,2,\ldots,p
\end{equation}
\begin{itemize}[topsep=2pt,itemsep=2pt]
\item ANOVA $ F $-Test:\index{ANOVA $ F $-test}
    
    We can examine  
    \[
        F=\dfrac{\mathrm{MSR}}{\mathrm{MSE}}\sim F_{p,n-p-1} 
    \]
    
\item General Linear Test (GLT)\index{GLT (General Linear Test)}
    
    First we introduce the examine models:
    \begin{itemize}[topsep=2pt,itemsep=2pt]
        \item Full model: 
        \[
            Y_i=X_i'\beta +\varepsilon _i=\beta _0+\sum_{j=1}^nX_{ij}\beta _j+\varepsilon _i
        \]

        And define $ \mathrm{SSE}_\mathrm{F} $ with $ dof_\mathrm{F}=n-p-1 $ under Full Model.
        \item Reduced model: 
        \[
         Y_i=\beta _0+\varepsilon _i 
        \]
        
        And define $ \mathrm{SSE}_\mathrm{R} $ with $ dof_\mathrm{R}=n-1 $ under Reduced Model.
    \end{itemize}

    and examine
    \begin{equation}
        F=\dfrac{(\mathrm{SSE_R-SSE_F})/(dof_\mathrm{R}-dof_\mathrm{F} )}{\mathrm{SSE_F}/dof_F} \sim F_{p,n-p-1}
    \end{equation}
\begin{rcode}
\begin{lstlisting}[language=R]
nullmodel <- lm(y ~ 1,df)
anova(nullmodel,lmfit)
\end{lstlisting}
\end{rcode}

\item Pearson Correlation Coefficient $ r $ and Coefficient of Multiple Determination $ R^2 $\index{CMD (Coefficient of Multiple Determination)}:

    Pearson's $ r $:
    \[
        r=\hat{cov}(Y,\hat{Y})=\dfrac{\sum\limits_{i=1}^n(Y_i-\bar{Y})(\hat{Y}_i-\bar{Y})}{\sqrt{\sum\limits_{i=1}^n(Y_i-\bar{Y})^2}\sqrt{\sum\limits_{i=1}^n(\hat{Y}_i-\bar{Y})^2}}=\sqrt{\dfrac{\sum\limits_{i=1}^n(\hat{Y}-\bar{Y})^2}{\sum\limits_{i=1}^n(Y_i-\bar{Y})^2}}
    \]
    
    CMD $ R^2 $:

    \[
        R^2=\dfrac{\mathrm{SSR}}{\mathrm{SST}}=1-\dfrac{\mathrm{SSE}}{\mathrm{SST}}
    \]

    Adjusted $ R^2 $:
    \[
        R^2_\mathrm{a}=1-\dfrac{\mathrm{MSE}}{\mathrm{MST}} =1-\dfrac{n-1}{n-p}\dfrac{\mathrm{SSE}}{\mathrm{SST}}
    \]
    
    \begin{itemize}[topsep=2pt,itemsep=2pt]
        \item Relation between $ r $ and $ R^2 $: Under Simple Linear Model, we have 
        \[
            R^2=r^2 
        \]
        \item Relation between $ R^2 $ and $ F $-Statistic:
        \[
            F=\dfrac{R^2}{1-R^2}\dfrac{n-p}{n-1} 
        \]
    \end{itemize}
\end{itemize}






\subsection{Model Assumption, Diagnostics and Remedies}

    To apply OLS, we need the basic Gauss–Markov Assumption eqa(\ref{EqaGaussMarkovAssumption}); or we further need better properties of the model, so need Normal Assumption.
    
    Assumptions:
    \begin{equation}
        \begin{aligned}
            \text{Zero-Mean: }&E(\epsilon_i|X_i)=0 \\
            \text{Homogeneity of Variance: }&var(\epsilon_i)=\sigma^2\\
            \text{Independent: }&\epsilon_i\text{ i.i.d. }\sim \varepsilon\\
            \text{Normal: }&Y_i\sim N(\beta _0+\beta _1X_i,\sigma^2)
        \end{aligned}
    \end{equation}
    
    Or sum up as 
    \begin{equation}
        \vec{\varepsilon }\sim N_n(\vec{0},\sigma^2I_n) 
    \end{equation}
    
    
    
    Thus we need to conduct Diagnostics and Remedies to 
    \begin{itemize}[topsep=2pt,itemsep=2pt]
        \item examine whether these assumptions are satisfies;
        \item perform correction to regression method.
    \end{itemize}
\subsubsection{Diagnostics}\label{SubSecDiagnostics}


    Preliminary Diagnostics:
\begin{rcode}
\begin{lstlisting}[language=R]
lmfit <- lm(y~x,lmfit)
par(mfrow = c(2, 2))
plot(lmfit)
\end{lstlisting}

\end{rcode}
    

\begin{point}
    Diagnostics to $ X $
\end{point}

    Considering the dependence of $ Y_i $ on $ X_i $, to get a more reliable $ \hat{\beta }_1 $, we cannot just focus on the (marginal) distribution of $ Y_i $, we would also need a better 'distribution' of $ X_i $
    \begin{itemize}[topsep=2pt,itemsep=2pt]
        \item 4 statistics(parameters);\footnote{See sec.\ref{SubSectionStatistics}}
        \begin{itemize}[topsep=2pt,itemsep=2pt]
            \item Mean: Location;
            \begin{equation}
                \bar{X}=\dfrac{1}{n}\sum_{i=1}^nX_i 
            \end{equation}
            \item Standard Deviation: Variability;
            \begin{equation}
                S^2=\dfrac{1}{n-1}\sum_{i=1}^n(X_i-\bar{X}) ^2
            \end{equation}
            
            
            \item Skewness: Lack of Symmertry;
            \begin{equation}
                \hat{g}_1=\dfrac{m_{n,3}}{m_{n,2}^{3/2}}=\dfrac{\frac{1}{n}\sum\limits_{i=1}^n(X_i-\bar{X})^3}{\left( \frac{1}{n}\sum\limits_{i=1}^n(X_i-\bar{X}) \right)^{3/2}} 
            \end{equation}

            Adjusted Skewness (Least MSE):
            \begin{equation}
                \dfrac{\sqrt{n(n-1)}}{n-2}\hat{g}_1 
            \end{equation}
            
            \begin{itemize}[topsep=2pt,itemsep=2pt]
                \item $ \hat{g}_1>0 $: Right skewness, longer right tail;
                \item $ \hat{g}_1<0 $: Left skewness, longer left tail.
            \end{itemize}
            
                
            Fisher-Pearson coefficient of skewness.


            \item Kurtosis: Heavy/Light Tailed.
            \begin{equation}
                \hat{g}_2=\dfrac{m_{n,4}}{m_{n,2}^2}-3= \dfrac{\frac{1}{n}\sum\limits_{i=1}^n(X_i-\bar{X})^4}{\left( \frac{1}{n}\sum\limits_{i=1}^n(X_i-\bar{X})^2 \right)^{2}} -3
            \end{equation}

            $ \hat{g}_2=0 \Rightarrow $ similar to normal.
            \begin{itemize}[topsep=2pt,itemsep=2pt]
                \item $ \hat{g}_2>0 $: Leptokurtic, heavy tail, slender;
                \item $ \hat{g}_2<0 $: Platykurtic, light tail, broad.
            \end{itemize}
            
            Note: In expression of $ \hat{g}_1 $ and $ \hat{g}_2 $, we already divide the variance. So Skewness and Kurtosis only reflect the difference from normal, but \textbf{not}  related to variance.
                
            Best tool to determine Kurtosis: \hyperlink{QQplot}{QQ-Plot}.
            
        \end{itemize}

\begin{rcode}
\begin{lstlisting}[language=R]
summary(df$x)
\end{lstlisting}

    Other moments use package \lstinline|moments|
\end{rcode}
        \item Useful Plots:
        \begin{itemize}[topsep=2pt,itemsep=2pt]
            \item BoxPlot: to examine the similarity of  distribution.
            
            Notation:
            \begin{enumerate}[topsep=2pt,itemsep=2pt]
                \item min point above 25\% quantile-1.5IQR;
                \item 25\% quantile;
                \item median;
                \item 75\% quantile;
                \item max point below 75\% quantile+1.5IQR.
            \end{enumerate}
            
                
            \begin{center}
                \begin{tikzpicture}
                    \draw(-2,-0.6)rectangle(2,0.6);
                    \draw (-5.5,0)--(-2,0);
                    \draw (2,0)--(5.5,0);
                    \draw(-5.5,-0.7)--(-5.5,0.7);
                    \draw(5.5,-0.7)--(5.5,0.7);
                    \draw(0,-0.6)--(0,0.6);
                    \node at (-2,1){$ 2 $};
                    \node at (2,1){$ 4 $};
                    \node at (-5.5,1.1){$ 1 $};
                    \node at (5.5,1.1){$ 5 $};
                    \node at (0,1){$ 3 $};
                \end{tikzpicture}
            \end{center}


    
            

            \item Histogram Plots: Frequency distribution (can deal with many-peak)
            \item Quantile-Quantile Plots\index{QQ-Plot (Quantile-Quantile Plots)}: Examine the similarity  between distribution.
            
            For two CDF $ q=F(x) $ and $ q=G(x) $(where $ q $ for quantile), with $ x=F^{-1}(q) $, $ x=G^{-1}(q) $. And Plot $ F^{-1}(q) $-$ G^{-1}(q) $.

            Usually test normality, take $ G=\Phi  $
        \end{itemize}
\begin{rcode}
\begin{lstlisting}[language=R]
boxplot(df$x)

hist(df$x)

hist(df$x,freq=FALSE)
lines(density(df$x))

stem(df$x)

qqnorm(df$x)
qqline(df$x,col='red)
\end{lstlisting}

\end{rcode}
            
        \item Normality;
        \item Bias:
        \begin{itemize}[topsep=2pt,itemsep=2pt]
            \item Selection Bias: Not completely random sampling;
            \item Information Bias: Difference between 'designed' and 'get', e.g. no response;
            \item Confounding: Exist another important variable, while the model actually focuses on a less important variable, or even reverse the causality.
        \end{itemize}
        
            
    \end{itemize}
    
\begin{point}
    Diagnostics to Residual
\end{point}

   

\begin{itemize}[topsep=2pt,itemsep=2pt]
    \item Linearity: use Residual Plot: Reflect the linearity and variance assumption.
\begin{rcode}
\begin{lstlisting}[language=R]
lmfit <- lm(y~x,df)
scatter(df$x,lmfit$residuals)
abline(h=0)
\end{lstlisting}

\end{rcode}
    \item The Assumption of Equal Variances:
    \begin{itemize}[topsep=2pt,itemsep=2pt]
        \item Bartlett's test:\index{Bartlett's test}
        
        Idea: divide the sample into groups $ g $, and get each MSE
        \begin{equation}
             \mathrm{MSE}_g=\dfrac{1}{n_g}\sum_{i=1}^{n_g}(Y_{gi}-\hat{Y}_g)^2
        \end{equation}
        
        and take statistic
        \begin{equation}
            S=-\dfrac{(N-g)\ln\left[ \sum\limits_g \dfrac{n_g}{N-n_g}\mathrm{MSE}_g \right]-\sum\limits_{g}(n_g-1)\ln \dfrac{n_g}{N-n_g}\mathrm{MSE}_g }{1+\dfrac{1}{3(G-1)}\sum\limits_g\left( \dfrac{1}{n_g-1}-\dfrac{1}{N-G} \right)} \sim \chi^2
        \end{equation}

        to conduct test. 

        Note: \textbf{sensitive}  to normal assumption, not robust. Used when normal assumption is satisfied.
        \item Levene's test: \index{Levene's Test}Divide the sample into $ G $ groups. Denote \textbf{mean}  of residual within each group as $ \tilde{e}_g $, and in each group compute
        \begin{equation}
            d_{ig}=|e_{ig}-\tilde{e}_g| \Rightarrow \bar{d}_{g}=\dfrac{1}{n_g}\sum_{j=1}^{n_g}d_{ig}
        \end{equation}

        Then conduct ANOVA to $ d_{ig} $.

        If $ G=2 $: 2-sample $ t $-test,
        \begin{equation}
            T=\dfrac{\bar{d}_1-\bar{d}_2}{s\sqrt{\frac{1}{n_1}+\frac{1}{n_2}}}\xrightarrow[]{\mathscr{L}} t_{n-2}\qquad s^2=\dfrac{\sum(d_{i1}-\bar{d}_1)^2+\sum(d_{i2}-\bar{d}_2 )^2}{n-2}
        \end{equation}
        

        
        
        \item Brown-Forsythe's Test\index{Brown-Forsythe's Test} (Modified Levene's test): For skewed sample, take the \textbf{mean} as \textbf{median}, more robust. 


        \item Breusch-Pagan Test:\index{B-P Test (Breusch-Pagan Test)}
        
        Assume variance of $ \varepsilon _i $ dependent on $ X_i $ as $ m^{\mathrm{th}} $ polynomial:
        \begin{equation}
            \sigma_i^2=\alpha _0+\sum_{k=1}^m\alpha _kX_i^k
        \end{equation}
        
        and test 
        \begin{equation}
            H_0:\alpha _k=0\,\forall k=1,2,\ldots,m\longleftrightarrow H_1 
        \end{equation}

        Method: First conduct OLS to get regression line $ \hat{l}_1 $ and residuals $ e_i $ and SSE, and conduct regression of $ e_i^2 $ over $ X_i $ to get another regression line $ \hat{l}_2 $ and corresponding SSR$ ^* $.

        Then statistic
        \begin{equation}
            S=\dfrac{\mathrm{SSR^*}/2}{(\mathrm{SSE}/n)^2}\xrightarrow[]{\mathscr{L}} \chi^2_m
        \end{equation}
        
    \end{itemize}
\begin{rcode}
    Example for $ G=2 $:
\begin{lstlisting}[language=R]
group <-factor(rep(c(1,2),length.out=length(df$x),
    each=(ceiling(length(df$x)/2))))

bartlett.test(lmfit$residuals~group,group)

library(car)
leveneTest(lmfit$residuals~group,group,center=mean)
leveneTest(lmfit$residuals~group,group,center=median)

library(lmtest)
bptest(lmfit)
\end{lstlisting}

\end{rcode}

    \item The Assumption of Normality:
    
    In most case we use S-W Test($ n<2000 $) and K-S Test($ n>2000 $):
    \begin{itemize}[topsep=2pt,itemsep=2pt]
        \item QQ-plot of ordered residuals;
        \item[$ \star  $] Shapiro-Wilk Test\index{S­-W Test(Shapiro-­Wilk Test)} (Most Powerful)\footnote{Detail of S-W Test and K-S Test see \hyperlink{testofnormality}{Test of Normality} in sec.\ref{SubSectionIntroToNonParametricHypothesisTesting}}:
        \begin{equation}
            R^2=\dfrac{\left(\sum_{i=1}^n(X_{(i)}-\bar{X})(m_i-\bar{m})\right)^2}{\sum_{i=1}^n(X_{i}-\bar{X})^2\sum_{i=1}^n(m_i-\bar{m})^2}=corr(X_{(i)},m_i) 
        \end{equation}
        
        \item Kolmogorov-Smirnov Test\index{K-S Test (Kolmogorov-Smirnov Test)}: 
        \begin{equation}
            D_n=\sum_{x}|F_n(x)-\Phi(x)|
        \end{equation}
        
        
        \item Cramér-von Mises Test\index{CvM Test (Cramér-von Mises Test)}:
        \begin{equation}
            T=n\int_{-\infty}^\infty (F_n(x)-\Phi (x)) ^2\,\mathrm{d}\Phi(x)
        \end{equation}
        \item Anderson-Darling Test:\index{A-D Test (Anderson-Darling Test)}
        \begin{equation}
            A^2-n\int _{-\infty}^\infty (F_n(x)-\Phi(x))^2\dfrac{1}{\Phi(x)(1-\Phi(x))} \,\mathrm{d}\Phi(x)
        \end{equation}
    \end{itemize}
\begin{rcode}
\begin{lstlisting}[language=R]
qqnorm(lmfit$residuals)
qqline(lmfit$residuals)

qqp <- qqnorm(lmfit$residuals)
cor(qqp$x,qqp$y)

shapiro.test(lmfit$residuals)

ks.test(jitter(lmfit$residuals),pnorm,mean(lmfit$residuals),
    sd(lmfit$residuals))

library(nortest)
cvm.test(lmfit$residuals)

ad.test(lmfit$residuals)
\end{lstlisting}

\end{rcode}


    \item The Assumption of Independence:
    \begin{itemize}
        \item Durbin-Watson Test:  \index{DW Test (Durbin-Watson Test)}
        \begin{equation}
            d=\dfrac{\sum_{j=2}^n(e_j-e_{j-1})^2}{\sum_{j=1}^ne_j^2} 
        \end{equation}
        
        $ d\in (1.5,2.5) $ is fine.
        \item Ljung-Box Test:\index{Ljung-Box Test}
        
        \begin{equation}
            Q=n(n+2)\sum_{k=1}^n\dfrac{\hat{\rho}_k^2}{n-k} 
        \end{equation}
        
        
    \end{itemize}

\begin{rcode}
\begin{lstlisting}[language=R]
dwtest(lmfit)
\end{lstlisting}

\end{rcode}   
        
\end{itemize}

\begin{point}
    Diagnostics to Influentials
\end{point}

    An intuitive explanation to extreme values:
    \begin{itemize}[topsep=2pt,itemsep=2pt]
        \item Outliers: Extreme case for $ Y $;
        \item High Leverage: Extreme case for $ X $;
        \item Influentials: Cases that influence the regression line.
    \end{itemize}

    Influentials = Outliers $ \cap $ High Leverage

    In OLS part, we got the $ \hat{\beta}  $ as $\hat{ \beta} = (X'X)^{-1}X'Y $ and got $ \hat{Y} $ as 
    \begin{equation}
        \hat{Y}= X\hat{\beta }=X(X'X)^{-1}X'y=\hat{H}Y
    \end{equation}
    
    where $ \hat{H} $ is the \textbf{Hat Matrix}\footnote{It can also be considered as the projection matrix onto span$ \{X\} $.} 

    Denote in matrix derivation as $ H=\dfrac{\partial^{} \hat{Y}}{\partial Y^{}} $. The diagonal elements of $ \hat{H} $ is self-sensitivity:
    \begin{equation}
    h_{ii}=\dfrac{\partial^{} \hat{Y}_i}{\partial Y_i^{}} =\dfrac{1}{n}+\dfrac{(X_i-\bar{X})^2}{S_{XX}}
    \end{equation}

    Note: the distribution of $ e_i $ in eqa.(\ref{EqaSamplingDistributionOfResiduals}) thus can be written in $ h_{ii} $ as 
    \begin{equation}
        e_i\sim (0,\sigma ^2(1-h_{ii})) 
    \end{equation}

    Some refined residuals to help conduct Diagnostics:
\begin{itemize}[topsep=2pt,itemsep=2pt]
    \item Standardized Residual:
    \begin{equation}
         \dfrac{e_i}{\sigma _{e_i}}=\dfrac{e_i}{\sigma \sqrt{1-h_{ii}}}
    \end{equation}
    \item (Internal) Studentized Residual: replace $ \sigma  $ with $ s=\hat{\sigma } $
    
    \begin{equation}
         r_i=\dfrac{e_i}{\hat{\sigma }\sqrt{1-h_{ii}}}
    \end{equation}
    \item (External) Studentized Residual: To avoid self-influence, take \textbf{deleted} residual:
    
    Delete the $ i^{\mathrm{th}} $ case and conduct regression to the $ n-1 $ sample cases, denote the regression parameter as 
    \begin{equation}
        \hat{\beta }_{1(\wedge i)}\qquad \hat{\beta }_{0(\wedge i)} 
    \end{equation}
    
    and deleted residual defined as 
    \begin{equation}
        d_i=Y_i-Y_{i(\wedge i)} =\dfrac{e_i}{1-h_{ii}}
    \end{equation}

    external studentized residual:
    \begin{equation}
         t_i=\dfrac{d_i}{\sigma _{(\wedge i)\sqrt{1-h_{ii}}}}
    \end{equation}
    
    
\end{itemize}


    Cook's Distance: 
    \begin{equation}
        D_i=\dfrac{\sum_{k=1}^n(Y_k-\hat{Y}_{k(\wedge i)})^2}{p\hat{\sigma }^2} =\dfrac{e_i^2}{p\hat{\sigma }^2}\left[ \dfrac{h_{ii}}{(1-h_{ii})^2} \right]
    \end{equation}

    Comment:
    \begin{equation}
        D_i=\dfrac{e_i^2}{p\hat{\sigma }^2}\left[ \dfrac{h_{ii}}{(1-h_{ii})^2} \right]=\dfrac{1}{p}\dfrac{h_{ii}}{1-h_{ii}}\times r_i^2
    \end{equation}

    where $ \dfrac{1}{p}\dfrac{h_{ii}}{1-h_{ii}} $ correponds to hige leverage, and $ r_{i}^2 $ correponds to outliers, multiply to get influentials.
    
    
    
    
    

    
    
    
    
    
    
        










\subsubsection{Remedies}
\begin{point}
    General Linear Model
\end{point}
    
    \begin{equation}
        E(Y)=g(\beta _0+\beta _1X_1+\beta _2X_2+\ldots) 
    \end{equation}
    
\begin{point}
    Remedies: Conduct Transformation
\end{point}

    \begin{itemize}[topsep=2pt,itemsep=2pt]
        \item Stablize Variance;
        \item Improve Normality;
        \item Simplify the Model.
    \end{itemize}

    Transformation Methods:
    \begin{itemize}[topsep=2pt,itemsep=2pt]
        \item Variance Stabilizing Transformations:
            For $ E(Y_X)=\mu_X$, $ var(Y_X)=h(\mu_X) $, take transformation $ f(Y) $ such that $ var(f(Y))=\mathrm{const} $, satisfies
            \begin{equation}
                f(\mu)=\int\dfrac{c\,\mathrm{d}\mu}{\sqrt{h(\mu)}} 
            \end{equation}

            Examples:
            \begin{align*}
                h(\mu)=&\mu^2\Rightarrow f(\mu )=\ln\mu\\
                h(\mu)=&\mu^{2\nu}\Rightarrow f(\mu )=\mu ^{1-\nu}
            \end{align*}
        
        \item Box-Cox Transformation: Take 
    \begin{equation}
        Y^*=\dfrac{Y^\lambda -1}{\lambda }
    \end{equation}

            Examples:
        \begin{align*}
            \lambda =1&\Rightarrow Y^*\sim Y\\
            \lambda =0.5&\Rightarrow Y^*\sim \sqrt{Y}\\
            \lambda =0&\Rightarrow Y^*\sim \ln Y\\
            \lambda =-1&\Rightarrow Y^*\sim 1/Y
        \end{align*}
    
        And conduct regression to model
        \begin{equation}
            Y^* =\beta _0+\beta _1X+\varepsilon_i 
        \end{equation}
        
        Likelihood Function
        \begin{equation}
            L(\beta ,\sigma ^2;\lambda )=\dfrac{1}{(2\pi\sigma ^2)^{n/2}}\exp\left( -\dfrac{1}{2\sigma ^2}\sum_{i=1}^n (Y_i^*-\beta _0-\beta _1X_i)^2 \right) J(\dfrac{\partial^{} Y^*}{\partial Y^{}})
        \end{equation}

        where the Jacobi Matrix denoted in Geometric Mean $ \mathrm{GM}(Y)=\prod_{i=1}^n Y_i^{1/n}$
        \begin{equation}
            J(\dfrac{\partial^{} Y^*}{\partial Y^{}})=\prod_{i=1}^nY_i^{\lambda -1}=\mathrm{GM}(Y)^{n(\lambda -1)}
        \end{equation}
        
        

        MLE Estiamtor:
        \begin{align*}
            \hat{\beta }^*&= (X'X)^{-1}X'Y^*\\
            \hat{\sigma }^2_n&=\dfrac{1}{n}\mathrm{SSE}^*\\
            \mathrm{SSE}^*&=\sum_{i=1}^n(Y_i^*-\hat{Y}_i^*)^2
        \end{align*}

        And when $ \beta  $, $ \sigma ^2 $ take MLE estimator, $ L(\beta ,\sigma ^2;\lambda ) $ can be regarded a function of $ \lambda  $:
        \begin{equation}
            \ln L(\beta ,\sigma ^2;\lambda )=l(\lambda )=-\dfrac{n}{2}\ln \dfrac{\hat{\sigma}^2_n}{\mathrm{GM}(Y)^{2(\lambda -1)}}+\mathrm{const}
        \end{equation}

        For simplification, denote $ Z=Y*/J^{1/n} $ and get
        \begin{equation}
            l(\lambda )=-n\ln \sigma^2_{n_Z}+\mathrm{const} 
        \end{equation}
        
        where 

        \begin{equation}
            Z_i^* =\begin{cases}
                \dfrac{Y_i^\lambda-1 }{\lambda }\dfrac{1}{\prod\limits_{k=1}^n Y_k^{\frac{\lambda -1}{n}}},&\lambda \neq 0\\
                \ln Y_i\prod\limits_{k=1}^n Y_k^{\frac{1}{n}},&\lambda =0
            \end{cases}
        \end{equation}

        Plot $ l(\lambda ) $-$ \lambda  $ to determine a proper $ \lambda  $ and transform $ Y^*=\dfrac{Y^\lambda -1}{\lambda } $:
        \begin{itemize}[topsep=2pt,itemsep=2pt]
            \item Selected $ \lambda $ should be closed to $ \lambda_{\arg\max l} $, at least within CI\footnote{Here CI can be derived using Wilk's Thm.}
            \begin{equation}
                \{\lambda |l(\lambda )\geq l(\lambda_{\arg\max l})-\dfrac{1}{2}\chi^2_{1,1-\alpha }\}
            \end{equation}
            \item Should pick a $ \lambda  $ which is \textbf{Interpretable}. e.g. If $ \lambda =1 $ is within range, then take $ \lambda =1 $ (does not transform).
            
            
        \end{itemize}
        
            
        
% \begin{rcode}
% \begin{lstlisting}[language=R]
% library(MASS)
% bctrans <- boxcox(y~x,df,lambda = seq(-1.5, 1.5, length = 15))
% bctrans$x[which.max(bctrans$y)]
% \end{lstlisting}

% \end{rcode}
        
        
%     Note: we can transform on $ X $ or $ Y $ or simultaneously to get better regression model.
    
\end{itemize}



    
\subsection{Multiple Linear Regression}

\begin{point}
    Sample Geometry Notation
\end{point}

    In sample matrix notation:
    \begin{equation}
        Y=X\beta+\varepsilon \,\leftrightharpoons\, Y_i=X\beta _i+\varepsilon _i,\,\forall i=1,2,\ldots,q
    \end{equation}
    
    where 
\begin{subequations}\label{EqaSampleNotationOfMultiLinear}
    \begin{align}
        \mathop{Y}\limits_{n\times q} =&\begin{bmatrix}
        y_{11}&y_{12}&\ldots&y_{1q}\\
        y_{21}&y_{22}&\ldots&y_{2q}\\
        \vdots&\vdots&\ddots&\vdots\\
        y_{n1}&y_{n2}&\ldots&y_{nq}\\
        \end{bmatrix}
        =\begin{bmatrix}
            y_1,y_2,\ldots,y_q
        \end{bmatrix}
        &
        y _i=&\begin{bmatrix}
                y _{1i}\\
                y _{2i}\\
                \vdots\\
                y _{ni}
            \end{bmatrix}\\
        \mathop{X}\limits_{n\times (p+1)}=&\begin{bmatrix}
        1&x_{11}&x_{12}&\ldots&x_{1p}\\
        1&x_{21}&x_{22}&\ldots&x_{2p}\\
        \vdots&\vdots&\ddots&\vdots\\
        1&x_{n1}&x_{n2}&\ldots&x_{np}\\
        \end{bmatrix}
        =
        \begin{bmatrix}
        x_1'\\x_2'\\\vdots\\x_n'   
        \end{bmatrix} 
        &
        x_i=&\begin{bmatrix}
            1\\
            x_{i1}\\
            \vdots\\
            x_{ip}
        \end{bmatrix}\\
        \mathop{\beta }\limits_{(p+1)\times q}=&\begin{bmatrix}
        \beta _{01}&\beta _{02}&\ldots&\beta _{0q}\\
        \beta _{11}&\beta _{12}&\ldots&\beta _{1q}\\
        \beta _{21}&\beta _{22}&\ldots&\beta _{2q}\\
        \vdots&\vdots&\ddots&\vdots\\
        \beta _{p1}&\beta _{p2}&\ldots&\beta _{pq}\\
        \end{bmatrix} =\begin{bmatrix}
            \beta _1,\beta _2,\ldots,\beta _q
        \end{bmatrix}
        & \beta _i=&\begin{bmatrix}
            \beta _{i0}\\
            \beta_{i1}\\
            \vdots\\
            \beta_{ip}
        \end{bmatrix}\\
        \mathop{\varepsilon }\limits_{n\times q} =&
        \begin{bmatrix}
        \varepsilon _{11}&\varepsilon _{12}&\ldots&\varepsilon _{1q}\\
        \varepsilon _{21}&\varepsilon _{22}&\ldots&\varepsilon _{2q}\\
        \vdots&\vdots&\ddots&\vdots\\
        \varepsilon _{n1}&\varepsilon _{n2}&\ldots&\varepsilon _{nq}\\
        \end{bmatrix}=
        \begin{bmatrix}
            \varepsilon _1,\varepsilon _2,\ldots,\varepsilon _q
        \end{bmatrix}
        &
        \varepsilon _i=&\begin{bmatrix}
                \varepsilon _{1i}\\
                \varepsilon _{2i}\\
                \vdots\\
                \varepsilon _{ni}
            \end{bmatrix}
    \end{align}

    
\end{subequations}


    Under matrix notation, model and assumptions eqa(\ref{EqaGaussMarkovAssumption}) can be expressed in condensed notation:

    \begin{equation}
        Y_i=X\beta_i +\varepsilon_i  \sim N_n(X\beta_i ,\sigma_i^2I_n),\quad i=1,2,\ldots,q
    \end{equation}

    To conduct OLS
    \begin{equation}
        \hat{\beta }=\mathop{ \arg\min }\limits_{\beta \in \mathbb{R}^{p+1} } (Y-X\beta )^T(Y-X\beta )
    \end{equation}
    
    Here we introduce two approaches:
\begin{itemize}[topsep=2pt,itemsep=2pt]
    \item Analytical: Take matrix differciation (See \hypertarget{MatrixDifferenciation}{sec.\ref{SubSubSectionMatrixNotationAndLemma}})
    

\begin{align*}
    0&=\dfrac{\partial^{} (Y-X\beta )^T(Y-X\beta ) }{\partial \beta ^{}} =\dfrac{\partial^{} }{\partial\beta  ^{}}(Y^TY- Y^TX\beta -\beta ^TX^TY+\beta ^TX^TX\beta )\\ 
    &=-X^TY-X^TY+(X^TX+XX^T)\beta 
    =-2X^T(Y-X\beta )
\end{align*}
    
    Thus we get OLS:
    \begin{equation}
        \hat{\beta }=(X'X)^{-1}X'Y 
    \end{equation}
    
    
    \item Geometric/Algebraical: Use hyper-projection.
    \begin{equation}
        \hat{\beta }=\mathop{ \arg\min }\limits_{\beta \in \mathbb{R}^{p+1} } d(Y,X\beta )
    \end{equation}

    i.e. $ \hat{\beta } $ is the (hyper-)projection of $ Y $ onto $ X $ (within Euclidean Space), naturally we have
    \begin{equation}
        (X\beta )^T(Y-X\beta )=0\Rightarrow \hat{\beta }=(X'X)^{-1}X'Y 
    \end{equation}

\end{itemize}

\begin{point}
    Matrix Notation of OLS Estimator:
    \begin{equation}
        \hat{\beta }=(X'X)^{-1}X'Y 
    \end{equation}
\end{point}

    (For simplificaiton, the following part consider multivariate $ \mathop{X}\limits_{n\times (p+1)}  $ with one $ \mathop{Y}\limits_{n\times 1}  $)

    Properties \& Extrapolation
\begin{itemize}[topsep=2pt,itemsep=2pt]
    \item Sampling Districution of $ \hat{\beta } $: (Here consider normal case $ Y\sim N(X\beta ,\sigma^2I_n) $, and use eqa(\ref{EqaTransformOfMultiNormal})) 
    \begin{equation}
        \hat{\beta }=(X'X)^{-1}X'Y \sim N_n(\beta,\sigma^2(X'X)^{-1})
    \end{equation}

    Comment: $ cov(\beta_i,\beta_j ) $ are generally not 0, $ \Rightarrow $ $ \beta _i,\beta _j $ dependent.
    \item Predicted Response \& Hat Matrix $ H $:
    \begin{equation}
        \hat{Y}=X\hat{\beta }=X(X'X)^{-1}X'Y\equiv  HY=P_XY
    \end{equation}

    where \textbf{Hat Matrix}/Influence matrix/Projection matrix $ H=P_X=X(X'X)^{-1}X' $, with properties
    \begin{itemize}[topsep=2pt,itemsep=2pt]
        \item Symmertric: $ H^T=H $;
        \item Idempotence: $ H^2=H $
        \item $ H $ and self-influene factor $ h_{ii} $: Note the linearity of $ \hat{Y} $ on $ Y $
        \begin{equation}
            \hat{Y}=HY \Rightarrow H=\dfrac{\partial^{} \hat{Y}}{\partial Y^{}}
        \end{equation} 
    
        The diagonal elements of $ H $ is 
        \begin{equation}
            h_{ii}=\dfrac{\partial^{}\hat{y}_i}{\partial y_i^{}}=X_i(X'X)^{-1}X'_i
        \end{equation}

        Comment on $ h_{ii} $: $ var(e_i) =\sigma ^2(1-h_{ii})$, for $ h_{ii}\to 1 $, i.e. the regression line always pass $ y_i $, thus it's `influential'.
        \item $ H $ and Residual $ e $
    \end{itemize}




    \item Residual:
    \begin{equation}
        e=Y-\hat{Y}=(I-H)Y\sim N_n\left(0 , \sigma ^2(I-H) \right)
    \end{equation}

    Covariance Matrix of Residual:
    \begin{equation}
        cov(e)=\sigma ^2(I-H)=
        \sigma ^2\begin{bmatrix}
        1-h_{11}&-h_{12}&\ldots&-h_{1n}\\
        -h_{21}&1-h_{22}&\ldots&-h_{2n}\\
        \vdots&\vdots&\ddots&\vdots\\
        -h_{n1}&-h_{n2}&\ldots&1-h_{nn}\\
        \end{bmatrix}
    \end{equation}
    
    
    \item Estimator and Distribution of  $ \sigma ^2 $:
    
    First use eqa(\ref{EqaExpectationOfQuadric}) to get \footnote{Also we need the property of idmpotnet matrix
    \begin{equation}
        \lambda_i=0\text{ or }1\Rightarrow tr(H)=\mathrm{rank}(H)=\sum_{i=1}^n\lambda _i=\# (\lambda =1) 
    \end{equation}
    }
    \begin{equation}
        E(\mathrm{SSE})=E(e'e)=E(Y'(I-H)Y)=(X\beta )'(I-H)X\beta +tr((I-H)\sigma ^2I_n) =\sigma ^2(n-p-1)
    \end{equation}

    $ dof $ of Residual $ e $ (use definition eqa(\ref{EqaDefinitionOfDegreeOfFreedom})):
    \begin{equation}
        dof_e=dof_{(I-H)Y}=\mathrm{rank}(I-H)=n-p-1 
    \end{equation}
    
    
    
    Thus the unbiased estimator of $ \sigma ^2 $ is 
    \begin{equation}
        \hat{\sigma }^2=\mathrm{MSE}=\dfrac{e'e}{n-p-1}=\dfrac{Y'(I-H)Y}{n-p-1}
    \end{equation}

    Distribution (under normal assumption):
    
    \begin{equation}
        \dfrac{(n-p-1)\hat{\sigma }^2}{\sigma ^2}\sim \chi^2_{n-p-1}
    \end{equation}
    
    \item Gauss–Markov Thm.: OLS Estimator of $ \beta  $ is the BLUE Estimator.
    \item Leverage and Mahalanobis Distance:
    
    Mahalanobis Distance between $ X $ and $ Y $ as defined in eqa(\ref{MahalanobisDistance})
    \begin{equation}
         d_M(\vec{x})=\sqrt{(\vec{x}-\vec{\mu})^TS ^{-1}(\vec{x}-\vec{\mu})} 
    \end{equation}

    And we can proof $ d_M $ of a case item $ X_{ i\cdot}=(1,X_{i1},X_{i2},\ldots,X_{ip}) $ is
    \begin{equation}
        d_{M}^2(X_{i\cdot})=(n-1)(h_{ii}-\dfrac{1}{n}) 
    \end{equation}
    
    
    
    


\end{itemize}

    






Test of Normality: Jarque-Bera Test \index{JB-test (Jarque-Bera test)}, using skewness $ \hat{g}_1 $ and kurtosis $ \hat{g}_2 $
\begin{equation}
    \mathrm{JB}=\dfrac{n}{6}(\hat{g}_1^2+\dfrac{1}{4}\hat{g}_2^2) \xrightarrow[]{\mathscr{L}} \chi^2_2
\end{equation}

\begin{rcode}
\begin{lstlisting}[language=R]
library(tseries)
jarque.bera.test(df$y)
\end{lstlisting}

\end{rcode}











        


\newpage

\section{多元统计分析部分}\label{SecMultivariateStatisticalAnalysis}
\begin{center}
    Instructor: Dong Li \& Tianying Wang
\end{center}
\subsection{Multivariate Data}
    In this section, we consider a \textbf{Multivariate Statistic Model}. Sample comes from $p$ dimension multivariate population $f(x_1,x_2,\ldots,x_p)$.

    \textbf{Notation }: In this section, we still denote random variable in upper case and observed value in lower case, specially express random vector in bold font. \textbf{But} in this section we usually omit the vector symbol $ \vec{\cdot} \,\,$. e.g.
    random vector with $ n $ \textbf{variable }is denoted as $\mathbf{X}=(X_{\cdot 1},X_{\cdot 2},\ldots ,X_{\cdot p})$; sample of size $ n $ from the multivariate population is a $ n\times p $ matrix $ \{x_{ij}\} $, each sample item (a row in sample matrix) is denoted as $ x_i' $ or $ x_i^T $.\footnote{Here sample item (or sample case) $x_i=[x_{i1},x_{i2},\ldots,x_{ip}]^T$ is a column vector.} 
    % In this section we use the upper case $ X_i $ means that it's a vector (not necessarily means an r.v.).
    %\footnote{In previous section, a multivariate r.v. is denoted $\vec{X}=(X_1,X_2,\ldots,X_p) $, and sample item is $ \vec{X_i}=(X_{i1},X_{i2},\ldots,X_{ip})  $}


\subsubsection{Matrix Representation}


    \begin{itemize}[topsep=0pt,itemsep=1pt]
        \item \hyperlink{RandomVariableRepresentation}{Random Variable Representation}
        \item \hyperlink{SampleRepresentation}{Sample Representation}
        \item \hyperlink{StatisticsRepresentation}{Statistics Representation}
        \item \hyperlink{SampleStatisticsProperties}{Sample Statistics Properties}
    \end{itemize}
    



\begin{point}
    \hypertarget{RandomVariableRepresentation}{Random Variable Representation}:
\end{point}
    \begin{itemize}[topsep=6pt,itemsep=4pt]
    \item Random Vector\index{r.v. (Random Variable or Random Vector)}: For a $ p\times 1 $ random vector $ \vec{X}=(X_{1},X_{2},\ldots,X_{p})^T  $, denote (Marginal) expectation and variance, and covariance, correlation coefficient between $ X_i,X_j $ as follows:\footnote{An intuition to avoid confusion of $ \sigma _{\cdot \cdot } $: two subscripts means quadratic.}
    \begin{align}
        \mu_i&=\mathbb{E}(X_i)\\
        \sigma _{ii}&=\sigma_i ^2=\mathbb{E}(X_i-\mu_i)^2\\
        \sigma_{ij}&=\mathbb{E}[(X_i-\mu_i)(X_j-\mu_j)]\\
        \rho _{ij}&=\dfrac{\sigma _{ij}}{\sqrt{\sigma _{ii}}\sqrt{\sigma _{jj}}}=\dfrac{\sigma _{ij}}{\sigma _i\sigma _j}
    \end{align}
    
    and we have covariance matrix (as defined in \autoref{SubSubSectionCovarianceAndCorrelation}, \autoref{covariancematrix})
    \begin{equation}
        \Sigma =\mathbb{E}[(X-\mu)(X-\mu)^T] =
        \begin{bmatrix}
        \sigma _{11}&\sigma _{12}&\ldots&\sigma _{1p}\\
        \sigma _{21}&\sigma _{22}&\ldots&\sigma _{2p}\\
        \vdots&\vdots&\ddots&\vdots\\
        \sigma _{1p}&\sigma _{p2}&\ldots&\sigma _{pp}\\
        \end{bmatrix}
    \end{equation}

    and Standard Deviation Matrix
    \begin{equation}\label{EqaStandardDeviationMatrix}
        V^{1/2}=diag\{\sqrt{\sigma _{ii}}\} 
    \end{equation}

    Based on $ \vec{X}=(X_{1},X_{2},\ldots,X_{p})  $, consider the linear combination:$ Y=c'X=c_1X_1+c_2X_2+\ldots c_pX_p $
    \begin{align}
        \mathbb{E}(y)=c'\mu\qquad var(Y)=c'\Sigma c
    \end{align}

    and $ Z_i=\sum_{j=1}^p c_{ij}X_j $ (i.e. $ Z=CX $):
    \begin{equation}
        \mu_Z=\mathbb{E}(Z)= C\mu_X\qquad \Sigma _Z=C\Sigma _XC^T
    \end{equation}
    
    
    
    

    and Correlation Matrix\footnote{Here the correlation matrix is the matrix of Pearson's Correlation Coefficients. Another frequently use correlation matrix called Cross Correlation Matrix is\index{Correlation Matrix@(Cross) Correlation Matrix} 
    \begin{align}
            \mathrm{cross}(X,Y)= \mathbb{E}\left[ X'Y \right]
    \end{align}

    and cross correlation matrix with $ Y=X $:
    \begin{align}
        \mathrm{cross}(X,X)=  \mathbb{E}\left[ X'X\right]
    \end{align}
    }
    \index{Correlation Matrix@(Pearson's) Correlation Matrix}
    \begin{equation}
        \varrho  =\begin{bmatrix}
        \rho _{11}&\rho _{12}&\ldots&\rho _{1p}\\
        \rho _{21}&\rho _{22}&\ldots&\rho _{2p}\\
        \vdots&\vdots&\ddots&\vdots\\
        \rho _{1p}&\rho _{p2}&\ldots&\rho _{pp}\\
        \end{bmatrix} 
        =V^{-1/2}\Sigma V^{-1/2}
    \end{equation}
    
    \item Random Matrix: Definition and basic properties of r.v. see \autoref{SectionPropertiesOfRandomVariableAndVector}. Now extend the definition to matrix $ X=\{X_{ij}\} $. 
    
    \begin{equation}
        X=\{X_{ij}\}=\begin{bmatrix}
        X_{11}&X_{12}&\ldots&X_{1p}\\
        X_{21}&X_{22}&\ldots&X_{2p}\\
        \vdots&\vdots&\ddots&\vdots\\
        X_{1n}&X_{n2}&\ldots&X_{np}\\
        \end{bmatrix} 
    \end{equation}

    And we can further define $ \mathbb{E}(X)=\{\mathbb{E}(X_{ij})\} $.
    For any const matrix $ A,B $ we have
    \begin{equation}
        \mathbb{E}(AXB)=A\mathbb{E}(X)B 
    \end{equation}

    Some more complex parameter can be expressed in language of tensors.
    
    \end{itemize}
    
        
\begin{point}
    \hypertarget{SampleRepresentation}{Sample Representation} (for random vector):
\end{point}
    
    Sample of $n$ items from population characterized by $ p $ variables
    % \begin{table}[H]
    %     \centering
    %     \begin{tabular}{|c|cccccc|}
    %         \hline
    %         \diagbox{Item}{Variable}&Variable 1&Variable 2&$\ldots$&Variable $j$&$\ldots$&Variable $p$\\
    %         \hline
    %         Item 1&$ x_{11} $&$ x_{12} $&$ \ldots $&$ x_{1j} $&$ \ldots $&$ x_{1p} $\\
    %         Item 1&$ x_{21} $&$ x_{22} $&$ \ldots $&$ x_{2j} $&$ \ldots $&$ x_{2p} $\\
    %         $\vdots$&$\vdots$&$\vdots$&$ \ddots $&$\vdots$&$ \ddots $&$\vdots$\\
    %         Item $j$&$ x_{i1} $&$ x_{i2} $&$ \ldots $&$ x_{ij} $&$ \ldots $&$ x_{ip} $\\
    %         $\vdots$&$\vdots$&$\vdots$&$ \ddots $&$\vdots$&$ \ddots $&$\vdots$\\            
    %         Item $n$&$ x_{n1} $&$ x_{n2} $&$ \ldots $&$ x_{nj} $&$ \ldots $&$ x_{np} $\\
    %         \hline
    %     \end{tabular}
    % \end{table}


\[
    \begin{pNiceMatrix}[first-row,first-col,nullify-dots]
        &\mathrm{var}\,1&\mathrm{var}\,2&\ldots &\mathrm{var}\,j&\ldots&\mathrm{var}\,p\\    
    \mathrm{item}\,1& x_{11}&x_{12}&\ldots&x_{1j}&\ldots&x_{1p}\\
    \mathrm{item}\,2&x_{21}&x_{22}&\ldots&x_{2j}&\ldots&x_{2p}\\
    \vdots&\vdots&\vdots&\ddots&\vdots&\ddots&\vdots\\
    \mathrm{item}\,i&x_{i1}&x_{i2}&\ldots&x_{ij}&\ldots&x_{ip}\\
    \vdots&\vdots&\vdots&\ddots&\vdots&\ddots&\vdots\\
    \mathrm{item}\,n&x_{n1}&x_{n2}&\ldots&x_{nj}&\ldots&x_{np}
        \end{pNiceMatrix} 
\]



    Or represented in condense notation:
    \begin{equation}
        X=\{x_{ij}\}=
        \begin{bmatrix}
            x_1^T\\x_2^T\\ \vdots \\ x_n^T
        \end{bmatrix}
        =
        \begin{bmatrix}
            x_{11}&x_{12}&\ldots&x_{1p}\\
            x_{21}&x_{22}&\ldots&x_{2p}\\
            \vdots&\vdots&\ddots&\vdots\\
            x_{n1}&x_{n2}&\ldots&x_{np}\\
        \end{bmatrix} 
        =
        \begin{bmatrix}
            x_{\cdot 1}&x_{\cdot 2}&\ldots &x_{\cdot p}
        \end{bmatrix}
    \end{equation}
\begin{point}
    \hypertarget{StatisticsRepresentation}{Statistics Representation}
\end{point}

    \begin{itemize}[topsep=6pt,itemsep=4pt]
        \item Unit 1 vector:
        \begin{equation}
            \mathbf{1}_k=(\underbrace{1,1,\ldots,1}_{k\text{ 1 in total}})^T
        \end{equation}

        Unit 1 matrix: (Sometimes I also use notation $ \mathcal{J}_n $)
        \begin{equation}\label{EqaAllOneMatrix}
            \mathcal{I}_n  = \{1\}_{n\times n}=\begin{bmatrix}
            1&1&\ldots&1\\
            1&1&\ldots&1\\
            \vdots&\vdots&\ddots&\vdots\\
            1&1&\ldots&1\\
            \end{bmatrix}_{n\times n}
        \end{equation}


        \item Sample mean of the $ j^\mathrm{th} $ variable:
        \begin{equation}
            \bar{x}_j=\dfrac{x_{1j}+x_{2j}+\ldots+x_{nj}}{n}=\dfrac{\mathbf{1}_n'x_{\cdot j}}{n},\quad j=1,2,\ldots,p
        \end{equation}
        
        \item Deviation of measurement of the $ j^\mathrm{th} $ variable:
        \begin{equation}
            d_j=\begin{bmatrix}
                x_{1j}-\bar{x}_j\\x_{2j}-\bar{x}_j\\\vdots\\x_{nj}-\bar{x}_j
            \end{bmatrix}=x_{.j}-\bar{x}_j\mathbf{1}_n = (I-\dfrac{1}{n}\mathcal{I}_n)x_{\cdot j},\quad j=1,2,\ldots,p
        \end{equation}
        \item Covariance Matrix:\index{Covariance Matrix}
            \begin{itemize}[topsep=6pt,itemsep=4pt]      
            \item Variance of $ x_{\cdot j} $:
            \begin{align}
                s_{jj}=s^2_j=&\dfrac{1}{n}d_j'd_j=\dfrac{1}{n}\sum_{k=1}^n (x_{kj}-\bar{x}_j)^2,\quad i=1,2,\ldots p\\
                =&x_{\cdot j}'(I-\dfrac{1}{n}\mathcal{I}_n)x_{\cdot j},\quad j=1,2,\ldots,p
            \end{align}
            \item Covariance between $ x_i $ and $ x_j $:
            \begin{align}
                s_{ij}=&\dfrac{1}{n}d_i'd_j=\dfrac{1}{n}\sum_{k=1}^n(x_{ki}-\bar{x}_i)(x_{kj}-\bar{x}_j),\quad i,j=1,2,\ldots p\\
                =&x_{\cdot i}'(I-\dfrac{1}{n}\mathcal{I}_n)x_{\cdot j},\quad i,j=1,2,\ldots,p
            \end{align}
            \item Pearson's Correlation Coefficient between $ x_i $ and $ x_j $:\index{Correlation Coefficient!Pearson's Correlation Coefficient $ r $}
            \begin{equation}\label{EqaEstimatorOfCorrelationCoefficient}
                r_{ij}=\dfrac{s_{ij}}{\sqrt{s_{ii}}\sqrt{s_{jj}}}=\dfrac{{\displaystyle\sum_{k=1}^n(x_{ki}-\bar{x}_i)(x_{kj}-\bar{x}_j)}}{\sqrt{{\displaystyle\sum_{k=1}^n(x_{ki}-\bar{x}_i)^2}}\sqrt{{\displaystyle\sum_{k=1}^n(x_{kj}-\bar{x}_j)^2}}},\quad i,j=1,2,\ldots p
            \end{equation}
            \end{itemize}
        
        In condense notation, define Covariance Matrix from sample of size $ n $:
        \begin{equation}\label{EqaSampleCovarianceMatrix}
            S_n=\begin{bmatrix}
            s_{11}&s_{12}&\ldots&s_{1p}\\
            s_{21}&s_{22}&\ldots&s_{2p}\\
            \vdots&\vdots&\ddots&\vdots\\
            s_{1p}&s_{p2}&\ldots&s_{pp}\\
            \end{bmatrix}
        \end{equation}

        and sample Correlation Coefficient Matrix:\index{Correlation Coefficient!Correlation Coefficient Matrix}
        \begin{equation}
            R_n=
            \begin{bmatrix}
            r_{11}&r_{12}&\ldots&r_{1p}\\
            r_{21}&r_{22}&\ldots&r_{2p}\\
            \vdots&\vdots&\ddots&\vdots\\
            r_{1p}&r_{p2}&\ldots&r_{pp}\\
            \end{bmatrix}
        \end{equation}
        \item Generalized sample variance: $ |S_n|=\lambda _1\lambda _2 \ldots \lambda _p$, where $ \lambda_i  $ are eigenvalues of $ S_n $.
        
        \item `Statistical Distance' between vectors: to measure the difference between two vectors $ x=(x_1,x_2,\ldots,x_p) $ and $ y=(y_1,y_2,\ldots,y_p) $.
        \begin{itemize}[topsep=6pt,itemsep=4pt]
            \item Euclidean Distance:\index{Euclidean Distance}
            \begin{equation}
                d_E(x,y) =\sqrt{(x-y)^T(x-y)}
            \end{equation}
            \item \textbf{Mahalanobis Distance}\index{Mahalanobis Distance}: Scale invariant distance, and include information about relativity position:
            \begin{equation}
                d_M(x,y)=\sqrt{(x-y)'S^{-1}(x-y)} 
            \end{equation}

            % Note: $ P,Q $ are from the same distribution with covariance matrix $ S_p $. When $ S=I $, return to Euclidean distance.
            
            Remark: Mahalanobis distance is actually the normalized Euclidean distance in principal component space. So we can actually define the Mahalanobis distance for one sample case $ \vec{x}=(x_1,x_2,\ldots ,x_p) $ from distribution of $ (\vec{\mu},\Sigma)  $
            \begin{equation}\label{MahalanobisDistance}
                d_M(\vec{x})=\sqrt{(\vec{x}-\vec{\mu})^T\Sigma ^{-1}(\vec{x}-\vec{\mu})} 
            \end{equation}

            Note: the hyper-sruface $ d_M(\vec{x})=\mathrm{const} $ forms a ellipsoid.

        \end{itemize}
    \end{itemize}

\begin{point}
    \hypertarget{SampleStatisticsProperties}{Sample Statistics Properties}
\end{point}

    Consider taking an $ n $ cases sample from r.v. population $ \vec{X}=(X_1,X_2,\ldots,X_p) $, population mean $ \mu $ and covariance matrix $ \Sigma  $. Basic statistics are sample mean and sample variance
    \begin{align}
        \bar{X}=\dfrac{1}{n}X'\mathbf{1}_n, \qquad  S_n=\dfrac{1}{n}\left(X-\dfrac{1}{n}\mathcal{I}_nX\right)'\left(X-\dfrac{1}{n}\mathcal{I}_nX\right)=\dfrac{1}{n}X'\left(I-\dfrac{1}{n}\mathcal{I}_n\right)X
    \end{align}

    Properties:
\begin{align}
     \mathbb{E}\left[ \bar{X} \right] = \mu \qquad cov(\bar{X})=\dfrac{1}{n}\Sigma \qquad \mathbb{E}\left[ S_n \right] = \dfrac{n-1}{n}\Sigma 
\end{align}


    
        


\subsubsection{Review: Some Matrix Notation \& Lemma}\label{SubSubSectionMatrixNotationAndLemma}

    \begin{itemize}[topsep=6pt,itemsep=4pt]
        \item Orthonormality\index{Orthonormality}: For square matrix $ P $ satisfies:
        \begin{equation}
            x_i^Tx_j=\delta _{ij} 
        \end{equation}

        where $ x_i,x_j $ are columns of $ P $.
        \item Eigenvalue and Eigenvector\index{Eigenvalue}: For square matrix $ A $, its eigenvalues $ \lambda_i $ and corresponding eigenvectors $ e_i $ satisfies:
        \begin{equation}
            Ae_i=\lambda_ie_i,\,\forall i=1,2,\ldots p 
        \end{equation}

        Denote $ P=[e_1,e_2,\ldots ,e_p] $, which is an orthonormal matrix. And denote $ \Lambda =diag\{\lambda _1,\lambda _2,\ldots,\lambda _p\} $.
        \begin{equation}
            A=\sum_{i=1}^p\lambda _ie_ie_i^T=P \Lambda P^T=P\Lambda P^{-1}
        \end{equation}

        is called the Spectral Decomposition of $ A $

        
        
        \item Square root matrix\index{Square Root Matrix}: Def. as
        \begin{equation}
            A^{1/2}=\sum_{i=1}^p\sqrt{\lambda _i}e_ie_i^T=P\Lambda ^{1/2}P^T 
        \end{equation}

        Properties:
        \begin{itemize}[topsep=0pt,itemsep=-2pt]
            \item $ {\displaystyle A^{1/2}A^{1/2}=A} $;
            \item $ {\displaystyle A^{-1/2}=(A^{1/2})^{-1}=PL^{-1/2}}P^T $;
            \item $ tr(A) =\sum_{i=1}^n\lambda _n$;
            \item $ |A|=\prod_{i=1}^n\lambda _n $.
        \end{itemize}
        
            
        \item (Symmetric) Positive Definite Matrix: Say $ A $ a Positive Definite Matrix if\index{Positive Definite Matrix}
        \begin{equation}
            x^TAx> 0,\,\forall x\in\mathbb{R}^p 
        \end{equation}

        where $ x^TAx $ is called a Quadric Form.

        Properties:
        \begin{itemize}[topsep=6pt,itemsep=4pt]
            \item Use the Spectral Decomposition of $ A $, we can write the Quadric Form as
            \begin{equation}
                x^TAx=x^TP\Lambda P^Tx=y^T\Lambda y=\sum_{i=1}^p\lambda_iy_i^2=\sum_{i=1}^p(\sqrt{\lambda_i}y_i)^2 
            \end{equation}
            
            
            \item Eigenvalues $ \lambda _i>0,\,\forall i=1,2,\ldots,p $
            \item $ A $ can be written as product of symmetric matrix: $ A= Q^TQ$ ($ Q $ is symmetric);
        \end{itemize}

        Positive Semi-definite matrix is one with $ \lambda_i\geq 0 $

        \item Trace of Matrix\index{Trace}: For $ p\times p $ square matrix $ A $
            
            \begin{equation}
                tr(A) =\sum_{i=1}^p a_{ii}
            \end{equation}
            
            Properties:
            \begin{itemize}[topsep=2pt,itemsep=2pt]
                \item $ tr(AB)=tr(BA)  $;
                \item $ x'Ax=tr(x'Ax)=tr(Axx') $
                \item $ tr(A)=\sum_i \lambda _i $
            \end{itemize}
            
                
        \item Matrix Partition: partition square matrix $ \mathop{A}\limits_{p\times p} $ as 
        \begin{equation}
            A=         
            \begin{bmatrix}
                \mathop{A_{11} }\limits_{q_1\times q_1}&\mathop{A_{12} }\limits_{q_1\times q_2} \\
                \mathop{A_{21} }\limits_{q_2\times q_1}&\mathop{A_{22} }\limits_{q_2\times q_2}   
            \end{bmatrix}   
        \end{equation}

        where $ p=q_1+q_2 $      
        

        Property:
        \begin{equation}
            |A|= |A_{22}\Vert A_{11}-A_{12}A_{22}^{-1}A_{21}|=|A_{11}\Vert A_{22}-A_{21}A_{11}^{-1}A_{12}|
        \end{equation}
        
        

            
                       
        \item Matrix Differentiation\index{Matrix Differentiation}
        
        Calculus Notations: Take derivative of $ y=(y_1,y_2,\ldots,y_q)^T $ over $ x=(x_1,x_2,\ldots,x_p)^T $; or similarly of matrix $ A $ over scalar, etc.
        
        We use 'Denominator-layout',\index{Denominator-layout} which means the result follows the shape of denominator:
        \begin{equation}\label{EqaMatrixDifferential}
            \dfrac{\partial^{}y }{\partial ^{}x}:=\dfrac{\partial^{} y^T}{\partial x^{}} =
            \begin{bmatrix}
            \dfrac{\partial^{} y_1}{\partial x_1 ^{}}&\dfrac{\partial^{} y_2}{\partial x_1 ^{}}&\ldots&\dfrac{\partial^{} y_q}{\partial x_1 ^{}}\\
            \dfrac{\partial^{} y_1}{\partial x_2 ^{}}&\dfrac{\partial^{} y_2}{\partial x_2 ^{}}&\ldots&\dfrac{\partial^{} y_2}{\partial x_p ^{}}\\
            \vdots&\vdots&\ddots&\vdots\\
            \dfrac{\partial^{} y_1}{\partial x_p ^{}}&\dfrac{\partial^{} y_2}{\partial x_p ^{}}&\ldots&\dfrac{\partial^{} y_q}{\partial x_p ^{}}\\
            \end{bmatrix}\Leftrightarrow \left(\dfrac{\partial^{} y}{\partial x^{}}\right)_{ij}=\dfrac{\partial^{} y_j}{\partial x_i^{}}
        \end{equation}
        
        \hypertarget{MatrixDifferenciation}{Properties (under denominator-layout):}\footnote{More matrix diffrenciation equation see book \cite{线性回归分析ref2} P49. Or can be easily derivated using Einstein sumation notation.
        
        An example:
        \begin{align*}
            \dfrac{\partial^{} |A|}{\partial A^{}}=&\dfrac{\partial^{} }{\partial A_{ij}^{}}\mathrm{Ex}_{i_1,i_2,\ldots,i_n}A_{1i_i}A_{2i_2}\ldots A_{ni_n}\\
            =&\sum_{k=1}^n\mathrm{Ex}_{i_1,\ldots, (\wedge i_k),\ldots, i_n}\delta _{ki}\delta _{i_kj}A_{1i_1}\ldots (\wedge A_{ki_k})\ldots A_{ni_n}\times (-1)^{(n-k)+(n-i_k)}\\
            =& (-1)^{i+j}\mathrm{Ex}_{i_1,\ldots, (\wedge j),\ldots, i_n}A_{1i_1}\ldots (\wedge A_{ij})\ldots A_{ni_n} \\
            =&|A|A^{-1}
        \end{align*}
        
        }
        \begin{itemize}[topsep=6pt,itemsep=2pt]
            \item $ \dfrac{\partial^{} }{\partial x^{}}Ax=A^T $;\\
            \item $ \dfrac{\partial^{} }{\partial x^{}}x^TA=A $;\\
            \item $ \dfrac{\partial^{} }{\partial x^{}}x^Tx=2x $;\\
            \item $ \dfrac{\partial^{} }{\partial x^{}}x^TAx=Ax+A^Tx $;\\
            \item $ \dfrac{\partial^{} }{\partial x^{}}\log(x^TAx)=\dfrac{2Ax}{x^TAx} $;\\
            \item $ \dfrac{\partial^{} |A|}{\partial A^{}}=|A|A^{-1} $;\\
            \item $ \dfrac{\partial^{} tr(AB)}{\partial A^{}}=B^T $;\\
            \item $ \dfrac{\partial^{} tr(A^{-1}B)}{\partial A^{}}=-A^{-1}B^TA^{-1} $
        \end{itemize}
          
        
        \item Kronecker Product: For matrix $ \mathop{A}\limits_{m\times n}=\{a_{ij}\},\,\mathop{B}\limits_{p\times q}=\{b_{ij}\} $. Their Kronecker product
        \begin{equation}
            A\otimes B=\begin{bmatrix}
            a_{11}B&a_{12}B&\ldots&a_{1n}B \\
            a_{21}B&a_{22}B&\ldots&a_{2n}B \\
            \vdots&\vdots&\ddots&\vdots\\
            a_{1m}B&a_{m2}B&\ldots&a_{mn}B \\
            \end{bmatrix} 
        \end{equation}

        \item \hypertarget{NormDefinition}{Norm}: \index{Norm}
        \begin{itemize}[topsep=2pt,itemsep=0pt]
            \item Vector Norm: for vector $ x,y\in \mathbb{C}^m $, norm $ \Vert \cdot \Vert  $ is a function $ \mathbb{C}^m\to \mathbb{R} $, with:
            \begin{align}
                \text{Semi-definiteness: }&\Vert x \Vert \geq 0,\, =\text{ for }x=0\\
                \text{Absolute homogeneity: }&\Vert kx \Vert = |k|\Vert x \Vert,\, k\in \mathbb{C} \\
                \text{Triangle inequality: }& \Vert x \Vert + \Vert y \Vert \geq \Vert x+y \Vert 
            \end{align}
        
        
        the $ \ell_p $-norm of $ x $ is
        \begin{equation}
            \Vert x\Vert _p\equiv \left( \sum_{i=1}^n| x_i|^p \right)^{1/p}
        \end{equation}

        Useful norm:
        \begin{itemize}[topsep=2pt,itemsep=2pt]
            \item $ \ell_0$-norm: \# of none-0 elements in $ x $;\footnote{Note: actually triangle inequality is not satisfied for $ \Vert \cdot \Vert _0 $}
            \item $ \ell_1 $-norm: $ \Vert x\Vert _1=\sum_{i=1}^n|x_i| $;
            \item $ \ell_2 $-norm/Euclidean norm: $ \Vert x\Vert _2=\sqrt{\sum_{i=1}^n x_i^2}  $;
            \item $ \ell_\infty $-norm: $ \max |x_i| $.
        \end{itemize}
        \item Matrix Norm:  for matrix $ A,B\in \mathbb{C}^{m\times n} $, norm $ \Vert \cdot \Vert  $ is a function $ \mathbb{C}^{m\times n}\to \mathbb{R} $, with:
        \begin{align}
            \text{Semi-definiteness: }&\Vert A \Vert \geq 0,\, =\text{ for }x=0\\
            \text{Absolute homogeneity: }&\Vert kA \Vert = |k|\Vert A \Vert,\, k\in \mathbb{C} \\
            \text{Triangle inequality: }& \Vert A \Vert + \Vert B \Vert \geq \Vert A+B \Vert 
        \end{align}

        further for $ m=n $, i.e. $ A,B\in \mathbb{C}^{m\times m} $, usually append
        \begin{align}
            \text{Sub-multiplicative: }&\Vert A \Vert \Vert B \Vert \geq \Vert AB \Vert \\
            \text{Hermite: }&\Vert A \Vert =\Vert A^* \Vert 
        \end{align}

        Matrix norm induced by vector norm:
        \begin{equation}
            \Vert A \Vert =\max \dfrac{\Vert Ax \Vert }{\Vert x \Vert } 
        \end{equation}
        
        e.g. $ \ell_p $ induced matrix norm:
        \begin{itemize}[topsep=2pt,itemsep=2pt]
            \item $ \ell_1 $-norm: $ \Vert A\Vert _1=\mathop{\max}\limits_{1\leq j\leq n}\sum\limits_{i=1}^m|A_{ij}| $
            \item $ \ell_2 $-norm/Euclidean norm: $ \Vert A\Vert _2=\sigma_{\max}(A) $;
            \item $ \ell_\infty $-norm: $ \Vert A\Vert _\infty=\mathop{\max}\limits_{1\leq i\leq m}\sum\limits_{j=1}^n|A_{ij}| $.
        \end{itemize}

        Non-induced matrix norm, e.g. 
        \begin{itemize}[topsep=2pt,itemsep=0pt]
            \item Frobenius norm: $ \Vert A \Vert _F=\left(\sum_{i=1}^m\sum_{j=1}^n |A_{ij}|^2 \right)^{1/2} =\sqrt{tr(A^*A)}$
            \item Weighted Frobenius norm: $ \Vert A \Vert _W=\Vert W^{-1/2}AW^{-1/2} \Vert _F $( or some textbooks uses $ \Vert W^{1/2}AW^{1/2} \Vert_F  $)
            \item Max norm: $ \Vert A \Vert_{\max}=\mathop{\max}\limits_{i,j} |A_{ij}|  $
        \end{itemize}
        
        % Useful Inequalities for $ A\in\mathbb{C}^{m\times n} $:
        % \begin{itemize}[topsep=2pt,itemsep=0pt]
        %     \item $ \Vert A \Vert _2\leq\Vert A \Vert _F\leq\sqrt{n}\Vert A \Vert _2 $\\
        % \end{itemize}
        
            
        \end{itemize}
    \item Sherman-Morrison Formula:
    \begin{align}
         \left(A+u^Tv\right)^{-1}=A^{-1}-\dfrac{A^{-1}u^TvA^{-1}}{1+v^TA^{-1}u} 
    \end{align}
    
    Or in matrix form:
    \begin{align}
         \left(A+B\right)^{-1}=A^{-1}-\dfrac{A^{-1}BA^{-1}}{1+tr(A^{-1}B)},\quad \mathrm{rank}(B)=1 
    \end{align}
    
    Application instances see \url{https://v1ncent19.github.io//texts/MahalanobisAndLeverage/} and \url{https://v1ncent19.github.io//texts/DeletedResidual/}.
    
    
    \end{itemize}
    
        


    \subsubsection{Useful Inequalities}
    \begin{itemize}[topsep=6pt,itemsep=4pt]
        \item Cauchy-Schwartz Inequality:\index{Inequality!Cauchy-Schwarz Inequality}
        
        Let $ b,d$ any $ p\times 1 $ vectors.
        \begin{equation}
            (b'd)^2\leq (b'b)(d'd) 
        \end{equation}
        
        \item Extended Cauchy-Schwartz Inequality: 
        
        Let $ B $ be a positive definite matrix.
        
        \begin{equation}
            (b'd)^2\leq(b'Bb)(d'B^{-1}d) 
        \end{equation}
        
        \item Maximazation Lemma:\index{Inequality!Maximazation Lemma}
        
        $ d $ be a given vector, for any non-zero vector $ x $,
        \begin{equation}
            \dfrac{(x'd)^2}{x'Bx}\leq d'B^{-1}d 
        \end{equation}

        Take Maximum when $ x=cB^{-1}d $.
        
        
    \end{itemize}

    % note: 无法用地位投影寻找高微离群值
        

\subsection{Statistical Inference to Multivariate Population}
    Statistics model: a $ n $ cases sample $ \mathbf{X}_1,\mathbf{X}_2,\ldots,\mathbf{X}_n $, where each $ \mathbf{X}_i $ i.i.d. from a multivariate population (usually consider a multi-normal). i.e.
    \begin{equation}\label{EqaNPSampleMatrixNotation}
        \mathbf{X}=\begin{bmatrix}
            X_{11}&X_{12}&\ldots&X_{1p}\\
            X_{21}&X_{22}&\ldots&X_{2p}\\
            \vdots&\vdots&\ddots&\vdots\\
            X_{1n}&X_{n2}&\ldots&X_{np}\\
            \end{bmatrix} 
            =
            \begin{bmatrix}
                \mathbf{X}_1'\\
                \mathbf{X}_2'\\
                \vdots\\
                \mathbf{X}_n'
            \end{bmatrix}
    \end{equation}



\subsubsection{Multivariate Normal Distribution}\label{SubSubSectionMultivariateNormalDistribution}
    Univariate Noraml Distribution: $ N(\mu,\sigma^2) $
    \begin{equation}
        f_X(x)=\dfrac{1}{\sqrt{2\pi\sigma ^2}}\exp{-\dfrac{(x-\mu)^2}{2\sigma ^2}} 
    \end{equation}
    
    Multivariate Normal Distribution: $X\sim N_p(\vec{\mu},\Sigma) $\footnote{Detailed derivation see \autoref{SubsectionDerivationMultivariateNormal}}\index{Distribution!Multivariate Normal Distribution}
    \begin{equation}
        f_\mathbf{X}(\vec{x})=\dfrac{1}{(2\pi)^{p/2}|\Sigma |^{1/2}}\exp\left({-\dfrac{(\vec{x}-\vec{\mu})'\Sigma^{-1}(\vec{x}-\vec{\mu})}{2}} \right)
    \end{equation}

    Note: Here in the $ \exp $, the $ (\vec{x}-\vec{\mu})'\Sigma^{-1}(\vec{x}-\vec{\mu}) $ is the Mahalanobis Distance $ d_M $ defined in \autoref{MahalanobisDistance}

    % Further denote $ \mathop{Y}\limits_{q\times 1}=\mathop{A}\limits_{q\times p}\mathop{X}\limits_{p\times 1} $, where $ A $ is a const matrix. Then 
    % \begin{equation}
    %     Y=AX\sim N_q(A\vec{\mu},A\Sigma A^T) 
    % \end{equation}
    
    

    Remark: A $ n $-dimension multivariate normal has $ \dfrac{p(p+1)}{2} $ free parameters. Thus for a very high dimension, contains too many free parameters to be determined! 
    
    Properties: Consider $ X\sim N_p(\mu,\Sigma) $
    \begin{itemize}[topsep=6pt,itemsep=4pt]
        \item Linear Transform:
        \begin{itemize}[topsep=6pt,itemsep=4pt]       
        \item For a $ p\times 1 $ vector $ a $:
        \begin{equation}
            X\sim N_p(\mu,\Sigma )\Leftrightarrow a'X\sim N(a'\mu,a'\Sigma a),\,\forall a\in\mathbb{R}^p 
        \end{equation}

        (Proof: use characteristic function.)
        
        \item For a $ q\times p $ const matrix $ A $:
        \begin{equation}\label{EqaTransformOfMultiNormal}
            AX+a\sim N_q(A\mu+a,A\Sigma  A')
        \end{equation}
        \item For a $ p\times p    $ square matrix $ A $:
        
        \begin{equation}\label{EqaExpectationOfQuadric}
            \mathbb{E}(X'AX)= \mu'A\mu +tr(A\Sigma )            
        \end{equation}
        
        
        \end{itemize}
        \item Conditional Distribution: Take partition of $ \mathop{X}\limits_{p\times 1}\sim N(\mathop{\mu}\limits_{p\times 1},\mathop{\Sigma }\limits_{p\times p}) $ into $ \mathop{X_1}\limits_{q_1\times 1} $ and $ \mathop{X_2}\limits_{q_2\times 1}  $, where $ q_1+q_2=p $. Write in matrix form:
        \begin{equation}
            \mathop{X}\limits_{p\times 1}=
            \begin{bmatrix}
                \mathop{X_1}\limits_{q_1\times 1}\\
                \mathop{X_2}\limits_{q_2\times 2}  
            \end{bmatrix}  
            \qquad 
            \mathop{\mu}\limits_{p\times 1}=
            \begin{bmatrix}
                \mathop{\mu_1 }\limits_{q_1\times 1}\\
                \mathop{\mu_2 }\limits_{q_2\times 2}  
            \end{bmatrix}  
            \qquad             
            \mathop{\Sigma }\limits_{p\times p}=
            \begin{bmatrix}
                \mathop{\Sigma_{11} }\limits_{q_1\times q_1}&\mathop{\Sigma_{12} }\limits_{q_1\times q_2} \\
                \mathop{\Sigma_{21} }\limits_{q_2\times q_1}&\mathop{\Sigma_{22} }\limits_{q_2\times q_2}   
            \end{bmatrix}  
            \qquad 
        \end{equation}
        
            i.e. 
        \begin{equation}
            \mathop{X}\limits_{p\times 1}=\begin{bmatrix}
                \mathop{X_1 }\limits_{q_1\times 1}\\
                \mathop{X_2 }\limits_{q_2\times 2}  
            \end{bmatrix}  
            \sim
            N_{q_1+q_2}\left(\begin{bmatrix}
                \mathop{\mu_1 }\limits_{q_1\times 1}\\
                \mathop{\mu_2 }\limits_{q_2\times 2}  
            \end{bmatrix},\begin{bmatrix}
                \mathop{\Sigma_{11} }\limits_{q_1\times q_1}&\mathop{\Sigma_{12} }\limits_{q_1\times q_2} \\
                \mathop{\Sigma_{21} }\limits_{q_2\times q_1}&\mathop{\Sigma_{22} }\limits_{q_2\times q_2}   
            \end{bmatrix}  
                \right)
        \end{equation}
            
        Independence: $ X_1\parallel X_2\Leftrightarrow \Sigma _{21}=\Sigma _{12}^T=0  $

        And the conditional dictribution $ X_1|X_2=x_2 $ is given by \footnote{In \autoref{EqaTransformOfMultiNormal}, take 
        \begin{equation}
            \mathop{A}\limits_{p\times p}=\begin{bmatrix}
                \mathop{I}\limits_{q\times q} & -\mathop{\Sigma _{12}\Sigma _{22}^{-1}}\limits_{q\times (p-q)} \\
                \mathop{0}\limits_{(p-q)\times q}&\mathop{I}\limits_{(p-q)\times (p-q)}  
            \end{bmatrix}  
        \end{equation}
        
        }
        \begin{equation}\label{EquConditionalPrForGaussian}
            X_1|_{X_2=x_2}\sim N_p(\mu_1+\Sigma _{12}\Sigma _{22}^{-1}(x_2-\mu_2),\,\Sigma _{11}-\Sigma _{12}\Sigma _{22}^{-1}\Sigma _{21})
        \end{equation}

        \item Multivariate Normal \& $ \chi^2 $
        
         Let $ X\sim N_p(\mu,\Sigma ) $, then 
         \begin{equation}
             (X-\mu)^T\Sigma ^{-1}(X-\mu)\sim \chi_p^2 
         \end{equation}
         
        %  \item Linear Combination:
        % Let $ X_1,X_2\ldots,X_n $ with $ X_i\sim N_p(\mu_i,\Sigma ) $ (different $ \mu_i $, same $ \Sigma  $). And denote $ V_1=\sum_{i=1}^nc_iX_i $, then
        % \begin{equation}
        %     V_1\sim N_p(\sum_{i=1}^n c_i\mu_i,\sum_{i=1}^nc_i^2\Sigma ) 
        % \end{equation}
        
        
        
        
        
        
    \end{itemize}
    
        






    % \begin{point}
    %     Problem: Property of 2-D Normal:
    %     \begin{equation}
    %         corr(X,Y)=\rho\Rightarrow corr(X^2,Y^2)=\rho ^2 
    %     \end{equation}
    % \end{point}

    
    
\subsubsection{MLE of Multivariate Normal}
        \index{MLE (Maximum Likelihood Estimation)}
    Under the notation in \autoref{EqaNPSampleMatrixNotation}, i.e. each sample case $ \mathbf{X}_i$ i.i.d. $\sim N_p(\mu,\Sigma ) $, we can get the joint PDF of $ \mathbf{X} $:
    \begin{equation}
        f_{\mathbf{X_1},\ldots,\mathbf{X_n};\mu,\Sigma }(x_1,\ldots,x_n)=\dfrac{1}{(2\pi)^{np/2}|\Sigma |^{n/2}}\exp\left( -\sum_{i=1}^n\dfrac{(x_i-\mu)'\Sigma ^{-1}(x_i-\mu)}{2} \right) 
    \end{equation}
  
    and at the same time get likelihood function\footnote{Here we need to use the property of trace
    \begin{equation}
        x'Ax=tr(x'Ax)=tr(Ax'x)
    \end{equation}    }:
    
    \begin{equation}
        L(\mu ,\Sigma;x_1,\ldots,x_n)=\dfrac{1}{(2\pi)^{np/2}|\Sigma |^{n/2}}\exp\left[ -\dfrac{1}{2}tr\left( \Sigma ^{-1} \left(\sum_{i=1}^n(x_i-\bar{x})(x_i-\bar{x})'+n(\bar{x}-\mu)(\bar{x}-\mu)' \right) \right) \right]
    \end{equation}
        And we can get the MLE of $ \mu $ and $ \Sigma  $ as follows\footnote{Detailed proof see '\textit{Applied Multivariate Statistical Analysis}' P130}:
        \begin{align}
            \hat{\mu}&= \dfrac{1}{n}\sum_{i=1}^n x_i=\bar{x} \\
            \hat{\Sigma  }&= \dfrac{1}{n}\sum_{i=1}^n(x_i-\bar{x})(x_i-\bar{x})'=\dfrac{n-1}{n}S
        \end{align}

        \fbox{
            \begin{minipage}{0.9\linewidth}
                $ \Delta $ \textbf{Note: }In this section, $ S $ is used to denote $ \hat{\Sigma } $, which is different from that in \autoref{SubSectionStatistics} ($ S^2 $ for $ \hat{\Sigma } $)
            \end{minipage}
        }\\
    
    And we can furthur construct MLE of function of $ \mu,\,\Sigma  $ (use invariance property of MLE), for example 
    
    \begin{equation}
        \widehat{|\Sigma |} =|\hat{\Sigma } | 
    \end{equation}
    
    
        Note: $ (\hat{\mu} , \hat{\Sigma} ) $ is sufficient statistic of multi-normal population.






%Consistency

    % Consistency: Ensuring that when we get more data point, weare 'closer' to the real case.
    % \begin{itemize}[topsep=2pt,itemsep=2pt]
    %     \item Weak consistency:
    %     \begin{equation}
    %         \lim_{n\to\infty}P(\Vert \hat{\mu}-\mu\Vert >\varepsilon )=0 
    %     \end{equation}
    %     \item Strong consistency:
    %     \begin{equation}
    %         \hat{\mu}\xrightarrow[]{\mathrm{a.s.}} \mu 
    %     \end{equation}
    % \end{itemize}
    
        
\subsubsection{Sampling distribution of $ \bar{X} $ and $ S $}\label{SubSubSectionMultivariateNormalSamplingDistribution}
        $ \hat{\mu}=\bar{X} $ and $ \hat{\Sigma}=\dfrac{n-1}{n}S $ are statistics, with sampling distribution.



    \begin{point}
        Sampling distribution of $ \bar{X} $
    \end{point}

    Similar to monovariate case:
    \begin{equation}
        \bar{X}\sim N_p(\mu,\dfrac{1}{n}\Sigma ) 
    \end{equation}
    
    \begin{point}
        Sampling distribution of $ S^2 $
    \end{point}
    
\begin{itemize}[topsep=2pt,itemsep=2pt]
    \item Monovariate case: Consider $ (X_1,X_2,\ldots,X_n) $ i.i.d. $ \sim N(\mu,\sigma ^2) $

    % Then $ \bar{x}=\dfrac{1}{n}\sum_{i=1}^nx_i $, $ S^2=\dfrac{1}{n-1}\sum_{i=1}^n(x_i-\bar{x})^2 $
    
    % Define an orthogonal matrix
    % \begin{equation}
    %     Q=\begin{bmatrix}
    %         \dfrac{1}{\sqrt{n}}&\dfrac{1}{\sqrt{n}}&\ldots&\dfrac{1}{\sqrt{n}}\\
    %         &&&\\
    %         &&&\\
    %         &&&
    %     \end{bmatrix} _{n\times n}
    % \end{equation}
    
    % and def 
    % \begin{equation}
    %     Y=QX\sim N(Q\mathbf{1}_n\mu,\sigma^2I) =N(\begin{bmatrix}
    %         \sqrt{n}\mu\\0\\ \vdots\\0
    %     \end{bmatrix})
    % \end{equation}

    Then 
    \begin{equation}
        \dfrac{(n-1)S}{\sigma ^2}\sim \chi^2_{n-1} 
    \end{equation}
    
    \item Multivariate case: Consider $ (\mathbf{X}_{1},\mathbf{X}_{2},\ldots,\mathbf{X}_{n})  $ i.i.d. $ \sim N_p(\mu,\Sigma ) $
    
    Then
    \begin{equation}
        (n-1) S\sim W_p(n-1,\Sigma )
    \end{equation}
    
    Where $ W_p(n-1,\Sigma ) $ is Wishart Distribution,\index{Distribution!Wishart Distribution}\index{Wishart Distribution} details as follows:

         For r.v. $ Z_1,Z_2,\ldots,Z_m $ i.i.d. $ \sim N_p(0,\Sigma  ) $, def $ p $ dimensional \textbf{Wishart Distribution } with dof $ m $ as $ W_p(m,\Sigma ) $.\footnote{$ W_p(m,\Sigma ) $ is a distribution defined on $ p\times p $ matrix space.}
        \begin{equation}
            W_p=\sum_{i=1}^nZ_iZ_i' 
        \end{equation}

        
        PDF of $ W_p(m,\Sigma ) $:
        \begin{equation}
            f_W(w;p,m,\Sigma )= \dfrac{|w|^{\frac{m-p-1}{2}}\exp\left( -\dfrac{1}{2}tr(\Sigma ^{-1}w) \right)}{2^{\frac{mp}{2}}|\Sigma |^{-1/2}\pi^{\frac{p(p-1)}{4}}{\displaystyle\prod_{i=1}^p\Gamma (\dfrac{m-i+1}{2})} }
        \end{equation}

 
        
        C.F.
        \begin{equation}
            \phi(T)=|I_p-2i\Sigma T|^{-\frac{m}{2}} 
        \end{equation}
    % \begin{align}
    %     \sum_{i=1}^nY_iY_i'=\sum_{i=1^n}X_iX_i'=\sum_{i=1}^n(X_i-\bar{X})(X_i-\bar{X})'+n\bar{X}\bar{X}'=(n-1)S+Y_1Y_1' \\
    %     \Rightarrow (n-1)S=\sum_{i=2}^nY_iY_i'\parallel \bar{X}=\dfrac{1}{\sqrt{n}}Y_1
    % \end{align} 

    % Then consider the distribution of $ {\displaystyle\sum_{i=2}^nY_iY_i'} \sum W_p(n-1,\Sigma )$, which is Wishart distribution.

    Properties:
    \begin{itemize}[topsep=2pt,itemsep=2pt]
        \item For independent $ A_1\sim W_p(m_1,\Sigma ) $ and $ A_2\sim W_p(m_2,\Sigma ) $, then 
        \begin{equation}
            A_{1}+A_{2} \sim W_p(m_1+m_2,\Sigma )
        \end{equation}
        
        \item For $ A\sim W_p(m,\Sigma ) $, then
        \begin{equation}
            CAC'\sim W_p(m,C\Sigma C') 
        \end{equation}
        \item Wishart distribution is the matrix generization of $ \chi^2_n $. When $ p=1 $, $ \Sigma =\sigma ^2=1 $, $ W_p(m,\Sigma ) $ naturally reduce to $ \chi^2_m $.
        \begin{equation}
            \chi^2_n=W_1(n,1) 
        \end{equation}
        

\begin{rcode}
    Distribution functions are in package \lstinline|MCMCpack|, or use \lstinline|rWishart()| function.
\end{rcode}
        

    
    
\end{itemize}
    
\end{itemize}

    

\begin{point}
    Large sample $ \bar{X} $ and $ S $
\end{point}
\begin{itemize}[topsep=2pt,itemsep=2pt]
    \item $ \sqrt{n}(\bar{X}-\mu)\xrightarrow[]{\mathrm{d}} N_p(0,\Sigma ) $;
    \item $ n(\bar{X}-\mu)'S ^{-1}(\bar{X}-\mu)\xrightarrow[]{\mathrm{d}} \chi_p^2 $
\end{itemize}

    


% Stein's method


\subsubsection{Hypothesis Testing for Normal Population}\label{SubSectionMultivariateHypothesisTesting}
\begin{itemize}[topsep=2pt,itemsep=2pt]
    \item \textbf{One-Population Hypothesis Testing}: 
    
    Conduct hypothesis testing to $ \mu $:
    \begin{equation}
        H_0: \mu=\mu_0\longleftrightarrow H_1:\mu\neq \mu_0
    \end{equation}

\begin{point}
    Hotelling's $ T^2 $ test
\end{point}

    
    \begin{itemize}[topsep=2pt,itemsep=2pt]
        \item One-Dimensional case: $ t $-test
        \begin{equation}
            T=\dfrac{\sqrt{n}(\bar{X}-\mu_0)}{S}\sim t_{n-1}
        \end{equation}
        
        i.e.
        \begin{equation}
            T^2=[\sqrt{n}(\bar{X}-\mu_0)]S^{-1}[\sqrt{n}(\bar{X}-\mu_0)] \sim t^2_{n-1}=F_{1,n-1}
        \end{equation}


        \item Multi-Dimensional case: Hotelling's $ T^2 $\index{Hotelling's $ T^2 $}
        \begin{equation}
            T^2 =[\sqrt{n}(\bar{X}-\mu_0)']S^{-1}[\sqrt{n}(\bar{X}-\mu _0)] \sim \dfrac{p}{n-p}(n-1)F_{p,n-p}
        \end{equation}

        And we can get the distribution of \textbf{Hotelling's $ T^2 $}:
         \begin{equation} \dfrac{n-p}{p}\dfrac{T^2}{n-1}\sim F_{p,n-p} \end{equation}

        Rejection Rule:
        \begin{equation}
            T^2>\dfrac{p(n-1)}{n-p}F_{p,n-p,\alpha } 
        \end{equation}
        
        

        Property:

        Invariant for $ X $ transform: For $ Y=CX+d $, then 
            
            \begin{equation}
                T^2_Y=n(\bar{X}-\mu_0)'S^{-1}(\bar{X}-\mu_0)=T^2_X 
            \end{equation}
    \end{itemize}      
            
\begin{point}
    \hypertarget{PartHotellingT2Test}{LRT of $ \hat{\mu} $}
\end{point}

    Monovariate case see \autoref{SubSectionLRT}.

    LRT uses the statistic:
    \begin{equation}
        \Lambda =\dfrac{\max_{H_0 }L(\mu_0,\Sigma)}{\max_{H_0\cup H_1}L(\mu,\Sigma)}=(1+\dfrac{T^2}{n-1})^{-n/2} 
    \end{equation}

    where $ T^2=n(\bar{x}-\mu_0)'S^{-1}(\bar{x}-\mu_0) $
    
\item \textbf{Two-Population Hypothesis Testing}:

    Conduct hypothesis testing to $ \delta =\mu _1-\mu _2 $:
\begin{equation}
    H_0: \delta =\delta _0\longleftrightarrow H_1:\delta \neq \delta _0
\end{equation}

    Notation: The two sample of size $ n_1,n_2 $, each denoted as
    \begin{equation}
        X_{1,ij}\qquad X_{2,ij} 
    \end{equation}
    
    with mean $ \mu_1,\mu_2 $ and covariance matrix $ \Sigma_1,\Sigma_2 $

    \begin{itemize}[topsep=2pt,itemsep=2pt]
        \item Paired Samples: $ n_1=n_2 $
        
        For two paires samples $ \{X_{1,ij}\} $, $ X_{2,ij} $, take subtraction as 
        \begin{equation}
            D_{ij}=X_{1,ij}-X_{2,ij} 
        \end{equation}

        denote $ \bar{D}=\dfrac{1}{n}\sum_{j=1}^nD_{ j} $, $ S^2_D=\dfrac{1}{n-1}\sum_{j=1}^n(D_j-\bar{D}'(D_j-\bar{D})) $
        
        and conduct test to 
        \begin{equation}
            H_0: \bar{D} =\delta _0\longleftrightarrow H_1:\bar{D} \neq \delta _0
        \end{equation}

        And the folloeing steps are as in One-population testing, test
        \begin{equation}
            T^2=n(\bar{D}-\delta )'(S^2_D)^{-1}(\bar{D}-\delta )\sim \dfrac{(n-1)p}{n-p}F_{p,n-p}
        \end{equation}
        
        \item Under Equal Unknown Variance: $ \Sigma_1=\Sigma_2 $
        
        \begin{align}
            \bar{X}_1&=\dfrac{1}{n_1}\sum_{j=1}^{n_1} X_{1,j}&\bar{X}_2&=\dfrac{1}{n_2}\sum_{j=1}^{n_2} X_{1,j}\\
            S_1&=\dfrac{1}{n_1-1}\sum_{j=1}^{n_1}(X_{1,j}-\bar{X}_1)(X_{1,j}-\bar{X}_1)'&S_2&=\dfrac{1}{n_2-1}\sum_{j=1}^{n_2}(X_{2,j}-\bar{X}_2)(X_{2,j}-\bar{X}_2)'
        \end{align}
        
        And denote pooled variance
        \begin{equation}\label{EqaPooledVariance}
            S_\mathrm{pooled}=\dfrac{1}{n_1+n_2-2}\left((n_1-1)S_1+(n_2-1)S_2 \right)  \sim \dfrac{W_p(n_1+n_2-2,\Sigma )}{n_1+n_2-2}
        \end{equation}
        
        
        Under $ H_0 $, we have 
        \begin{equation}
            T^2= \dfrac{1}{\frac{1}{n_1}+\frac{1}{n_2}}(\bar{X}_1-\bar{X}_2-\delta _0)'S_\mathrm{pooled}^{-1}(\bar{X}_1-\bar{X}_2-\delta _0)\sim \dfrac{p(n_1+n_2-2)}{n_1+n_2-p-1}F_{p,n_1+n_2-p-1}
        \end{equation}
        

    \end{itemize}
    
    


\end{itemize}




    


\subsubsection{Confidence Region}
    \index{Confidence Region}

    Estimate the confidence region for $ \mu $ of $ X\sim N_p(\mu,\Sigma ) $, Monovariate case see \autoref{SubSectionConfidenceIntervalForDistributions}
\begin{itemize}
    \item Confidence Region:

    Also use Hotelling's $ T^2 $
    \begin{equation}
         \dfrac{n-p}{p}\dfrac{T^2}{n-1}\sim F_{p,n-p}
    \end{equation}
    
    And take $ 100(1-\alpha )\% $ confidence region of $ \mu $ as
    \begin{equation}
        R(x)=\{x|T^2\leq c^2\}\qquad c^2=\dfrac{p}{n-p}(n-1)F_{p,n-p,\frac{\alpha }{2}} 
    \end{equation}

    The shape of $ R(x) $ is an ellipsoid.
    


    \item Individual Converage Interval\index{Confidence Region!Individual Converage Interval}
    
    Use the decomposition of $ S^2 $ as a positive finite matrix $ S^2=A^TA $, where $ A $ is some $ p\times p $ matrix, then
    \begin{equation}
        T^2= [\sqrt{n}(\bar{X}-\mu_0)']S^{-1}[\sqrt{n}(\bar{X}-\mu _0)]=[A^{-1\prime}\sqrt{n}(\bar{X}-\mu_0)]'[A^{-1\prime}\sqrt{n}(\bar{X}-\mu_0)]
    \end{equation}

    Thus denote $ Z=A^{-1\prime}(X-\mu_0) \sim N_p(0,A^{-1\prime}\Sigma A^{-1}) $, the $ T^2 $ estimator of $ Z $ would be
    \begin{equation}
        T_Z^2=[\sqrt{n}\bar{Z}]'S^{-1}_Z [\sqrt{n}\bar{Z}]=n\bar{Z}'\bar{Z}=\dfrac{1}{n}\sum_{i=1}^n\bar{Z}_i^2\sim F_{p,n-p}
    \end{equation}

    As a simplified case, we can take the \textbf{Individual Converage Interval} of $ Z_i $, which is 
    \begin{equation}
        \dfrac{\sqrt{n}Z_i}{s_{Z_i}}\sim t_{n-1} 
    \end{equation}
    
    And we can take the Confidence Region\footnote{The confidence region of $ Z $ can be transformed to that of $ X $ using $ \hat{Z}=A^{-1\prime}(\hat{X}-\bar{X}) $. } as
    \begin{equation}\label{EqaConfidenceRegionUsingBonferroni}
        R(z)= \bigotimes\limits_{i=1}^n(\bar{Z}_i\pm s_{Z_i}t_{n-1,\frac{\beta }{2}})
    \end{equation}
    
    where $ \beta  $ taken with Bonferroni correction\index{Bonferroni Correction}
    \begin{equation}
        1-p\beta =1-\alpha  
    \end{equation}
    
    Note: Consider that
    \begin{equation}
        P(\text{all }Z_i\text{ in CI}_i)\geq 1-m\beta =1-\alpha  
    \end{equation}

    So the real CR for $ \mu $ should be larger.
    

    The shape of $ R(x) $ is an oblique cubold.
    
\end{itemize}
        





\subsubsection{Large Sample Multivariate Inference}
    Basic point:
    \begin{equation}
        \bar{X}\xrightarrow[]{\mathrm{d}} \mu\qquad S\xrightarrow[]{\mathrm{d}} \Sigma  
    \end{equation}
\begin{itemize}[topsep=2pt,itemsep=2pt]
    \item One-sample Mean:
    
    \begin{equation}
        n(\bar{X}-\mu)S^{-1}(\bar{X}-\mu)\xrightarrow[]{\mathrm{d}} \chi^2_p 
    \end{equation}

    \item Unequal Variance Two-sample Mean:
    
    \begin{equation}
        \bar{X}_1-\bar{X}_2\xrightarrow[]{\mathrm{d}} N\left(\mu_1-\mu _2,\dfrac{1}{n_1}\Sigma _1+\dfrac{1}{n_2}\Sigma _2\right) \qquad \dfrac{1}{n_1}S_1+\dfrac{1}{n_2}S_2\xrightarrow[]{\mathrm{d}} \dfrac{1}{n_1}\Sigma _1+\dfrac{1}{n_2}\Sigma _2
    \end{equation}

    Test:
    \begin{equation}
        T^2=\left[(\bar{X}_1-\bar{X}_2)-(\mu _1-\mu _2) \right]'(\dfrac{1}{n_1}S_1+\dfrac{1}{n_2}S_2)^{-1}\left[(\bar{X}_1-\bar{X}_2)-(\mu _1-\mu _2) \right]\xrightarrow[]{\mathrm{d}} \chi^2_p
    \end{equation}
    
    
    
    
\end{itemize}

    

    
    
    
    

\subsection{Principal Component Analysis}\label{EqaCurseOfDimensionality}
\fbox{
    \begin{minipage}{0.9\linewidth}
    PCA and next subsection \hyperlink{SubSectionFA}{FA} focus on data dimension reduction. Why?

\begin{point}
    `Curse of Dimensionality'
\end{point}
\index{Curse of Dimensionality}
    \begin{itemize}[topsep=2pt,itemsep=2pt]
        \item Difficulty in computation complexity: Many algorithms has complexity $ O(n^2) $ or more, high dimension $ n $ cause high complexity.
        \item Hughes Phenomenon: As the number of feature dimension increases, the classifier's performance increases as well until an optimal dimension. Adding more features based on the same size as the training set will then degrade the classifier's performance. 
        \footnote{Example: Volumn of unit sphere in $ n $-dim space
        \begin{equation}
            V_n=\pi^{n/2}\dfrac{1}{\Gamma (1+n/2)}\to \left(\dfrac{2\pi e}{n} \right)^{n/2}\to 0
        \end{equation}
        
        i.e. data will naturally become `sparse' in high dimension data $ \to $ difficult to extract information.}
    \end{itemize}

    \end{minipage}
}\\

    Key Idea of PCA: Find the components most powerful in explaining variance. (Similar to the idea of ANOVA)\index{PCA (Principal Component Analysis)}

\subsubsection{Population Principal Component}
    For population $ \vec{X}=(X_{1},X_{2},\ldots,X_{p})\sim (\mu,\Sigma )_p  $, conduct spectrum decomposition to $ \Sigma  $ such that
    \begin{equation}
        \Sigma P=P\Lambda \qquad P=\begin{bmatrix}
            e_1,e_2,\ldots,e_p
        \end{bmatrix} 
        \quad \Lambda =\mathrm{diag} \{\lambda _1,\lambda _2,\ldots,\lambda _p\},\, \lambda _1\geq\lambda _2\geq\ldots\geq\lambda _p
    \end{equation}

    where $ (\lambda _i,e_i) $ is the $ i^\mathrm{th} $ eigenvalue-eigenvector pair of $ \Sigma  $, large $ \lambda _i $ suggests $ X $ is more `extended' in $ e_i $ direction(large variance).
    
    Then the \textbf{Principal Components} $ Y=\{Y_i\} $ 
    \begin{align}
        Y&=P'X\sim (P'\mu,P'\Sigma P)_p=(P'\mu,\Lambda )\\
        &\begin{cases}
            Y_1=e_1'X\sim (e_1'\mu,\lambda _1)\\
            \vdots\\
            Y_p=e_p'X\sim (e_p'\mu,\lambda _p)
        \end{cases}
    \end{align}

    Properties \& Definitions:
    \begin{itemize}[topsep=2pt,itemsep=2pt]
        \item Trace of cov. matrix:
        \begin{equation}
            \sum_{i=1}^p\sigma _{ii}=\sum_{i=1}^pvar(X_i)=\sum_{i=1}^pvar(Y_i)=\sum_{i=1}^p\lambda_i
        \end{equation}
        \item $ corr $ between $ Y_i,X_j $:
        \begin{equation}
            \rho_{Y_i,X_j}=\dfrac{cov(Y_i,X_j)}{\sqrt{\lambda _i}\sqrt{\sigma _{jj}}}=\dfrac{(e_{i})_j\sqrt{\lambda _i}}{\sqrt{\sigma _{jj}}} 
        \end{equation}
        \item Factor Loading:\index{Factor Loading}
        \begin{equation}
             \mathrm{FL}_{ij}=(e_{i})_j\sqrt{\lambda _i}
        \end{equation}
        \item PC Score:
        \begin{equation}
            \mathrm{PC\, Score}_i=Y_i=e_i'X \text{ or } Y_i=e_i'(X-\mu)
        \end{equation}
        
        
    \end{itemize}
    
    In practice, we pick the first several $ m $ PC such that
    \begin{equation}
        \sum_{i=1}^m\dfrac{\lambda _i}{\sum\limits_{k=1}^p\lambda _k}\,\text{large enough} 
    \end{equation}
    
    

    Note: Another important point for PCA is the \textbf{interpretability} of principal components.

    A coutinuous version of PCA in stochastic process is Karhunen-Loève Expansion in \autoref{SubSubSectionKLExpansion}.

    \begin{point}
        Standardized Principal Component
    \end{point}

    To cancel out the influence due to scale, we can also obtain standardized PC from $ Z=(V)^{-1/2}(X_\mu) $, where $ V $ is standard deviation matrix as def. in \autoref{EqaStandardDeviationMatrix}.

    And we have $ \vec{Z}=(Z_{1},Z_{2},\ldots,Z_{p}) \sim N_p(0,V^{-1/2}\Sigma V^{-1/2\prime})=N_p(0,\rho ) $. Then obtain $ (\lambda _i,e_i) $ pairs\footnote{The eigenvalue-eigenvector pairs obtained from $ \rho  $ is generally \textbf{different}  from $ \Sigma  $.} from $ \rho  $ to form PC.
    \begin{equation}
        \rho  P=P\Lambda \qquad P=\begin{bmatrix}
            e_1,e_2,\ldots,e_p
        \end{bmatrix} 
        \quad \Lambda =diag\{\lambda _1,\lambda _2,\ldots,\lambda _p\},\, \lambda _1\geq\lambda _2\geq\ldots\geq\lambda _p
    \end{equation}
    Then the Principal Components $ W=\{W_i\} $ 
    \begin{align}
        W&=P'Z\sim (0,P'\rho  P)_p=(0,\Lambda  )\\
        &\begin{cases}
            W_1=e_1'Z\sim (0,\lambda _1)\\
            \vdots\\
            W_p=e_p'Z\sim (0,\lambda _p)
        \end{cases}
    \end{align}

    Properties:
    \begin{itemize}[topsep=2pt,itemsep=2pt]
        \item Trace of cov. matrix:
        \begin{equation}
            \sum_{i=1}^pvar(Z_i)=\sum_{i=1}^pvar(W_i)=\sum_{i=1}^p\lambda_i=p
        \end{equation}
        \item $ corr $ between $ Y_i,X_j $:
        \begin{equation}
            \rho_{W_i,Z_j}=(e_{i})_j\sqrt{\lambda _i}
        \end{equation}     
    \end{itemize}
    


    
\subsubsection{Sample Principal Component}
    For sample matrix $ X $ denoted in \autoref{EqaNPSampleMatrixNotation}, with cov. matrix $ S $ in \autoref{EqaSampleCovarianceMatrix}. Then conduct the above spectrum decomposition to $ S $ to get sample PCs.

    
    \begin{equation}
        \hat{Y}=\hat{P}\hat{\Lambda }\hat{P}' \qquad \hat{P}=\begin{bmatrix}
            \hat{e}_1,\hat{e}_2,\ldots,\hat{e}_p
        \end{bmatrix} 
        \quad \hat{\Lambda} =diag\{\hat{\lambda} _1,\hat{\lambda} _2,\ldots,\hat{\lambda} _p\},\, \hat{\lambda} _1\geq\hat{\lambda} _2\geq\ldots\geq\hat{\lambda} _p
    \end{equation}
    
    Properties and Definitions 
    \begin{itemize}[topsep=2pt,itemsep=2pt]
        \item Trace of cov. matrix:
        \begin{equation}
            \sum_{i=1}^ps _{ii}=\sum_{i=1}^p\hat{\lambda}_i
        \end{equation}
        
        
        \item Sample corr \& factor load:
        \begin{equation}
             \rho (\hat{y}_i,x_j)=\dfrac{(\hat{e}_{i})_j\sqrt{\hat{\lambda _j}}}{\sqrt{s_{jj}}}
        \end{equation}
    \end{itemize}
    
    \begin{point}
        Large Sample \& Normal PCA
    \end{point}
    
    Under normal assumption or large sample case, i.e. 
    \begin{equation}
        X\sim N_p(\mu,\Sigma ) \text{ or } X\xrightarrow[]{\mathrm{d}} N_p(\mu,\Sigma )
    \end{equation}
    
    We can examine the (asymptotic) distribution of $ (\hat{\lambda} _{1},\hat{\lambda} _{2},\ldots,\hat{\lambda} _{p})  $ and $ (\hat{e}_{1},\hat{e}_{2},\ldots,\hat{e}_{p})  $:
    \begin{itemize}[topsep=2pt,itemsep=2pt]
        \item $ \hat{\lambda } $:
        \begin{equation}
            \sqrt{n}(\hat{\lambda }-\lambda )\sim N_p(0,2\Lambda ^2) 
        \end{equation}
        \item $ \hat{e}_i $:
        \begin{equation}
            \sqrt{n}(\hat{e}_i-e_i)\sim N_p(0,E_i),\quad E_i=\lambda _i\sum_{k\neq i}\dfrac{\lambda _k}{(\lambda _k-\lambda _i)^2}e_ke_k' 
        \end{equation}
        \item Independence:
        \begin{equation}
            \hat{\lambda }_i\independent \hat{e}_i 
        \end{equation}
        
    \end{itemize}
    
        










\subsection{Factor Analysis}
    \hypertarget{SubSectionFA}{Key idea of FA:}\index{FA (Factor Analysis)} For a model with $ p $ variable $ X=(X_{1},X_{2},\ldots,X_{p})\sim (\mu ,\Sigma )_p  $ (especially when $ p $ large and $ X_i $ interrelated), there would be some internal, latent \textbf{factors} $ F $ behind $ X $ determining the model structure.\footnote{As the most simplified case, here only consider $ X $ linear dependent on $ F $.}


\subsubsection{Orthogonal Factor Model}
    
    \begin{equation}
        \mathop{X-\mu}\limits_{p\times 1}=\mathop{L}\limits_{p\times m}\mathop{F}\limits_{m\times 1} +\mathop{\varepsilon }\limits_{p\times 1} ,\,\, m<p 
    \end{equation}
    
    where $ L $ is the $ \mathrm{const} $ \textbf{loading matrix} ; $ F $ is r.v. \textbf{factor}; and $ \varepsilon  $ is r.v. error.
    \begin{align}
        X=\begin{bmatrix}
            X_1\\X_2\\\vdots\\X_p
        \end{bmatrix}\quad
        L=\begin{bmatrix}
        \ell_{11}&\ell_{12}&\ldots&\ell_{1m}\\
        \ell_{21}&\ell_{22}&\ldots&\ell_{2m}\\
        \vdots&\vdots&\ddots&\vdots\\
        \ell_{p1}&\ell_{p2}&\ldots&\ell_{pm}\\
        \end{bmatrix}\quad
        F=\begin{bmatrix}
            F_1\\F_2\\\vdots\\F_m
        \end{bmatrix}
        \quad
        \varepsilon =\begin{bmatrix}
            \varepsilon _1\\\varepsilon _2\\\vdots\\\varepsilon _p
        \end{bmatrix}
    \end{align}

    Note: Intuitively, we cannot estimate $ (m+p) $ (unobserable) r.v. from $ p $ r.v., so we need the following assumptions on $ F $ and $ \varepsilon  $
    \begin{equation}
        \begin{aligned}
            \mathbb{E}(F)&=0&cov(F)&=I_n\\
            \mathbb{E}(\varepsilon )&=0&cov(\varepsilon )&=\Psi =diag\{\psi _1,\psi _2,\ldots ,\psi _p\}\\
            \varepsilon \independent& F\Leftrightarrow cov(F,\varepsilon )=0&&
        \end{aligned}
    \end{equation}

    Derived Conclusions:
    \begin{itemize}[topsep=2pt,itemsep=2pt]
        \item Representation of $ \Sigma  $:
        \begin{equation}
            cov(X)=\Sigma =LL'+\Psi  
        \end{equation}
        \begin{itemize}[topsep=2pt,itemsep=2pt]
            \item Diagonal Elements:
            \begin{equation}
                var(X_i)=\sum_{k=1}^m\ell^2_{ik}+\psi _i=h^2_{i}+\psi _i
            \end{equation}

            where $ h_{i}^2 $ is Communality, $ \psi _i $ is Specific variance.
            \item NonDiagonal Elements:
            \begin{equation}
                cov(X_i,X_j)=\sum_{k=1}^m\ell_{im}\ell_{jm} 
            \end{equation}
        \end{itemize}
        \item relation bet. $ X $ and $ F $:
        \begin{equation}
            cov(X,F)=L 
        \end{equation}
    \end{itemize}
    
    \begin{point}
        Factor Rotation
    \end{point}
    For any orthonormal rotation/reflection matrix $ \mathop{T}\limits_{m\times m}  $, $ \tilde{L}=LT $ satisfies the same factor model (with a different $ \tilde{F} $):
    \begin{equation*}
        \begin{aligned}
            X=&LF+\varepsilon =LTT'F+\varepsilon =\tilde{L}\tilde{F}+\varepsilon \qquad \tilde{L}=LT,\,\tilde{F}=T'F\\
            \Sigma =& LL'+\Psi =\tilde{L}\tilde{L}'+\Psi 
        \end{aligned}
    \end{equation*}

    Comment: Factor rotation reflects the arbitrariness of selection of $ L $, allowing us to choose an \textbf{interpretable} $ L $ for FA model.

        
\subsubsection{Principal Component Approach}
    Origin: when $ m=p $, factor decomposition reduces to spectrum(PC) decomposition.(At the same time $ \Psi  $ can be taken $ 0 $.)
    \begin{equation}
        \begin{aligned}
        X=&LF+\varepsilon =PY &\Rightarrow \Psi &=0\\
        \Sigma =&LL'+\Psi =P\Lambda P'&\Rightarrow L&=P\Lambda ^{1/2}
        \end{aligned}
    \end{equation}
    
    Then take the first $ m $ eigenvectors to form $ L $, and use $ \psi _i=\sigma  _{ii}-\sum\limits_{k=1}^m\ell_{ik}^2$ as an approximation.
    \begin{equation}
        \Sigma =LL'+\Psi \qquad L=\begin{bmatrix}
            \sqrt{\lambda_1 }e_1,\sqrt{\lambda_2 }e_2,\ldots,\sqrt{\lambda _m}e_m
        \end{bmatrix} \qquad \Psi =diag\{\psi _{i}\}
    \end{equation}

    \begin{point}
        Sample Factor Decomposition
    \end{point}

    From sample cov. matrix $ S $ and eigenvalue-eigenvector pairs $ (\hat{\lambda }_i,e_i) $, pick the first $ m $ paris to form $ L=\{\ell_{ij}\} $:
    \begin{equation}
        \hat{L}=\{\hat{\ell}_{ij}\}=\begin{bmatrix}
            \sqrt{\hat{\lambda }_1}\hat{e}_1,\sqrt{\hat{\lambda }_2}\hat{e}_2,\ldots,\sqrt{\hat{\lambda }_m}\hat{e}_m
        \end{bmatrix} 
        \qquad
        \hat{\Psi }=diag\{s_{ii}-\sum_{k=1}^m\hat{\ell}_{ik}^2\}
    \end{equation}

    \begin{itemize}[topsep=2pt,itemsep=2pt]
        \item Selection of $ m $: Construct Residual Matrix
        \begin{equation}
            \hat{E}=S-(\hat{L}\hat{L}'+\hat{\Psi }) 
        \end{equation}
        
        Residual matrix is trace $ 0 $, pick $ m $ such that
        \begin{equation}
            \text{Sum of All Elements in }\hat{E}<\sum_{k=m+1}^p\hat{\lambda }_{k}^2\text{ small enough} 
        \end{equation}
        
        
        
    \end{itemize}
    
        
    
    
    
    
    
\subsubsection{MLE Method}
    Assumption: Factor $ F $ and error $ \varepsilon  $ are normal.(Then also $ X \sim N_p(\mu ,\Sigma )$ is normal)
    
    \begin{equation}
        F\sim N_m(0,I_m)\quad \varepsilon \sim N_p(0,\Psi )\quad X\sim N_p(\mu,\Sigma ) 
    \end{equation}

    Likelihood Function:
    \begin{equation}
        L(\mu ,\Sigma ) =(2\pi)^{-np/2}|\Sigma |^{-n/2}\exp\left( -\dfrac{1}{2}tr\left[ \Sigma ^{-1}\left( \sum_{k=1}^n(x_k-\bar{x})(x_k-\bar{x})'+n(\bar{x}-\mu)(\bar{x}-\mu)' \right) \right] \right)
    \end{equation}

    Maximize $ L $ to get $ \hat{L}  $ and $ \hat{\Psi } $, usually for convenient (and to counteract the arbitrariness of factor rotation) we further assume
    \begin{equation}
        L'\Psi ^{-1}L=\varXi \text{ (diagonal matrix)} 
    \end{equation}
    
    \begin{itemize}[topsep=2pt,itemsep=2pt]
        \item Estimtor of communality variance $ h_i^2 $:
        \begin{equation}
            \hat{h}_i^2=\sum_{k=1}^m \hat{l}_{ik}^2 
        \end{equation}
        
        
    \end{itemize}
    
        
    
    
    
    











\subsection{Canonical Correlation Analysis}
    Key idea of CCA\index{CCA (Canonical Correction Analysis)}: For a model with two multivariate population $ X^{(1)}=(X^{(1)}_{1},X^{(1)}_{2},\ldots,X^{(1)}_{p})  $, \\$ X^{(2)}=(X^{(2)}_{1},X^{(2)}_{2},\ldots,X^{(2)}_{q})  $ with covariance
    \begin{equation}
        \mathop{\Sigma }\limits_{(p+q)\times (p+q)}  =\begin{bmatrix}
            \Sigma _{11}&\Sigma _{12}\\
            \Sigma _{21}&\Sigma _{22}
        \end{bmatrix} 
    \end{equation}
    
    find a few condensed variable to measure their similarity.
    
    
\subsubsection{Canonical Variate Pair}
    By using the linear combination, we can construct a pair of vector $ \mathop{a}\limits_{p\times 1 }  $ and $ \mathop{b}\limits_{q\times 1}  $ such that $ corr(a'X^{(1)},b'X^{(2)}) $ large, i.e.
    \begin{equation}\label{EqaFirstCanonicalVariatePair}
        \{a,b\}=\mathop{\arg\max}\limits_{a,b\neq 0}\dfrac{a'\Sigma _{12}b}{\sqrt{a'\Sigma _{11}a}\sqrt{b'\Sigma _{22}b}}  
    \end{equation}

    
    
    where $ U_1=a'X^{(1)} $, $ V_1=b'X^{(2)} $ with $ var(U_1)=var(V_1)=1 $ are the \textbf{(first) canonical variate pair}, and $ \rho^* _1=corr(U_1,V_1) $ is the \textbf{(first) canonical correlation}.\index{Canonical Variable Pair}
    
    Similarly, the $ k^{\mathrm{th}} $ canonical pair $ (U_k,V_k) $ satisfy the same criterion as \autoref{EqaFirstCanonicalVariatePair} but with $ a_k\in \mathrm{span}\{a_1,\ldots,a_{k-1}\}^\perp,$ $b_k\in\mathrm{span}\{b_1,\ldots,b_{k-1}\}^\perp $, $ k\leq \min\{p,q\} $.
    
    Result: $ U_k $, $ V_k $ can be expressed as 
    \begin{equation}
        U_k=a_k'X^{(1)}=e_k'\Sigma _{11}^{-1/2}X^{(1)}\qquad V_k=b_k'X^{(2)}=f_k'\Sigma _{22}^{-1/2}X^{(2)} 
    \end{equation}
    
    where $ e_k $ is the $ k^\mathrm{th} $ eigen vector of $ \Sigma _{11}^{-1/2}\Sigma _{12}\Sigma _{22}^{-1}\Sigma _{21}\Sigma _{11}^{-1/2} $, $ f_k $ is the $ k^\mathrm{th} $ eigenvector of $ \Sigma _{22}^{-1/2}\Sigma _{21}\Sigma _{11}^{-1}\Sigma _{12}\Sigma _{22}^{-1/2} $. $ e_k $ and $ f_k $ satisfies:
    \begin{equation}
        f_k=\dfrac{1}{\rho ^*_k}\Sigma _{22}^{-1/2}\Sigma _{21}\Sigma _{11}^{-1/2}e_k\quad e_k=\dfrac{1}{\rho _k^*}\Sigma _{11}^{-1/2}\Sigma _{12}\Sigma _{22}^{-1/2}f_k
    \end{equation}
       
    
\subsubsection{Canonical Correlation based on Standardized Variables}
    Using standardized variable of $ X $:
    \begin{equation}
        Z_k^{(\nu )}=\dfrac{X_k^{(\nu )}-\mu _k^{(\nu )}}{\sqrt{\sigma _{kk}^{(\nu )}}},\,k=1,2,\ldots p\text{ or }q,\, \nu =1,2
    \end{equation}
    
    with covariance
    \begin{equation}
        \mathop{\rho  }\limits_{(p+q)\times (p+q)}=V^{-1/2}\Sigma V^{-1/2}=\begin{bmatrix}
            \rho _{11}&\rho _{12}\\
            \rho _{21}&\rho _{22}
        \end{bmatrix}
    \end{equation}

    And similarly, the CCA pair is
    \begin{equation}
        U_k=a_k'Z^{(1)}=e_k'\rho _{11}^{-1/2}Z^{(1)}\qquad V_{k}=b_k'Z^{(2)}=f_k'\rho _{22}^{-1/2}Z^{(2)}
    \end{equation}
    
    with $ e_k $ is the $ k^\mathrm{th} $ eigenvector of $ \rho _11 ^{-1/2}\rho _{12}\rho _{22}^{-1}\rho_{21}\rho _{11}^{-1/2} $, $ f_[k] $ is the $ k^\mathrm{th} $ eigenvector of $ \rho _{22}^{-1/2}\rho _{21}\rho _{11}^{-1}\rho _{12}\rho _{22}^{-1/2} $, and
    \begin{equation}
        f_k=\dfrac{1}{\rho ^*_k}\rho _{22}^{-1/2}\rho _{21}\rho _{11}^{-1/2}e_k\quad e_k=\dfrac{1}{\rho _k^*}\rho _{11}^{-1/2}\rho _{12}\rho _{22}^{-1/2}f_k
    \end{equation}
    
    
\subsubsection{Sample Canonical Correlation}
    Replacement:
    \begin{equation}
        \Sigma \longrightarrow S\qquad \rho \longrightarrow R 
    \end{equation}
    
    to get 
    \begin{equation}
        \hat{U}=\hat{A}x^{(1)}\qquad \hat{V}=\hat{B}x^{(2)} 
    \end{equation}

    and we can use $ \hat{U},\hat{V},\, \hat{A},\hat{B} $ to express $ S_{12} $ as
    \begin{equation}
        S_{12}=\hat{A}^{-1}\begin{bmatrix}
        \hat{\rho }^*_{1}&0&\ldots&0\\
        0&\hat{\rho }^*_{2}&\ldots&0\\
        \vdots&\vdots&\ddots&\vdots\\
        0&0&\ldots&\hat{\rho }^*_{p}\\
        \end{bmatrix}(\hat{B}^{-1})'
    \end{equation}
    
    When applying CCA, we pick the first $ r $ canonical variable, thus some infomation is lost. But we hope the first $ r $ canonical variables can contain enough information of $ X^{(1)} $ and $ X^{(2)} $.

    Determine of $ r $: consider the error if approximation by expressing
    \begin{equation}
        \hat{A}^{-1}=[\alpha _1,\alpha _2,\ldots,\alpha _p]\qquad \hat{B}^{-1}=[\beta _1,\beta _2,\ldots,\beta _p] 
    \end{equation}

    and 
    \begin{align}
        S_{12}=&\sum_{i=1}^p \hat{\rho }^*_i\alpha _i\beta _i'\\
        S_{11}=&\hat{A}^{-1}(\hat{A}^{-1})'=\sum_{i=1}^p\alpha _i\alpha _i'\\
        S_{22}=&\hat{B}^{-1}(\hat{B}^{-1})'=\sum_{i=1}^p\beta _i\beta _i'
    \end{align}
    
    Total sample variance explained by the first $ r $ canonical variables:
    \begin{equation}
        \dfrac{\sum_{i=1}^r\alpha _i'\alpha _i}{tr(S_{11})}\qquad \dfrac{\sum_{i=1}^r\beta _i'\beta _i}{tr(S_{22})} 
    \end{equation}
    
    



\subsection{Discriminant Analysis}\label{SubSectionDiscriminantAnalysis}
    \index{DA (Discriminant Analysis)}
    Key idea of DA: for $ \mathop{X}\limits_{n\times p}  $ with an extra column labeling the classification, we want to determine a rule to assign new objects. More specifically, determine the classification region $ R_i $ for each class $ \pi_i $.

\subsubsection{Classification Criterion}
\begin{itemize}[topsep=2pt,itemsep=2pt]
    \item Two-category classification case:Each row of $ X $ is labeled in $ \pi_1 $ or $ \pi_2 $, for two-category, only one of $ R_1 $, $ R_2 $ is needed.

    Some basic concept in classification model:
    \begin{itemize}[topsep=2pt,itemsep=2pt]
        \item Prior Possibility $ p_i $, $ i=1,2 $;
        \item Penalty for misclassification $ c(i|j) $, $ i,j=1,2 $: cost if a $ \pi_j $ object is classified in $ R_i $.
        \item Conditional Probability $ \mathbb{P}(i|j) $, $ i,j=1,2 $: probability that a $ \pi_j $ object falls in region $ R_i $
    \end{itemize}

    \begin{point}
Determination Criterion:
    \end{point}
\begin{itemize}[topsep=2pt,itemsep=2pt]
    \item Expected Cost of Misclassification (ECM) Criterion\index{ECM (Expected Cost of Misclassification)}: Minimizing $ \mathrm{ECM} $,
    \begin{equation}
        \mathrm{ECM}=c(2|1)\mathbb{P}(2|1)p_1+c(1|2)\mathbb{P}(1|2)p_2 
    \end{equation}
    
    For two-category problem, $ R_1 $, $ R_2 $ can be determined as
    \begin{align}
        R_1=&\dfrac{f_{\pi_1}(x)}{f_{\pi_2}(x)}\geq \dfrac{c(1|2)}{c(2|1)}\dfrac{p_2}{p_1}\\
        R_2=&\complement_{R_x}^{R_1}=\mathop{\arg}\limits_{x\in R} \,  \dfrac{f_{\pi_1}(x)}{f_{\pi_2}(x)}<\dfrac{c(1|2)}{c(2|1)}\dfrac{p_2}{p_1}
    \end{align}
    \item Total Probability of Misclassification (TPM) Criterion: Minimizing $\mathrm{TPM}$,\index{TPM (Total Probability of Misclassification)}
    \begin{equation}
        \mathrm{TPM}=\mathbb{P}(\mathrm{misclass})=\mathbb{P}(2|1)p_1+\mathbb{P}(1|2)p_2 
    \end{equation}

    actually $ \arg\min\mathrm{TPM}=\mathop{\arg\min}\limits_{c(1|2)=c(2|1)} \mathrm{ECM} $
    \item Posterior Probability Criterion: Maximize posterior probability $ P(\pi_i|x_0) $,
    \begin{equation}
        \mathbb{P}(X\in\pi_i|X=x_0)=\dfrac{p_i f_{\pi_i(x_0)}}{p_1f_{\pi_1}(x_0)+p_2f_{\pi_2}(x_0)},\, i=1,2
    \end{equation}
    
    Also equivalent to $ \mathrm{ECM} $ for $ c(1|2)=c(2|1) $
\end{itemize}
    \item Here only introduce ECM: $ \{R_i\}=\arg\min\mathrm{ECM} $
    \begin{align}
        \mathrm{ECM}(i)=&\sum_{j\neq i}c(j|i)\mathbb{P}(j|i)\\
        \mathrm{ECM}=&\sum_{i=1}^gp_i\mathrm{ECM}_i=\sum_{i=1}^g\sum_{j\neq i} c(j|i)p(j|i)p_i
    \end{align}
\end{itemize}

    
    




    
\subsubsection{Linear \& Quadratic Discriminant Analysis}
    Now take two-category ECM criterion as example. An estimation to $ \mathbb{P}(1|2),\,\mathbb{P}(2|1) $, i.e. to $ f_{\pi_1},\,f_{\pi_2} $ is needed.

    Assumption: for $ \pi_1:X\sim N(\mu _1,\Sigma _1) $,$ \pi_2:X\sim N(\mu _2,\Sigma _2) $, further for
    \begin{itemize}[topsep=2pt,itemsep=2pt]
        \item $ \Sigma _1=\Sigma _2=\Sigma  $: Linear Discriminant Analysis (LDA)\index{LDA (Linear Discriminant Analysis)}.
        
        \begin{equation}
            f_{\pi_i}(x)=\dfrac{1}{(2\pi)^{p/2}|\Sigma |^{1/2}}\exp\left( -\dfrac{1}{2}(x-\mu _i)'\Sigma ^{-1}(x-\mu _i) \right),\,i=1,2
        \end{equation}
        
        then 
        \begin{align}
            R_1=&\mathop{\arg}\limits_{x\in R} \,(\mu_1-\mu _2)'\Sigma ^{-1}x-\dfrac{1}{2}(\mu _1-\mu _2)'\Sigma ^{-1}(\mu _1-\mu _2)\geq \ln\left(\dfrac{c(1|2)}{c(2|1)}\dfrac{p_2}{p_1}\right)\\
            R_2=&\mathop{\arg}\limits_{x\in R}\,(\mu_1-\mu _2)'\Sigma ^{-1}x-\dfrac{1}{2}(\mu _1-\mu _2)'\Sigma ^{-1}(\mu _1-\mu _2)< \ln\left(\dfrac{c(1|2)}{c(2|1)}\dfrac{p_2}{p_1}\right)
        \end{align}

        Note that $  \mathrm{L.H.S.} $ is a linear combination of $ x $, thus called LinearDA.

        Sample estimation to $ \Sigma  $: use pooled variance in \autoref{EqaPooledVariance}.

        \item $ \Sigma _1\neq \Sigma _2 $: Quadratic Discriminant Analysis (QDA)\index{QDA (Quadratic Discriminant Analysis)}.
        
        \begin{equation}
            f_{\pi_i}(x)=\dfrac{1}{(2\pi)^{p/2}|\Sigma |^{1/2}}\exp\left( -\dfrac{1}{2}(x-\mu _i)'\Sigma_i ^{-1}(x-\mu _i) \right),\,i=1,2
        \end{equation}
        
        then 
        \begin{align}
            R_1=&-\dfrac{1}{2}x'(\Sigma _1^{-1}-\Sigma _2^{-1})x+(\mu _1'\Sigma _1^{-1}-\mu _2'\Sigma _2^{-1})x-\dfrac{1}{2}\ln\left(\dfrac{|\Sigma _1|}{|\Sigma _2|}\right)+\dfrac{1}{2}(\mu _1'\Sigma _1^{-1}\mu _1-\mu _2'\Sigma _2^{-1}\mu _2)\geq \ln\left(\dfrac{c(1|2)}{c(2|1)}\dfrac{p_2}{p_1}\right)\\
            R_2=&-\dfrac{1}{2}x'(\Sigma _1^{-1}-\Sigma _2^{-1})x+(\mu _1'\Sigma _1^{-1}-\mu _2'\Sigma _2^{-1})x-\dfrac{1}{2}\ln\left(\dfrac{|\Sigma _1|}{|\Sigma _2|}\right)+\dfrac{1}{2}(\mu _1'\Sigma _1^{-1}\mu _1-\mu _2'\Sigma _2^{-1}\mu _2)< \ln\left(\dfrac{c(1|2)}{c(2|1)}\dfrac{p_2}{p_1}\right)
        \end{align}

        Note that $  \mathrm{L.H.S.} $ is a quadric form of $ x $, thus called QuadraticDA.
        \item Two extension: allow more flexible estimation to variance:
        \begin{itemize}[topsep=2pt,itemsep=2pt]
            \item $ \hat{\Sigma }_i(\alpha )=\alpha\hat{\Sigma }_i+(1-\alpha )\hat{\Sigma } $, shrink between QDA and LDA;
            \item $ \hat{\Sigma }_i(\gamma )=\gamma \hat{\Sigma }+(1-\gamma )\hat{\sigma }^2I $, shrink toward scalar cov.
        \end{itemize}
        
            
    \end{itemize}
    

\subsubsection{Fisher's Discriminant Analysis}
    \index{FDA (Fisher's Discriminant Analysis)}Project $ X $ onto some hyperplane and conduct low-dimensional classification.

    Project $ x $ onto some hyperplane by $ y=a'x  $, then we maximize $ \psi =\dfrac{\text{mean of treatment}^2}{\text{variance}} $\footnote{MANOVA Model:    
    For $ g $ groups with same $ \Sigma  $, consider an MANOVA model: $ X_{ij}=\mu +\tau_i+e_{ij} $. Then MANOVA table gives Sum of Squares and cross Products (SSP):
    \begin{align}
        \text{Treatment:}&B=\sum_{i=1}^gn_i(\bar{x}_i-\bar{x})(\bar{x}_i-\bar{x})'\\
        \text{Residual:}&W=\sum_{i=1}^g\sum_{j=1}^{n_i}(x_{ij}-\bar{x}_i)(x_{ij}-\bar{x}_i)'\\
        \text{Total:}&T=B+W=\sum_{i=1}^g\sum_{j=1}^{n_i}(x_{ij}-\bar{x})(x_{ij}-\bar{x})'
    \end{align}

    use $ B $ and $ W $ to measure the variance of sample.
    }. i.e.
    \begin{align}
        \psi =\dfrac{\sum\limits_{i=1}^g(\mu_{iY}-\mu_Y)^2}{\sigma ^2_Y}=\dfrac{a'\left(\sum\limits_{i=1}^g(\mu_i-\mu)(\mu_i-\mu)'\right)a}{a'\Sigma a}=\dfrac{a'B_\mu a}{a'\Sigma a}
    \end{align}
    
    Result: $ a $ is the largest eigen vector of $ W^{-1}B $.

    Relation between FDA and LDA: in FDA, take the first $ \xi  $ eigenvectors to conduct classification, thus loses more information. But when $ \xi =g-1 $, FDA $ \equiv $ LDA.\footnote{Because $ a $ is eigenvector of $ W^{-1}B $, while $ \mathrm{rk}(B)=g-1 $, thus there are $ g-1 $ non-zero eigenvalues at most.}
   
    
        
\subsubsection{Evaluation of Discriminant Model}
\begin{point}
    Judging Index:
\end{point}

\begin{itemize}[topsep=2pt,itemsep=2pt]
    \item Total Probability of Misclassification (TPM)\index{TPM (Total Probability of Misclassification)}:
    \begin{equation}
        \mathrm{TPM}=p_1\mathbb{P}(2|1)+p_2\mathbb{P}(1|2)=p_1\int _{R_2}f_{\pi_1}(x) \,\mathrm{d}x+p_2\int _{R_1}f_{\pi_2}(x) \,\mathrm{d}x
    \end{equation}
    
    \item APparent Error Rate (APER)\index{APER (Apparent Error Rate)}: used with cross validation (CV). The fraction of misclassification in training set. 
    
\end{itemize}

    
    
    




\subsection{Clustering Analysis}\label{SubSectionClusteringAnalysis}
    \index{Clustering Analysis}
    Key idea of CA: Group a collection of data according to  similarity and relation of objects.

\subsubsection{Agglomerative Clustering Algorithm}
    \index{Agglomerative Clustering Algorithm}\index{Hierarchical Clustering}

\begin{point}
    Clustring Algorithm
\end{point}

    Hierarchical clustering: start with individual points and combine them to form groups. 
    \begin{algorithm}{Hierarchical Clustering}
        \begin{enumerate}[topsep=2pt,itemsep=2pt]
        \item All $ k=n $ points are individual clusters;
        \item In each iteration step $ k $: 
        \begin{enumerate}[topsep=2pt,itemsep=2pt]
            \item Use a distance/dissimilarity matrix $ \mathop{D}\limits_{k\times k}  $ to express distances between clusters; the 'distance' between clusters is diversified, choice of which see the \hyperlink{BetweenClusterD}{following part};
            \item merge the closest pair of clusters(or points) to form a larger cluster, and now number of clusters 
            \item $ k = k-1 $;
        \end{enumerate}
        \item Only $ k=1 $ cluster is left
        \item Choose a proper threshold of distance to determine $ K $
        \end{enumerate}
    \end{algorithm}
        

    
    
\begin{point}
    \hypertarget{BetweenClusterD}{Choice of between-cluster distance:} To express distance between two clusters $ A $ and $ B $,
\end{point}

\begin{itemize}[topsep=2pt,itemsep=0pt]
    \item Choice of distance functional $ D(\cdot,\cdot) $:
\begin{itemize}[topsep=2pt,itemsep=2pt]
    \item Euclidean Distance $ D_\mathrm{E} $;
    \item Mahalanobis Distance $ D_\mathrm{M} $;
    \item Jaccard Distance $ D_\mathrm{J}=1-\dfrac{|A\cap B|}{|A\cup B|} $;
    \item etc.
\end{itemize}
    \item Location choice of cluster:    

    \begin{itemize}[topsep=2pt,itemsep=2pt]
        \item Complete link: $ \max D(a\in A,b\in B) $;
        \item Single link: $ \min D(a\in A,b\in B) $;
        \item Centroid distance: $ D(A\,\mathrm{centroid},B\,\mathrm{centroid}) $;
        \item Group average: $ \langle D(a\in A,b\in B)\rangle $
    \end{itemize}
    \item Note: pros-and-cons of agglomerative clustering algorithm
    \begin{itemize}[topsep=2pt,itemsep=2pt]
        \item No assumptions for final $ k $ needed;
        \item Intuitive display of relations;
        \item Large computational requirement: $ \sim O(n^3) $;
        \item Sensitive to noise and outliers.
    \end{itemize}
\end{itemize}

\begin{figure}[H]
    \centering
    \includegraphics[width=0.7\linewidth]{sections/images/Hierarchical Clustering.png}
    \caption{Illustration of Hierarchical Clustering}
    \label{}
\end{figure}


\subsubsection{$ K $-Means Clustering Algorithm}
    \index{k-Means Clustering Algorithm@$ k $-Means Clustering Algorithm}
    Assume we have a preset number $ K $ of clusters , we can use $ K $-means clustering.
\begin{algorithm}{$ K $-Means Clustering}
    \begin{enumerate}[topsep=2pt,itemsep=2pt]
        \item Choose/Preset number of clusters $ K $;
        \item Select $ K $ points as initial centroids, useful methods:
        \begin{itemize}[topsep=2pt,itemsep=2pt]
            \item Randomly select;
            \item Use Centroid of agglomerative algorithm;
            \item Successively pick the farthest point from others. 
        \end{itemize}
        \item In each iteration of centroids:
        \begin{enumerate}[topsep=2pt,itemsep=2pt]
            \item For all points $ i $, calculate its distance from the $ l^\mathrm{th} $ centroid $ D(i,l) $
            \item Classify each $ i $ point to the nearest centroid cluster;
            \item Re-calculate the centroid of new $ K $ clusters;
        \end{enumerate}
        \item Repeat until convergence.(Convergence criterion can be e.g. $ <\sum_i D(i\in g_l,l)>\to \mathrm{const} $)
    \end{enumerate}
\end{algorithm}
    





Note: pros­and­cons of $ K $-Means clustering algorithm
\begin{itemize}[topsep=2pt,itemsep=2pt]
    \item Efficient:$ \sim O(n) $;
    \item Sensitive to outliers;
    \item Ineffective for non-convex shapes.
\end{itemize}

    


\subsubsection{Gaussian Mixture Model with Expectation Maximization Algorithm}\label{SubSubSectionEMAlgorithmForGMM}
\index{E-M Algorithm (Expectation Maximization Algorithm)}
    % We can also label each object, then conduct discriminant analysis (e.g. QDA).

    %  Assumption and notation:  
    % \begin{align}
    %     Y\sim& \sum_{l=1}^k\pi_lN()\text{ where }\pi\text{ is a set }\{\pi_l\}\\
    %     X|Y=l\sim& N(\mu_l,\Sigma _l)=N(\theta_l),\, 1\leq l\leq k
    % \end{align}

    % Our aim is to estimate $ \pi,(\mu_i,\Sigma _i) ,\,1\leq i\leq k$. 
    % Take two-component gaussion mixture model $ k=2 $ as example.
        The Gaussian Mixture Model (GMM)\index{GMM (Gaussian Mixture Model)} for clustering assumes $ X $ is generated from a mixed distribution of $ K $ normal, i.e. $ X $ has probability $ \pi_l $ to be generated from corresponding normal $ N(\mu _l,\Sigma _l) $:
    \begin{equation}
        X\sim \sum_{l=1}^K\pi_lN(\mu_l,\Sigma _l)=\sum_{l=1}^K\pi_lN(\theta _l),\quad \sum_{l=1}^K\pi_l=1,\,\pi_l\geq 0.
    \end{equation}
    
    Use its likelihood function $ L(\theta;x) $ and maximize posterior probability by $ \dfrac{\partial^{} \ell}{\partial \theta ^{}} $:
    \begin{equation}
        L(\{\pi_l\},\{\theta_l \};x)=\prod_{i=1}^N \sum_{l=1}^K\pi_l   \dfrac{1}{(2\pi)^{p/2}|\Sigma _l|^{1/2}}\exp\left(   -\dfrac{1}{2}(x_i-\mu _l)'\Sigma^{-1} _l(x_i-\mu _l)\right)
    \end{equation}
    
    E-M Algorithm uses the ELBO maximizing method, detail see \autoref{SubSectionExpectationMaximumAlgorithm}. For simplification express $ \theta \equiv \{ \cup \pi_l,\cup \mu_l,\cup \Sigma _l \} $. The maximizing function $ Q(\theta |\theta ^{(t)}) $ for GMM model and corresponding iteration: 
\begin{align}
    \theta ^{(t+1)}=\mathop{\arg\max}\limits_{\theta }  Q(\theta |\theta ^{(t)})=&\mathop{\arg\max}\limits_{\theta }\sum_{i=1}^N\sum_{l=1}^K\gamma _{il}^{(t)}\log \pi_l\phi (x_i|\mu _l,\Sigma _l),\quad \gamma _{il}^{(t)}\equiv \dfrac{\pi_l^{(t)}\phi(x_i|\mu _l^{(t)},\Sigma _l^{(t)})}{\sum\limits_{j=1}^K\pi_j^{(t)}\phi (x_i|\mu _j^{(t)},\Sigma _j^{(t)})}
\end{align}
    Lagrange Multiplier: Extreme value $ \mathop{\arg\max}\limits_{\theta }Q(\theta |\theta ^{(t)})  $ with constraint $ \sum_{l=1}^K \pi_l=1 $ requires 
    \begin{equation}
         \dfrac{\partial^{}  Q(\theta |\theta ^{(t)})}{\partial \mu _l^{}}=0\quad \dfrac{\partial^{}  Q(\theta |\theta ^{(t)})}{\partial \Sigma ^{-1}_l}=0 \quad \dfrac{\partial^{}  Q(\theta |\theta ^{(t)})+\lambda (\sum_{j=1}^K\pi_l-1)}{\partial \pi_j^{}}=0,\quad \forall l=1,2,\ldots,K
    \end{equation}
    
    Result:
    \begin{align}\label{EqaGMMEMIteration}
        % &\sum_{i=1}^N\dfrac{\pi_l\phi _{\theta_l}(x_i)}{\sum\limits_{j=1}^k\pi_j\phi _{\theta_j}(x_i)}
        \begin{cases}
        \mu _l^{(t+1)}=&\dfrac{\sum\limits_{i=1}^N\gamma _{il}^{(t)}x_i}{\sum\limits_{i=1}^N\gamma^{(t)}_{il}}\\
        \Sigma _l^{(t+1)}=&\dfrac{\sum\limits_{i=1}^N\gamma^{(t)} _{il}(x_i-\mu _l)(x_i-\mu _l)'}{\sum\limits_{i=1}^N\gamma ^{(t)}_{il}}\\
        \pi_l^{(t+1)}=&\dfrac{1}{N}\sum_{i=1}^N\gamma^{(t)}_{il}
        \end{cases}
    \end{align}
\begin{equation}
        \gamma ^{(t)}_{il}\equiv \dfrac{\pi_l^{(t)}\phi(x_i|\mu _l^{(t)},\Sigma _l^{(t)})}{\sum\limits_{j=1}^K\pi_j^{(t)}\phi (x_i|\mu _j^{(t)},\Sigma _j^{(t)})} 
\end{equation}  
    
    
    
    where $ \gamma _{il} $ is the posterior probability that the $ i^\mathrm{th} $ object belongs to the $ l^\mathrm{th} $ group.

    The above constraint equations are difficult to solve, use iteration algorithm:
\begin{algorithm}{EM-Algorithm for Gaussian Mixture Model}
    \begin{enumerate}[topsep=2pt,itemsep=2pt]
        \item Use e.g. $ K $-means method to set an initial estimation as $ (\hat{\mu}^{(0)}_l,\hat{\Sigma }_l^{(0)}),\,\hat{\pi}_l^{(0)}=1/K$;
        \item Repeat Expectation \& Maximization:
        \begin{enumerate}[topsep=2pt,itemsep=2pt]
            \item $ \mathrm{E_{xpectation}} $-Step: Compute posterior of latent variable on each point;
        \begin{equation}
            \hat{\gamma }_{il}^{(t)}=\dfrac{\pi_l^{(t)}\phi(x_i|\mu _l^{(t)},\Sigma _l^{(t)})}{\sum\limits_{j=1}^K\pi_j^{(t)}\phi (x_i|\mu _j^{(t)},\Sigma _j^{(t)})} ,\quad  1\leq i\leq N,\,\, 1\leq l\leq K
        \end{equation}
        \item $ \mathrm{M_{aximize}} $-Step: Re-calculate parameters $ \{\mu_l,\Sigma _l,\pi_l\} $ by \autoref{EqaGMMEMIteration}.
        \end{enumerate}
        \item Repeat until convergence.
    \end{enumerate}
\end{algorithm}
    
    Note: EM method for Gaussion Mixture Model is a greedy algorithm $ \longrightarrow $ local maximum.
    
\subsubsection{DBSCAN \& OPTICS Density Clustering Algorithm}
    \index{Density Clustering}
    \hyperlink{DBSCAN}{DBSCAN} algorithm (Density-Based Spatial Clustering of Application with Noise) is a kind of density clustering algorithm. \hyperlink{OPTICS}{OPTICS} algorithm (Ordering Point To Indentify the Cluster Structure) is its improved version.


\begin{point}
    \hypertarget{DBSCAN}{DBSCAN} Algorithm
\end{point}
\index{DBSCAN (Density-Based Spatial Clustering of Application with Noise)}
    Key (preset) index in DBSCAN:
    \begin{itemize}[topsep=2pt,itemsep=2pt]
        \item Eps $ \varepsilon  $: Radius of neighbourhood of a point;
        \item MinPts $ M $: Minimum number of points to be indentified as cluster core point, usually choose $ M\geq \mathrm{dim}+1 $;
        \item (Also, a distance norm is needed, e.g. Euclidean $ D $).
    \end{itemize}
    
    Notation:
    \begin{itemize}[topsep=2pt,itemsep=2pt]
        \item $ \varepsilon  $ neighbourhood of point $ x_i $:
        \begin{equation}
             \mathcal{N}_\varepsilon (x_i)\equiv \{y\in \mathbb{R} ^n: 0<D(y,x)<\varepsilon \}
        \end{equation}
        \item `Density' (is actually an integer):
        \begin{equation}
            \rho _\varepsilon (x_i)\equiv \#  x_j\in \mathcal{N}_\varepsilon (x_i) 
        \end{equation}
        \item Three types of Points: $ X_c $, $ X_{bd} $, $ X_{noi} $.
        \begin{itemize}[topsep=2pt,itemsep=2pt]
            \item Core Point: label an $ x_i $ as core point if
        \begin{equation}
            \rho _\varepsilon (x_i)\geq M
        \end{equation}

        Denote the set of core point as $ X_c $, and set of non-core point as $ X_{nc} $
        
        \item Border Point: label an $ x_j\in X_{nc} $ as border point if
        \begin{equation}
             \exists (x_i\in X_c)\in \mathcal{N}_\varepsilon (x_j) \& x_j\in X_{nc}
        \end{equation}
        Denote the set of border point as $ X_{bd} $
        \item Noise Point: the set of noise point is 
        \begin{equation}X_{noi}\equiv\displaystyle{\complement_X^{X_{c}\cup X_{bd}} }\end{equation}
        \end{itemize}
    \item Point Relations: DDR, DR, DC
    \begin{itemize}[topsep=2pt,itemsep=2pt]
    \item Directly Density Reachable: For $ x_i,x_j\in X $, if $ x_i\in X_c $, $ x_j\in\mathcal{N}_\varepsilon (x_i) $, then say $ x_j $ is DDR from $ x_i $;
    \item Density Reachable: For point chain $ x_{i_1},x_{i_2},\ldots,x_{i_m} $, $ m\geq 2 $. If $ x_{i_{\kappa+1} } $ is DDR from $ x_{i_\kappa } $, $ \forall 1\leq \kappa \leq m-1 $, then say $ x_{i_m} $ is DR from $ x_{i_1} $.
    \item Density Connected: For point $ x_{i_1} $, $ x_{i_2} $, $ x_{i_3} $, if $ x_{i_2} $ and $ x_{i_3} $ are both DR from $ x_{i_1} $, then say $ x_{i_2} $ and $ x_{i_3} $ are DC. 
    \end{itemize}
    
    Note: DR is not symmetric for $ x_{i_1} $ and $ x_{i_m} $; while DC is.       
    \end{itemize}
    
    DBSCAN algorithm classify all points that are Density Connected to each other into a cluster $ C\subset X $, i.e.
    \begin{align}
        \text{Maximality:}& x\in C\&\& y\text{ DR from } x \Rightarrow y\in C\\
        \text{Connectivity:}& x,y\in C\Rightarrow x,y\text{ DC}.
    \end{align} 
        
    Pros and cons of DBSCAN:
\begin{itemize}[topsep=2pt,itemsep=2pt]
    \item Insensitive to noise;
    \item Based on density, with no constraint on the shape of cluster; 
    \item Suitable for clusters with uniformly densed data, otherwise difficult to choose proper Eps $ \varepsilon  $;
    \item Complexity $ \sim O(n^2) $, at least $ O(n\log n) $.
\end{itemize}


\begin{point}
    \hypertarget{OPTICS}{OPTICS} Algorithm\index{OPTICS (Ordering Point To Indentify the Cluster Structure)}
\end{point}

    OPTICS is based on DBSCAN and shares most of the basic concepts and ideas. Further define the following distance (preset $ \varepsilon  $ and $ M $):
    \begin{itemize}[topsep=2pt,itemsep=2pt]
        \item Core Distance: For $ x_i\in X_c $, the smallest distance allowing $ x_i $ to become core point.
        \begin{equation}
            \mathrm{CD}(x_i)=D(x_i,N^M_\varepsilon (x_i)),\, \rho _\varepsilon (x_i)\geq M
        \end{equation}

        where $ N^M_\varepsilon (x_i) $ is the $ M^\mathrm{th} $ closest point from $ x_i $;
        \item Reachablity Distance: For $ y\in X,\,x_i \in X_c\subset X$, 
        \begin{equation}
            \mathrm{RD}(y,x_i)=\max\{CD(x_i,D(y,x_i))\}
        \end{equation}

        Or equivlantly
        \begin{equation}
            \mathrm{RD}(y,x_i)=\mathop{\arg\min}\limits_{\rho _d(x_i)\geq M, y\in \mathcal{N}_d(x_i)}  d
        \end{equation}
        
        
    \end{itemize}
    
        
    Algorithm flow:
\begin{algorithm}{OPTICS }

\begin{enumerate}[topsep=2pt,itemsep=2pt]
    \item Construct $ X_c $ based on preset $ M $, $ \varepsilon  $;
    % \item Denoting two sequence: \lstinline|Order seq| and \lstinline|Outcome seq|;
    \item Pick an `unprocessed' point $ x_{n_i}\in X_c $ and calculate $ \mathrm{RD}(x_j,x_{n_i} ) $, $ \forall \text{`unprocessed'} x_j\in \mathcal{N}_\varepsilon (x_{n_i}) \cap X_c $. Pick the $ x_j\in X_c $ with smallest RD and label as $ x_{n_{i+1}} $ processed;
    \item Repeat step 2 until all points are processed. Output $ \{x_{n_i}\}=(x_{n_1},x_{n_2},\ldots,x_{n_{|X_c|}}) $. Each $ x_{n_i} $ is attached with a $ \mathrm{CD}(x_{n_i}) $ and a $ r(x_{n_i}):=\mathrm{RD}(x_{n_{i-1}},x_{n_i}) $\footnote{For $ i=1 $, just define as $ 0 $}.
\end{enumerate}
    
\end{algorithm}
    

    Then break the ordering sequence $ n_i $ according to $ r(x_{n_i}) $, .e.g. break $ n_i $ if $ r(x_{n_i})\geq \tilde{\varepsilon } $

    Comment: OPTICS is more stable than DBSCAN, capable of dealing with multi-density clustering.






\newpage

\section{数据科学导论部分}\label{SecDataScience}
\begin{center}
    Instructor: Sheng Yu
\end{center}

\begin{point}
    Road to Data Scientist
\end{point}

\begin{figure}[H]
    \centering
    \includegraphics[width=\linewidth]{pic/RoadToDataScientist1.png}
    \caption{Road to Data Scientist}
    \label{RoadToDataScience}
\end{figure}


Comparison of \lstinline|R|, \lstinline|python|:focus on different aspects of `Statistics':
\begin{itemize}[topsep=2pt,itemsep=0pt]
    \item Differnece in programming philosophy: \lstinline|R| for data analysis and \lstinline|python| for data processing
    \item Difference in operating domain: \lstinline|R| for statistical programming while \lstinline|python| for general programming.
\end{itemize}



\subsection{Basic R. Manipulation}


\subsubsection{Installation and Maintenance of R.}

\noindent Installing and updating \lstinline|R.|: update by delete old version and install new version.
\begin{itemize}[topsep=2pt,itemsep=0pt]
    \item In CRAN (The Comprehensive \lstinline|R| Archive Network):\index{CRAN (The Comprehensive R Archive Network)} \url{https://cran.r-project.org}
    \item In Mirror@TUNA: \url{https://mirrors.tuna.tsinghua.edu.cn/CRAN}
\end{itemize}

\noindent Installing and updating RStudio: \url{https://www.rstudio.com}


\noindent Running \lstinline|R.| command:
\begin{itemize}[topsep=2pt,itemsep=0pt]
    \item In \lstinline|R.| GUI\index{GUI (Graphical User Interface)};
    \item In \lstinline|R.| command line terminal;
    \item \lstinline|R. CMD BATCH|;
    \item \lstinline|Rscript|;
        \begin{itemize}[topsep=2pt,itemsep=0pt]
        \item Use \lstinline|>| to redirect output(overwrite);
        \item Use \lstinline|>>| to append output.
        \end{itemize}
\end{itemize}



\noindent \lstinline|R.| package library: packages are collection of \lstinline|R.| functions (as well as test data and sample code).
\begin{itemize}[topsep=2pt,itemsep=0pt]
    \item \lstinline|.libPaths()| show package library location\footnote{Unlike in \lstinline|C| or \lstinline|python| where \lstinline|.| is an operator, \lstinline|.| in \lstinline|R.| is just a common character, without special meaning.
    
    This feature can be used in naming self-defined functions: use \lstinline|.FUN_NAME1| for within-project function while \lstinline|FUN_NAME2| for external interface.} ;
    \item \lstinline|library('PACKAGE_NAME1','PACKAGE_NAME2',...)| load packages.
    \item \lstinline|install.packages('PACKAGE_NAME1','PACKAGE_NAME2',...)| install package from CRAN/mirrors;
    \item \lstinline|installed.packages()| show all installed packages;
    \item \lstinline|updata.packages(checkBuilt = TRUE, ask = FALSE)| update installed packages;
\end{itemize}

\noindent Working Directory manipulation:
\begin{itemize}[topsep=2pt,itemsep=0pt]
    \item \lstinline|getwd()| get current working directory;
    \item \lstinline|setwd('TARGET_PATH')| set working directory as an existing path.  
\end{itemize}

\noindent Recommended \lstinline|R.| project organization: working directory organized like
\begin{itemize}[topsep=2pt,itemsep=0pt]
    \item \lstinline|data/| folder for structured original dataset;
    \item \lstinline|result/| folder for output result;
    \item \lstinline|presentation/| folder for result representing slides/reports/etc.;
    \item \lstinline|.r| project file $ \times n $.
\end{itemize}



\subsubsection{Data Manipulation in R.}

\begin{point}
    Looking for help/example of function:
\end{point}

\begin{itemize}[topsep=2pt,itemsep=0pt]
    \item \lstinline|?FUN_NAME()|;
    \item \lstinline|help('FUN_NAME')|;
\end{itemize}





\begin{point}
    Atomic Classes
\end{point}
\begin{itemize}[topsep=2pt,itemsep=0pt]
    \item \lstinline|'abc'| Character;
    \item \lstinline|3L| Integer;
    \item \lstinline|2.4| Numeric;
    \item \lstinline|TRUE,FALSE,T,F| Logical;
    \item Special types: \lstinline|NA|, \lstinline|NaN|, \lstinline|NULL|, \lstinline|Inf|
\end{itemize}

\begin{point}
    Operators
\end{point}
\begin{itemize}[topsep=2pt,itemsep=0pt]
    \item Numerical Operators: \lstinline|+|,\lstinline|-|,\lstinline|*|(multiply by column),\lstinline|/|,\lstinline|%*%|(matrix multiply),\lstinline|^|;
    \item Logical Operators: \lstinline|==|,etc.; \lstinline|&| and \lstinline{|} for common operator, \lstinline|&&| and \lstinline{||} for comparing the first element;
    \item Round a numeric:
    \begin{itemize}[topsep=2pt,itemsep=0pt]
        \item \lstinline|as.integer()|, round towards 0
        \item \lstinline|trunc()|
        \item \lstinline|ceiling()|
        \item \lstinline|floor()|
        \item \lstinline|round(NUMBER_TO_ROUND,digits = DIGITS)|
    \end{itemize}
    
        
\end{itemize}


\begin{point}
    Data Structure
\end{point}
\begin{itemize}[topsep=2pt,itemsep=0pt]
    \item Vector: Column vector is the \textbf{basic} data structure in \lstinline|R.| (scalar is length=1 vector).
    
    Only data of the same class can be held in one vector.
    
    Initialization:
    \begin{itemize}[topsep=2pt,itemsep=0pt]
        \item Ordinary way: 
        \begin{itemize}[topsep=2pt,itemsep=0pt]
            \item \lstinline|c(1,2,3)|, \lstinline|c(T,FALSE,TRUE)|, \lstinline|c('a',NA,'b')|
            \item \lstinline|vector(mode = MODE,length = LENGTH)|
        \end{itemize}
        
        where \lstinline|c()| for `combine'; 
        
        \lstinline|c()| combines all things into one vector, e.g. \lstinline|c(c(1,2,3),c(1,2))=(1,2,3,4,5)|.
        \item Sequence vector: 
        \begin{itemize}[topsep=2pt,itemsep=0pt]
            \item \lstinline|1:3.5=c(1,2,3)|, \lstinline|3:1=c(3,2,1)|
            \item \lstinline|seq(from, to ,by, length.out)|, \lstinline|length.out| for total vector length;
            \item \lstinline|rep(SEQ_TO_REP, times, lenght.out ,each)|, used in $ k $-fold cross validation labelling.
        \end{itemize}
    \end{itemize}
    
    Operations:
    \begin{itemize}[topsep=2pt,itemsep=0pt]
        \item between vectors of different length \lstinline|SHORT| and \lstinline|LONG|: First \lstinline|SHORT <- rep(SHORT,|\\\lstinline| length.out=length(LONG))|. Then operate \lstinline|SHORT| and \lstinline|LONG|.
        \item Element access: \lstinline|a[i]|
    \end{itemize}
    
         
    \fbox{
        \begin{minipage}{0.9\linewidth}

    \textbf{Vectorized Operation}: All operation in \lstinline|R.| are based on vector, and vectorized operation is \textbf{Parallel Arithmetic}, which is \textbf{much faster} than loop such as \lstinline|for| $ \longrightarrow $ Consider using vectorized opertion when writing code for \textbf{Speed}! 
        \end{minipage}
    }

    \item Factor: A special kind of `vector' in \lstinline|R.|, used to label discrete categorical data.\footnote{Factor vector is stored as integer vector.}
    
    Initialization:

    \lstinline|factor(FACTOR_SEQ, levels = FACTOR_LEVEL, labels = ...)|, \lstinline|FACTOR_LEVEL| is the `rank' of each factor, \lstinline|labels| is the `tag' of levels. 

    \item Matrix: Only data of the same class can be held in one matrix.
    
    Initilaization:
    
    \lstinline|matrix(DATA_SEQ, nrow, ncol, byrow = FALSE|
    
    If \lstinline|length(DATA_SEQ) < nrow*ncol|, then \lstinline|DATA_SEQ| is repeated. Default: fill by column (because matrix is stored by column).

    Operation:
    \begin{itemize}[topsep=2pt,itemsep=0pt]
        \item Common operators \lstinline|+-*/^| etc. operate in column-by-column mode (vectorized operation).
        \item Binding matrix: \lstinline|cbind| for \lstinline|[A,B]| and \lstinline|rbind| for \lstinline|[A;B]|
        \item Transpose: \lstinline|t()|
        \item Matrix multiplication: \lstinline|%*%|
        \item Inverse matrix: \lstinline|solve()| (The essence of inversion is solving linear equations)
        \item Diagonal matrix:
        \begin{itemize}[topsep=2pt,itemsep=0pt]
            \item \lstinline|diag(VECTOR)| returns a matrix $ \mathrm{diag}\{ $\lstinline|VECTOR|$ \} $
            \item \lstinline|diag(MATRIX)| returns the diagonal element vector
        \end{itemize}
        \item Element access: \lstinline|a[i,j]|, \lstinline|a$OBJECT_NAME|
        \item Dimension of matrix: \lstinline|dim()|, \lstinline|nrow()|, \lstinline|ncol()|
    \end{itemize}
    
    \item List: A pack containing various datatype
    
    Initialization: \lstinline|list(OBJECT1,OBJECT2,...)|

    Element access: \lstinline|a[[i]]|
    \item \lstinline|data.frame|: `Mixture' of matrix and list. \lstinline|data.frame| is actually a kind of list(with some constraint), organized in the shape of matrix (but allowing different datatype for different columns, each column is a list object).
    
    Each column of \lstinline|data.frame| has name: \lstinline|names(DATA_FRAME)|, \lstinline|colnames(DATA_FRAME)|

    \item Element access: \lstinline|a[i,j]|, \lstinline|a[[i]]|, \lstinline|a$COL_NAME|
\end{itemize}

    

    
    
        

    





    




     




    

    









\newpage

\addcontentsline{toc}{section}{参考文献}
\begin{thebibliography}{99}
    \bibitem{讲义}
    清华大学统计学研究中心 辅修课程课件与讲义. W. Deng, J. Wang, Z. Zhou, D. Li, T. Wang, S. Yu, P. Yang.
    \bibitem{Springer参考书系列}
    Springer Series in Statistics (SSS). \url{https://www.springer.com/series/692}
    \bibitem{RStudioCheatSheets}
    RStudio Cheatsheets \url{https://www.rstudio.com/resources/cheatsheets}
    \bibitem{概率论ref1}
    概率导论(第二版·修订版). Dimitri P. Bertsekas, John N. Tsitsiklis. 人民邮电出版社.
    \bibitem{概率论ref2}
    北京大学《概率统计A》课程讲义. 李东风. \url{https://www.math.pku.edu.cn/teachers/lidf/course/probstathsy/probstathsy.pdf}
    \bibitem{统计推断ref1}
    数理统计(第二版). 韦来生. 科学出版社.
    \bibitem{统计推断ref2}
    Statistical Inference(2nd Edition). George Casella, Roger L. Berger. Duxbury Press.
    \bibitem{线性回归分析ref1}
    Applied Linear Statistical Models(5th Edition). Michael H. Kutner, Christopher J. Nachtsheim, John Neter, William Li. McGraw-Hill Compaines, Inc.
    \bibitem{线性回归分析ref2}
    线性模型引论. 王松桂 et. al. 科学出版社.
    \bibitem{线性回归分析ref3}
    Linear Models with R(2nd Edition). Julian J. Faraway. CRC Press.
    \bibitem{线性回归分析ref4}
    Generalized Linear Model Lecture Note. Germán Rodríguez. \url{https://data.princeton.edu/wws509/notes}
    \bibitem{多元统计分析ref1}
    实用多元统计分析(第六版). Richard A. Johnson, Dean W. Wichern. 清华大学出版社.
    \bibitem{数据科学导论ref1}
    R In Action: Data Analysis and Graphics with R(2nd Edition). Robert I. Kabacoff. Manning Publications Co.
    \bibitem{数据科学导论ref2}
    R Programming For Data Science. Roger D. Peng. Lean Publishing.
    \bibitem{统计计算与软件ref1}
    Numerical Linear Algebra. I Lloyd N. Trefethen, David Bau Ill. Society for Industrial and Applied Mathematics
    \bibitem{统计计算与软件ref2}
    Numerical Optimization(2nd Edition). J. Nocedal, Stephen J. Wright. Springer Science+Business Media, LLC. 
    \bibitem{统计计算与软件ref3}
    北京大学《统计计算》课程讲义. 李东风. \url{https://www.math.pku.edu.cn/teachers/lidf/docs/statcomp/html/_statcompbook/statcomp2ndv.pdf}
    \bibitem{生存分析ref1}
    生存分析与可靠性. 陈家鼎. 北京大学出版社.
    \bibitem{机器学习ref1}
    机器学习. 周志华. 清华大学出版社.
    \bibitem{机器学习ref2}
    机器学习公式详解. 谢文睿, 秦州. 人民邮电出版社.
    \bibitem{机器学习ref3}
    神经网络与深度学习. 邱锡鹏. \url{https://nndl.github.io/}
    \bibitem{时间序列ref1}
    Time Series Analysis With Applications in R(2nd Edition). Jonathan D. Cryer, Kung-Sik Chan. Springer Science+Business Media, LLC.
    \bibitem{时间序列ref2}
    北京大学《应用时间序列分析》课程讲义. 李东风. \url{https://www.math.pku.edu.cn/teachers/lidf/course/atsa/atsanotes/html/_atsanotes/atsanotes.pdf}
    \bibitem{时间序列ref3}
    Forecasting: Principles and Practice (2nd Edition). Hyndman, R.J., Athanasopoulos, G. \url{https://otexts.com/fppcn}
    \bibitem{因果推断ref1}
    Causal Inference for Statistics, Social, and Biomedical Sciences: An Introduction. Guido W. Imbens \& Donald B. Rubin. Cambridge University Press.
    \bibitem{因果推断ref2}
    Causal Inference in Statistics - A Primer. Judea Pearl. Wiley.
    \bibitem{随机过程ref1}
    Random Processes for Engineers. Bruce Hajek. Cambridge University Press
    \bibitem{随机过程ref2}
    应用随机过程. 林元烈. 清华大学出版社.


\end{thebibliography}
\newpage

\newpage
\addcontentsline{toc}{section}{索引}
\printindex


\end{document}


